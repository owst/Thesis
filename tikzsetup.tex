\pgfdeclarelayer{background}
\pgfdeclarelayer{foreground}
\pgfsetlayers{background,main,foreground}

\def\invertTFtrue{false}
\def\invertTFfalse{true}
% Allow us to turn off overlay, with overlay=false
\tikzoption{overlay}[true]{\csname pgf@relevantforpicturesize\csname
invertTF#1\endcsname\endcsname}

\tikzset{
    overlabel/.style={label={[overlay]#1}}
}

\tikzstyle{commarr}=[ ->, shorten <=0.75cm, shorten >=0.75cm, >=stealth', semithick ]

\tikzstyle{dummy}=[inner sep=0pt]

\tikzstyle{hider}=[draw=white, arrows=-, line width=4pt]

\tikzstyle{box}=
    [rectangle, rounded corners, thick, draw=blue!50!black,
       text height=1ex, text depth=0.1ex, minimum size=4ex,
       top color=white, bottom color=blue!15]

\tikzstyle{petriplace}=[circle, draw, inner sep=0pt, minimum size=6mm]
\tikzstyle{petritransition}=[ rectangle, draw=black!50 %, fill=black!20
                            , inner sep=0pt , minimum height=7mm
                            , minimum width=1.8mm
                            ]
\tikzstyle{petriarr}=[ ->, shorten <=1pt, >=stealth', semithick ]

%\tikzstyle{pnbplace}=[circle, draw, inner sep=0pt, minimum size=6mm]
%\tikzstyle{pnbarr}=[ -, semithick, shorten <=1pt, shorten >=1pt]

\tikzstyle{loose1.5}=[looseness=1.5]

%\tikzstyle{nwbjoin}=
%    [rectangle, fill, inner sep=0pt , minimum height=3mm, minimum width=0.6mm]

\tikzstyle{cross at}=
    [postaction={
        decorate,
        decoration={
            markings,
            mark=at position #1 with {\draw[-, double=black, white, semithick] (-0.05,0) -- (0.05,0);}
        }
    }]

\tikzstyle{mark inside}=
    [postaction={
        decorate,
        decoration={
            markings,
            mark=at position #1 with {\draw[-, solid] (0,-0.12) -- (0,0.12);}
        }
    }]

\newcommand{\marknode}[1]{\draw[-] (#1.center) -- ++(-0.12,0) -- ++(0.24,0);}

\tikzstyle{pn}=[node distance=1cm]
\tikzstyle{pnb}=[node distance=2cm]
\tikzstyle{nfa}=[ ->, >=stealth', shorten >=1pt, auto, node distance=2.5cm
                , semithick, inner sep=0, initial text=]
\tikzset{init/.style={ initial text=, initial above}}
%, every initial by arrow/.style={overlay}}

\tikzstyle{boundaryPort}=[fill,shape=rectangle,inner sep=0pt, outer sep=0pt,
bPortSize=0.2cm]
\tikzstyle{boundaryPortHidden}=[boundaryPort, draw=none, fill=none]
\tikzset{bPortSize/.append style={minimum width=#1, minimum height=#1}}
\tikzstyle{dotsBPort}=[boundaryPort, minimum height=0.5cm, fill=white, label={[rotate=90]center:$\dots$}]

% Taken from http://tex.stackexchange.com/questions/129547/tikz-surrounding-box-with-automatically-drawn-border-ports/129555?noredirect=1#129555
% #1 = y padding
% #2 = x padding
% #3 = left ports
% #4 = right ports
\DeclareDocumentCommand{\drawBoundaries}{ O{0.8} O{0.8} m m }{
    \node[inner sep=0pt, outer sep=0pt] (foo1) at ($(current bounding box.south west)+(-#1,-#2)$) {};
    \node[inner sep=0pt, outer sep=0pt] (foo2) at ($(current bounding box.north east)+(#1,#2)$) {};

    \node[draw, fit=(foo1) (foo2), outer sep=0pt] (bbox) {};

    % coordinates of bounding box
    \drawBoundaryPortsOnInternal{bbox}{#3}{#4}{}
}

\newcommand{\drawBoundaryPortsOnInternal}[4]{
    % Define these once, since e.g. when we use the bounding box to draw the
    % boundaries, the bounding box can grow if we add more nodes after drawing
    % the boundaries
    \coordinate(#1-nw)at(#1.north west);
    \coordinate(#1-ne)at(#1.north east);
    \coordinate(#1-sw)at(#1.south west);
    \coordinate(#1-se)at(#1.south east);

    \drawLRBoundaryPorts{#2}{#3}{#4}{#1}
}

\newcommand{\drawBoundaryPortsOn}[3]{%
    \drawBoundaryPortsOnInternal{#1}{#2}{#3}{#1-}
}

% Draws boundary ports, assuming coordinates nw,sw,ne and se to delimit
% "border" bounding box.
\newcommand{\drawLRBoundaryPorts}[4]{%
    \drawPorts{#1}{#4-nw}{#4-sw}{#3l}
    \drawPorts{#2}{#4-ne}{#4-se}{#3r}
}

% Draws #1 ports between top, #2 and bottom, #3 with prefix #4.
\newcommand{\drawPorts}[4]{%
    \ifnum#1=0,,
    \else
    \foreach \portloopvar in {1,...,#1} {
        \pgfmathparse{\portloopvar/(#1+1)}
        % Generate a style for every port, such that we can customise
        % individual boundary ports.
        \tikzset{boundaryPort-#4\portloopvar/.append style={}}
        \node(#4\portloopvar)[boundaryPort, boundaryPort-#4\portloopvar]at($(#2)!\pgfmathresult!(#3)$){};
    }
    \fi
}

\newcommand{\connectedBetweenL}[3]{%
    \rPortAtHeightOf{#1}{#2}{#3}
    \draw (#1-r#3) -- (#2);
}

\newcommand{\connectedBetweenR}[3]{%
    \lPortAtHeightOf{#1}{#2}{#3}
    \draw (#2) -- (#1-l#3);
}


% 1 = node name (optional)
% 2 = from
% 3 = to
% 4 = label
% 5 = edge properties
% 6 = node properties
\DeclareDocumentCommand{\labelledpnbarr}{ O{joinnode} m m m m m }{
    \draw (#2) edge[pnbarr,o0i180,#5] node (#1) [pnbjoin,pnblabel=#4,#6] {} (#3);
}

\DeclareDocumentCommand{\pnbarr}{ O{joinnode} m m }{
    \labelledpnbarr[#1]{#2}{#3}{}{}{}
}

% 1 = edge style (optional)
% 2 = node style (optional)
% 3 src
% 4 tgt
% 5 left label component
% 6 left label component
\DeclareDocumentCommand{\nfaArr}{ O{} O{} m m m m }{%
    \draw (#3) edge [#1] node [#2] {$\lbl{#5}{#6}$} (#4);
}

\def\mainsize{3cm}

\tikzset{schematic/.append style={node distance=0.4cm,
                                  boundaryPort/.append style={bPortSize=0.125cm}}}

\tikzstyle{schematicnode}=[draw, rectangle, inner sep=0.25cm, outer sep=0pt
                          , minimum height=1.3cm]

\tikzset{alignbot/.style={right=of #1.south east, anchor=south west}}
\tikzset{aligntop/.style={right=of #1.north east, anchor=north west}}
\tikzset{alignright/.style={below=of #1.south east, anchor=north east}}
\tikzset{alignleft/.style={below=of #1.south west, anchor=north west}}

\tikzset{
    slimmer head/.style args={#1 at #2}{
        -triangle 45, line width=#1*.4, #2,
        postaction={draw, line width=#1, shorten >=0.4cm, -}
    }
}

\newcommand{\translateArr}[3][]{\draw[-stealth, line width=0.15cm, gray!40] (#2) to (#3);}

% From: http://tex.stackexchange.com/a/64454
\makeatletter
\tikzset{
    samewidthas/.style args={#1 and #2}{
    samewidthasc={#1}{#2},minimum width=\mylength
    },
    samewidthasc/.code 2 args={
        \pgfpointdiff{\pgfpointanchor{#1}{west}}{\pgfpointanchor{#2}{east}}
        \xdef\mylength{\the\pgf@x}
    }
}
\makeatother

\newcommand\boundaryAlignedWith[4]{
    \path let \p1 = (#1), \p2=(#2) in node (#3#4) [boundaryPort] at (\x1,\y2) {};
}
\newcommand\rBAlignedWith[2]{\boundaryAlignedWith{bbox-ne}{#1}{r}{#2}}
\newcommand\lBAlignedWith[2]{\boundaryAlignedWith{bbox-nw}{#1}{l}{#2}}

% percentage, num
\newcommand{\rBPortAt}[2]{
    \bPortAtHelper{#1}{#2}{bbox-ne}{r}
}
\newcommand{\lBPortAt}[2]{
    \bPortAtHelper{#1}{#2}{bbox-nw}{l}
}

\newcommand{\bPortAtHelper}[4]{
    \portAtHeightOf{#3}{$(bbox-nw)!#1!(bbox-sw)$}{#4#2}
}

% node, percentage, num
\newcommand{\leftBPort}[3]{
    \coordinate(nw)at(#1.north west);
    \coordinate(sw)at(#1.south west);
    \node(#1-l#3)[boundaryPort]at($(nw)!#2!(sw)$){};
}
\newcommand{\rightBPort}[3]{
    \coordinate(ne)at(#1.north east);
    \coordinate(se)at(#1.south east);
    \node(#1-r#3)[boundaryPort]at($(ne)!#2!(se)$){};
}

% x-pos, height of, name
\newcommand{\portAtHeightOf}[3]{
    \path let \p1=(#1),\p2=(#2) in node (#3)[boundaryPort] at (\x1, \y2) {};
}

\newcommand{\rPortAtHeightOf}[3]{
    \portAtHeightOf{#1.east}{#2}{#1-r#3}
}

\newcommand{\lPortAtHeightOf}[3]{
    \portAtHeightOf{#1.west}{#2}{#1-l#3}
}

\makeatletter
\pgfdeclareshape{nwbplaceshape}
{
    \inheritsavedanchors[from=circle]
    \inheritanchorborder[from=circle]
    \inheritanchor[from=circle]{center}
    \inheritanchor[from=circle]{north west}
    \inheritanchor[from=circle]{north}
    \inheritanchor[from=circle]{north east}
    \inheritanchor[from=circle]{east}
    \inheritanchor[from=circle]{south east}
    \inheritanchor[from=circle]{south}
    \inheritanchor[from=circle]{south west}
    \inheritanchor[from=circle]{west}
    \inheritbackgroundpath[from=circle]
    \anchor{in}{%
        \centerpoint
        \pgf@xa=\radius
        \advance\pgf@x by-1.18\pgf@xa
    }
    \anchor{out}{%
        \centerpoint
        \pgf@xa=\radius
        \advance\pgf@x by1.23\pgf@xa
    }
    \foregroundpath{
       \centerpoint%
       \pgf@xc=\pgf@x % \pgf@xc = x value of centerpoint
       \pgf@yc=\pgf@y % \pgf@yc = y value of centerpoint
       \pgfutil@tempdima=\radius% \pgfutil@tempdima = radius
       % top left of triangle pointing out l -> r
       \pgfpathmoveto{\pgfpointadd{\pgfqpoint{\pgf@xc}{\pgf@yc}}
            {\pgfqpoint{0.8\pgfutil@tempdima}{-0.25\pgfutil@tempdima}}}
       % bottom left of triangle pointing out l -> r
       \pgfpathlineto{\pgfpointadd{\pgfqpoint{\pgf@xc}{\pgf@yc}}
            {\pgfqpoint{0.8\pgfutil@tempdima}{0.25\pgfutil@tempdima}}}
       % right point of triangle pointing out l -> r
       \pgfpathlineto{\pgfpointadd{\pgfqpoint{\pgf@xc}{\pgf@yc}}
            {\pgfqpoint{1.3\pgfutil@tempdima}{0mm}}}
       \pgfpathclose
       % top left of triangle pointing in l -> r
       \pgfpathmoveto{\pgfpointadd{\pgfqpoint{\pgf@xc}{\pgf@yc}}
            {\pgfqpoint{-1.2\pgfutil@tempdima}{-0.25\pgfutil@tempdima}}}
       % bottom left of triangle pointing in l -> r
       \pgfpathlineto{\pgfpointadd{\pgfqpoint{\pgf@xc}{\pgf@yc}}
            {\pgfqpoint{-1.2\pgfutil@tempdima}{0.25\pgfutil@tempdima}}}
       % right point of triangle pointing in l -> r
       \pgfpathlineto{\pgfpointadd{\pgfqpoint{\pgf@xc}{\pgf@yc}}
            {\pgfqpoint{-0.7\pgfutil@tempdima}{0mm}}}
       \ifx\tikz@strokecolor\pgfutil@empty
       \def\my@strokecolor{black}
       \else
       \let\my@strokecolor\tikz@strokecolor
       \fi \pgfpathclose
       \pgfsetfillcolor{\my@strokecolor}
       \pgfusepath{fill}
    }
}
\makeatother

\makeatletter
\pgfdeclareshape{pnbjoinshape}
{
    \inheritsavedanchors[from=rectangle]
    \inheritanchor[from=rectangle]{center}
    \inheritanchor[from=rectangle]{north west}
    \inheritanchor[from=rectangle]{north}
    \inheritanchor[from=rectangle]{north east}
    \inheritanchor[from=rectangle]{east}
    \inheritanchor[from=rectangle]{south east}
    \inheritanchor[from=rectangle]{south}
    \inheritanchor[from=rectangle]{south west}
    \inheritanchor[from=rectangle]{west}

    \inheritbackgroundpath[from=rectangle]
    \inheritforegroundpath[from=rectangle]
}
\makeatother

 % TODO: tweak anchors so that they are on the triangle west/east
 %       align center of triangles on border of place
\makeatletter
\tikzset{pnbplace/.style ={on grid, nwbplaceshape, draw, inner sep=0pt, minimum size=6mm}}

\tikzset{every node/.add code={}{\def\nodedist{\tikz@node@distance}}}
\def\nodedist{\tikz@node@distance}
\makeatother

\tikzset{
    pnblabel/.style={
        label={#1},
        prefix after command= {\pgfextra{\tikzset{every label/.append style={opacity=1, overlay, black!60}}}}
    }
}

\tikzstyle{pnbhidden}=[pnbplace, stroke=white, draw=white, fill=white]

\tikzset{pnbarr/.style ={-, semithick}}
\tikzset{pnbarrstyle1/.append style={pnbarr,densely dotted}}
\tikzset{pnbarrstyle2/.append style={pnbarr,densely dashdotted}}
\tikzset{pnbarrstyle3/.append style={pnbarr,loosely dotted}}
\tikzset{pnbarrstyle4/.append style={pnbarr,dashed}}
% make "boundary" arrows lighter
\tikzset{barr/.append style={pnbarr, opacity=0.3}}
%Problems with meeting points when using this
\tikzset{overlap/.style ={white, double=black, thin}}
\newcommand{\joinwidth}{7pt}
\tikzstyle{pnbjoin}=[pnbjoinshape, draw, solid, sloped, inner sep=0pt, rotate=90, minimum height=0pt, minimum width=\joinwidth, pos=0.5]

% For some reason this needs a minimum height that isn't 0 whereas nwbjoin
% doesn't
\tikzstyle{nwbjoinExtra}=[fill, inner sep=0pt, minimum height=0.5pt, minimum width=\joinwidth]

\tikzstyle{nfa}=[->,>=stealth', shorten >=1pt, auto, node distance=2.1cm, semithick, state/.append style={inner sep=0.02cm}]

\tikzstyle{brace}=[decoration={brace}, decorate]
\tikzstyle{mirrorbrace}=[brace, decoration={mirror}]
%
\tikzstyle{nwblbl}=[black!70,scale=0.85]

\tikzstyle{boundarylbl}=[nwblbl,overlay]

% Draw labels on boundary ports
\newcommand{\lBoundaryLabel}[2]{\node (blbl) [label={[boundarylbl]west:#2}] at (l#1) {};}
\newcommand{\rBoundaryLabel}[2]{\node (blbl) [label={[boundarylbl]0:#2}] at (r#1) {};}

\tikzstyle{wantedTokenPlace}=[pattern=north east lines, pattern color=gray!50]

\newcommand{\drawBuf}{
    \pgfmathtruncatemacro\prev{int(\x-1)}

    \node[petriplace] (p\x0) [above right of=t\prev, tokens=1] {};
    \node[petriplace,wantedTokenPlace] (p\x1) [below right of=t\prev] {};

    \node[petritransition] (t\x) [below right of=p\x0, rotate=90] {}; %, label={above:$t_\x$}] {};

    \draw [petriarr] (t\prev) to [bend left=30] (p\x0);
    \draw [petriarr] (p\x1) to [bend left=30] (t\prev);
    \draw [petriarr] (t\x) to [bend left=30] (p\x1);
    \draw [petriarr] (p\x0) to [bend left=30] (t\x);
}

\tikzset{wanted style/.style={}}

% #1: draw marked?
% #2: draw labels?
% #3: draw boundary ports?
% #4: buffer count
\DeclareDocumentCommand{\nBufPNB}{ O{0} O{0} O{0} m }{
{%
    \begin{tikzpicture}[pnb, place label/.style={}]

    \ifthenelse{#1 = 1}{%
        \tikzset{wanted style/.style=wantedTokenPlace}
    }{}

    \ifthenelse{#2 = 1}{%
        \tikzset{place label/.style={pnblabel=above:##1}}
    }{}

    \node[pnbplace] (p10) [tokens=1, place label=$\aPlace_1$]{};
    \node[pnbplace] (p11) [wanted style, below of=p10, rotate=180, place label=$\aPlace_2$] {};

    \def\tlblA{}
    \def\tlblB{}
    \ifthenelse{#2 = 1}{%
        \def\tlblA{$\aTrans_2$}
        \def\tlblB{$\aTrans_1$}
    }{}

    \labelledpnbarr[p10p11]{p10.out}{p11.in}{below left:\tlblA}{out=0, in=0}{}
    \labelledpnbarr[p11p10]{p11.out}{p10.in}{below left:\tlblB}{out=180, in=180}{}

    \ifthenelse{#4 > 1}{%
    \foreach \x in {2, ..., #4} {%
        \pgfmathtruncatemacro\prev{int(\x-1)}
        \pgfmathtruncatemacro\tname{int(\x + 1)}
        \pgfmathtruncatemacro\pname{int((\x - 1) * 2 + 1)}
        \pgfmathtruncatemacro\nextpname{int(\pname + 1)}

        \node[pnbplace] (p\x0) [right of=p\prev0, tokens=1, place label=$\aPlace_\pname$] {};
        \node[pnbplace] (p\x1) [wanted style, below of=p\x0, rotate=180, place label=$\aPlace_\nextpname$]{};

        \def\tlbl{}
        \ifthenelse{#2 = 1}{%
            \def\tlbl{$\aTrans_\tname$}
        }{}

        \labelledpnbarr[p\x0p\x1]{p\x0.out}{p\x1.in}{below left:\tlbl}{out=0, in=0}{}

        \draw (p\x1.out) edge[pnbarr, out=180, in=-90] (p\prev0p\prev1);
        \draw (p\x0.in) edge[pnbarr, out=180, in=90] (p\prev0p\prev1);
    }
    }{}

    \ifthenelse{#3 = 1}{%
        \drawBoundaries{1}{1}
        \draw (p11p10) edge[barr, out=90, in=0] (l1);
        \draw (p#40p#41) edge[barr, out=90, in=180] (r1);
    }{%
        \drawBoundaries{0}{0}
    }
\end{tikzpicture}
}
}

% Spacing for "tree-like" PNBs
\newcommand{\levelSepA}{1cm}
\newcommand{\levelSepB}{0.6cm}

\tikzset{wiringTree/.style={level distance=1.5cm,sibling distance=0.2cm,
every tree node/.append style={anchor=north}}}

\tikzset{loop below right/.style={looseness=8, min distance=5mm, out=-30, in=-60}}
\tikzset{loop below left/.style={looseness=8, min distance=5mm, out=-120, in=-150}}

% BDDs
\newcommand{\bddedge}[3]{\draw (#1) edge [->, shorten <=3pt, shorten >=3pt, #3] (#2.north);}
\newcommand{\falseEdge}[3][]{\bddedge{#2}{#3}{dashed,#1}}
\newcommand{\trueEdge}[3][]{\bddedge{#2}{#3}{solid,#1}}

\tikzset{bdd/.style={node distance=2cm
        ,every edge/.append style={}
        ,var/.append style={draw, circle, inner sep=0.05cm}%
        ,leaf/.append style={draw, rectangle, minimum height=0.75cm}}}

% between nodes
\newcommand{\bn}[3]{($(#1)!#2!(#3)$)}


\makesavebox{\lendOneBox}{%
\begin{tikzpicture}[pnb]
    \node[pnbhidden] {};
    \drawBoundaries{0}{1}

    \coordinate (l1) at ([xshift=-0.5 * \nodedist]r1);
    \labelledpnbarr{l1}{r1}{}{barr}{}
\end{tikzpicture}
}
\makesavebox{\rendOneBox}{%
\begin{tikzpicture}[pnb]
    \node[pnbhidden] {};
    \drawBoundaries{1}{0}

    \coordinate (r1) at ([xshift=0.5 * \nodedist]l1);
    \labelledpnbarr{l1}{r1}{}{barr}{}
\end{tikzpicture}
}
\makesavebox{\rtermOneBox}{%
\begin{tikzpicture}[pnb]
    \node[pnbhidden] {};
    \drawBoundaries{1}{0}
\end{tikzpicture}
}

\makesavebox{\bufferBox}{%
\begin{tikzpicture}[pnb]
    \node[pnbplace] (p0) [tokens=1, pnblabel=above:$\aPlace_0$] {};
    \node[pnbplace] (p1) [wantedTokenPlace, below=of p0, rotate=180, pnblabel=above:$\aPlace_1$] {};

    \drawBoundaries[1.2]{1}{1}

    \foreach \from/\to/\side/\in/\out in {p0/p1/r1/180/0, p1/p0/l1/0/180}{%
        \labelledpnbarr[\from\to]{\from.out}{\to.in}{}{out=\out, in=\out}{}
        \draw (\from\to) edge[barr, out=90, in=\in] (\side);
    }
\end{tikzpicture}
}

\newcommand{\disLeaf}[1]{%
\begin{tikzpicture}[pnb]
    \node (hidden) [pnbhidden] {};
    \ifthenelse{#1 = 1}{%
        \tikzset{wanted style/.style=wantedTokenPlace}
    }{}

    \node (tok) [pnbplace, wanted style, yshift=0.5\nodedist, below=of hidden] {};

    \drawBoundaries{0}{0}
    \lBAlignedWith{hidden}{1}
    \rBAlignedWith{hidden}{1}

    \labelledpnbarr{l1}{tok.in}{}{barr}{}
    \labelledpnbarr{l1}{r1}{}{barr}{}
\end{tikzpicture}
}

\makesavebox{\disNodeBox}{%
\begin{tikzpicture}[pnb]
    \node (hidden) [pnbhidden] {};
    \node (tok) [pnbplace, yshift=0.5\nodedist, below=of hidden] {};

    \drawBoundaries{0}{0}

    \lBAlignedWith{hidden}{1}
    \rBAlignedWith{hidden}{1}
    \rBAlignedWith{tok}{2}

    \labelledpnbarr{l1}{r1}{}{barr}{}
    \labelledpnbarr{l1}{tok.in}{}{barr}{}
    \labelledpnbarr{tok.out}{r2}{}{barr}{}
\end{tikzpicture}
}

% #1 = 1: is bigger row (first node goes left), 0: smaller row, 2: same row
% #2 = rowid
% #3 = row size
\newcommand{\nodeRow}[3][1]{%
    \pgfmathtruncatemacro{\prevrow}{#2 - 1}

    \ifthenelse{#1=1}{
        \node (#20) [state, below left=of \prevrow0] {};
    }{\ifthenelse{#1=0}{
        \node (#20) [state, below right=of \prevrow0] {};
        \nfaArr{\prevrow1}{#20}{1}{0}
    }{
        \node (#20) [state, below=of \prevrow0] {};
    }}

    \nfaArr{\prevrow0}{#20}{1}{0}
    \nfaArr[loop left, overlay]{#20}{#20}{\aLbl}{\aLbl}

    \foreach \nodeId in {1,...,#3}{%
        \pgfmathtruncatemacro{\prevNodeId}{\nodeId - 1}

        \ifthenelse{#1=0}{
            \def\belowof{\prevrow\nodeId}
        }{
            \def\belowof{\prevrow\prevNodeId}
        }

        \ifthenelse{#1=0}{
            \pgfmathtruncatemacro{\added}{\nodeId + 1}
            \def\prevRowNext{\prevrow\added}
        }{
            \def\prevRowNext{\prevrow\nodeId}
        }

        \ifthenelse{#1=2}{
            \node (#2\nodeId) [state, below=of \prevrow\nodeId] {};
        }{
            \node (#2\nodeId) [state, below right=of \belowof] {};
        }

        \nfaArr[loop left]{#2\nodeId}{#2\nodeId}{\aLbl}{\aLbl}

        \nfaArr{\belowof}{#2\nodeId}{1}{0}
        \def\nodeRowHelper{\nfaArr{\prevRowNext}{#2\nodeId}{1}{0}}

        % If we're getting bigger, there's no ``wider'' node to draw from
        \ifthenelse{#1=1}{
            \ifthenelse{\nodeId<#3}{
                \nodeRowHelper
            }{}
        }{
            \nodeRowHelper
        }
    }
}

% #1 = first place
% #2 = second place
% #3 = name
\newcommand{\xHalfwayYFirst}[3]{
    \path let \p1 = (#1.out), \p2 = ($(#1.out)!.5!(#2.in)$) in coordinate (#3) at (\x2, \y1);}
