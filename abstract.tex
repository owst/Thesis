\begin{abstract}

Concurrent systems are frequently scrutinised using automated model checking, routinely using Petri
nets as a model. While for small system models, it is often sufficient to give the system
specification in a monolithic manner, for larger systems this approach is infeasible. Instead, a
compositional, or component-wise specification can be used. However, while existing model checking
techniques sometimes allow the specification of nets in terms of components, the techniques used
for checking properties of the system all consider the composed, global net.

In this thesis, we investigate and advocate compositional system specification and an alternative
approach to model checking that uses the structural compositional information to its advantage,
vastly improving efficiency in many examples. In particular, we examine the categorical structure
of component nets and their semantics, illustrating the functoriality of a map between the
categories as compositionality. We introduce contextual Petri Nets with Boundaries (PNBs), adding
read arcs, which naturally model behaviour that non-destructively reads the token state of a place.
Furthermore, we introduce a type-checked specification language that allows us to compositionally
construct systems to be modelled using PNBs, whilst ensuring that only correct compositions are
expressible. We then discuss and implement compositional statespace generation, which can be used
to check reachability.  Via optimisations using weak language equivalence and memoisation, we
obtain substantial speed ups and demonstrate that our checker outperforms the current
state-of-the-art for several examples. A final contribution is the compositional specification of
existing benchmark examples, in more natural, component-wise style.
\end{abstract}
