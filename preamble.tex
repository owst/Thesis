\documentclass{ecsthesis}
\usepackage{show2e}
\usepackage[T1]{fontenc}
\usepackage{helvet}
\renewcommand{\rmdefault}{phv}
\usepackage{bm}
\usepackage{calc}
\usepackage{amsfonts}
\usepackage{amsmath}
\usepackage{amssymb}
\usepackage{amsthm}
\usepackage{latexsym}
\usepackage{mathpartir}
% Used for capitalisewords
\usepackage{mfirstuc}
\usepackage{subcaption}
\usepackage{graphicx}
\usepackage{listings}
\usepackage{fixltx2e}
\usepackage{float}
\usepackage{multicol}
\usepackage{multirow}
\usepackage{bigdelim}
% Lets us colour tables (must be before tikz)
\usepackage[table]{xcolor}
\usepackage{tikz}
\usepackage{pgfplots}
\usepackage{tikz-qtree}
\usetikzlibrary{fadings,arrows,fit,positioning,shapes.symbols,petri,automata,decorations,decorations.pathmorphing,calc,shapes.geometric,decorations.markings,shapes,decorations.pathreplacing, matrix, patterns}
% IEEEtran and enumitem both define labelindent. IEEEtran is only defining it
% for backwards-compatability, so this is fine.
\let\labelindent\relax
\usepackage[shortlabels]{enumitem}
% Allows us to rotate figures.
\usepackage{rotfloat}
% Used to define commands with multiple optional params
\usepackage{xparse}
\usepackage{colonequals}
\usepackage{nicefrac}
\usepackage{stmaryrd}
% CWD-relative imports
\usepackage{import}
% Provides ifdefempty to check for empty params
\usepackage{etoolbox}
% Provides ifmtarg to check for empty args, see http://tex.stackexchange.com/questions/33749/xparse-define-new-command-with-multiple-optional-parameters
\usepackage{ifmtarg}
% Provides the framed environment
\usepackage{framed}
\usepackage[chapter]{algorithm}
% Make algorithm references use chapter number too
\renewcommand{\thealgorithm}{\arabic{chapter}.\arabic{algorithm}}
\usepackage{algpseudocode}
% For scaling delimiters
\usepackage{scalerel}
% For better vector notation
\usepackage{esvect}
% Increase size of math symbols
\usepackage{relsize}

% Setup tikz out/in styles
\foreach \out in {-360,-355,...,360}{
    \foreach \in in {-360,-355,...,360}{
        \globaldefs=1
        \edef\dotikzset{\noexpand\tikzset{o\out i\in/.style={out=\out, in=\in}}}
        \dotikzset
    }
}

% To import graph data from sub-directories (assumes top-level path)
\usepackage{currfile}
