Before introducing and specifying our example net systems, we describe a
collection of ``wiring'' nets that we frequently use in specifications. Such
wiring nets do not contain places, only transitions connecting the boundary
ports in various ways. Consider the ``left-end'' components shown in
\figref{fig:leftendComponents} (such named since they are to be composed on the
left of other components). Each illustrated component is a representation of a
parametric family of nets over $\aN \in \N \setminus \setof{0}$; for notational
convenience we drop the subscript when $\aN = 1$. The $\etaC{\aN}$ net, shown
in \figref{fig:etaComponent}, has $2\aN$ right boundary ports, with transitions
connecting the ports $0 \le \bN < \aN$ to $2\aN - 1 - \bN$.

\makesavebox{\etaBox}{%
\begin{tikzpicture}[pnb, boundaryPort-r5/.style={boundaryPortHidden}]

    \drawBoundaries[2][4]{0}{9}

    \path node [dotsBPort] at (r3) {};
    \path node [dotsBPort] at (r7) {};

    \foreach \i in {1,2,4}{%
        \pgfmathtruncatemacro{\other}{10 - \i}
        \labelledpnbarr{r\i}{r\other}{}{barr, out=180, in=180, looseness=1.5}{}
    }
\end{tikzpicture}
}

\makesavebox{\lendBox}{%
\begin{tikzpicture}[pnb]
    \drawBoundaries[2][4]{0}{4}

    \path node [dotsBPort] at (r3) {};

    \foreach \i in {1,2,4}{%
        \coordinate (l\i) at ([xshift=-1cm]r\i);
        \labelledpnbarr{l\i}{r\i}{}{barr}{}
    }
\end{tikzpicture}
}

\makesavebox{\ltermBox}{%
\begin{tikzpicture}[pnb]
    \drawBoundaries[2][4]{0}{4}

    \path node [dotsBPort] at (r3) {};
\end{tikzpicture}
}

\begin{figure}[ht]
\centering
\begin{subfigure}{0.33\textwidth}
\centering
\usebox\etaBox
\caption{\etaC{\aN} \withNetType{0}{2\aN}}
\label{fig:etaComponent}
\end{subfigure}%
\begin{subfigure}{0.33\textwidth}
\centering
\usebox\lendBox
\caption{\lendC{\aN} \withNetType{0}{\aN}}
\label{fig:lendComponent}
\end{subfigure}%
\begin{subfigure}{0.33\textwidth}
\centering
\usebox\ltermBox
\caption{\ltermC{\aN} \withNetType{0}{\aN}}
\label{fig:ltermComponent}
\end{subfigure}%
\caption{Left-end Component Families}
\label{fig:leftendComponents}
\end{figure}

The $\lendC{\aN}$ net, shown in \figref{fig:lendComponent}, has $\aN$ right
boundary ports, each with a single corresponding transition; composing with
such a net completes the partial specification of any transitions connected to
the shared boundary port, but without adding further connections to places.
Finally, the $\ltermC{\aN}$ net, shown in \figref{fig:ltermComponent}, has
$\aN$ right boundary ports, but no transitions; composing with this net
terminates any transitions, $\aTrans$, connected to the shared boundary ports
in the other component --- there are no transitions in $\ltermC{}$ to
synchronise with and thus $\aTrans$ cannot appear as part of a transition in
the composite net.

\makesavebox{\epsilonBox}{\reflectbox{\usebox\etaBox}}
\makesavebox{\rendBox}{\reflectbox{\usebox\lendBox}}
\makesavebox{\rtermBox}{\reflectbox{\usebox\ltermBox}}

Similarly, we have right-end component nets, as shown in
\figref{fig:rightendComponents} which are reflections of their left-end
counterparts.

\begin{figure}[ht]
\centering
\begin{subfigure}{0.33\textwidth}
\centering
\usebox\epsilonBox
\caption{\epsilonC{\aN} \withNetType{2\aN}{0}}
\label{fig:epsilonComponent}
\end{subfigure}%
\begin{subfigure}{0.33\textwidth}
\centering
\usebox\rendBox
\caption{\rendC{\aN} \withNetType{\aN}{0}}
\label{fig:rendComponent}
\end{subfigure}%
\begin{subfigure}{0.33\textwidth}
\centering
\usebox\rtermBox
\caption{\rtermC{\aN} \withNetType{\aN}{0}}
\label{fig:rtermComponent}
\end{subfigure}%
\caption{Right-end Component Families}
\label{fig:rightendComponents}
\end{figure}

Finally, we often use identity net components: such components have $\aN$ left
and right boundary ports, and $\aN$ transitions, connecting each left boundary
port to the corresponding right boundary port, as illustrated in
\figref{fig:idComponent}. Such nets preserve the transitions of any components
they are composed with (in a formal sense: PNBs form a category, with this
family of nets as the identity morphisms --- see \secref{sec:pnbCat}).

\begin{figure}[ht]
\centering
\begin{tikzpicture}[pnb]
    \drawBoundaries[2][4]{4}{4}

    \path node [dotsBPort] at (l3) {};
    \path node [dotsBPort] at (r3) {};

    \foreach \i in {1,2,4}{%
        \labelledpnbarr{l\i}{r\i}{}{barr}{}
    }
\end{tikzpicture}
\caption{$\idC{\aN}$ \withNetType{\aN}{\aN}}
\label{fig:idComponent}
\end{figure}

\newcommand{\subsectionImport}[2]{%
    \subsection{#1}
    \label{sec:example-#2}
    \subimport{examples/}{#2}
}
\newcommand{\importList}[1]{%
    \foreach \name/\file in {#1} {%
        \subsectionImport{\name}{\file}
    }
}

We now introduce two basic net systems, the $\aN$-bit Buffer and Token Ring,
before introducing a convenient programming language to specify such systems.

\importList{$\aN$-bit Buffer/buffer,Token Ring/token-ring}

\subsection{A Language for Net Composition}
\label{sec:motivation-lang}
\subimport{}{motivation-lang}

% Require the trailing % so we don't introduce spurious spaces, breaking labels
\importList{ Complete Trees/trees%
           , Cliques/cliques%
           , Powersets/powersets%
           }

\section{Benchmark Systems}

In this section, we introduce several \emph{more realistic} benchmark systems
than those introduced thus far. Several of the systems are taken from
Corbett's~\cite{Corbett1996} benchmark suite, but have been reformulated in
terms of compositional specifications. To arrive at the component-based
specifications, we examined the original monolithic Petri net and Ada models,
as they existed in the Petri net community, and decomposed them into components
by hand. By doing so, the natural system descriptions in terms of component
structure and component compositions can be explicitly presented.

\importList{ Overtake Protocol/over%
           , Hartstone/hartstone%
           , Iterated Choice/iter-choice%
           , Replicator/replicator%
           , DAC: Divide and Conquer/dac%
           , Dining Philosophers/philos%
           , Milner's Cyclic Scheduler/cyclic%
           , $\aN$-bit Counter/counter%
           }
