\chapter{Benchmarks and a Domain Specific Language for Net Compositions} \label{chp:benchmarksAndLang}

In this chapter we introduce the example systems that we will use to
demonstrate and benchmark our reachability checking techniques. Several of
these examples are taken from Corbett's benchmark suite~\cite{Corbett1996},
commonly used as reference benchmarks in the Petri net literature. However,
here we give alternative, \emph{parameterised} specifications in a
\emph{component-wise} manner; we propose that such specifications are more
natural and easier to construct and reason about, relative to the standard
monolithic definitions. Indeed, for this purpose, we motivate and introduce a
\emph{Domain Specific Language} (DSL) for constructing such systems, proving
that it ensures that invalid constructions cannot be formed. This chapter is
formed from two of the author's papers; the core was presented at Petri Nets
2014~\cite{Sobocinski2014}, while some of the example systems are taken
from~\cite{Rathke2013}.

\section{Component-wise Specification of Nets}
\label{sec:benchmarks}
\subimport{}{examples}

\section{Specification Domain Specific Language}
\label{sec:lang}
\subimport{}{lang}

\section{Summary}
In this section we introduced the example systems we use to demonstrate and
benchmark our reachability checking techniques. We illustrated that using a
suitable DSL for net system specification gives convenient representations,
especially for parametric systems, such as the example systems we consider. We
illustrated our net system DSL, \DSL{}, and proved that by using a static type
system, we can ensure that well-typed expressions are guaranteed to terminate
and construct well-composed net systems.
