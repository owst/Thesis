The \iteratedchoiceSys{\aN} system models a simple sequence of components that
each have a choice between two transitions to fire; the choice they make is
recorded in the resulting marking. Such a system has a simple component-wise
specification as a sequence of suitable ``choice'' components.

A single choice component is illustrated in \figref{fig:iter-choiceComponent};
the choice of transition is between those consuming tokens from $\aPlace_1$:
either $\aTrans_1$ or $\aTrans_2$. When $\aTrans_2$ is chosen, the choice is
recorded by a token being present in $\aPlace_2$. A single token is injected
into a sequence of single choice components by the \addtokC{} component, as
shown in \figref{fig:addtokC}.

\begin{figure}[ht]
    \centering
    \begin{subfigure}{0.5\textwidth}
    \centering
    \begin{tikzpicture}[pnb]
        \node (tok) [pnbplace, tokens=1] {};

        \drawBoundaries{0}{1}

        \labelledpnbarr{tok.out}{r1}{}{barr}{}
    \end{tikzpicture}
    \caption{\addtokC \withNetType{0}{1}}
    \label{fig:addtokC}
    \end{subfigure}%
    \begin{subfigure}{0.5\textwidth}
    \centering
    \begin{tikzpicture}[pnb]
        \node (init) [pnbplace, pnblabel=$\aPlace_1$] {};
        \node (chosen) [pnbplace, pnblabel=$\aPlace_2$, below right=of init, rotate=-90] {};

        \drawBoundaries{0}{0}
        \lBAlignedWith{init}{1}
        \rBAlignedWith{init}{1}

        \labelledpnbarr{init.out}{r1}{$\aTrans_1$}{barr, out=20, in=160}{}

        \labelledpnbarr[half]{init.out}{chosen.in}{$\aTrans_2$}{bend left=45}{}

        \draw (half) edge[barr, out=-20, in=180] (r1);

        \labelledpnbarr{l1}{init.in}{}{barr}{}
    \end{tikzpicture}
    \caption{\choiceC \withNetType{1}{1}}
    \label{fig:iter-choiceComponent}
    \end{subfigure}%
\end{figure}

The component-wise specification of a sequence of choice components is then:
\[
    \iteratedchoiceSys{\aN} \defeq
    \compExpr{\addtokC}{\compExpr{\sequenceExpr{\aN}{\choiceC}}{\rendC{}}}
\]
the schematic of such a system is almost trivial, e.g.  the schematic of
\iteratedchoiceSys{2} is illustrated in \figref{fig:iterchoiceschematic}, with
the corresponding composite net shown in \figref{fig:iterchoicecomposed}.

\begin{figure}[ht]%
\centering
\begin{tikzpicture}[schematic]
    \node[schematicnode] (lend) []              {\addtokC};
    \node[schematicnode] (choice1) [right=of lend] {\choiceC};
    \node[schematicnode] (choice2) [right=of choice1] {\choiceC};
    \node[schematicnode] (rend) [right=of choice2] {\rendC{}};

    \newcommand{\offset}{0.5}

    \rightBPort{lend}{\offset}{1}
    \foreach \i in {1,2} {
        \leftBPort{choice\i}{\offset}{1}
        \rightBPort{choice\i}{\offset}{1}
    }
    \leftBPort{rend}{\offset}{1}

    \draw (choice1-r1) -- (choice2-l1);

    \draw (lend-r1) -- (choice1-l1);
    \draw (choice2-r1) -- (rend-l1);
\end{tikzpicture}
\caption{Schematic of \iteratedchoiceSys{2}} \label{fig:iterchoiceschematic}
\end{figure}

\begin{figure}[ht]
    \centering
    \begin{tikzpicture}[pnb]
    \node (init0) [pnbplace, tokens=1] {};
    \node (init1) [pnbplace, right=of init0] {};
    \node (chosen1) [pnbplace, below right=of init1, rotate=-90] {};
    \node (init2) [pnbplace, above right=of chosen1] {};
    \node (chosen2) [pnbplace, below right=of init2, rotate=-90] {};

    \pnbarr{init0.out}{init1.in}

    \labelledpnbarr{init1.out}{init2.in}{}{bend left=30}{pos=0.8}
    \labelledpnbarr[c1]{init1.out}{init2.in}{}{bend right=30}{pos=0.2}
    \draw (c1) edge[pnbarr, out=-20, in=90] (chosen1.in);

    \coordinate (urk) at ([xshift=0.5cm, yshift=0.5cm]init2.out);

    \labelledpnbarr{init2.out}{urk}{}{out=0, in=180}{}
    \labelledpnbarr{init2.out}{chosen2.in}{}{out=0, in=90}{pos=0.3}

    \drawBoundaries{0}{0}
    \end{tikzpicture}
\caption{\iteratedchoiceSys{2}}
\label{fig:iterchoicecomposed}
\end{figure}
