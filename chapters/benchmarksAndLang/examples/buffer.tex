A $\aN$-buffer net (taken from~\cite{Esparza1996}) models a system of $\aN$
buffer cells that can each either be ``full'' or ``empty''. Such a system is
naturally described in a compositional manner: to form an $\aN$-buffer, simply
compose $\aN$ suitable $1$-buffer components.

\begin{figure}[ht]
\centering
\nBufPNB{3}
\caption{\bufferSys{3}}
\label{fig:buffer3}
\end{figure}

For example, \bufferSys{3} is illustrated in \figref{fig:buffer3}. Taking a
component-wise view, and using the single-buffer component illustrated in
\figref{fig:buffer1}, the schematic for \bufferSys{3} is illustrated in
\figref{fig:bufferschematic}: with suitable terminators, we simply synchronously
compose 3 buffers together.

\begin{figure}[ht]
\centering
\setlength{\fboxsep}{0pt}
\begin{subfigure}{0.33\textwidth}
\centering
\begin{tikzpicture}[pnb]
    \node[pnbplace] (p0) [tokens=1] {};
    \node[pnbplace] (p1) [below=of p0, rotate=180] {};

    \drawBoundaries[1.2]{1}{1}

    \foreach \from/\to/\side/\in/\out in {p0/p1/r1/180/0, p1/p0/l1/0/180}{%
        \labelledpnbarr[\from\to]{\from.out}{\to.in}{}{out=\out, in=\out}{}
        \draw (\from\to) edge[barr, out=90, in=\in] (\side);
    }
\end{tikzpicture}
\caption{\bufferC{} \withNetType{1}{1}}
\end{subfigure}%
\caption{\bufferSys{\aN} component net}
\label{fig:buffer1}
\end{figure}

Recall that to aid intuition, we often mark (parts of) partially-specified
transitions using a lighter shade, emphasising the fully specified structure of
a component net.

\begin{figure}[ht]%
\centering
\begin{tikzpicture}[schematic]
    \node[schematicnode] (lend) []              {\lendC{}};
    \node[schematicnode] (buf1) [right=of lend] {\bufferC};
    \node[schematicnode] (buf2) [right=of buf1] {\bufferC};
    \node[schematicnode] (buf3) [right=of buf2] {\bufferC};
    \node[schematicnode] (rend) [right=of buf3] {\rendC{}};

    \newcommand{\offset}{0.5}

    \rightBPort{lend}{\offset}{1}
    \foreach \i in {1,2,3} {%
        \leftBPort{buf\i}{\offset}{1}
        \rightBPort{buf\i}{\offset}{1}
    }
    \leftBPort{rend}{\offset}{1}

    \foreach \i in {1,2} {%
        \pgfmathtruncatemacro{\next}{\i + 1}
        \draw (buf\i-r1) -- (buf\next-l1);
    }

    \draw (lend-r1) -- (buf1-l1);
    \draw (buf3-r1) -- (rend-l1);

\end{tikzpicture}
\caption{Schematic of \bufferSys{3}}
\label{fig:bufferschematic}
\end{figure}

Arbitrarily-sized buffers are defined:
\[
    \bufferSys{\aN} \defeq \lendC{} \comp \bufferC^\aN \comp \rendC{}
\]
