\newcommand{\cliqueC}{\component{clique1}}

A (directed) clique is a (directed) graph where there is an edge between every
pair of nodes. Similarly, a clique net has transitions such that for every pair
of distinct places, there are transitions to-and-from the places.

\begin{figure}[ht]
\centering
\begin{tikzpicture}[pnb, every edge/.append style={barr}]
    \node (p0) [pnbplace] {};

    \drawBoundaries[1][1.5]{0}{0}

    \rBPortAt{0.2}{1}
    \rBPortAt{0.8}{2}
    \lBPortAt{0.2}{1}
    \lBPortAt{0.8}{2}

    \pnbarr{l1}{r1}
    \labelledpnbarr{p0.out}{l1}{}{o0i0}{pos=0.7}
    \labelledpnbarr{r1}{p0.in}{}{o180i180,pnbarrstyle1}{pos=0.3}

    \pnbarr{l2}{r2}
    \pnbarr{p0.out}{r2}
    \pnbarr{l2}{p0.in}
\end{tikzpicture}
\caption{$\cliqueC \withNetType{2}{2}$}
\label{fig:cliqueC}
\end{figure}

Perhaps surprisingly, a compact, component-wise specification can be given for
cliques; if we specify a single ``clique component'', we can construct a
(suitably terminated) sequence of such components to arrive at the
specification of a clique. Indeed, a single clique component, \cliqueC{}, is
illustrated in \figref{fig:cliqueC}. Such a component has transitions that
allow it to produce tokens at the next and previous components, consume tokens
from the next and previous components, and pass tokens through from left to
right boundaries and vice-versa (it is these last transitions that allow, say,
the $3^{\text{rd}}$ component to produce tokens at the $1^\text{st}$, only via
connections to the $2^\text{nd}$. Using the previously-defined $\injectorC$
component, we can inject a single token into the clique system.

\begin{figure}[ht]
\centering
\begin{tikzpicture}[pnb]

\node (0) [pnbplace] {};
\node (1) [pnbplace, right=of 0] {};
\node (2) [pnbplace, right=of 1] {};
\node (i) [pnbplace, left=of 0, tokens=1] {};

\drawBoundaries[1][1.5]{0}{0}

\labelledpnbarr{0.out}{1.in}{}{}{}
\labelledpnbarr{0.out}{2.in}{}{bend left=45}{}
\labelledpnbarr{1.out}{2.in}{}{}{}
\labelledpnbarr{1.out}{0.in}{}{o-60i240}{}
\labelledpnbarr{2.out}{0.in}{}{o-60i240}{}
\labelledpnbarr{2.out}{1.in}{}{o60i120,pnbarrstyle1}{}
\labelledpnbarr{i.out}{0.in}{}{}{}


\end{tikzpicture}
\caption{\cliqueSys{3}}
\label{fig:clique3}
\end{figure}

\begin{figure}[ht]%
\centering
\begin{tikzpicture}[schematicnode/.append style={minimum height=2.2cm}, schematic]
    \node[schematicnode] (1) []           {$\ltermC{2}$};
    \node[schematicnode] (3) [minimum height=1cm, alignbot=1] {\injectorC};
    \node[schematicnode] (2) [minimum height=1cm, aligntop=1, samewidthas={3 and 3}] {\idC{}};
    \node[schematicnode] (4) [aligntop=2] {\cliqueC};
    \node[schematicnode] (5) [right=of 4] {\cliqueC};
    \node[schematicnode] (6) [right=of 5] {\cliqueC};
    \node[schematicnode] (7) [right=of 6] {$\rtermC{2}$};

    \newcommand{\leftPorts}[1]{
        \drawBoundaryPortsOn{#1}{2}{0}
    }
    \newcommand{\rightPorts}[1]{
        \drawBoundaryPortsOn{#1}{0}{2}
    }

    \drawBoundaryPortsOn{3}{1}{1}
    \drawBoundaryPortsOn{2}{1}{1}

    \foreach \i in {1,4,5,6}{%
        \rPortAtHeightOf{\i}{2-r1}{1}
        \rPortAtHeightOf{\i}{3-r1}{2}
    }
    \foreach \i in {4,5,6,7}{%
        \lPortAtHeightOf{\i}{2-l1}{1}
        \lPortAtHeightOf{\i}{3-l1}{2}
    }

    \draw (1-r1) -- (2-l1);
    \draw (1-r2) -- (3-l1);
    \draw (2-r1) -- (4-l1);
    \draw (3-r1) -- (4-l2);

    \foreach \i in {4,...,6} {
        \pgfmathtruncatemacro{\next}{\i + 1}
        \foreach \p in {1,2}{
            \draw (\i-r\p) -- (\next-l\p);
        }
    }
\end{tikzpicture}
\caption{Schematic of \cliqueSys{3}} \label{fig:cliqueschematic}
\label{fig:cliqueSchematic}
\end{figure}

As an example, \cliqueSys{3} is illustrated in \figref{fig:clique3}, with its
schematic shown in \figref{fig:cliqueSchematic}. Cliques of arbitrary size are
defined:
\begin{flalign*}
    \cliqueSys{\aN} \defeq
    \compExpr{\ltermC{2}}{\compExpr{\parens{\tensExpr{\idC{}}{\injectorC}}}{\compExpr{\sequenceExpr{\aN}{\cliqueC}}{\rtermC{2}}}}
\end{flalign*}
