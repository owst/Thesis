\newcommand{\startPlace}{S}
\newcommand{\powersetC}{\component{powerset1}}

A \powersetSys{\aN} net has $\aN + 1$ places. There is a chosen start place,
\startPlace, with the remaining places being the elements of $\ordinal{\aN}$.
There are $2^\aN$ transitions, all with a single source, \startPlace, and with
targets being the elements of $2^{\ordinal{\aN}}$. For example, \powersetSys{3}
is illustrated in \figref{fig:powerset3}, where transition $t_X$ connects $S$
to all $x \in X$.

\begin{figure}[ht]
    \centering
    \def\thevsep{6cm}
    \def\thehsep{4cm}
    \begin{tikzpicture}[pnb, node distance=\thevsep]

    \node[pnbplace] (S) [label={[overlay]above:$\startPlace$}, rotate=-90] {};
    \node[pnbplace] (1) [label={[overlay]above:$1$}, below of=S, rotate=-90] {};
    \node[pnbplace] (0) [label={[overlay]above:$0$}, node distance=\thehsep, left of=1, rotate=-90] {};
    \node[pnbplace] (2) [label={[overlay]above:$2$}, node distance=\thehsep, right of=1, rotate=-90] {};

    \coordinate (e) at ([xshift=-1cm, yshift=0.25cm]S.out);

    \newcommand{\coordBetween}[5]{
        \coordinate [yshift=#1cm] (#2) at ($ (#3.in) !.#4! (#5.in) $);
    }

    \coordBetween{2}{two1}{0}{3}{1}
    \coordBetween{2}{two2}{1}{7}{2}
    \coordBetween{3}{two3}{0}{7}{1}
    \coordBetween{3}{three}{1}{3}{2}

    \drawBoundaries{0}{0}

    \newcommand{\transSet}[1]{\aTrans_{\scriptstyle\setof{#1}}}

    \labelledpnbarr{S.out}{e}{{$\aTrans_\emptyset$}}{o180i0}{}
    \labelledpnbarr{S.out}{0.in}{{left:$\transSet{0}$}}{o200i80}{}
    \labelledpnbarr{S.out}{1.in}{{[yshift=0.2cm, xshift=0.1cm]left:$\transSet{1}$}}{o-90i90}{}
    \labelledpnbarr{S.out}{2.in}{{right:$\transSet{2}$}}{o-20i100}{}

    \labelledpnbarr{S.out}{two1}{{[pos=1]right:$\transSet{0,1}$}}{pnbarrstyle2, out=225, in=90}{pos=0.99}
    \draw (two1)  edge[pnbarrstyle2,  pnbarr, out=-90, in=80] (0.in);
    \draw (two1)  edge[pnbarrstyle2,  pnbarr, out=-90, in=135] (1.in);

    \labelledpnbarr{S.out}{two2}{{[pos=1]left:$\transSet{1,2}$}}{pnbarrstyle3, out=-45, in=90}{pos=0.99}
    \draw (two2)  edge[pnbarrstyle3, pnbarr, out=-90, in=45] (1.in);
    \draw (two2)  edge[pnbarrstyle3, pnbarr, out=-90, in=110] (2.in);

    \labelledpnbarr{S.out}{two3}{{[pos=1]left:$\transSet{0,2}$}}{pnbarrstyle4, o-90i90}{pos=0.99}
    \draw (two3)  edge[pnbarrstyle4, pnbarr, out=-90, in=60] (0.in);
    \draw (two3)  edge[pnbarrstyle4, pnbarr, out=-90, in=130] (2.in);

    \labelledpnbarr{S.out}{three}{{[pos=1]right:$\transSet{0,1,2}$}}{pnbarrstyle1, o-90i90}{pos=0.99}
    \draw (three) edge[pnbarrstyle1, out=-110, in=50] (0.in);
    \draw (three) edge[pnbarrstyle1, out=-100, in=60] (1.in);
    \draw (three) edge[pnbarrstyle1, out=-90, in=125] (2.in);

    \end{tikzpicture}
    \caption{\powersetSys{3}}
    \label{fig:powerset3}
\end{figure}

To obtain a component-wise specification, consider partitioning the non-start
places of the net into individual components. Then, the global transitions must
be constructed from compositions of suitable local transitions.
Locally, each incoming partially specified transition should be connected to
the place of the component (and other components) in three ways:
\begin{enumerate}
    \item To the local place and any further specified places
    \item To the local place alone,
    \item Only to the further specified places.
\end{enumerate}

\begin{figure}[ht]
\centering
\begin{subfigure}{0.5\textwidth}
\centering
\begin{tikzpicture}[pnb]
    \node[pnbplace] (p0) {};

    \drawBoundaries{1}{1}

    \coordinate[node distance=0.75cm, above right=of p0.out] (e) ;
    \labelledpnbarr{p0.out}{e}{}{o0i180}{}

    \draw[barr, mark inside=0.5] (p0.out) to (r1);
    \draw[barr, mark inside=0.5] (l1) to (p0.in);
\end{tikzpicture}
\caption{\rootC{} \withNetType{1}{1}}
\label{fig:powersetRootComponent}
\end{subfigure}%
\begin{subfigure}{0.5\textwidth}
\centering
\begin{tikzpicture}[pnb]
    \node (p0) [pnbplace] {};

    \drawBoundaries[1.5][1]{1}{1}

    \labelledpnbarr{l1}{p0.in}{}{barr, bend right=45}{}
    \labelledpnbarr{l1}{r1}{}{barr, bend right=45}{}

    \labelledpnbarr[l1r1]{l1}{r1}{}{barr, bend left=45}{pos=0.15}
    \draw (l1r1) edge[barr, out=0, in=180] (p0.in);
\end{tikzpicture}
\caption{$\powersetC \withNetType{1}{1}$}
\label{fig:powersetComponent}
\end{subfigure}%
\caption{\powersetSys{\aN} component nets}
\label{fig:powersetNets}
\end{figure}

Such a component is illustrated in \figref{fig:powersetComponent}. With the
\rootC{} component illustrated in \figref{fig:powersetRootComponent} (which
either passes its token into a sequence of \powersetC{} components or not) and
the \injectorC{} component from $\tokenringSys{\aN}$, we have:
\[
    \powersetSys{\aN} \defeq
    \compExpr{\ltermC{}}{\compExpr{\injectorC}{\compExpr{\rootC}{\compExpr{\sequenceExpr{\aN}{\powersetC}}{\rtermC{}}}}}
\]

The schematic for the \powersetSys{3} net illustrated in
\figref{fig:powerset3}, is shown in \figref{fig:powersetschematic}.

\begin{figure}[ht]%
\centering
\makebox[\textwidth][c]{
\begin{tikzpicture}[schematic]
    \node[schematicnode] (-1) []           {$\ltermC{}$};
    \node[schematicnode] (0) [right=of -1] {$\injectorC$};
    \node[schematicnode] (1) [right=of 0]  {$\rootC$};
    \node[schematicnode] (2) [right=of 1]  {\powersetC};
    \node[schematicnode] (3) [right=of 2]  {\powersetC};
    \node[schematicnode] (4) [right=of 3]  {\powersetC};
    \node[schematicnode] (5) [right=of 4]  {$\rtermC{}$};

    \newcommand{\offset}{0.5}

    \rightBPort{-1}{\offset}{1}
    \foreach \i in {0,1,2,3,4} {
        \leftBPort{\i}{\offset}{1}
        \rightBPort{\i}{\offset}{1}
    }
    \leftBPort{5}{\offset}{1}

    \foreach \i in {-1,...,4} {
        \pgfmathtruncatemacro{\next}{\i + 1}
        \draw (\i-r1) -- (\next-l1);
    }
\end{tikzpicture}
}
\caption{Schematic of \powersetSys{3}}
\label{fig:powersetschematic}
\end{figure}
