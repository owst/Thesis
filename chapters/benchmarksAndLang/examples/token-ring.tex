Consider a model of a simple token ring network, taken from~\cite{Abdulla2004}.
Such a network consists of $\aN$ worker processes sharing a single token; a
worker may only work if it holds the token. As an example, $\tokenringSys{3}$
is illustrated in~\figref{fig:tokenring}. All workers are initially ``able to
work''; a token is ``injected'' into the system --- upon receiving the token, a
worker may choose to do work, or pass the token onto the next worker.

\newcommand{\drawTokComponent}[1]{%
    \def\x{#1}

    \node[pnbplace] (sync\x) {};
    \node[pnbplace] (working\x) [above of=sync\x, rotate=180] {};
    \node[pnbplace] (want\x) [tokens=1,yshift=0.5 * \nodedist,above right of=working\x] {};
    \node[pnbplace] (given\x) [yshift=0.5 * \nodedist,above left of=working\x, rotate=0] {};

    \pnbarr{given\x.out}{want\x.in}

    \labelledpnbarr[wantworking\x]{want\x.out}{working\x.in}{}{out=0,in=0}{pos=0.7}
    \labelledpnbarr{sync\x.out}{wantworking\x}{}{out=0, in=45}{}
    \labelledpnbarr[workinggiven\x]{working\x.out}{given\x.in}{}{out=180,in=180}{pos=0.3}
    \labelledpnbarr{workinggiven\x}{sync\x.in}{}{out=150, in=180}{}
}
\begin{figure}[ht]
    \centering
    \begin{tikzpicture}[pnb, node distance=1.3cm]
    \foreach \i in {0,1,2}{%
        \begin{scope}[shift={(\i*3*\nodedist, 0cm)}]
            \drawTokComponent{\i}
        \end{scope}
    }

    \node (inject) [pnbplace, tokens=1, left=of sync0, xshift=-1 * \nodedist, yshift=0.5 * \nodedist] {};

    \pnbarr{inject.out}{sync0.in}

    \draw (sync0.out) edge[pnbarr, mark inside=0.5] (sync1.in);
    \draw (sync1.out) edge[pnbarr, mark inside=0.5] (sync2.in);

    \node (left) [pnbhidden, below=of sync0] {};
    \draw (left.in) edge[pnbarr, bend left=90] (sync0.in);

    \node[pnbhidden] (right) [below of=sync2] {};
    \draw (right.out) edge[pnbarr, bend right=90] (sync2.out);
    \draw (left.in) edge[pnbarr, mark inside=0.5] (right.out);

    \drawBoundaries{0}{0}
\end{tikzpicture}
\caption{\tokenringSys{3}\label{fig:tokenring}}
\end{figure}

Again, a token ring system is naturally described in a compositional manner:
each worker is a single component, as illustrated in \figref{fig:workerC}, that
may interact (via transitions connected to its left/right boundary ports) with
the workers on its left and right, in order to receive/send the token. The
injector component, shown in \figref{fig:injectorC}, simply holds a token that
can be injected, before acting as \idC{}. The last worker is connected to the
first, completing the ring. The schematic of \tokenringSys{3} is illustrated in
\figref{fig:tokenringschematic}.

\begin{figure}[ht]
    \centering
    \begin{subfigure}{0.5\textwidth}
        \centering
        \begin{tikzpicture}[pnb]
            \node (hidden) [pnbhidden] {};
            \node (tok) [pnbplace, tokens=1, above=of hidden] {};

            \drawBoundaries[0.8][0.6]{1}{1}

            \labelledpnbarr{tok.out}{r1}{}{barr}{}

            \labelledpnbarr{l1}{r1}{}{barr}{}
        \end{tikzpicture}
        \caption{\injectorC{} \withNetType{1}{1}}
        \label{fig:injectorC}
    \end{subfigure}%
    \begin{subfigure}{0.5\textwidth}
        \centering
    \begin{tikzpicture}[pnb, node distance=1.3cm]
        \drawTokComponent{0}

        \drawBoundaries{0}{0}

        \lBAlignedWith{sync0}{1}
        \rBAlignedWith{sync0}{1}

        \labelledpnbarr{l1}{sync0.in}{}{barr}{}
        \labelledpnbarr{sync0.out}{r1}{}{barr}{}
    \end{tikzpicture}
    \caption{\workerC{} \withNetType{1}{1}}
    \label{fig:workerC}
    \end{subfigure}
    \caption{\tokenringSys{\aN} components}
\end{figure}

\begin{figure}[ht]%
\centering
\begin{tikzpicture}[node distance=0.3cm, schematic]
    \def\upperSizeRatio{5/8}
    \def\lowerSizeRatio{2/8}

    \tikzset{main/.append style={minimum height=\mainsize}}
    \tikzset{int21t/.append style={minimum height=\upperSizeRatio * \mainsize}}
    \tikzset{int21b/.append style={minimum height=\lowerSizeRatio * \mainsize}}

    \node [schematicnode] (eta) [main] {\etaC{}};
    \node [schematicnode] (injector) [int21t, aligntop=eta] {\injectorC};
    \node [schematicnode] (worker1) [int21t, aligntop=injector] {\workerC};
    \node [schematicnode] (worker2) [int21t, aligntop=worker1] {\workerC};
    \node [schematicnode] (worker3) [int21t, aligntop=worker2] {\workerC};

    \node [schematicnode] (idbot) [int21b, alignbot=eta, samewidthas={injector and worker3}]   {\idC{}};
    \node [schematicnode] (epsilon) [main, aligntop=worker3] {\epsilonC{}};

    \drawBoundaryPortsOn{idbot}{1}{1}
    \drawBoundaryPortsOn{injector}{0}{0}


    \foreach \i in {1,2,3}{
        \drawBoundaryPortsOn{worker\i}{1}{1}
    }

    \foreach \from/\to in {injector/worker1, worker1/worker2, worker2/worker3}{
        \draw (\from) -- (\to);
    }

    \foreach \which in {injector, idbot}{
        \drawBoundaryPortsOn{\which}{1}{1}
    }

    \connectedBetweenL{eta}{injector-l1}{1}
    \connectedBetweenL{eta}{idbot-l1}{2}

    \connectedBetweenR{epsilon}{worker3-r1}{1}
    \connectedBetweenR{epsilon}{idbot-r1}{2}
\end{tikzpicture}
\caption{Schematic of \tokenringSys{3}}
\label{fig:tokenringschematic}
\end{figure}

{\tokenringSys{\aN}} is defined:
\[
    \tokenringSys{\aN} \defeq \etaC{} \comp \parens{\parens{\injectorC \comp \workerC^\aN} \tensor \idC{}} \comp \epsilonC{}
\]
