\mkC{onebit}
\mkC{zerobit}
\mkC{tester}
The \counterSys{\aN} system models a ``counter'' system that counts from $0-n$
allowing increment/decrement. Intuitively, a $\aN$-bit counter is
formed by synchronously composing $\aN$ of the single-bit counter components
from \figref{fig:onebitC}, and terminating with a \zerobitC{} component, shown
in \figref{fig:zerobitC}, which always reports as being full and can't be
incremented or decremented. The intuitive description of a $1$-bit component is
that it is either ``empty'' or ``full''. A full component may be directly
decremented, or may pass its token to the next component, in either case it
becomes empty. Passing tokens along a chain of components allows the chain to
become full --- a $\aN$-bit counter is not full if any component has a token in
the empty place; it is full if all places have tokens in the full place.

\begin{figure}[ht]%
\centering
\begin{subfigure}{0.7\textwidth}
\centering
\begin{tikzpicture}[pnb]
\node[pnbplace] (zero)  [tokens=1, rotate=180] {};
\node[pnbplace] (one)  [below of=zero] {};

\drawBoundaries[2][1.25]{4}{4}

\lBoundaryLabel{1}{linc}
\lBoundaryLabel{2}{ldec}
\lBoundaryLabel{3}{lnotfull}
\lBoundaryLabel{4}{lfull}
\rBoundaryLabel{1}{rinc}
\rBoundaryLabel{2}{rdec}
\rBoundaryLabel{3}{rnotfull}
\rBoundaryLabel{4}{rfull}

\labelledpnbarr[zeroone]{zero.out}{one.in}{}{o180i180}{pos=0.2}
\draw (l1) edge[barr, pnbarr,out=0,in=180] (zeroone);

\labelledpnbarr[onezero1]{one.out}{zero.in}{}{pnbarrstyle1, bend right=70}{pos=0.6}

\useasboundingbox (current bounding box.north west) rectangle (current bounding box.south east);

\labelledpnbarr[onezero2]{one.out}{zero.in}{}{bend right=90, looseness=1.5}{pos=0.8}
\draw (onezero2) edge[barr, o0i180] (r1);
\draw (l2) edge[barr, pnbarrstyle1, out=0, in=150] (onezero1);
\labelledpnbarr{l2}{r2}{}{barr, pnbarrstyle1, bend left=45}{}
\labelledpnbarr{l3}{r3}{}{barr, pnbarrstyle1, bend right=35}{pos=0.7}
\labelledpnbarr{l3}{zero.north}{}{barr, pnbarrstyle2, o0i-90}{pos=0.3}
\labelledpnbarr[l4r4]{l4}{r4}{}{barr, bend right=20}{pos=0.4}

\draw (l4r4) edge[barr,out=0,in=-90] (one.south);

\end{tikzpicture}%
\caption{\onebitC{} \withNetType{4}{4}}
\label{fig:onebitC}
\end{subfigure}%
\begin{subfigure}{0.3\textwidth}
\centering
\begin{tikzpicture}[pnb]
    \node[pnbhidden] (zero) [] {};
    \node[pnbhidden] (one)  [below of=zero] {};

    \drawBoundaries[0.75][1.25]{4}{0}

    \lBoundaryLabel{4}{full}

    \node[coordinate] (full) [right of=l4]{};

    \labelledpnbarr{l4}{full}{}{barr}{}
\end{tikzpicture}
\caption{\zerobitC{} \withNetType{4}{0}}
\label{fig:zerobitC}
\end{subfigure}%
\caption{$\aN$-bit Counter component nets}
\end{figure}

\begin{figure}[ht]
    \centering
    \begin{tikzpicture}[pnb]
        \node[pnbplace] (0)  [tokens=1, rotate=-90] {};
        \node[pnbplace] (1)  [right of=0, rotate=90] {};
        \node[pnbplace] (2)  [below of=0, rotate=-90] {};

        \drawBoundaries{0}{4}

        \labelledpnbarr[01]{0.out}{1.in}{}{bend right}{}
        \labelledpnbarr[10]{1.out}{0.in}{}{bend right}{}
        \labelledpnbarr[02]{0.out}{2.in}{}{relative}{}

        \draw (10) edge[barr, o30i180] (r1);
        \draw (01) edge[barr, o-10i180] (r3);
        \draw (02) edge[barr, o-60i180] (r4);
    \end{tikzpicture}
    \caption{\testerC{} component net}
    \label{fig:testerC}
\end{figure}

The specification of \counterSys{\aN} is simple:

\[
    \counterSys{\aN} \defeq
        \compExpr{\testerC}{\compExpr{\sequenceExpr{\aN}{\onebitC}}{\zerobitC}}
\]

Clients of \counterSys{\aN} interact with the 4 boundary ports on its left
boundary: they may increment, decrement and test for not-full/full ( if the
counter is full to capacity, transitions connected to $\mathit{lfull}$ may fire
and otherwise not). Indeed, while the counter sequence ``reports'' that it is
not full, the tester component repeatedly fires the sequence's $\mathit{inc}$
transition and then tests that the sequence ``reports'' as being full. The
schematic for \counterSys{3} is shown in \figref{fig:counterschematic}.

\begin{figure}[ht]%
\centering
\begin{tikzpicture}[schematic]
    \node[schematicnode] (0)              {\testerC};
    \node[schematicnode] (1) [right=of 0] {\onebitC};
    \node[schematicnode] (2) [right=of 1] {\onebitC};
    \node[schematicnode] (3) [right=of 2] {\onebitC};
    \node[schematicnode] (4) [right=of 3] {\zerobitC};

    \foreach \name in {0,...,3}{%
        \drawBoundaryPortsOn{\name}{4}{4}
    }
    \drawBoundaryPortsOn{4}{4}{0}

    \foreach \x in {0,...,3} {%
       \pgfmathtruncatemacro{\next}{\x + 1}

        \foreach \port in {1,...,4} {%
           \draw(\x-r\port) -- (\next-l\port);
        }
    }
\end{tikzpicture}
\caption{Schematic of \counterSys{3}}
\label{fig:counterschematic}
\end{figure}
