A replicator is a simple component that is able to fire the transition
connected to its right boundary port an unbounded number of times, once the
transition connected to its left boundary port has fired once. Sequences of
such replicators can be formed into a chain, similarly to the
\iteratedchoiceSys{\aN} system.

\newcommand{\drawRepPlaces}[1]{%
    \node (p0#1) [pnbplace] {};
    \node (p1#1) [pnbplace, below=of p0#1, rotate=180] {};

    \labelledpnbarr{p1#1.out}{p0#1.in}{}{overlay, bend left=45}{}
    \labelledpnbarr[p0p1#1]{p0#1.out}{p1#1.in}{}{overlay, bend left=45}{pos=0.1}
}

\begin{figure}[ht]
    \centering
    \begin{subfigure}{0.5\textwidth}
    \centering
    \begin{tikzpicture}[pnb]
        \drawRepPlaces{}

        \drawBoundaries{0}{0}

        \lBAlignedWith{p0}{1}
        \rBAlignedWith{p0}{1}

        \labelledpnbarr{l1}{p0.in}{}{barr}{}
        \draw (p0p1) edge[barr, out=-30, in=180] (r1);
    \end{tikzpicture}
    \caption{\replicatorC \withNetType{1}{1}}
    \label{fig:replicatorC}
    \end{subfigure}%
    \begin{subfigure}{0.5\textwidth}
    \centering
    \begin{tikzpicture}[pnb]
        \node (tok) [pnbplace] {};

        \drawBoundaries{1}{0}

        \labelledpnbarr{l1}{tok.in}{}{barr}{}
    \end{tikzpicture}
        \caption{\taketokC \withNetType{1}{0}}
    \end{subfigure}%
\end{figure}

The \replicatorsSys{\aN} system is defined:
\[
    \replicatorsSys{\aN} \defeq
    \compExpr{\addtokC}{\compExpr{\sequenceExpr{\aN}{\replicatorC}}{\taketokC}}
\]
the schematic of such a system is again almost trivial, e.g. the
schematic for \replicatorsSys{3} is illustrated in
\figref{fig:replicatorsschematic}, with the corresponding composite net shown
in \figref{fig:replicatorscomposed}.

\begin{figure}[ht]%
\centering
\makebox[\textwidth][c]{
\begin{tikzpicture}[schematic]
    \node[schematicnode] (lend) []              {\addtokC};
    \node[schematicnode] (rep1) [right=of lend] {\replicatorC};
    \node[schematicnode] (rep2) [right=of rep1] {\replicatorC};
    \node[schematicnode] (rep3) [right=of rep2] {\replicatorC};
    \node[schematicnode] (rend) [right=of rep3] {\taketokC};

    \newcommand{\offset}{0.5}

    \foreach \from/\to in {lend/rep1, rep1/rep2, rep2/rep3, rep3/rend}{
        \rightBPort{\from}{\offset}{1}
        \leftBPort{\to}{\offset}{1}

        \draw (\from-r1) -- (\to-l1);
    }
\end{tikzpicture}
}
\caption{Schematic of \replicatorsSys{3}}
\label{fig:replicatorsschematic}
\end{figure}

\begin{figure}[ht]
    \centering
    \begin{tikzpicture}[pnb]
    \node (init0) [pnbplace, tokens=1] {};

    \coordinate (p0p10) at (init0.out);

    \foreach \i in {1,2,3}{
        \pgfmathtruncatemacro{\prev}{\i - 1}
        \def\prevjoin{p0p1\prev}
        \begin{scope}[shift={(\i*\nodedist, 0cm)}]
            \drawRepPlaces{\i}
        \end{scope}

        \draw (\prevjoin) edge[pnbarr, out=-20, in=180] (p0\i.in);
    }

    \node (consume) [pnbplace, right=of p03] {};

    \draw (p0p13) edge[pnbarr, out=-20, in=180] (consume.in);

    \drawBoundaries{0}{0}
    \end{tikzpicture}
    \caption{\replicatorsSys{3}}
    \label{fig:replicatorscomposed}
\end{figure}
