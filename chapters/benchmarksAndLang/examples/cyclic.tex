\mkC{scheduler}
Milner's \cyclicschedulerSys{\aN} system models a set of $\aN$ processes
arranged in a cycle; each process starts in turn and notifies the next process
in the cycle to start. Upon finishing, the processes may start again, but only
if they have been signalled to do so by the previous process; similarly, if a
process is signalled whilst still running, it must wait to finish before
starting and passing the signal along.

\begin{figure}[ht]%
\centering
\centering
\begin{tikzpicture}[pnb]
\node[pnbplace] (init)  [tokens=1, rotate=270] {};

\node[pnbplace] (ready)  [below of=init, rotate=270] {};
\node[pnbplace] (startedNext) [below of=ready, rotate=270] {};
\node[pnbplace] (fws)  [below left of=startedNext, rotate=180] {};
\node[pnbplace] (swf)  [below right of=startedNext, rotate=0] {};

\drawBoundaries{1}{3}

\lBoundaryLabel{1}{signalMe}
\rBoundaryLabel{1}{startTask}
\rBoundaryLabel{2}{taskFinished}
\rBoundaryLabel{3}{signalNext}

\labelledpnbarr[initready]{init.out}{ready.in}{}{relative}{}
\draw (l1) edge[barr, out=30, in=120] (initready);

\labelledpnbarr[readystartedNext]{ready.out}{startedNext.in}{}{pnbarrstyle1, relative}{}

\labelledpnbarr[startedNextfws]{startedNext.out}{fws.in}{}{pnbarrstyle2, out=-135, in=0}{pos=0.7}

\labelledpnbarr[startedNextswf]{startedNext.out}{swf.in}{}{out=-45, in=180}{pos=0.7}

\labelledpnbarr[fwsready]{fws.out}{ready.in}{}{pnbarrstyle2, out=180, in=120}{pos=0.6}
\draw (l1) edge[barr, pnbarrstyle2, out=0, in=-150] (fwsready);

\draw (l1) edge[barr, out=-60, in=135] (startedNextswf);

\labelledpnbarr[swfready]{swf.out}{ready.in}{}{out=0, in=60}{pos=0.6}
\draw (swfready) edge[barr, out=-30, in=180] (r2);

\draw (r1) edge[barr, pnbarrstyle1, out=-120, in=70] (readystartedNext);
\draw (r3) edge[barr, pnbarrstyle1, out=120, in=-70] (readystartedNext);
\draw (r2) edge[barr, pnbarrstyle2, out=-120, in=45] (startedNextfws);

\end{tikzpicture}
\caption{\schedulerC{} \withNetType{2}{2}}
\label{fig:schedulerC}
\end{figure}

The component-wise specification of \cyclicschedulerSys{\aN} makes use of the
same task component net as \hartstoneSys{\aN} (illustrated in
\figref{fig:taskC}). Each \schedulerC{} component, illustrated in
\figref{fig:schedulerC} has a corresponding \taskC{} component that it controls.
Each scheduler/task component feeds its signalNext into the next component; the
cycle is initially started by the $\injectorC$ component from
\tokenringSys{\aN} (shown in \figref{fig:injectorC}). The specification is
then:
\begin{flalign*}
    \cyclicschedulerSys{\aN} \defeqsp
    &\bindExpr
        {\varName{schedTask}}
        {\compExpr{\schedulerC}{\tensExpr{\taskC}{\idC{}}}}
        {}\\
    &\bindExpr
        {\varName{tasks}}
        {\compExpr{\injectorC}{\sequenceExpr{\aN}{\varName{schedTask}}}}
        {} \\
    & \compExpr
        {\etaC{}}
        {\compExpr
            {\parens{\tensExpr{\varName{tasks}}{\idC{}}}}
            {\epsilonC{}}}
\end{flalign*}
The schematic of \cyclicschedulerSys{2} is shown in
\figref{fig:cyclicschematic}.

\begin{figure}[ht]%
\centering
\makebox[\textwidth][c]{
\begin{tikzpicture}[schematic]
    \def\mainsize{4cm}
    \def\upperSizeRatio{5/8}
    \def\lowerSizeRatio{2/8}

    \tikzset{main/.append style={minimum height=\mainsize}}
    \tikzset{int21t/.append style={minimum height=\upperSizeRatio * \mainsize}}
    \tikzset{int21b/.append style={minimum height=\lowerSizeRatio * \mainsize}}
    \tikzset{int22t/.append style={minimum height=\upperSizeRatio * \upperSizeRatio * \mainsize}}
    \tikzset{int22b/.append style={minimum height=\upperSizeRatio * \lowerSizeRatio * \mainsize}}

    \node [schematicnode] (eta) [main] {\etaC{}};
    \node [schematicnode] (injector) [int21t, aligntop=eta] {\injectorC};
    \node [schematicnode] (sched1) [int21t, aligntop=injector] {\schedulerC};

    \node [schematicnode] (task1) [int22t, aligntop=sched1] {\taskC};
    \node [schematicnode] (id1) [int22b, alignbot=sched1, samewidthas={task1 and task1}] {\idC{}};

    \node [schematicnode] (sched2) [int21t, aligntop=task1] {\schedulerC};
    \node [schematicnode] (task2) [int22t, aligntop=sched2] {\taskC};
    \node [schematicnode] (id2) [int22b, alignbot=sched2, samewidthas={task2 and task2}] {\idC{}};

    \node [schematicnode] (idbot) [int21b, alignbot=eta, samewidthas={injector and id2}] {\idC{}};
    \node [schematicnode] (epsilon) [main, aligntop=task2] {\epsilonC{}};

    \drawBoundaryPortsOn{idbot}{1}{1}
    \drawBoundaryPortsOn{injector}{0}{0}

    \connectedBetweenL{eta}{idbot-l1}{2}

    \foreach \i in {1,2}{%
        \drawBoundaryPortsOn{id\i}{1}{1}
        \drawBoundaryPortsOn{task\i}{2}{0}

        \foreach \j in {1,2}{%
            \connectedBetweenL{sched\i}{task\i-l\j}{\j}
        }

        \connectedBetweenL{sched\i}{id\i-l1}{3}

        \lPortAtHeightOf{sched\i}{sched\i-r3}{1}
    }

    \connectedBetweenR{epsilon}{id2-r1}{1}
    \connectedBetweenR{epsilon}{idbot-r1}{2}

    \connectedBetweenL{injector}{sched1-l1}{1}
    \draw (id1-r1) -- (sched2-l1);
    \lPortAtHeightOf{injector}{injector-r1}{1}
    \connectedBetweenL{eta}{injector-l1}{1}
\end{tikzpicture}
}
\caption{Schematic of \cyclicschedulerSys{2}}
\label{fig:cyclicschematic}
\end{figure}
