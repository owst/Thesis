\newcommand{\mkRuleName}[3]{%
\expandafter\def\csname #1#2RuleName\endcsname{#3\capitalisewords{#1}}%
}

\newcommand{\mkRuleNames}[1]{%
    \mkRuleName{#1}{Ty}{T}
    \mkRuleName{#1}{Ev}{E}
}

\mkRuleNames{var}
\mkRuleNames{bind}
\mkRuleNames{lam}
\mkRuleNames{app}
\mkRuleNames{seq}
\mkRuleNames{ten}
\mkRuleNames{nat}
\mkRuleNames{net}
\mkRuleNames{fold}

\newcommand{\inferenceRule}[1]{\RefTirName{#1}}

In this section we give a formal description of the language \DSL{} for
specifying net compositions. In particular, we show how \DSL{} expressions are
evaluated to nets, and how the type system ensures that only valid expressions
are processed, via the slogan \emph{well-typed programs are well-composed}
(\thmref{thm:typeSafety}), in the style of Milner~\cite{Milner1978}.

\DSL{} is a monomorphic, call-by-value functional language; it follows the
tradition in classic functional programming languages, such as ML, of extending
the syntax of the lambda calculus with convenient programming constructs. In
particular, we add natural numbers, net literals/composition, syntax for
variable binding and recursion over natural numbers.

The abstract syntax of \DSL{} is given in~\figref{fig:syntax}: $\varExpr$ is
drawn from a countable set of variables, $\netExpr$ is a net literal (defined
using a low-level syntax that we describe in \secref{sec:netliteralspec}).
Natural number literals are written as standard: $0,1,\dots$, but semantically
are in Peano form: $\natExpr$ is either $0$ or $\succNat{\natExpr'}$ for some
natural literal $\natExpr'$.  Function application is indicated by simple
juxtaposition: $\appExpr{\expr_1}{\expr_2}$. Following mathematical convention,
`$\comp$' and `$\tensor$' associate to the left, with `$\tensor$' binding
tighter than `$\comp$'.

\begin{figure}[ht]
\begin{align*}
    \expr &=    \varExpr
                    & (\text{variable}) \\
          &\mid \natExpr
                    & (\text{natural number literal}) \\
          &\mid \netExpr
                    & (\text{net literal}) \\
          &\mid \bindExpr{\varExpr}{\expr_1}{\expr_2}
                    & (\text{variable binding})\\
          &\mid \lamExpr{\varExpr}{\type}{\expr}
                    & (\text{function abstraction})\\
          &\mid \appExpr{\expr_1}{\expr_2}
                    & (\text{function application})\\
          &\mid \compExpr{\expr_1}{\expr_2}
                    & (\text{sequential net composition})\\
          &\mid \tensExpr{\expr_1}{\expr_2}
                    & (\text{tensor net composition})\\
          &\mid \natRecursionExpr{\expr_n}{\expr_z}{\expr_s}
                    & (\text{recursion over natural numbers})\\
\end{align*}
\caption{Syntax of \DSL{}\label{fig:syntax}}
\end{figure}

Note that the sequence construction, $\sequenceExpr{\expr_n}{\expr_1}$ for an
expression $\expr_n$, discussed
in~\secref{sec:motivation-lang}, does not appear in the formal syntax, since it
is a syntactic sugar for a fold:
\[
    \sequenceExpr{\expr_n}{\expr_1} \defeq
\natRecursionExpr
    {\parens{\expr_n - 1}}
    {\expr_1}
    {\parens{\lamExpr{\varExpr}{\netType{\aN}{\aN}}{\compExpr{\expr_1}{\varExpr}}}}
\] where $\succNat{\natExpr} - 1 \defeq \natExpr$.
Notice however, that $\sequenceExpr{0}{\expr_1}$ cannot appear in any
expression: $0 - 1$ is undefined\footnote{An alternative, would be to consider
$n^0$ as $\idC{\aN}$, assuming $n \withNetType{\aN}{\aN}$}. We discuss the
importance of the type annotation, $\netType{\aN}{\aN}$ on the lambda
expression in \secref{sec:typechecking}, but briefly, for the sequence to be
correct, $\expr_1$ must be an expression representing a net with $\aN$ boundary
ports on each side.

\subsection{Operational Semantics}\label{sec:operationalSemantics}

We define the big-step operational semantics of \DSL{} in \figref{fig:opsem},
using an explicit variable binding environment $\opEnv$, which we discuss
shortly. The language is call-by-value, and the evaluation rules reduce \DSL{}
expressions to values, which have three forms:
\begin{inlinelist}
    \item (Composite) nets,
    \item Natural numbers, and
    \item Function closures (consisting of an environment and lambda
        abstraction)
\end{inlinelist}.  A variable binding environment, $\opEnv$, is simply a map
from variables, $\varExpr$, to values, $\val$, as shown in \figref{fig:values}.

\begin{figure}[ht]
\begin{align*}
    \val &= \netExpr
                   & (\text{Net Literal}) \\
         &\mid \natExpr
                   & (\text{Natural Number Literal}) \\
         &\mid \closure{\opEnv}{\lamExpr{\varExpr}{\type}{\expr}}
                   & (\text{Function Abstraction})\\
\\
    \opEnv &= \emptyEnv
               & (\text{Empty Environment}) \\
           &\mid \extendEnv{\opEnv}{\varExpr}{\val}
               & (\text{Environment Extension})
\end{align*}
\caption{\DSL{} Values and Operational Environments\label{fig:values}}
\end{figure}

The evaluation rules are of the form:
\(
    \opEnv \entails \reducesTo{\expr}{\val}
\), meaning that in environment $\opEnv$, expression $\expr$ reduces to value
$\val$. Evaluation intuitively proceeds as follows: variables are simply looked
up in the environment, lambdas evaluate to a closure over the current
environment, and numeric/net literals evaluate to themselves. Bindings evaluate
their body in an extended environment, similarly for applications. Synchronous
and tensor composition evaluate both expressions to net literals, before
applying the appropriate net operation. Structural recursion over the natural
numbers (commonly referred to as a \emph{fold}, e.g. see~\cite{Hinze2005}) is
implemented via repeated function application of $\expr_s$ to $\expr_z$.

\begin{figure}[ht]
    \newcommand{\opRule}[2]{\opEnv \entails \reducesTo{#1}{#2}}
    \newcommand{\opExtendRule}[5]{\extendEnv{#1}{#2}{#3} \entails \reducesTo{#4}{#5}}
\begin{gather*}
    \ruleitem{(\varEvRuleName)}{}{\placeholder}{\opRule{\varExpr}{\opEnv(\varExpr)}}
    \hspace{1cm}
    \ruleitem{(\lamEvRuleName)}{}{\placeholder}
        {\opRule{\lamExpr{\varExpr}{\type}{\expr}}
                {\closure{\opEnv}{\lamExpr{\varExpr}{\type}{\expr}}}}
    \\[\inferrulespace]
    \ruleitem{(\bindEvRuleName)}{}
        {\opRule{\expr_1}{\val_1} \\
         \opExtendRule{\opEnv}{\varExpr}{\val_1}{\expr_2}{\val_2}}
         {\opRule{\bindExpr{\varExpr}{\expr_1}{\expr_2}}{\val_2}}
    \\[\inferrulespace]
    \ruleitem{(\appEvRuleName)}{}
    {\opRule{\expr_1}{\closure{\opEnv'}{\lamExpr{\varExpr}{\type}{\expr_3}}} \\
         \opRule{\expr_2}{\val_2} \\
         \opExtendRule{\opEnv'}{\varExpr}{\val_2}{\expr_3}{\val_3}}
         {\opRule{\appExpr{\expr_1}{\expr_2}}{\val_3}}
    \\[\inferrulespace]
    \ruleitem{(\natEvRuleName)}{}{\placeholder}{\opRule{\natExpr}{\natExpr}}
    \hspace{1cm}
    \ruleitem{(\netEvRuleName)}{}{\placeholder}{\opRule{\netExpr}{\netExpr}}
    \\[\inferrulespace]
    \ruleitem{(\seqEvRuleName)}{}
        {\opRule{\expr_1}{\netExpr_1} \\
         \opRule{\expr_2}{\netExpr_2} }
         {\opRule{\compExpr{\expr_1}{\expr_2}}{\netExpr_1 \comp \netExpr_2}}
    \\[\inferrulespace]
    \ruleitem{(\tenEvRuleName)}{}
        {\opRule{\expr_1}{\netExpr_1} \\
         \opRule{\expr_2}{\netExpr_2} }
         {\opRule{\tensExpr{\expr_1}{\expr_2}}{\netExpr_1 \tensor \netExpr_2}}
    \\[\inferrulespace]
    \ruleitem{(\foldEvRuleName{}z)}{}
    {\opRule{\expr_n}{0} \\ \opRule{\expr_z}{\val}}
    {\opRule{\natRecursionExpr{\expr_n}{\expr_z}{\expr_s}}{\val}}
    \\[\inferrulespace]
    \ruleitem{(\foldEvRuleName{}s)}{}
        {\opRule{\expr_n}{\succNat{\natExpr}} \\
         \opRule
            {\appExpr{\expr_s}{(\natRecursionExpr{\natExpr}{\expr_z}{\expr_s}})}
            {\val}}
        {\opRule{\natRecursionExpr{\expr_n}{\expr_z}{\expr_s}}{\val}}
\end{gather*}
\caption{Operational Semantics of \DSL{}}
\label{fig:opsem}
\end{figure}

For example, considering the \DSL{} expression given at the end of
\secref{sec:motivation-lang}; after expanding syntactic sugar, the following
example reduction holds.

\begin{example}
\setlength{\abovedisplayskip}{0pt}
\setlength{\belowdisplayskip}{0pt}
For expression
\begin{flalign*}
    \expr &=
\begin{aligned}[t]
    &\bindExpr
        {\varName{makeRing}}
        {\lamExpr{\varExpr}{\netType{1}{1}}
                 {\compExpr{\etaC{}}
                           {\compExpr{\parens{\tensExpr{\varExpr}{\idC{}}}}
                                     {\epsilonC{}}}}}
        {}\\
    &\bindExpr
        {\varName{procs}}
        {\natRecursionExpr{2}{\workerC}{\parens{\lamExpr{\varExpr}{\netType{1}{1}}{\compExpr{\workerC}{\varExpr}}}}}
        {}\\
    &\appExpr{\varName{makeRing}}{\parens{\compExpr{\injectorC}{\varName{procs}}}}
        &
\end{aligned}\\
\text{ and value }
\\
    \val &= \etaC{} \comp \parens{\injectorC{} \comp \parens{\workerC
    \comp \parens{\workerC \comp \workerC}}} \tensor \idC{} \comp \epsilonC{}
\\
\text{ we have that }
\\
\emptyset &\entails \reducesTo{\expr}{\val}
\end{flalign*}
\begin{proof}
    Direct, by the definition of the evaluation rules.
\end{proof}
\end{example}

\subsection{Static Type Checking} \label{sec:typechecking}

As mentioned in \secref{sec:motivation-lang}, \emph{run-time} synchronous composition
errors are encountered when attempting to synchronously compose nets with
differently-sized boundaries. Such errors can be ruled out \emph{statically},
that is, before evaluation is performed, by using a \emph{type system}.
Such a type system ascribes a \emph{type} to each expression, with
a type being a static approximation of the dynamic (after evaluation, at
run-time) form of the expression~\cite{Cousot1977}. Indeed, other run-time
errors can be ruled out by using a static type system; such errors include
treating a net as a function or trying to tensor two lambda expressions.

In this subsection, we introduce a simple, \emph{monomorphic} type
system, giving rules for ascribing types to expressions, before proving a
safety property: if a expression can be typed, it cannot fail to evaluate to a
value with the same type.

\subsubsection{Monomorphic Type System}

A monomorphic type system ascribes a \emph{single} type to any particular
expression. Our monomorphic type system uses only two types, as shown in
\figref{fig:mono-types}.  As we will prove, these types can be used to rule out
\emph{all} possible run-time failures for \DSL{} expressions. The novel feature
is that component boundary sizes are tracked, achieved by parameterising the
net base type: $\netType{l}{r}$ by $l, r\in \N$, which describe the left and
right boundary sizes respectively. Indeed, the net component type,
$\netType{\aN}{\bN}$, will be ascribed to any expression, that, when evaluated,
results in a (composite) net with boundaries $\aN$ and $\bN$, for a particular
choice of $\aN$ and $\bN$. Thus the type $\netType{\aN}{\bN}$ approximates the
form of the resulting expression in that it only records the boundary sizes.

The deductive rules for assigning types to \DSL{} expressions are shown in
\figref{fig:monomorphic-typing}. As is standard (e.g.~\cite{Pierce2002}) for a
monomorphic language, $\tyEnv$ is a sequence of bindings from variables to
types (shown in \figref{fig:mono-types}), which is extended in the
\inferenceRule{\bindTyRuleName} and \inferenceRule{\lamTyRuleName} rules and
inspected in the \inferenceRule{\varTyRuleName} rule.

It is important to note is that lambda expressions parameters are annotated
with their type. This is necessary to \emph{infer} the type of arbitrary
expressions; inference involves reading type rules ``in reverse'', from
conclusion to premise, and regarding $\tyEnv$ and $\expr$ as \emph{inputs}, and
$\type$ as an output.  Indeed, in \inferenceRule{\lamTyRuleName}, we require a
single \emph{monomorphic} type for $\varExpr$, with which to extend the type
environment; without the annotation, we have no such type. For such a reverse
reading to be correct, the rules must be \emph{syntax-directed}: each syntactic
form of the expression language corresponds to exactly one type rule.

\begin{figure}[ht]
\begin{align*}
    \type &= \natType
              & (\text{Natural Number}) \\
          &\mid \netType{\aN}{\bN}\;\;(\aN,\bN \in \N)
              & (\text{Net Component}) \\
          &\mid \arrType{\type_1}{\type_2}
              & (\text{Function Type}) \\
    \\
    \tyEnv &= \emptyEnv
              & (\text{Empty Environment}) \\
           &\mid \extendEnv{\tyEnv}{\varExpr}{\type}
              & (\text{Environment Extension})
\end{align*}
\caption{Monomorphic Types and Type Environments of \DSL{}}
\label{fig:mono-types}
\end{figure}

\begin{figure}[ht]
\begin{gather*}
    \ruleitem{(\varTyRuleName)}{}{\tyEnv(\varExpr) = \type}{\tyRule{\varExpr}{\type}}
    \hspace{1cm}
    \ruleitem{(\bindTyRuleName)}{}
        {\tyRule{\expr_1}{\type_1} \\
         \tyExtendRule{\varExpr}{\type_1}{\expr_2}{\type_2}}
        {\tyRule{\bindExpr{\varExpr}{\expr_1}{\expr_2}}{\type_2}}
    \\[\inferrulespace]
    \ruleitem{(\lamTyRuleName)}{}
        {\tyExtendRule{\varExpr}{\type_1}{\expr}{\type_2}}
        {\tyRule{\lamExpr{\varExpr}{\type_1}{\expr}}{\arrType{\type_1}{\type_2}}}
    \hspace{1cm}
    \ruleitem{(\appTyRuleName)}{}
        {\tyRule{\expr_1}{\arrType{\type_1}{\type_2}} \\
         \tyRule{\expr_2}{\type_1}}
        {\tyRule{\appExpr{\expr_1}{\expr_2}}{\type_2}}
    \\[\inferrulespace]
    \ruleitem{(\natTyRuleName)}{}{\placeholder}{\tyRule{\natExpr}{\natType}}
    \hspace{1cm}
    \ruleitem{(\netTyRuleName)}{}{\netExpr \withNetType{\aN}{\bN}}{\tyRule{\netExpr}{\netType{\aN}{\bN}}}
    \\[\inferrulespace]
    \ruleitem{(\tenTyRuleName)}{}{\tyRule{\expr_1}{\netType{\aN}{\bN}} \\
                        \tyRule{\expr_2}{\netType{\cN}{\dN}}}
                       {\tyRule{\tensExpr{\expr_1}{\expr_2}}{\netType{\aN+\cN}{\bN+\dN}}}
    \\[\inferrulespace]
    \ruleitem{(\seqTyRuleName)}{}{\tyRule{\expr_1}{\netType{\aN}{\bN}} \\
                        \tyRule{\expr_2}{\netType{\cN}{\dN}} \\ \bN = \cN}
                       {\tyRule{\compExpr{\expr_1}{\expr_2}}{\netType{\aN}{\dN}}}
    \\[\inferrulespace]
    \ruleitem{(\foldTyRuleName)}{}
        {\tyRule{\expr_n}{\natType} \\
         \tyRule{\expr_z}{\type} \\
         \tyRule{\expr_s}{\arrType{\type}{\type}}}
        {\tyRule{\natRecursionExpr{\expr_n}{\expr_z}{\expr_s}}{\type}}
\end{gather*}
\caption{Syntax-directed Monomorphic Typing Rules for \DSL{}}
\label{fig:monomorphic-typing}
\end{figure}

We say an expression $\expr$ is \emph{well-typed}, if there exists a type
environment binding all free variables appearing in $\expr$, $\tyEnv$, and a
type, $\type$, such that $\tyEnv \entails \hasType{\expr}{\type}$.

As an example use of the typing rules, consider composing the lambda expression
that forms a loop from a sequence of workers, introduced in
\secref{sec:motivation-lang}, reproduced here:
\[
    \lamExpr{\varExpr}{\netType{1}{1}}
        {\compExpr{\etaC{}}
                  {\compExpr{\parens{\tensExpr{\varExpr}{\idC{}}}}
                            {\epsilonC{}}}}
\]
We can confirm that this expression is well-composed (i.e.\ that it contains no
invalid compositions, a consequence of being well-typed) with the proof
illustrated in \figref{fig:typingproof} (using $\eta,i,\epsilon$ in place of
$\component{eta},\component{id},\component{epsilon}$).
\begin{figure}[ht]
    \centering
\newcommand{\theTyEnv}{\extendEnv{\emptyEnv}{\varExpr}{\netType{1}{1}}}
\[
\inferrule*
    {
    \inferrule*
        {
            \inferrule*
                {\eta \withNetType{0}{2}}
                {
                    \tyEnv
                    \entails
                    \hasType{\eta}{\netType{0}{2}}
                }
            \\
            \inferrule*
                {
                    \inferrule*
                        {
                        \inferrule*
                            {\tyEnv(\varExpr) = \netType{1}{1}}
                            {
                                \tyEnv
                                \entails
                                \hasType{\varExpr}{\netType{1}{1}}
                            }
                        \\
                        \inferrule*
                            {i \withNetType{1}{1}}
                            {
                                \tyEnv
                                \entails
                                \hasType{i}{\netType{1}{1}}
                            }
                        }
                        {
                            \tyEnv
                            \entails
                            \hasType{\tensExpr{\varExpr}{i}}
                                {\netType{2}{2}}
                        }
                    \\
                    \inferrule*
                        {\epsilon \withNetType{2}{0}}
                        {
                            \tyEnv
                            \entails
                            \hasType{\epsilon}{\netType{2}{0}}
                        }
                }
                {
                    \tyEnv
                    \entails
                    \hasType{\compExpr
                                {\parens{\tensExpr{\varExpr}{i}}}
                                {\epsilon}}
                            {\netType{2}{0}}
                }
        }
        {\theTyEnv
         \entails
 \hasType{\compExpr{\eta}{\compExpr{\parens{\tensExpr{\varExpr}{i}}}{\epsilon}}}{\netType{0}{0}}}
    }
    {
        \emptyEnv
        \entails
        \hasType{\lamExpr{\varExpr}{\netType{1}{1}}{\compExpr{\eta}{\compExpr{\parens{\tensExpr{\varExpr}{i}}}{\epsilon}}}}{\arrType{\netType{1}{1}}{\netType{0}{0}}}
    }
\]
For compact presentation, let $\tyEnv \defeq \theTyEnv$.
\caption{Example typing proof}
\label{fig:typingproof}
\end{figure}

Values can be ascribed types; we write $\dentails \hasType{\val}{\type}$ if
value $\val$ has type $\type$. Furthermore, an operational environment (mapping
lambda-bound-variables to values), $\opEnv$, respects a type environment,
$\tyEnv$, iff the domains of $\tyEnv$ and $\opEnv$ are equal and $\opEnv$
point-wise respects $\tyEnv$, as per \figref{fig:typingvalues}.

\begin{figure}[ht]
    \begin{gather*}
        \ruleitem{}{}
            {\placeholder}
            {\dentails \hasType{\emptyEnv}{\emptyEnv}}
            \hspace{1cm}
        \ruleitem{}{}
            {\dentails \hasType{\opEnv}{\tyEnv} \\ \dentails \hasType{\val}{\type}}
            {\dentails \hasType{\extendEnv{\opEnv}{\varExpr}{\val}}
                              {\extendEnv{\tyEnv}{\varExpr}{\type}}}
        \\[\inferrulespace]
        \ruleitem{}{}
            {\placeholder}
            {\dentails \hasType{\natExpr}{\natType}}
        \hspace{1cm}
        \ruleitem{}{}
            {\netExpr \withNetType{\aN}{\bN}}
            {\dentails \hasType{\netExpr}{\netType{\aN}{\bN}}}
        \hspace{1cm}
        \ruleitem{}{}
            {\dentails \hasType{\opEnv}{\tyEnv} \\
             \tyExtendRule{\varExpr}{\type_1}{\expr}{\type_2}}
            {\dentails \hasType{\closure{\opEnv}
                                       {\lamExpr{\varExpr}{\type_1}{\expr}}}
                              {\arrType{\type_1}{\type_2}}}
    \end{gather*}
    \caption{Typing of Values and Operational Environments}
    \label{fig:typingvalues}
\end{figure}

Now we can state our formal notion of well-typed expressions can always be
evaluated to well-composed values:

\begin{theorem}\label{thm:typeSafety}
    For a well-typed expression, $\tyEnv \entails \hasType{\expr}{\type}$, and
    operational environment $\opEnv$ such that $\dentails
    \hasType{\opEnv}{\tyEnv}$, we have:
    \[
        \opEnv \entails \reducesTo{\expr}{\val}
        \text{ with } \dentails \hasType{\val}{\type}
    \]
\end{theorem}
\begin{proof}
    Induction over the structure of the proof of $\tyEnv \entails
    \hasType{\expr}{\type}$, uniquely determined by the structure of $\expr$.
    We only state the non-standard cases here:
    \begin{itemize}
        \item In the base cases of $\expr$ being a net or natural number
            literal the result is trivial.
        \item In the case of a tensor composition,
            $\tensExpr{\expr_1}{\expr_2}$, we apply the \IH{} to $\expr_1$ and
            $\expr_2$, obtaining values $\val_1$ and $\val_2$ such that
            $\dentails \hasType{\val_1}{\netType{\aN}{\bN}}$ and
            $\dentails \hasType{\val_2}{\netType{\cN}{\dN}}$ for some $\aN,
            \bN, \cN$ and $\dN$. Now, by the inversion of the syntax-directed
            typing rules, $\val_1$ and $\val_2$ must be net
            literals, $\netExpr_1 \withNetType{\aN}{\bN}$ and $\netExpr_2
            \withNetType{\cN}{\dN}$. Thus the premises for
            \inferenceRule{\tenEvRuleName} are satisfied and we have
            $\reducesTo{\tensExpr{\expr_1}{\expr_2}}{\netExpr_1 \tensor
            \netExpr_2}$, where $\netExpr_1 \tensor \netExpr_2 \withNetType{\aN
            + \cN}{\bN + \dN}$, giving $\dentails \hasType{\netExpr_1 \tensor
            \netExpr_2}{\netType{\aN + \cN}{\bN + \dN}}$, as required.
        \item The case of synchronous composition is similar:
            we apply the \IH{} to $\expr_1$ and
            $\expr_2$, obtaining values $\val_1$ and $\val_2$ such that
            $\dentails \hasType{\val_1}{\netType{\aN}{\bN}}$ and
            $\dentails \hasType{\val_2}{\netType{\cN}{\dN}}$ for some $\aN,
            \bN, \cN$ and $\dN$. Observe that the conclusion of
            $\inferenceRule{\seqEvRuleName}$ is only defined when $\bN = \cN$,
            which is ensured by the final premise on
            \inferenceRule{\seqTyRuleName}. Thus, by the definition of
            synchronous composition on net components, $\val$ is a net literal,
            $\netExpr_1 \comp \netExpr_2 \withNetType{\aN}{\dN}$, giving
            $\dentails \hasType{\netExpr_1 \comp
            \netExpr_2}{\netType{\aN}{\dN}}$, as required.
        \item Finally, for a fold,
            $\natRecursionExpr{\expr_n}{\expr_z}{\expr_s}$, we apply the \IH{}
            three times, obtaining values $\val_n, \val_z$ and $\val_s$ such
            that $\dentails \hasType{\val_n}{\N}$, $\dentails
            \hasType{\val_z}{\type}$ and $\dentails
            \hasType{\val_s}{\arrType{\type}{\type}}$.

            Now, we have to consider the the form of $\val_n$, which, by the
            inversion of the typing rules, must be a natural number literal,
            either $0$ or $\succNat{\natExpr}$:
            in the former case, $\opEnv \entails
            \reducesTo{\natRecursionExpr{0}{\expr_z}{\expr_s}}{\val_z}$, with
            $\dentails \hasType{\val_z}{\type}$, as required.
            In the latter case, where $\val_n$ is $\succNat{\natExpr}$, by our
            assumption that $\tyEnv \entails
            \hasType{\natRecursionExpr{\expr_n}{\expr_z}{\expr_s}}{\type}$, we
            have that $\hasType{\expr_s}{\arrType{\type}{\type}}$, and
            furthermore, that $\tyEnv \entails
            \hasType{\natRecursionExpr{\natExpr}{\expr_z}{\expr_s}}{\type}$ (by
            substituting a leaf for $\natExpr$ for the sub-tree corresponding
            to $\expr_n$). We therefore have the premises for
            \inferenceRule{\appTyRuleName}, giving the conclusion that
            $\tyEnv \entails \hasType{\appExpr{\expr_s}
                {\parens{\natRecursionExpr {\natExpr} {\expr_z}{\expr_s}}}}
                {\type}$.
            Now, by the \IH{}, we have that $\opEnv \entails
            \reducesTo{\appExpr{\expr_s}
                {\parens{\natRecursionExpr{\natExpr}{\expr_z}{\expr_s}}}}{\val}$
            and therefore
            $\reducesTo{\natRecursionExpr{\succNat{\natExpr}}{\expr_z}{\expr_s}}{\val}$,
            such that $\dentails \hasType{\val}{\type}$, as required.
    \end{itemize}
\end{proof}

\subsection{Net Literal Specification}
\label{sec:netliteralspec}

To finish this chapter, for completeness, we briefly discuss how component nets
are specified, such that they can be referred to within \DSL{} expressions.
Later in this thesis, we will be concerned with checking reachability of net
systems specified by \DSL{} expressions, therefore our component specification
language incorporates the initial/target markings of component places.

The grammar of the net specification language is as illustrated in
\figref{fig:literalSpecGrammar}.

\begin{figure}[ht]
\begin{lstlisting}
    pnbdef ::= "NET", name, "PLACES", places, "LBOUNDS", bounds,
                  "RBOUNDS", bounds, "TRANS", transitions ;

    name ::= ?sequence of alphanumeric characters?

    places ::= "[", { place, "," }, place, "]"
            |  "[", "]" ;

    place ::= "<", name, ",", initialmarking, ",", targetmarking ">" ;

    bounds ::= "[", { name, "," }, name, "]"
            |  "[", "]" ;

    transitions ::= "{", { transition, "," }, transition, "}"
                 |  "{", "}";

    transition ::= "{", { port, "," }, port "}"
                |  "{", "}" ;

    port ::= ">", name
          |  name, ">"
          |  name, "?"
          |  name ;

    initialmarking ::= "0"
                    |  "1" ;

    targetmarking ::= "0"
                   |  "1"
                   |  "*" ;

\end{lstlisting}
\caption{BNF grammar for the input language}
\label{fig:literalSpecGrammar}
\end{figure}

Each component definition specifies a list of places, a list of left and right
boundary port names, and a set of transitions. Each place is specified as a
triple of a (unique) name, initial marking and target marking, where the target
marking can be one of 3 values: ``yes'' (1), ``no'' (0) or ``don't care'' (*).
N.B.  since we allow ``don't care'' target markings, we are able to specify
\emph{coverability} problems in addition to \emph{reachability}.  Each
transition is specified as the set of ports that it connects, where each port
is one of 5 different types:
\begin{enumerate}
    \item Left boundary (denoted by a (left) boundary name),
    \item Right boundary (denoted by a (right) boundary name),
    \item Place input ({\bf P}roduce)
        (denoted by a place name preceded by '>'),
    \item Place output ({\bf C}onsume)
        (denoted by a place name followed by '>'),
    \item Place {\bf R}ead
        (denoted by a place name followed by '?').
\end{enumerate}

As an example of the component specification format, we show the definition of
the \philoC{} component, from \secref{sec:example-philos}.

\newcommand{\colA}{blue}
\newcommand{\colB}{black}
\newcommand{\colC}{teal}
\newcommand{\colD}{red}
\newcommand{\colE}{orange}

\begin{figure}[ht]
\begin{subfigure}{0.45\textwidth}
\centering
\begin{lstlisting}
NET philo
PLACES
    [ <none,  1,*>
    , <left,  0,*>
    , <right, 0,*>
    , <both,  0,1>
    ]
LBOUNDS [takeL, putL]
RBOUNDS [takeR, putR]
TRANS
    { {(*@ \textcolor{\colA}{none>,  takeL,      >left} @*)}
    , {(*@ \textcolor{\colB}{none>,  takeR,      >right} @*)}
    , {(*@ \textcolor{\colC}{left>,  takeR,      >both} @*)}
    , {(*@ \textcolor{\colD}{right>, takeL,      >both} @*)}
    , {(*@ \textcolor{\colE}{both>,  putL, putR, >none} @*)}
    }
\end{lstlisting}
\end{subfigure}%
\begin{subfigure}{0.55\textwidth}
    \centering
    \begin{tikzpicture}[pnb]
    \node[pnbplace] (none)  [tokens=1, rotate=270, pnblabel=above:none] {};
    \node[pnbplace] (left)  [below left of=none, rotate=270, pnblabel=below:left] {};
    \node[pnbplace] (right) [below right of=none, rotate=270, pnblabel=above:right] {};
    \node[pnbplace] (both)  [below left of=right, rotate=270, pnblabel=above:both] {};

    \drawBoundaries{2}{2}

    \lBoundaryLabel{1}{takeL}
    \lBoundaryLabel{2}{putL}
    \rBoundaryLabel{1}{takeR}
    \rBoundaryLabel{2}{putR}

    \labelledpnbarr[noneright]{none.out}{right.in}{}{\colB, pnbarrstyle1, o-90i90}{pos=0.9}
    \draw (r1) edge[\colB, pnbarrstyle1, out=180, in=120] (noneright);

    \labelledpnbarr[noneleft]{none.out}{left.in}{}{\colA, pnbarrstyle2, o-90i90}{pos=0.9}
    \draw (l1) edge[\colA, pnbarrstyle2, out=0, in=60] (noneleft);

    \labelledpnbarr[rightboth]{right.out}{both.in}{}{\colD, pnbarrstyle1, o-90i90}{}
    \draw (l1) edge[\colD, pnbarrstyle1, out=30, in=160] (rightboth);

    \labelledpnbarr[leftboth]{left.out}{both.in}{}{\colC, pnbarrstyle2,o-90i90}{}
    \draw (r1) edge[\colC, pnbarrstyle2, out=150, in=20] (leftboth);

    \labelledpnbarr[bothnone]{both.out}{none.in}{}{\colE, out=-150, in=150}{pos=0.1}

    \draw (l2) edge[\colE, out=-70, in=150] (bothnone);
    \draw (r2) edge[\colE, out=200, in=-45] (bothnone);
    \end{tikzpicture}
\end{subfigure}
\caption{\philoC{} using component specification format}
\end{figure}
