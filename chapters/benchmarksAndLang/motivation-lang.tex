In the previous examples we demonstrated the algebraic description of Petri net
systems in terms of their component PNBs. We now motivate using a Domain
Specific Language (DSL), \DSL{}, that evaluates to the algebra of PNBs, but
adds expressive high-level functional programming language features.

Consider again the algebraic description of a token ring network; how might we
generate the algebraic expression representing a similar network of, say, 10
worker components? The simple approach given is to explicitly write the term
\[
    \etaC{} \comp ((\injectorC \comp \workerC \comp \workerC \dots \comp
    \workerC) \tensor \idC{}) \comp \epsilonC{}
\]
where $\workerC$ appears precisely 10 times. However, this is clearly not
scalable: for large numbers of components, or more complex components (that may
themselves be formed of component compositions) it becomes a nuisance to
construct such low-level expressions, and furthermore, ensure that they are
correctly composed.

Consider the following expression that we might (accidentally) write
when composing worker nets, $\workerC{} \withNetType{1}{1}$:
\[
    \workerC \comp (\workerC \tensor \workerC)
\]
the resulting net is undefined, since the
two nets being synchronously composed have different size boundaries ---
composition on such nets is not definable in a unique way\footnote{In general
there will be many ways to align the boundaries, giving different nets.}.
Indeed, it is easy to observe that  $\workerC \tensor \workerC
\withNetType{2}{2}$. Yet, for the synchronous composition to be well-defined, we
must have that $\workerC \tensor \workerC$ has boundaries $(1, \aN)$ for any
$\aN$, a contradiction, indicating an invalid expression. To ensure that we
disallow such invalid expressions, we use an appropriate notion of \emph{type},
which can be ascribed to expressions, to ensure that incompatible nets are
never composed during evaluation. We will return to the description of the type
system later in this section.

When describing complex components, we would like any repeated sub-components
to be described only once, rather than each time they are used. Specifically,
we would like to bind a name to a sub-expression and be able to use that name
to refer to the sub-expression.
For example, we might consider an extended token ring network model where each
worker task, \workerC, is comprised of two sub-components: $\workerC_1$
and $\workerC_2$. Using name binding, we might describe a sequence of such
tasks as:
\newcommand{\workerVar}{\varName{w}}
\[
    \bindExpr
        {\workerVar}
        {(\compExpr
            {\workerC_1}
            {\workerC_2})}
        {\compExpr
            {\workerVar}
            {\compExpr{\workerVar}
                      {\compExpr{\dots}{\workerVar}}}}
\]

Another improvement that we can make is to introduce a compact form for
expressions that feature sequences of a repeated component, with
\emph{parameterised}
length.
\begin{align*}
    &\bindExpr
        {\workerVar}
        {(\compExpr
            {\workerC_1}
            {\workerC_2})}{} \\
    &\sequenceExpr{\aN}{\workerVar}
\end{align*}
where $\sequenceExpr{\aN}{c}$ represents $((c \comp c) \comp c) \dots \comp c$
with $c$ appearing $\aN$ times, similarly to the syntax sugar introduced in
\remref{rem:exponentiationSyntax}.

Finally, consider the procedure that takes an arbitrary-length sequence of
workers and forms a ring by connecting the first to the last. Such a procedure
can be written, using a lambda notation common to functional programming
languages, as:
\[
    \lamExpr{\varExpr}{\netType{1}{1}}
        {\compExpr{\etaC{}}
                  {\compExpr{\parens{\tensExpr{\varExpr}{\idC{}}}}
                            {\epsilonC{}}}}
\]
that is, take a suitable sequence of workers (with left and right boundaries of
size 1), represent it by the variable $\varExpr$ and perform the appropriate
compositions to form the ring.

Other examples, such as those in the remainder of this section, are naturally
\emph{parametric} and can thus be compactly represented for any particular
parameter choice. As examples, we may represent the \bufferSys{\aN} system
as:
\[
    \compExpr{\lendC{}}{\compExpr{\sequenceExpr{\aN}{\bufferC}}{\rendC{}}}
\]
and the \tokenringSys{3} system shown in \figref{fig:tokenring}, with the
expression:
\begin{align*}
    &\bindExpr
        {\varName{makeRing}}
        {\lamExpr{\varExpr}{\netType{1}{1}}
                 {\compExpr{\etaC{}}
                           {\compExpr{\parens{\tensExpr{\varExpr}{\idC{}}}}
                                     {\epsilonC{}}}}}
        {}\\
    &\bindExpr
        {\varName{procs}}
        {\sequenceExpr{3}{\workerC}}
        {}\\
    &\appExpr{\varName{makeRing}}{\parens{\compExpr{\injectorC}{\varName{procs}}}}
\end{align*}

We defer formally introducing \DSL{} to \secref{sec:lang}. Instead, we now use
it to give encodings of several well-known example systems.
