Recall that the \TLTS{} semantics of a PNB $\aPNB \withNetType{\aN}{\bN}$, has
labels of the form $\lbl{\aLbl}{\bLbl}$, syntactic sugar for $\parens{\aLbl,
\bLbl} \in \B^\aN \times \B^\bN$. Via concatenation, these labels can be
written as a single binary string of length $\aN+\bN$. Therefore, to represent
the transitions of the \TLTS{} we require a data structure representing
$\aLTSAllStates \times \B^{\aN+\bN} \times \aLTSAllStates$, where
$\aLTSAllStates$ is the set of states of the \TLTS{}. The nature of the
transition relation means it can be encoded as a function: $\aLTSAllStates \to
\B^{\aN+\bN} \to \powerset{\aLTSAllStates}$, i.e. for each source state,
$\aNFAState \in \aLTSAllStates$, we have a function $\B^{\aN+\bN} \to
\powerset{\aLTSAllStates}$ encoding the set of states reachable from
$\aNFAState$, via transitions with labels in $\B^{\aN+\bN}$.

A well-known, efficient representation of functions $\B^\aN \to \B$ (i.e.\
$\aN$ binary valued-variables that determine a binary value) are Reduced
Ordered Binary Decision Diagrams (simply, BDDs)~\cite{Bryant1986}. However,
to use BDDs directly, the \TLTS{} transition function co-domain must be
suitably encoded as a binary string, to extend the binary label, giving a
target of a single binary bit. A more natural representation uses
Multi-Terminal BDDs~\cite{Clarke1993}, a generalisation of BDDs that represent
functions with a co-domain other than the booleans.

MTBDDs encode functions that take an assignment of boolean variables to a
\emph{set} of values. Specifically, a MTBDD is a ROBDD, where the two-element
Boolean algebra of $\B$ has been replaced with the Boolean algebra of subsets
of some non-empty set, $S$.

For our purposes, $S$ is the set of LTS states. The MTBDD's variables are
\emph{boundary positions}: $\setBuilder{\varL{i}}{0 \le i < \aN} \union
\setBuilder{\varR{j}}{0 \le j < \bN}$, where, e.g. $\varL{0}$ is the 0th left
boundary port, $\varR{1}$ the first right boundary port and so on. MTBDDs, like
BDDs, require that the variables have a fixed ordering, which we take to be the
simple ordering top-down, left-before-right, i.e. the lexicographic
order (where $L < R$):
\[
    \pairof{lr_1}{\aN} < \pairof{lr_2}{\bN} \implies lr_1 < lr_2
    \,\disjunction\, (lr_1 = lr_2 \,\conjunction\, \aN < \bN)
\]
\begin{example}
    Consider the \TNFA{} semantics of a single \bufferC{} component, as
    illustrated in \figref{fig:bufferCNFA}.

    The MTBDD representations of the transitions from states $0$ and $1$
    are shown in \figref{fig:MTBDDsbufferCNFA}, where we use the convention
    that the false branch is a dashed line and the true branch is solid. State
    $0$'s MTBDD efficiently encodes that there are no transitions from state
    $0$ with a $\lbl{1}{-}$ label.
\end{example}

\begin{figure}[ht]
    \centering
    \begin{tikzpicture}[nfa]
        \node (0) [state, init]                  {$0$};
        \node (1) [state, accepting, below=of 0] {$1$};
        \path (0) edge [bend left]  node {$\lbl{0}{1}$} (1);
        \path (0) edge [loop right] node {$\lbl{0}{0}$} (0);
        \path (1) edge [bend left]  node {$\lbl{1}{0}$} (0);
        \path (1) edge [loop right] node {$\lbl{0}{0}$} (1);
    \end{tikzpicture}
    \caption{\TNFA{} semantics of \bufferC{}}
    \label{fig:bufferCNFA}
\end{figure}

\begin{figure}[ht]
    \begin{subfigure}[b]{0.5\textwidth}
        \centering
        \begin{tikzpicture}[bdd]
            \node[var]                    (0) {\varL{0}};
            \node[var, below right of=0]  (1) {\varR{0}};
            \node[leaf, below left of=1]  (3) {$\setof{0}$};
            \node[leaf, below right of=1] (4) {$\setof{1}$};
            \node[leaf, left of=3]        (2) {$\emptyset$};

            \falseEdge{0}{1}
            \trueEdge{0}{2}
            \falseEdge{1}{3}
            \trueEdge{1}{4}
        \end{tikzpicture}
        \caption{$0 \rightarrow$}
    \end{subfigure}
    \begin{subfigure}[b]{0.5\textwidth}
        \centering
        \begin{tikzpicture}[bdd]
            \node[var] (0L)                   {\varL{0}};
            \node[var, below left of=0L]  (1) {\varR{0}};
            \node[var, below right of=0L] (2) {\varR{0}};
            \node[leaf, below left of=1]  (3) {$\setof{1}$};
            \node[leaf, below right of=2] (5) {$\setof{0}$};
            \node[leaf] (4) at \bn{3}{.5}{5}  {$\emptyset$};

            \falseEdge{0}{1}
            \trueEdge{0}{2}
            \falseEdge{1}{3}
            \trueEdge{1}{4}
            \trueEdge{2}{4}
            \falseEdge{2}{5}
        \end{tikzpicture}
        \caption{$1 \rightarrow$}
    \end{subfigure}
    \caption{MTBDD representations of the transitions for states $0$ and $1$,
    in \figref{fig:bufferCNFA}.}
    \label{fig:MTBDDsbufferCNFA}
\end{figure}

Similarly to BDDs, the particular ordering chosen for the variables can
drastically affect the representation size of MTBDDs. For example, consider the
MTBDD illustrated in \figref{fig:MTBDDVarOrdering}, where we get a small size
decrease when re-ordering variables. In general the effect can be much more
pronounced. Clearly it would be preferable to chose the ordering that gave the
most compact MTBDD; unfortunately, the problem of choosing an optimal ordering
is NP-complete~\cite{Bollig1996}, we therefore use the lexicographic ordering
since it is simple, and empirical results suggest it performs well.

\begin{figure}[ht]
    \begin{subfigure}[b]{0.5\textwidth}
        \centering
        \begin{tikzpicture}[bdd]
            \node[var]                    (0) {\varL{0}};
            \node[var, below right of=0]  (1) {\varR{0}};
            \node[leaf, below left of=1]  (3) {$\setof{0}$};
            \node[leaf, below right of=1] (4) {$\setof{1}$};
            \node[leaf, left of=3]        (2) {$\emptyset$};

            \falseEdge{0}{1}
            \trueEdge{0}{2}
            \falseEdge{1}{3}
            \trueEdge{1}{4}
        \end{tikzpicture}
        \caption{$\varL{0} < \varR{0}$.}
    \end{subfigure}
    \begin{subfigure}[b]{0.5\textwidth}
        \centering
        \begin{tikzpicture}[bdd]
            \node[var] (0L)                   {\varR{0}};
            \node[var, below left of=0L]  (1) {\varL{0}};
            \node[var, below right of=0L] (2) {\varL{0}};
            \node[leaf, below left of=1]  (3) {$\setof{0}$};
            \node[leaf, below right of=2] (5) {$\setof{1}$};
            \node[leaf] (4) at \bn{3}{.5}{5}  {$\emptyset$};

            \falseEdge{0}{1}
            \trueEdge{0}{2}
            \falseEdge{1}{3}
            \trueEdge{1}{4}
            \trueEdge{2}{4}
            \falseEdge{2}{5}
        \end{tikzpicture}
        \caption{$\varR{0} < \varL{0}$.}
    \end{subfigure}
    \newcommand{\capt}{MTBDD size affected by variable ordering}
    \caption[\capt]{\capt:
        MTBDDs representing $0 \rightarrow$ in \figref{fig:bufferCNFA}, with
        different variable orderings.}
    \label{fig:MTBDDVarOrdering}
\end{figure}

\subsection{Synchronising Composition with MTBDDs}

As we have shown, the use of MTBDDs gives efficient representation of
transitions in \TNFA{} semantics, and furthermore many operations on
transitions, such as union of transitions, or $\epsilon$-closure have efficient
implementations. However, not all operations are natural and efficient, for
example, computing the sequential composition of two states requires computing
the \emph{synchronisation} of their transition MTBDDs. Intuitively, we want to
compute the MTBDD cross product since the synchronising composition corresponds
to taking a transition in each component, before restricting the cross product
to transitions that agree on the shared boundary. Thus, when composing a
\NFAB{\aN}{\bN} and a \NFAB{\bN}{\cN}, a simple but naive approach is:
\begin{enumerate}
    \item \label{step:n1} Generate each of the $2^\bN$ variable assignments for
        the common boundary
    \item \label{step:n2} In turn, instantiate the left and right MTBDDs using
        each assignment, before performing the cross product on the resulting
        pair of MTBDDs, generating a \emph{collection} of MTBDDs
    \item \label{step:n3} Finally, collapse the collection of MTBDDs using a
        pairwise union.
\end{enumerate}
Step~\ref{step:n2}, involves generating a collection of MTBDDs that record the
effect on the non-shared boundaries (i.e. the \emph{outer} boundaries, the left
boundary of the first component and the right boundary of the second) when the
two \TNFA{}s synchronise in each of the $2^\bN$ ways. Taking the union of these
MTBDDs in step~\ref{step:n3} is intuitively ignoring \emph{which} particular
synchronisation occurred. This approach is inefficient: it involves explicitly
generating a large collection of variable instantiations, even for common
variables that do not take part in the synchronisation (and thus do not appear
in the MTBDDs), thus we may unnecessarily perform substantial work.

A more efficient approach, that we explore now, is as follows:
\begin{enumerate}
    \item \label{step:b1} \emph{Pre-process} the two MTBDDs, renaming variables
        to tag \emph{synchronisation} variables (those on the common
        boundaries),
    \item \label{step:b2} Perform the cross product on these MTBDDs,
    \item \label{step:b3} \emph{Remove} any invalid sub-graphs of the MTBDD. We
        consider a sub-graph to be invalid if it assumes conflicting
        assignments to a particular boundary variable.
    \item \label{step:b4} Remove the remaining common variables, by taking the
        union of their sub-graphs.
\end{enumerate}

The pre-processing of \stepref{step:b1} renames the first component's variable
identifiers from $L/R$ to $L/S_R$, and the second component's from $L/R$ to
$S_L/R$, i.e. \emph{tagging} synchronisation variables to identify which
component they originated from. Importantly, the order on the resulting pairs
is changed, to form the ordering that places all Left variables before the
\emph{interleaving} of the Synchronisation variables before all Right
variables. For example, this ordering has the following:
\[
    \varL{1} < \commonL{0} < \commonR{0} < \commonR{1} < \commonL{2} < \varR{0}
\]
This ordering means that in a synchronised MTBDD, synchronisation variables
appear in subsequent nodes in the MTBDD, and can easily be checked for
conformance.

As an example, we will consider synchronously composing the \TNFA{}s of two \bufferC{}
components, with the left component's \TNFA{} in state 0, and
the right's in state 1. Thus we must compose the MTBDDs shown in
\figref{fig:MTBDDsbufferCNFA}. First, in \stepref{step:b1}, the MTBDDs are
pre-processed, leading to the MTBDDs illustrated in
\figref{fig:MTBDDsbufferCNFAPrepro}. Next, \stepref{step:b2} performs the cross
product of these two MTBDDs, leading to the MTBDD shown in
\figref{fig:fullCrossProduct}. This MTBDD contains \emph{semantically invalid}
sub-graphs, those highlighted red, which are reached by following a path where
the synchronisation variables have taken different values. Therefore, in
\stepref{step:b3}, we remove all such sub-graphs, as shown in
\figref{fig:strippedCrossProduct}. Finally, the synchronisations are removed in
\stepref{step:b4} by taking the union of the sub-graphs, giving the final
result illustrated in \figref{fig:syncsRemoved}, which encodes that the only
transitions possible from state $\pairof{0}{1}$ are $\tau$ transitions, either
staying in the same state or reaching $\pairof{1}{0}$.

\begin{figure}[ht]
    \begin{subfigure}[b]{0.5\textwidth}
        \centering
        \begin{tikzpicture}[bdd]
            \node[var]                    (0) {\varL{0}};
            \node[var, below right of=0]  (1) {\commonR{0}};
            \node[leaf, below left of=1]  (3) {$\setof{0}$};
            \node[leaf, below right of=1] (4) {$\setof{1}$};
            \node[leaf, left of=3]        (2) {$\emptyset$};

            \falseEdge{0}{1}
            \trueEdge{0}{2}
            \falseEdge{1}{3}
            \trueEdge{1}{4}
        \end{tikzpicture}
        \caption{$0 \rightarrow$.}
        \label{fig:processedMTBDD1}
    \end{subfigure}
    \begin{subfigure}[b]{0.5\textwidth}
        \centering
        \begin{tikzpicture}[bdd]
            \node[var] (0L) {\commonL{0}};
            \node[var, below left of=0L]  (1) {\varR{0}};
            \node[var, below right of=0L] (2) {\varR{0}};
            \node[leaf, below left of=1]  (3) {$\setof{1}$};
            \node[leaf, below right of=2] (5) {$\setof{0}$};
            \node[leaf] (4) at \bn{3}{.5}{5}  {$\emptyset$};

            \falseEdge{0}{1}
            \trueEdge{0}{2}
            \falseEdge{1}{3}
            \trueEdge{1}{4}
            \trueEdge{2}{4}
            \falseEdge{2}{5}
        \end{tikzpicture}
        \caption{$1 \rightarrow$.}
        \label{fig:processedMTBDD2}
    \end{subfigure}
    \caption{``Pre-processed'' MTBDDs of \figref{fig:MTBDDsbufferCNFA}}
    \label{fig:MTBDDsbufferCNFAPrepro}
\end{figure}

\begin{figure}[ht]
    \centering
    \begin{tikzpicture}[bdd, node distance=1.5cm]
        \node (6) [red, leaf]             {$\setof{\pairof{0}{0}}$};
        \node (8) [leaf, right=of 6]      {$\setof{\pairof{1}{0}}$};
        \node (1) [leaf, right=of 8]      {$\emptyset$};
        \node (0) [leaf, right=of 1]      {$\setof{\pairof{0}{1}}$};
        \node (3) [red, leaf, right=of 0] {$\setof{\pairof{1}{1}}$};

        \node (7)  [red, var, yshift=1.3*\nodedist] at \bn{6}{1/5}{3} {\varR{0}};
        \node (9)  [var, yshift=1.3*\nodedist] at \bn{6}{2/5}{3} {\varR{0}};
        \node (2)  [var, yshift=1.3*\nodedist] at \bn{6}{3/5}{3} {\varR{0}};
        \node (4)  [red, var, yshift=1.3*\nodedist] at \bn{6}{4/5}{3} {\varR{0}};

        \node (10) [var, yshift=1.3*\nodedist] at \bn{7}{1/3}{4} {\commonR{0}};
        \node (5)  [var, yshift=1.3*\nodedist] at \bn{7}{2/3}{4} {\commonR{0}};
        \node (11) [var, yshift=1.3*\nodedist] at \bn{10}{.5}{5} {\commonL{0}};
        \node (12) [var, above right=of 11, xshift=-1cm] {\varL{0}};

        \falseEdge{2}{0}
        \trueEdge{2}{1}
        \falseEdge[red]{4}{3}
        \trueEdge[red]{4}{1}
        \falseEdge{5}{2}
        \trueEdge{5}{4}
        \falseEdge[red]{7}{6}
        \trueEdge[red]{7}{1}
        \falseEdge{9}{8}
        \trueEdge{9}{1}
        \falseEdge{10}{7}
        \trueEdge{10}{9}
        \falseEdge{11}{5}
        \trueEdge{11}{10}
        \falseEdge{12}{11}
        \trueEdge{12}{1}
    \end{tikzpicture}
    \caption{MTBDD constructed using the cartesian product of those in
        \figref{fig:processedMTBDD1} and \figref{fig:processedMTBDD2}, with
        invalid sub-graphs highlighted}
    \label{fig:fullCrossProduct}
\end{figure}

\begin{figure}[ht]
    \centering
    \begin{tikzpicture}[bdd, node distance=1.5cm]
        \node (8) [leaf]             {$\setof{\pairof{1}{0}}$};
        \node (1) [leaf, right=of 8] {$\emptyset$};
        \node (0) [leaf, right=of 1] {$\setof{\pairof{0}{1}}$};

        \node (9)  [var, yshift=1.3*\nodedist] at \bn{8}{.5}{1} {\varR{0}};
        \node (2)  [var, yshift=1.3*\nodedist] at \bn{1}{.5}{0} {\varR{0}};

        \node (10) [var, above=of 9] {\commonR{0}};
        \node (5)  [var, above=of 2] {\commonR{0}};

        \node (11) [var, yshift=1.3*\nodedist] at \bn{10}{.5}{5} {\commonL{0}};
        \node (12) [var, above right=of 11, xshift=-0.9cm] {\varL{0}};

        \falseEdge{2}{0}
        \trueEdge{2}{1}
        \falseEdge{5}{2}
        \falseEdge{9}{8}
        \trueEdge{9}{1}
        \trueEdge{10}{9}
        \falseEdge{11}{5}
        \trueEdge{11}{10}
        \falseEdge{12}{11}
        \trueEdge{12}{1}
    \end{tikzpicture}
    \caption{MTBDD of \figref{fig:fullCrossProduct}, with invalid sub-graphs
        removed}
    \label{fig:strippedCrossProduct}
\end{figure}

\begin{figure}[ht]
    \centering
    \begin{tikzpicture}[bdd, node distance=1.5cm]
        \node (12) [var]                    {$\pairof{L}{0}$};
        \node (9)  [var, below left=of 12]  {$\pairof{R}{0}$};
        \node (8)  [leaf, below left=of 9]  {$\setof{\pairof{0}{1}, \pairof{1}{0}}$};
        \node (1)  [leaf, below right=of 9] {$\emptyset$};

        \falseEdge{12}{9}
        \trueEdge{12}{1}
        \falseEdge{9}{8}
        \trueEdge{9}{1}
    \end{tikzpicture}
    \caption{MTBDD of \figref{fig:strippedCrossProduct}, with synchronisations
        removed}
    \label{fig:syncsRemoved}
\end{figure}

Having discussed the three key implementation details for the efficiency of our
implementation, we now move onto empirically comparing the resulting
performance with existing state-of-the-art tools.
