We discuss three key design points of our implementation, each of which
strongly influence its efficiency and performance. As discussed in the previous
chapter, our technique uses weak \emph{language-preserving} reductions to
reduce component semantics, while ensuring all interactions with neighbouring
components are preserved (explored in~\secref{sec:boundaryProtocol}).
Furthermore, since the component structure is exposed, we are able to use
\emph{memoisation} to avoid repeating work by recognising already-computed
compositions (up-to language-equivalence of the component \TNFA{}s) as
discussed in~\secref{sec:memoisation}. Finally, as discussed in the related
work section (\secref{sec:decision-diagrams}), using an efficient
representation of transitions can radically improve performance; to this end we
use MTBDDs to represent the transitions of \TNFA{}s.

The three key design points are elaborated as follows:
\begin{enumerate}
    \item The algorithm for language-preserving reduction of
        \TNFA{}s in \secref{sec:nfaReduction},
    \item The algorithm for checking language-equvialence of
        \TNFA{}s in \secref{sec:checkNfaLangEquiv},
    \item The data structure used to represent \TNFA{}
        transitions in \secref{sec:mtbdds}.
\end{enumerate}

Our tool, \penrose{}, is written in Haskell~\cite{PeytonJones2003}, a
statically-typed, functional programming language. Commonly, model checkers are
implemented using C or similar low-level programming languages, so the use of
Haskell is unorthodox in this respect. Despite a highly performant compiler,
the levels of abstraction that are present in Haskell (garbage-collection, data
structures, etc.) can lead to an overhead that is not present in C. However,
the two major benefits Haskell brings outweigh the potential lack of outright
performance (which, in practice, is often very slight):
\begin{enumerate}
    \item A powerful static type system is particularly beneficial when
        prototyping new implementation details: a large number of run time bugs
        are ruled out, before the code is executed.
    \item High level of abstraction and concise coding style, leads to more
        direct translations of abstract design, using data structures such as
        sets and BDDs as first-class abstractions.
\end{enumerate}

Later, we will see that \penrose{} performs favourably against other tools,
implemented in standard languages, such as C.

\subsection{NFA Reduction}\label{sec:nfaReduction}

A key component of our technique is the algorithm for NFA reduction, that
improves the performance of \TNFA{} composition by reducing component \TNFA{}
sizes, as discussed in~\secref{sec:boundaryProtocol}. Indeed, as we showed at
the end of \chpref{chp:improveEfficiency}, the size of \TNFA{}s encountered in
certain examples grows rapidly and it is therefore important that we attempt to
minimise this growth, to avoid performance degradation. Of course, this state
growth phenomena is the well-known statespace explosion problem and
minimisation is our approach to avoiding it.

An obvious first choice of NFA minimisation algorithm is to use DFA
minimisation techniques: by first converting the NFA to a language-equivalent
DFA, for example, using the well-known powerset-construction, we can apply DFA
minimisation techniques. There are several well-known techniques for DFA
minimisation, such as the partition-refinement approaches of
Moore~\cite{Moore1956} and Hopcroft~\cite{Hopcroft1971}. A simple-to-describe
alternative that does not require prior determinisation is that of
Brzozowski~\cite{Brzozowski1962}; Brzozowski showed that reversing an input
NFA, determinising, reversing the resulting DFA (the result of which is not
necessarily deterministic) and finally determinising generates a minimal DFA,
language equivalent to the input NFA.

Indeed, due to its simplicity, we chose Brzozowski's method in the first
implementation of our technique~\cite{Sobocinski2013}. However, implicit in the
use of this minimisation technique is the possibility of exponential blowup due
to the (double) determinisation.

On the other hand, a positive aspect of the DFA-minimisation approach is that
it renders language equivalence checking almost trivial. Since the minimal DFA
for a NFA with a particular language is unique, to check language equivalence
of two NFA, it is sufficient to simply convert them both to the corresponding
unique minimal DFA. If the resulting DFA are equal, the original NFA are
language equivalent. However, as we further discuss in
\secref{sec:checkNfaLangEquiv}, checking language equivalence in this way is
still inefficient, since every pair of states of the determinised automata must
be checked.

Due to the potential exponential blowup effect of determinisation, it is
preferable to directly minimise NFA, rather than determinise NFA before
minimising the resulting DFA. Unfortunately, the minimisation problem is more
difficult for NFA than DFA, indeed, it is NP-hard~\cite{Bjorklund2011} and
furthermore, even approximating minimisation of NFA is
intractable~\cite{Gramlich2007}.  However, practical algorithms with
generally-high performance do exist, such as the recent technique due to
{Clemente and Mayr}~\cite{Mayr2013}, which outperforms all prior techniques.
Clemente and Mayr's algorithm operates by removing transitions \emph{and}
states, whilst preserving the recognised language: transitions are
pruned---those transitions that are subsumed by ``better'' transitions are
removed---and states quotiented by an efficiently-computed under-approximation
to language equivalence.

Using NFA minimisation means that we can no-longer rely on uniqueness of
minimal automata for checking language equivalence: whereas there necessarily
exists a minimal DFA for a particular language, there may be several
equally-minimal NFA (an example of such NFA is given by {Arnold et
al.}~\cite{Arnold1992}), thus we must use an alternative method of checking
language equivalence; in the next subsection, we describe such a method.

\subsection{Checking NFA Language Equivalence} \label{sec:checkNfaLangEquiv}

As previously mentioned, a naive method of checking language equivalence of NFA
is to determinise and check for equality of the resulting DFA. However, once
the minimisation technique does not use determinisation, we cannot employ this
method. An alternative approach, originally due to {Hopcroft and
Karp}~\cite{Hopcroft1971}, and further improved by {Bonchi and
Pous}~\cite{Bonchi2013}, is to use an on-the-fly determinisation procedure to
check equivalence, aiming to only \emph{partially} calculate the determinised
automata. {Bonchi and Pous}' approach exploits the notion of \emph{bisimulation
up-to context}; put briefly, if (sets of) states $X$ and $Y$ have equal
language, as do $Z$ and $W$, then bisimulation up-to context asserts that $X
\union Z$ has equal language to $Y \union W$. The advantage of such a
compositional technique is that the language equivalence of $X \union Z$ and $Y
\union W$ is not \emph{explicitly} checked. Thus, in the setting of NFA (where
a set of NFA states forms a single set of its determinisation), the full
corresponding DFA need not be explored.

In practice, we can further improve the performance of checking language
equivalence when reaching \emph{fixed-points} (which were introduced and
discussed in \secref{sec:fixedPoints}). If we \emph{tag} all \TNFA{} with
identifiers (that can be checked for equality in constant time), we can
identify fixed-points \emph{without} explicitly checking language equivalence
of the underlying \TNFA{}. Recall from~\secref{sec:memoisation} that we
maintain two memoisation maps:
\begin{enumerate}
    \item From PNBs to their (tagged) \TNFA{} semantics,
    \item From pairs of (tagged) \TNFA{} to their (tagged) \TNFA{} composition.
\end{enumerate}
We use the first to ensure that a given PNB is translated to its \TNFA{}
semantics once, and that we use \TNFA{} as the representative of its
equivalence class. In particular, any member of the same equivalence class will
be assigned the same identifier. The second mapping ensures that we only
perform a given (up-to language-equivalence) composition once.

Now, consider composing a (left-associated) chain of PNBs: $\aPNB \comp \aPNB
\comp \ldots \comp \aPNB$ that reaches a fixed-point after a single
composition. The first memoisation map will contain a single entry\footnote{We
write $N_i$ for a \TNFA{} $N$ tagged with identifier $i$.}:
\[
    \aPNB \mapsto \PNBToTNFA{\aPNB}_0
\]
When performing \TNFA{} compositions, if, having composed two \TNFA{} that
\emph{do not} appear in the composition memoisation map, we find a
language-equivalent \TNFA{} in the first map, we use it as the result of the
composition (and the entry in the second memoisation map). To demonstrate, the
first composition of \TNFA{} is $\PNBToTNFA{\aPNB}_0 \comp
\PNBToTNFA{\aPNB}_0$, however, since we reach a fixed point after one
composition, we have that $\PNBToTNFA{\aPNB} \comp \PNBToTNFA{\aPNB} \TNFAIso
\PNBToTNFA{\aPNB}$, and thus the composition memoisation map is updated to
contain a single entry:
\[
    \PNBToTNFA{\aPNB}_0 \comp \PNBToTNFA{\aPNB}_0 \mapsto \PNBToTNFA{\aPNB}_0
\]
Now, for every remaining composition in the chain, we will be considering
$\PNBToTNFA{\aPNB}_0 \comp \PNBToTNFA{\aPNB}_0$ and thus simply perform two
\emph{identifier} comparisons with the single entry in the composition map,
before returning the result $\PNBToTNFA{\aPNB}_0$. In general then,
fixed-points w.r.t. language-equivalence lead to (cheap) comparison of
identifiers, rather than (expensive) checking of language equivalence, a clear
optimisation in the case where the fixed-point is non-trivial, for example
in~\figref{fig:philoFixedPoint}.

\subsection{Representing \TNFA{} transitions}\label{sec:mtbdds}
\subimport{mtbdds}{}
