\chapter{Implementation and Comparison}\label{chp:comparisonAndDiscussion}

In Chapters~\ref{chp:compChecking} and~\ref{chp:improveEfficiency}, we
introduced a \emph{compositional} approach to checking reachability in systems
specified using \DSL. Exploiting \emph{weak language equivalence}, we
demonstrated effective statespace reductions, whilst preserving nets'
\emph{boundary protocol}. In this chapter we discuss the implementation of our
tool, \penrose{}\footnote{\penrose{}: Petri Net Reachability Ose. According to
Wikipedia, an \emph{Ose} is a demon that gives true answers to secret things.},
which uses this approach. We go on to compare and contrast its performance with
existing state-of-the-art tools, giving empirical demonstration of its
favourable performance in a variety of example systems.

\section{Implementation}
\subimport{}{implementation}

\section{Comparison with Related Tools}
\subimport{}{comparison}

\section{Summary}

In this chapter, we have shown that in addition to being a theoretical nicety,
compositionality also leads to impressive reachability checking performance of
our tool, \penrose, for many example systems. However, while in many cases
fixed-points of behaviour exist, and lead to exponential increases in
performance, there are examples where the statespace explosion problem still
presents itself, hampering our compositional approach. Future work will
investigate integrating existing approaches to avoiding statespace explosion
with our compositional approach, but we think that the initial results
presented show that compositionality, and the component-wise approach, is a
promising \emph{orthogonal} method of attacking statespace explosion.
