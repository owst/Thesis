\section{Non-deterministic Finite Automata}

In this thesis, we will use NFAs to give the semantics of \emph{marked} Petri
Nets, those nets with certain initial and target markings. The initial marking
will be the initial state of the NFA and the target marking will be the one
accepting state of the NFA.

\begin{definition}[NFA]
    A non-deterministic finite automaton is 5-tuple:
    \(
        ( \aNFAAllStates
        , \aNFAAllLabels
        , \aNFATransitionRel
        , \aNFAInitState
        , \aNFAAcceptStates
        )
    \)
    where $(\aNFAAllStates, \aNFAAllLabels, \aNFATransitionRel)$ is an LTS,
    $\aNFAInitState \in \aNFAAllStates$ and $\aNFAAcceptStates \subseteq
    \aNFAAllStates$.
\end{definition}

Graphically, initial states have a small, unlabelled arrow pointing into them,
and accepting states are drawn as double circles.

The \emph{language} of an NFA is the traces of its runs that begin with the
initial state, and end with an accepting state. We refer to individual traces
in the language as \emph{words}:

\begin{definition}
    Given an NFA, $\aNFA$, its language, written $\langOf{\aNFA}$, is the set
    of all traces (i.e. $\sequenceof{\aNFALabel_0, \aNFALabel_1,\dots,
    \aNFALabel_{n - 1}}$, written as $\word{\aNFALabel}$) corresponding to runs
    of the underlying LTS:
    \[ \sequenceof{
        \aNFAState_0 \LabelledTrans{\aNFALabel_0} \bNFAState_0,\,
        \aNFAState_1 \LabelledTrans{\aNFALabel_1} \bNFAState_1,\,
        \dots,\,
        \aNFAState_{n-1} \LabelledTrans{\aNFALabel_{n-1}} \bNFAState_{n-1}
        }
    \] such that $\aNFAState_0 = \aNFAInitState$ and $\bNFAState_{n-1} \in
    \aNFAAcceptStates$. By \defnref{defn:ltsRun}, we must have $\aLTSState_i =
    \bLTSState_{i-1}$, for all $1 \le i < n$.
\end{definition}

\begin{example}
    An example NFA is illustrated in \figref{fig:exampleNFA}. Its language is the
    (infinite) set:
    \[
        \setBuilder{tmw}{w \in \setof{s}^\ast}
        \union
        \setBuilder{cmw}{w \in \setof{s}^\ast}
    \]
\end{example}

\begin{figure}[ht]
    \centering
    \begin{tikzpicture}[nfa]
        \node (0) [state, initial=left, initial text=] {$0$};
        \node (1) [state, above right=of 0]      {$1$};
        \node (2) [state, below right=of 0]      {$2$};
        \node (3) [state, accepting, right=of 1] {$3$};
        \node (4) [state, accepting, right=of 2] {$4$};

        \draw (0) edge node {$t$} (1);
        \draw (0) edge node {$c$} (2);
        \draw (1) edge node {$m$} (3);
        \draw (2) edge node {$m$} (4);
        \draw (3) edge [loop right] node {$s$} (3);
        \draw (4) edge [loop right] node {$s$} (4);
    \end{tikzpicture}
    \caption{An example NFA}
    \label{fig:exampleNFA}
\end{figure}


NFAs may have different structure, yet recognise the same language; we say that
such NFAs are \emph{language-equivalent}:
\begin{definition}
    $\aNFA$ and $\bNFA$ are language-equivalent, written $\aNFA \langEquiv
    \bNFA$, if $\langOf{\aNFA} = \langOf{\bNFA}$.
\end{definition}

\begin{example}
    We have that $\aNFA \langEquiv \bNFA$, where $\aNFA$ is illustrated in
    \figref{fig:exampleNFA}, and $\bNFA$ in \figref{fig:EquivalentNFA}.
\end{example}

\begin{figure}[ht]
    \centering
    \begin{tikzpicture}[nfa]
        \node (0) [state, initial=left, initial text=] {$0$};
        \node (1) [state, right=of 0]            {$1$};
        \node (2) [state, accepting, right=of 1] {$2$};

        \draw (0) edge node {$\setof{t,c}$} (1);
        \draw (1) edge node {$m$} (2);
        \draw (2) edge [loop right] node {$s$} (2);
    \end{tikzpicture}
    \caption{A NFA that is language equivalent to that of
        \figref{fig:exampleNFA}}
    \label{fig:EquivalentNFA}
\end{figure}

\subsection{Isomorphism of NFAs}

A homomorphism between two NFAs, $\aNFA_1$ and $\aNFA_2$, is a homomorphism on
the underlying LTSs that additionally preserves the initial and accepting
states.

\begin{definition}
    Given two NFAs, $\aNFA_1$:
    \(
        ( \aNFAAllStates_1
        , \aNFAAllLabels_1
        , \aNFATransitionRel_1
        , \aNFAInitState_1
        , \aNFAAcceptStates_1
        )
    \) and $\aNFA_2$:
    \(
        ( \aNFAAllStates_2
        , \aNFAAllLabels_2
        , \aNFATransitionRel_2
        , \aNFAInitState_2
        , \aNFAAcceptStates_2
        )
        \), a homomorphism, $f$ (comprised of components
        $\sbij : \aNFAAllStates_1 \to \aNFAAllStates_2$ and $\lbij :
        \aNFAAllLabels_1 \to \aNFAAllLabels_2$ ) on the underlying LTSs $(
        \aNFAAllStates_1, \aNFAAllLabels_1, \aNFATransitionRel_1)$ and
        $(\aNFAAllStates_2, \aNFAAllLabels_2, \aNFATransitionRel_2)$, is a NFA
        homomorphism if the following additional conditions are satisfied:
    \begin{enumerate}[(i)]
        \item $\sbij(\aNFAInitState_1) = \aNFAInitState_2$
        \item $\setBuilder{\sbij(x)}{x \in \aNFAAcceptStates_1} =
            \aNFAAcceptStates_2$
    \end{enumerate}
\end{definition}

Clearly, if an NFA-homomorphism is an LTS-isomorphism, it is also a
NFA-isomorphism.

\subsection{\TNFA{}s} \label{sec:TNFA}

If the underlying LTS of an NFA $\aNFA$ is a $\LTSB{\aN}{\bN}$ we say that
$\aNFA$ is a $\NFAB{\aN}{\bN}$, or, if the particular $\aN, \bN$ are
unimportant, a \TNFA{}.

We define composition operations on \TNFA{}s in a similar manner to how we
defined them on \TLTS{}s: the underlying operation is performed on the \TLTS,
and the initial/accepting states are suitably modified.

\begin{definition}[Synchronous composition of \TNFA{}s]
\label{defn:sequentialCompositionTNFA}
    Given a $\NFAB{\aN}{\bN}$, $\aNFA$:
    $( \aNFAAllStates_1
     , \aNFAAllLabels_1
     , \aNFATransitionRel_1
     , \aNFAInitState_1
     , \aNFAAcceptStates_1
     )$ and a $\NFAB{\bN}{\cN}$, $\bNFA$:
    $( \aNFAAllStates_2
     , \aNFAAllLabels_2
     , \aNFATransitionRel_2
     , \aNFAInitState_2
     , \aNFAAcceptStates_2
     )$ their synchronous composition is a $\NFAB{\aN}{\cN}$:
    \[( \aNFAAllStates
     , \aNFAAllLabels
     , \aNFATransitionRel
     , \pairof{\aNFAInitState_1}{\aNFAInitState_2}
     , \aNFAAcceptStates_1 \times \aNFAAcceptStates_2
 )\] where
    $( \aNFAAllStates
     , \aNFAAllLabels
     , \aNFATransitionRel)$ is the synchronous composition of the underlying
     \TLTS{}s (\defnref{defn:sequentialCompositionTLTS}).
\end{definition}

\begin{definition}[Tensor composition of \TNFA{}s]
\label{defn:tensorCompositionTNFA}
    Given a $\NFAB{\aN}{\bN}$, $\aNFA$:
    $( \aNFAAllStates_1
     , \aNFAAllLabels_1
     , \aNFATransitionRel_1
     , \aNFAInitState_1
     , \aNFAAcceptStates_1
     )$ and a $\NFAB{\cN}{\dN}$, $\bNFA$:
    $( \aNFAAllStates_2
     , \aNFAAllLabels_2
     , \aNFATransitionRel_2
     , \aNFAInitState_2
     , \aNFAAcceptStates_2
     )$ their tensor composition is a $\NFAB{\aN + \cN}{\bN + \dN}$:
    \[
        ( \aNFAAllStates
        , \aNFAAllLabels
        , \aNFATransitionRel
        , \pairof{\aNFAInitState_1}{\aNFAInitState_2}
        , \aNFAAcceptStates_1 \times \aNFAAcceptStates_2
        )
 \] where
    $( \aNFAAllStates
     , \aNFAAllLabels
     , \aNFATransitionRel)$ is the tensor composition of the underlying
     \TLTS{}s (\defnref{defn:tensorCompositionTLTS}).
\end{definition}

Now, we turn to Petri nets, the semantics of which we give using LTSs/NFAs.
