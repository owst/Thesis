\section{Petri nets}

Petri nets are a mathematical model with a vivid graphical presentation, used to
model concurrent and distributed systems. A Petri net consists of a set of
places, and a set of transitions that connect sets of places to sets of places.
\begin{definition}[Petri net]
A Petri net is a 4-tuple:
$(\aPnAllPlaces,\aPnAllTrans,\pre{-},\post{-})$, where:
\begin{itemize}
\item $\aPnAllPlaces$ is the set of places,
\item $\aPnAllTrans$ is the set of transitions,
\item $\pre{-},\,\post{-}:\aPnAllTrans \to 2^{\aPnAllPlaces}$ give,
    respectively, the pre- and post-sets of each transition.
\end{itemize}
\end{definition}

For $\aTrans \in \aPnAllTrans$, its preset, $\pre{\aTrans}$ is the set of places that
connect to $\aTrans$, and its postset, $\post{\aTrans}$, is the set of
places to which $\aTrans$ connects. The
\emph{neighbourhood} of $\aTrans$ is $\prepost{\aTrans} \defeq \pre{\aTrans}
\union \post{\aTrans}$.  We write $\places{\aPn}$ and $\trans{\aPn}$
respectively, for the place and transition sets of $\aPn$.

In this thesis, we consider the class of Petri nets where each place can only contain zero or one
\emph{token}. Other classes of Petri nets (e.g. P/T nets~\cite{Desel1998}) lift this restriction,
but we do not use this generalisation, instead focussing on \emph{1-bounded}, or \emph{safe} nets.
The definitions in this chapter essentially follow those of~\cite{Engelfriet1998}, albeit with a
slightly different presentation. For example, we use a functional, rather than relational
presentation of transition connections; this alternative presentation makes the extension of Petri
nets to Petri nets with boundaries clearer (indeed, every Petri net is a degenerate Petri net with
boundaries).

\begin{definition}[Marking]
    A marking of a net $N$ is a function $m : \places{N} \to \setof{0,1}$.
\end{definition}

Graphically, a marking is illustrated by drawing black circles inside each
place that is assigned a token by the marking.

\begin{example}
An example Petri net, $\aPn$, is illustrated in \figref{fig:examplePetriNet},
with 7 places, 5 transitions, and with the marking function:
\(
    \markedAt{\aPnMarking}{\aPlace_i} \defeq \begin{cases}
        1 & \mbox{if } i \le 1 \\
        0 & \mbox{otherwise} \\
    \end{cases}
\)

In $\aPn$, for example, $\pre{\aTrans_4}=\setof{\aPlace_2, \aPlace_4}$,
$\pre{\aTrans_1}=\setof{\aPlace_1}$ and
$\post{\aTrans_1}=\setof{\aPlace_2,\aPlace_5}$.
\end{example}

\begin{figure}[ht]
\centering
\begin{tikzpicture}[pn]
\node[petriplace]      (p0) [pnblabel=right:$\aPlace_0$, tokens=1]              {};
\node[petritransition] (t0) [above right=of p0, pnblabel=above:$\aTrans_0$]     {};
\node[petriplace]      (p1) [right=of t0, pnblabel=above:$\aPlace_1$, tokens=1] {};
\node[petritransition] (t1) [right=of p1, pnblabel=above:$\aTrans_1$]           {};
\node[petriplace]      (p2) [right=of t1, pnblabel=above:$\aPlace_2$]           {};
\node[petritransition] (t2) [below right=of p0, pnblabel=above:$\aTrans_2$]     {};
\node[petriplace]      (p3) [right=of t2, pnblabel=above:$\aPlace_3$]           {};
\node[petritransition] (t3) [right=of p3, pnblabel=above:$\aTrans_3$]           {};
\node[petriplace]      (p4) [right=of t3, pnblabel=above:$\aPlace_4$]           {};
\node[petriplace]      (p5) [pnblabel=above:$\aPlace_5$] at ($(p2)!.5!(p4)$)    {};
\node[petritransition] (t4) [above right=of p4, pnblabel=left:$\aTrans_4$]      {};
\node[petriplace]      (p6) [right=of t4, pnblabel=above:$\aPlace_6$]           {};
\draw [petriarr] (p0) to [bend left=45]  (t0);
\draw [petriarr] (p0) to [bend right=45] (t2);
\draw [petriarr] (t0) to                 (p1);
\draw [petriarr] (p1) to                 (t1);
\draw [petriarr] (t1) to                 (p2);
\draw [petriarr] (t1) to [out=0,in=170]  (p5);
\draw [petriarr] (t2) to                 (p3);
\draw [petriarr] (p3) to                 (t3);
\draw [petriarr] (t3) to                 (p4);
\draw [petriarr] (t3) to [out=0,in=190]  (p5);
\draw [petriarr] (p2) to [bend left=45]  (t4);
\draw [petriarr] (p4) to [bend right=45] (t4);
\draw [petriarr] (t4) to                 (p6);
\end{tikzpicture}
\caption{Example Petri net}
\label{fig:examplePetriNet}
\end{figure}


The marking of a Petri net is transformed by \emph{firing} transitions; we say
one marking is \emph{reachable} from another if there is a sequence of
transition firings that transforms one marking to the other. Intuitively, when
fired, a transition, $\aTrans$, consumes tokens from all $\aPlace \in
\pre{\aTrans}$ and produces a token at all $\bPlace \in \post{\aTrans}$, giving
a new marking.

However, it is not the case that arbitrary subsets of net transitions can
necessarily be fired at the same time. Two transitions cannot fire
simultaneously if they attempt to produce a token at the same place (e.g.
$\aTrans_1$ and $\aTrans_3$ in \figref{fig:examplePetriNet}), or both consume
the token from a place, (e.g. $\aTrans_0$ and $\aTrans_2$ in
\figref{fig:examplePetriNet}). To specify the transitions that may fire
concurrently, we define the notion of \emph{contention} between transitions.

\begin{definition}[Transition contention]
    For a Petri net $\aPn$, we say that $\aTrans, \bTrans \in
    \trans{N}$ are in contention (written $\aTrans \contention \bTrans$)
    when $\pre{\aTrans} \cap \pre{\bTrans} \neq \emptyset$ or
    $\post{\aTrans} \cap \post{\bTrans} \neq \emptyset$.
\end{definition}

Note that contention is a property of the net's topology and does \emph{not}
depend on any particular marking. We say a set of transitions is
\emph{contention-free} if it contains no pair of \emph{distinct} transitions in
contention:

\begin{definition}[Contention-free transition sets]\label{defn:pnContentionFreeSet}
    For a Petri net $\aPn$, we say that $\aPnTransSet \subseteq \trans{\aPn}$
    is contention-free if $\,\forall \aTrans, \bTrans \in \aPnTransSet$, we
    have:
    \[
        \aTrans \contention \bTrans \implies \aTrans = \bTrans
    \]
\end{definition}

For example, in the net shown in \figref{fig:examplePetriNet}, $\setof{\aTrans_0,
\aTrans_3}$ is contention-free, but $\setof{\aTrans_0,\aTrans_2}$ is not.

A transition may only be fired if it is \emph{enabled}: when each place of its
pre-set is marked, and no places in its post-set are marked:

\begin{definition}[EN enabled transitions]\label{defn:pnEnabledTrans}
For a net $\aPn$, $\aTrans \in \trans{\aPn}$ is enabled for a marking
$\aPnMarking$, written $\enabledTrans{\aTrans}{\aPnMarking}$, if for all
$\aPlace \in \pre{\aTrans}$, $\markedAt{\aPnMarking}{\aPlace} = 1$ and
for all $\bPlace \in \post{\aTrans}$, $\markedAt{\aPnMarking}{\bPlace}
= 0$.
\end{definition}

\begin{remark}\label{rem:contentionEnabled}
    A trivial observation is that a particular transition will be enabled at
    some, but not all, markings of the same net, whereas a given pair of
    transitions will either always be in contention, or never. Put differently,
    contention is a static property of the net, while a particular transition
    being enabled is a \emph{dynamic} property of the net. Determining if we
    are able to fire a particular set of transitions thus relies on a mixture
    of static and dynamic information: we must have no transitions in
    contention in the set, and the current marking should enable all of the
    transitions. In some net models, such as P/T nets, the enabled condition of
    \defnref{defn:pnEnabledTrans} is relaxed, allowing transitions to fire even
    if a token is already present on the place. However, in this thesis we do
    not consider such nets: their semantics are more complex, and their
    statespaces are often infinite.
\end{remark}

We may lift several of the previous definitions from individual transitions to
sets of transitions, in a straightforward manner. In the following,
$\aPnTransSet \subseteq \trans{\aPn}$, for some net, $\aPn$:
\begin{align*}
    \pre{\aPnTransSet} &\defeq \bigcup_{\aTrans \in
        \aPnTransSet}\pre{\aTrans}\\
    \post{\aPnTransSet} &\defeq \bigcup_{\aTrans \in
        \aPnTransSet}\post{\aTrans}\\
    \enabledTrans{\aPnTransSet}{\aPnMarking} &\defeq \bigwedge_{\aTrans
            \in \aPnTransSet}\enabledTrans{\aTrans}{\aPnMarking}
\end{align*}

\begin{remark}\label{rem:classes}
Before continuing, we briefly mention that the class of all Petri nets can be
sub-divided based upon either structure, token numbers and transition firing
rules. Common net classes relating to structure are \emph{pure}, \emph{simple}
or \emph{finite} nets, to tokens and structure are \emph{bounded} or
\emph{safe} nets and to transition firing rules are \emph{elementary} nets.
In turn, we have that a Petri net, $\aPn$ is:
\begin{itemize}
    \item \emph{pure} if, for any $\aTrans \in \trans{\aPn}$, $\pre{\aTrans}
        \intersect \post{\aTrans} = \emptyset$
    \item \emph{simple} if, for any $\aTrans, \bTrans \in \trans{\aPn}$,
        $\pre{\aTrans} = \pre{\bTrans} \implies \aTrans = \bTrans$,
    \item $\aN$-\emph{bounded} if, every reachable marking, $\aPnMarking$, has
        that: $\forall \aPlace \in \places{\aPn},
        \markedAt{\aPnMarking}{\aPlace} \le \aN$,
    \item \emph{safe} if it is 1-bounded,
    \item \emph{finite} if both $\places{\aPn}$ and $\trans{\aPn}$ are finite,
\end{itemize}
\end{remark}

We may now define the \emph{step firing} semantics of a Petri net, in which
(sets of) transitions are fired: a set of transitions may fire if it is both
contention-free (a static property) and enabled in the current marking of the
net (a dynamic property):

\begin{definition}[EN Firing Semantics] \label{defn:firePetriTransitions}
Given a net $\aPn$ and a marking $\aPnMarking$, a set $\aPnTransSet \subseteq
\trans{\aPn}$ can fire iff $\aPnTransSet$ is contention-free and
$\enabledTrans{\aPnTransSet}{\aPnMarking}$ (by \defnref{defn:pnEnabledTrans}).
Firing $\aPnTransSet$ creates a new marking $\bPnMarking$, defined thus:
\[
    \markedAt{\bPnMarking}{\aPlace_i} \defeq \begin{cases}
        0 & \mbox{if } \aPlace_i \in \pre{\aPnTransSet} \\
        1 & \mbox{if } \aPlace_i \in \post{\aPnTransSet} \\
        \markedAt{\aPnMarking}{\aPlace_i} & \mbox{otherwise}
    \end{cases}
\]
that is, tokens are consumed from all places in the pre-set of $\aPnTransSet$,
produced at all places in the post-set of $\aPnTransSet$, and otherwise
unchanged from $\aPnMarking$. If $\aPnTransSet$ can be fired to transform
$\aPn$ from $\aPnMarking$ to $\bPnMarking$, we write
$\PnFiringSemantics{\aPn}{\aPnTransSet}{\aPnMarking}{\bPnMarking}$.
\end{definition}

For the remainder of this thesis, we only consider finite, pure and simple
(\remref{rem:classes}) Elementary Nets~\cite{Thiagarajan1987}, which are
defined in \defnref{defn:EN}.

\begin{definition}[Elementary Net]\label{defn:EN}
    An elementary net (EN) is a Petri net,
    $(\aPnAllPlaces,\aPnAllTrans,\pre{-},\post{-})$, that uses the firing rule
    in \defnref{defn:firePetriTransitions}.
\end{definition}

Since ENs do not allow transitions to be enabled when there are tokens in any
of their post-set places, safeness (\remref{rem:classes}) is guaranteed.
As a consequence, the state space---set of reachable markings---of an EN $\aPn$
is \emph{finite} since the markings of $\aPn$ put at most one token in each
place.  Furthermore, markings can be identified with the sets of places $M
\subseteq \places{\aPn}$ they characterise.

In standard presentations, an EN $\aPn$ must have no unconnected places (i.e
$\forall \aPlace \in \places{\aN}$, $\exists \aTrans \in\trans{\aPn} \text{ s.t.
} \aPlace \in \prepost{\aTrans}$) or transitions (i.e $\forall \aTrans \in
\trans{\aN}, \prepost{\aTrans} \neq \emptyset$).  However in this thesis we
lift this restriction, allowing transitions to not connect to places, as will
be discussed when introducing Petri Nets With Boundaries.

\subsection{Labelled transition system semantics of Petri
Nets}\label{sec:PNLTSSemantics}

The firing semantics of a Petri net can be used to construct an LTS---known as
its \emph{reachability graph}~\cite{Murata1989}---representing the
relationships between all markings of the net.

The LTS semantics for a net $\aPn$ has states the markings of $\aPn$, with
transitions between markings $\aPnMarking$ and $\bPnMarking$ being labelled by
sets of (enabled, contention-free) transitions:

\begin{definition}[LTS semantics of a Petri net]
    \newcommand{\firedRel}{\mathit{fired}}
    The LTS semantics of a net $\aPn$ is defined to be the LTS
    $(2^{\places{\aPn}}, 2^{\trans{\aPn}}, \firedRel)$, where $(\aPnMarking,
    \aPnTransSet, \bPnMarking) \in \firedRel$ iff $\aPnTransSet$ is
    contention-free and enabled by $\aPnMarking$, and
    $\PnFiringSemantics{\aPn}{\aPnTransSet}{\aPnMarking}{\bPnMarking}$.
\end{definition}

\begin{example}
The LTS semantics of the Petri net of \figref{fig:examplePetriNet} is shown in
\figref{fig:exampleLTSSemantics}. Recall that net markings are identified
with the set of net states to which they assign a token, and that LTS states
correspond to markings of the underlying net, therefore, we draw LTS states
containing sets of net states. The LTS transitions are labelled with sets of
net transitions, fired to transform between the two markings. Note that each
state of the LTS has a self-loop: the empty set of transitions can always be
fired. For clarity, the 122 (out of 128 total) states that are not reachable
from the illustrated marking $\setof{\aPlace_0, \aPlace_1}$ have not been
drawn.

For example, we have that
\(\setof{\aPlace_0,\aPlace_1} \LabelledTrans{\setof{\aTrans_1,\aTrans_2}}
\setof{\aPlace_2,\aPlace_3,\aPlace_5} \) and $\setof{\aPlace_1,\aPlace_3} \LabelledTrans{\emptyset}
\setof{\aPlace_1,\aPlace_3}$.  Observe that the net transition $\aTrans_4$ doesn't appear in the
label of any LTS transition - it can never be fired since no reachable marking places a token in
both $\aPlace_2$ and $\aPlace_4$.
\end{example}

\begin{figure}[!ht]
\centering
\begin{tikzpicture}[nfa, node distance=3cm, every state/.append style={inner
    sep=1mm, minimum size=1.35cm, font=\footnotesize}]
\node[state] (01)                       {$\setof{\aPlace_0,\aPlace_1}$};
\node[state] (13)  [above right of=01]  {$\setof{\aPlace_1,\aPlace_3}$};
\node[state] (025) [below right of=01]  {$\setof{\aPlace_0,\aPlace_2,\aPlace_5}$};
\node[state] (235) [below right of=13]  {$\setof{\aPlace_2,\aPlace_3,\aPlace_5}$};
\node[state] (145) [above right of=235] {$\setof{\aPlace_1,\aPlace_4,\aPlace_5}$};
\node[state] (125) [below right of=235] {$\setof{\aPlace_1,\aPlace_2,\aPlace_5}$};
\path (01)  edge [bend left]  node           {$\setof{\aTrans_2}$}     (13)
            edge [bend right] node [swap]    {$\setof{\aTrans_1}$}     (025)
            edge              node           {$\setof{\aTrans_1,\aTrans_2}$} (235)
            edge [loop left]  node [overlay] {$\emptyset$}       (01)
      (13)  edge [bend left]  node [swap]    {$\setof{\aTrans_1}$}     (235)
            edge [loop above] node           {$\emptyset$}       (13)
            edge              node           {$\setof{\aTrans_3}$}     (145)
      (025) edge [bend right] node           {$\setof{\aTrans_2}$}     (235)
            edge [loop below] node           {$\emptyset$}       (025)
            edge              node [swap]    {$\setof{\aTrans_0}$}     (125)
      (235) edge [loop right] node [overlay] {$\emptyset$}       (235)
      (145) edge [loop right] node [overlay] {$\emptyset$}       (145)
 ;
\end{tikzpicture}
\caption{LTS semantics of the Petri net in \figref{fig:examplePetriNet}. }
\label{fig:exampleLTSSemantics}
\end{figure}

Finally, as a consequence of the definition of Petri net LTS semantics, we
observe that trace membership of an LTS semantics implies a valid corresponding
sequence of firings in the underlying net:

\begin{remark}\label{rem:validTraceImpliesFiring}
    Consider a net $\aPn$ and its LTS semantics $\aLTS$. Any trace,
    $\aLTSTrace = \sequenceof{\aLTSLabel_0,\aLTSLabel_1,\dots,\aLTSLabel_n}$,
    of $\aLTS$ is a valid sequence of firings of $\aPn$, between the markings
    corresponding to the states of \emph{any} run with trace $\aLTSTrace$.
\end{remark}

\section{Reachability and Coverability in Petri Nets}

A natural question to ask about a Petri net is whether it is possible to reach
a particular target marking from some initial marking. That is, does there
exist a firing sequence that transforms the net from the initial marking to the
target marking. This question is known as the \emph{reachability
problem}~\cite{Peterson1977}, and is PSPACE-complete~\cite{Cheng1995}.

Due to the correspondence between Petri net firing sequences and traces of
their LTS semantics (\remref{rem:validTraceImpliesFiring}), reachability
checking coincides with checking the existence of a trace between the states
corresponding to the initial and target markings. For example, there is a
trace between $\setof{\aPlace_0,\aPlace_1}$ and
$\setof{\aPlace_1,\aPlace_4,\aPlace_5}$ in
\figref{fig:exampleLTSSemantics}.

We often want to consider a particular net and pair of its markings, and refer
to such a triple as a \emph{marked} net:
\begin{definition}[Marked Net]
    A marked net, $(\aPn, \aMarkedNetInitial, \aMarkedNetTarget)$, is comprised
    of a Petri net $\aPn$ and the initial/target markings:
    $\aMarkedNetInitial,\aMarkedNetTarget \subseteq \places{\aPn}$.
\end{definition}

A related concept from the literature is a \emph{net
system}~\cite{Engelfriet1998}, which is simply a net along with a chosen
initial marking.

Graphically, we represent a marked net as a Petri net with tokens on each place
in the initial marking, and shading on each place in the target marking. For
example, consider the net illustrated in \figref{fig:markedPNet}, representing
a fork-join system containing 2 tasks; the initial marking is $\setof{\aPlace_0}$
(representing the initial, ready state of the system) and the target marking is
$\setof{\aPlace_3,\aPlace_4}$, where both tasks have finished, but have not
been joined.

\begin{figure}[ht]
    \centering
    \begin{tikzpicture}[pn]
    \node[petriplace]      (p0) [pnblabel=below:$\aPlace_0$, tokens=1]        {};
    \node[petritransition] (t0) [right=of p0, pnblabel=below:$\aTrans_{\text{fork}}$]   {};
    \node[petriplace]      (p1) [above right=of t0, pnblabel=below:$\aPlace_1$]   {};
    \node[petriplace]      (p2) [below right=of t0, pnblabel=below:$\aPlace_2$] {};
    \node[petritransition] (t1) [right =of p1, pnblabel=below:$\aTrans_{\text{finish1}}$]   {};
    \node[petritransition] (t2) [right =of p2, pnblabel=below:$\aTrans_{\text{finish2}}$] {};
    \node[petriplace, wantedTokenPlace] (p3) [right =of t1, pnblabel=below:$\aPlace_3$]   {};
    \node[petriplace, wantedTokenPlace] (p4) [right =of t2, pnblabel={[overlay=false]below:$\aPlace_4$}] {};
    \node[petritransition] (t3) [below right=of p3, pnblabel=below:$\aTrans_{\text{join}}$] {};

    \draw (p0) edge[petriarr] (t0);
    \foreach \x/\t/\y in {p1/t1/p3, p2/t2/p4} {
        \draw (t0) edge[petriarr, o0i180] (\x);
        \draw (\x) edge[petriarr] (\t);
        \draw (\t) edge[petriarr] (\y);
        \draw (\y) edge[petriarr, o0i180] (t3);
    }
    \draw (t3) edge[petriarr,out=80, in=90,looseness=1.2] (p0);

\end{tikzpicture}
    \caption{Marked Petri net representing a fork-join system}
    \label{fig:markedPNet}
\end{figure}

Given a marked Petri net and the LTS semantics of the net, we can construct
a NFA, recognising the set of traces between the pair of states:

\begin{definition}[NFA Semantics of a marked Petri Net]\label{defn:PnNFA}
    Given a marked net, $(\aPn, \aMarkedNetInitial, \aMarkedNetTarget)$, the
    NFA semantics extends the LTS semantics: the initial state is
    $\aMarkedNetInitial$ and the set of accepting states
    is the singleton set $\setof{\aMarkedNetTarget}$.
\end{definition}

\begin{example}
    Consider the LTS shown in \figref{fig:exampleLTSSemantics}, representing
    the semantics of the net in \figref{fig:examplePetriNet}. Take the initial
    state to be the original marking shown in \figref{fig:examplePetriNet},
    i.e. $\setof{\aPlace_0,\aPlace_1}$, and the accepting states to be
    $\setof{\setof{\aPlace_2,\aPlace_3,\aPlace_5}}$. Now, the language of the
    NFA, $\aNFA$, obtained from the LTS and chosen initial/accepting states is
    simple:
    \[
        \langOf{\aNFA} =
            \setof{ \sequenceof{\setof{t_1,t_2}}
                  , \sequenceof{\setof{t_2},\setof{t_1}}
                  , \sequenceof{\setof{t_1},\setof{t_2}}
                  }
    \] that is, we can either fire $t_2$ and $t_1$ in order, or concurrently,
    to reach the desired marking.
\end{example}

A simple observation following from \remref{rem:validTraceImpliesFiring}
is that a trace being a member of the language of an NFA for a particular
reachability problem implies that the reachability problem is satisfiable:

\begin{remark}\label{rem:nonEmptyLanguageImpliesReachability}
If $\aNFAWord \in \langOf{\aNFA}$, where $\aNFA$ is the NFA representing the
reachability problem for a given marked net, then $\aNFAWord$ is a
\emph{witness} of the final marking being reachable from the initial marking in
the net. Generally, if $\langOf{\aNFA} \neq \emptyset$, then the reachability
problem \emph{is} satisfiable, whereas if $\langOf{\aNFA} = \emptyset$, then
the reachability problem is \emph{not} satisfiable.
\end{remark}

Finally, we close this section by giving the definition of a slight
modification of the reachability problem, that of \emph{coverability}.
Coverability of a marking, $\aPnMarking$, checks if any marking that
\emph{contains} $\aPnMarking$ as a \emph{sub-marking} is reachable:

\begin{definition}[Sub-marking]
    Given a Petri net, $\aPn$, and two of its markings, $\aPnMarking$ and
    $\bPnMarking$, we say that $\aPnMarking$ is a sub-marking of $\bPnMarking$,
    written $\aPnMarking \subPnMarking \bPnMarking$, iff $\forall \aPlace \in
    \aPnMarking, \bPlace \in \bPnMarking$.
\end{definition}

\begin{definition}[Coverability]
    Given a marked Petri net, $(\aPn, \aMarkedNetInitial, \aMarkedNetTarget)$,
    we say that $\aMarkedNetTarget$ is coverable if $\exists \aPnMarking \,.\,
    \aMarkedNetTarget \subPnMarking \aPnMarking$ such that the reachability
    problem $(\aPn, \aMarkedNetInitial, \aPnMarking)$ is satisfiable.
\end{definition}
