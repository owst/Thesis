\section{Petri nets with boundaries}

Petri nets with boundaries (PNBs) extend the definition of the EN nets introduced in the previous
section by adding left and right boundaries, to which transitions of the net can also connect. PNBs
are composed using two operations: a synchronising composition (an intuitive example is illustrated
in \figref{fig:examplePNBcomposition}; the graphical notation will be explained shortly), and a
non-interacting, parallel composition. Indeed, by allowing transitions to connect to boundary
ports, their behaviour can be \emph{partially} specified; synchronously composing PNBs
\emph{completes} the specification of transitions by
supplementing the places that the transition connects to.

\begin{definition}[Petri net with Boundaries]
    $ $\\
    A PNB is a 9-tuple:
    $(\aPNBAllPlaces, \aPNBAllTrans, \pre{-}, \post{-}, l, r, \source{-},
    \target{-}, \contention)$, where:
    \begin{itemize}
        \item $(\aPNBAllPlaces, \aPNBAllTrans, \pre{-}, \post{-})$ is an EN,
        \item $l,r \in \N$ are respectively, the left and the right
            boundaries,
        \item $\source{-} : \aPNBAllTrans \to \powerset{\ordinal{l}}$ and
            $\target{-} : \aPNBAllTrans \to \powerset{\ordinal{r}}$ connect a
            transition to the left and right boundaries,
        \item $\contention$ is a contention relation (see
            Definition~\ref{defn:contentionRelation} below).
    \end{itemize}
\end{definition}

We will refer to PNBs simply as nets herein; when we need to be precise we will
explicitly name the type of nets being considered. As the definitions of
$\pre{-},\post{-}$ were lifted to sets of transitions, so are $\source{-},
\target{-}$: for a PNB, $\aPNB$, and $\aPNBTransSet \subseteq \trans{\aPNB}$:
\[
    \source{\aPNBTransSet} \defeq \bigcup_{\aTrans \in \aPNBTransSet}\source{\aTrans}
\] and similarly for
$\target{\aPNBTransSet}$. We frequently need to state the boundaries of a given
PNB, writing $\aPNB \withNetType{\aN}{\bN}$ to indicate that $\aPNB$ has left
and right boundaries $\aN$ and $\bN$, respectively.

\begin{figure}[ht]
\centering
\scalebox{0.9}{
\begin{subfigure}{0.3\textwidth}
\captionsetup{labelformat=empty}
    \centering
\begin{tikzpicture}[node distance=1.25cm]
	\node[pnbplace, tokens=1] (p0) {};
	\node[pnbplace, below of=p0, rotate=180] (p1) {};
	\node[node distance=0.75cm, below of=p1, pnbplace, draw=white, fill=white] (p2) {};
	\node[below of=p2, pnbplace, draw=white] (p3) {};

	\drawBoundaries{0}{0}
	\rBAlignedWith{p0}{1}
	\rBAlignedWith{p1}{2}
	\rBAlignedWith{p2}{3}
	\rBAlignedWith{p3}{4}

    \labelledpnbarr{p0.out}{r1}{$t$}{o0i180}{}
    \labelledpnbarr{p1.in}{r2}{$u$}{o0i180}{}
    \labelledpnbarr{r4}{p3}{$v$}{out=180, in=-60}{}
\end{tikzpicture}
\caption{$\aPNB\withNetType{0}{4}$}
\end{subfigure}%
\begin{subfigure}{0.3\textwidth}
\captionsetup{labelformat=empty}
    \centering
\begin{tikzpicture}[node distance=1.25cm]
	\node[pnbplace] (p0) {};
	\node[pnbplace, below of=p0, draw=white, fill=white] (p1) {};
	\node[node distance=0.75cm, below of=p1, pnbplace, rotate=180, tokens=1] (p2) {};
	\node[below of=p2, pnbplace, rotate=180, tokens=1] (p3) {};

	\drawBoundaries{0}{0}
	\lBAlignedWith{p0}{1}
	\lBAlignedWith{p1}{2}
	\lBAlignedWith{p2}{3}
	\lBAlignedWith{p3}{4}

    \labelledpnbarr{p0.in}{l1}{$a$}{o180i0}{}
    \labelledpnbarr{l1}{l2}{right:$b$}{bend left=60}{}
    \labelledpnbarr{p2.out}{l3}{$c$}{o180i0}{}
    \labelledpnbarr{p3.out}{l4}{$d$}{o180i0}{}
\end{tikzpicture}
\caption{$\bPNB\withNetType{4}{0}$}
\end{subfigure}%
\begin{subfigure}{0.4\textwidth}
\captionsetup{labelformat=empty}
\centering
\begin{tikzpicture}[node distance=1.25cm]
	\node[pnbplace, tokens=1] (p0) {};
	\node[pnbplace, below of=p0, rotate=180] (p1) {};
	\node[node distance=0.75cm, below of=p1, pnbplace, draw=white, fill=white] (p2) {};
	\node[below of=p2, pnbplace, draw=white] (p3) {};

	\node[pnbplace, node distance=1.5cm, right of=p0] (r0) {};
	\node[pnbplace, below of=r0, draw=white, fill=white] (r1) {};
	\node[node distance=0.75cm, below of=r1, pnbplace, rotate=180, tokens=1] (r2) {};
	\node[below of=r2, pnbplace, rotate=180, tokens=1] (r3) {};
	\drawBoundaries{0}{0}

    \labelledpnbarr{r0.in}{p0.out}{{[yshift=0.25cm]$\pairof{\setof{t}}{\setof{a}}$}}{}{}
    \labelledpnbarr{p0.out}{p1.in}{below right:$\pairof{\setof{t,u}}{\setof{b}}$}{bend left=60}{}
    \labelledpnbarr{r3.out}{p3.out}{below left:$\pairof{\setof{v}}{\setof{d}}$}{o-180i-60}{}
\end{tikzpicture}
\caption{$\aPNB \comp \bPNB\withNetType{0}{0}$}
\end{subfigure}
}
\caption{Example Synchronous Composition}
\label{fig:examplePNBcomposition}
\end{figure}


\subsection{Graphical Notation}\label{sec:PNBgraphicalNotation}

PNBs are represented using an alternative to the classic Petri net graphical
notation. The alternative notation is lighter weight than the classic notation,
and is more intuitive when reasoning about and presenting PNB compositions.

In the alternative graphical representation, each place is drawn as
\emph{directed}, having an \textit{in} and \textit{out} port, drawn as a
triangle pointing into and out of the place, respectively. Transitions are
undirected links that connect an arbitrary set of boundary and place ports.
The standard graphical presentation of Petri nets has directed arcs, where the
direction is either ``place to transition'' or ``transition to place'', as
determined by the pre- and post-sets of each transition.

With our alternative notation, then, the pre-set of a transition is just the
set of places to which the transition is connected via the \textit{out} port,
symmetrically, its post-set is the set of places to which the
transition is connected via the \textit{in} port.

In order to distinguish individual transitions and increase legibility,
individual transitions are drawn with a small perpendicular mark.

We demonstrate the graphical notation in \figref{fig:PNBExamplePetriNet}, which
represents the same Petri net as that of \figref{fig:examplePetriNet}.

\begin{figure}[ht]
    \centering
    \begin{tikzpicture}[pnb]
        \node (p0) [pnbplace] {};
        \node (p1) [pnbplace, above right=of p0] {};
        \node (p2) [pnbplace, below right=of p0] {};
        \node (p3) [pnbplace, right=of p1] {};
        \node (p4) [pnbplace] at (p0 -| p3) {};
        \node (p5) [pnbplace, right=of p2] {};
        \node (p6) [pnbplace, right=of p4] {};

        \labelledpnbarr{p0.out}{p1.in}{}{}{}
        \labelledpnbarr{p0.out}{p2.in}{}{}{}

        \labelledpnbarr[p1p3]{p1.out}{p3.in}{}{}{pos=0.2}
        \labelledpnbarr[p2p5]{p2.out}{p5.in}{}{}{pos=0.2}

        \path (p1p3) edge[pnbarr, o0i180] (p4.in);
        \path (p2p5) edge[pnbarr, o0i180] (p4.in);

        \coordinate (joinp6) at ([xshift=-0.25cm]p6.in);

        \path (p3.out) edge[pnbarr, o0i180] (joinp6);
        \path (p5.out) edge[pnbarr, o0i180] (joinp6);
        \path (joinp6) edge[pnbarr, mark inside=0] (p6.in);

        \drawBoundaries{0}{0}
    \end{tikzpicture}
    \caption{PNB representation of the Petri net in
    \figref{fig:examplePetriNet}.}
    \label{fig:PNBExamplePetriNet}
\end{figure}

To emphasise the ``internal'' structure of a PNB, we often draw transitions (or
parts thereof) that connect to boundary ports in a lighter colour; of course,
formally there is no distinction, and therefore lighter sections can always be
ignored. Additionally, where transitions cross, we sometimes use different
patterns to make the structure of the transitions clearer.

Though we are yet to define composition of PNBs, we can give a graphical
intuition, which we will make precise later in this chapter: synchronous
composition corresponds to fusing the \emph{inner} boundary ports of two PNBs,
suitably synchronising the transitions connecting to those boundary ports.
Parallel composition, on the other hand, corresponds to stacking two PNBs upon
one another without fusing transitions.

\subsection{PNB Firing Semantics}\label{sec:PNBfiring}

As for explained in \remref{rem:contentionEnabled}, transitions in contention
cannot be fired concurrently; in Petri nets, two transitions are in contention
precisely when they compete for a resource, i.e. they consume a token from, or
produce a token at the same place.

For PNBs, there are \emph{two} additional sources of contention between transitions; essentially,
if the two transitions connect to the same boundary port, or were created by \emph{fusing} multiple
transitions that connected to the same boundary port, then they should be in contention.  For now,
we outline the definition of a contention relation, but defer discussion of its role in composition
until \secref{sec:nwbComposition}. Note that a contention relation \emph{must} contain all
pairs of transitions that have structural contention (i.e. connecting to a common boundary/place
port), but \emph{may} also contain additional pairs, as
in~\exref{exm:nonCompositionalWithoutContention}.

\begin{definition}[Contention Relation]\label{defn:contentionRelation}
    For a net, $\aPNB$, a symmetric relation, $\contention$, on
    $\trans{\aPNB}$ is said to be a contention relation, if for all $(\aTrans,
    \bTrans) \in \trans{\aPNB} \times \trans{\aPNB}$, $\aTrans \neq \bTrans$,
    when any of the following hold:

    \makebox[\textwidth][c]{%
    \begin{minipage}[t]{.75\textwidth}
    \begin{multicols}{2}
    \begin{enumerate}[(i)]
        \item $\pre{t} \intersection \pre{u} \neq \emptyset$,
        \item $\post{t} \intersection \post{u} \neq \emptyset$,
        \item\label{item:contentionItem1} $\source{t} \intersection \source{u} \neq \emptyset$,
        \item\label{item:contentionItem2} $\target{t} \intersection \target{u} \neq \emptyset$.
    \end{enumerate}
    \end{multicols}
    \end{minipage}}
    then $\aTrans \contention \bTrans$.
\end{definition}

We can lift contention from individual transitions to sets of transitions; indeed, abusing
notation, we write $\aPNBTransSet \contention \bPNBTransSet$ for $\aPNBTransSet, \bPNBTransSet
\subseteq \trans{\aPNB}$, if $\aTrans \contention \bTrans$ for some $\aTrans \in \aPNBTransSet$,
$\bTrans\in \bPNBTransSet$.

We define contention-free (\defnref{defn:pnContentionFreeSet}) and enabled
(\defnref{defn:pnEnabledTrans}) \emph{sets} of transitions in the same was as
for Petri nets. Firing is also defined in the same way, though we repeat the
definition here, for clarity:

\begin{definition}[Firing Semantics for PNBs] \label{defn:firePNBTransitions}
Given a PNB $\aPNB$ and a marking $\aPNBMarking$, a set $\aPNBTransSet
\subseteq \trans{\aPNB}$ can fire iff $\aPNBTransSet$ is contention-free and is
enabled by marking $\aPNBMarking$.
Firing $\aPNBTransSet$ creates a new marking $\bPNBMarking$, defined as follows:
\[
    \markedAt{\bPNBMarking}{\aPlace_i} \defeq \begin{cases}
        0 & \mbox{if } \aPlace_i \in \pre{\aPNBTransSet} \\
        1 & \mbox{if } \aPlace_i \in \post{\aPNBTransSet} \\
        \markedAt{\aPNBMarking}{\aPlace_i} & \mbox{otherwise}
    \end{cases}
\]
that is, tokens are consumed from all places in the pre-set of $\aPNBTransSet$,
produced at all places in the post-set of $\aPNBTransSet$, and otherwise
unchanged from $\aPNBMarking$. If $\aPNBTransSet \subseteq \trans{\aPNB}$ is
enabled and contention-free at $\aPNBMarking$, and transforms the
marking to $\bPNBMarking$ when fired, we write
$\PNBFiringSemantics{\aPNB}{\aPNBTransSet}{\aPNBMarking}{\bPNBMarking}$.
\end{definition}

To illustrate the use of boundary ports, we make precise the synchronous
composition of PNBs, which connects the boundary ports of suitable pairs of
PNBs.

\subsection{Synchronous composition of PNBs}
\label{sec:nwbComposition}

In order to synchronously compose two PNBs, they must share a boundary with the
same number of ports. Intuitively, composition forms a new net which has the
places of both nets, and transitions that are formed of sets of transitions,
one from each of the underlying nets. We often call nets that are to be
composed \emph{component nets}, or simply \emph{components}.

To define composition, we require the notion of a \emph{synchronisation}
between two nets, which is a particular choice of transitions from the two
components:

\begin{definition}[Synchronisations]\label{defn:synch}
    A synchronisation between two PNBs, $\aTypedPNB$ and $\bTypedPNB$ is a pair
    \[
        \pairof{\aPNBTransSet}{\bPNBTransSet} \in \powerset{\trans{\aPNB}} \times
    \powerset{\trans{\bPNB}}
    \]
    of contention-free sets of transitions, such that $\target{\aPNBTransSet} =
    \source{\bPNBTransSet}$.
\end{definition}

Synchronisations inherit an ordering from the subset ordering, pointwise:
\[
    \aPNBSync \subseteq \bPNBSync \defeq \aPNBTransSet \subseteq \aPNBTransSet'
\,\conjunction\, \bPNBTransSet \subseteq \bPNBTransSet'
\] We call the empty synchronisation, $(\emptyset, \emptyset)$, \emph{trivial}.
A synchronisation $\aPNBSync$ is \emph{minimal} when it is not trivial, and for
all synchronisations $\bPNBSync \subseteq \aPNBSync$, then $\bPNBSync$ is
trivial or equal to $\aPNBSync$. Intuitively, minimal synchronisations are the
smallest collections of transitions from both nets that share a common set of
boundary ports; put differently, we only retain those transitions that
\emph{must} fire together, to synchronise the two nets.

\begin{figure}[ht]
\centering
\begin{subfigure}{0.5\textwidth}
    \centering
\begin{tikzpicture}[pnb]
	\node[pnbplace, tokens=1] (p0) {};
	\node[pnbplace, tokens=1, below of=p0] (p1) {};

	\drawBoundaries{0}{0}
	\rBAlignedWith{p0}{1}
	\rBAlignedWith{p1}{2}

    \labelledpnbarr{p0.out}{r1}{$\aTrans$}{}{}
    \labelledpnbarr{p1.out}{r2}{$\bTrans$}{}{}
\end{tikzpicture}
\end{subfigure}%
\begin{subfigure}{0.5\textwidth}
    \centering
\begin{tikzpicture}[pnb]
    \node[pnbplace] (p0) [tokens=1] {};
    \node[pnbplace] (p1) [right=of p0] {};
    \node[pnbplace] (p2) [below=of p0] {};
    \node[pnbplace] (p3) at ($(p2.out)!.5!(p1.in)$) {};

    \coordinate (join) at ([xshift=-0.5cm]p3.in);

	\drawBoundaries{0}{0}
	\lBAlignedWith{p0}{1}
	\lBAlignedWith{p2}{2}

    \labelledpnbarr{p0.out}{p1.in}{$\cTrans$}{}{}
    \labelledpnbarr{l2}{p2.in}{below:$\eTrans$}{}{}
    \labelledpnbarr{join}{p3.in}{$\dTrans$}{pos=0}{pos=0}

    \draw (l1) edge[pnbarr,o0i180] (join);
    \draw (l2) edge[pnbarr,o0i180] (join);
\end{tikzpicture}
\end{subfigure}%
\caption{Components to be composed}
\label{fig:exampleSeqComp}
\end{figure}

\begin{example}
    Consider the two nets in~\figref{fig:exampleSeqComp}; we have the
    following:
    \begin{itemize}
        \item $\pairof{\setof{\aTrans}}{\emptyset}$ is not a
            synchronisation since $\target{\setof{\aTrans}} = \setof{0} \neq
            \emptyset = \source{\emptyset}$,
        \item $\pairof{\setof{\bTrans}}{\setof{\cTrans,\eTrans}}$ and
            $\pairof{\setof{\aTrans, \bTrans}}{\setof{\cTrans, \dTrans}}$ are
            synchronisations, but are not minimal,
        \item $\pairof{\setof{\bTrans}}{\setof{\eTrans}}$,
            $\pairof{\setof{\aTrans, \bTrans}}{\setof{\dTrans}}$ and
            $\pairof{\emptyset}{\setof{\cTrans}}$ are all minimal
            synchronisations.
    \end{itemize}
\end{example}

Indeed, since we wish to use minimal synchronisations as the transitions of
composed PNBs, we require the notion of contention between synchronisations:

\begin{definition}[Contention between synchronisations]\label{defn:synchcontention}
    Contention is lifted to synchronisations between $\aPNB$ and $\bPNB$:
    \[
        \aPNBSync \contention \bPNBSync \defeq \aPNBTransSet \contention_\aPNB
        \aPNBTransSet' \, \disjunction\, \bPNBTransSet \contention_\bPNB
        \bPNBTransSet'
    \]
    where $\contention_X$ is the contention relation of net $X$.
\end{definition}

For nets $\aPNB \withNetType{\aN}{\bN}$ and $\bPNB \withNetType{\bN}{\cN}$, let
$\minsync{\aPNB}{\bPNB}$ be the set of minimal synchronisations between them.

\begin{remark}
    For any $\aPNBSync \in \minsync{\aPNB}{\bPNB}$, we have that
    $\cardinalityof{\aPNBTransSet} > 1$, or $\cardinalityof{\bPNBTransSet} > 1$
    only if there exists a transition in $\aPNB$ or $\bPNB$ that is connected
    to more than one common boundary port. If this were not the case,
    $\aPNBSync$ would not be a \emph{minimal} synchronisation, since we could
    remove a transition while preserving the synchronisation property.
    Properties concerning composition of nets with transitions that are only
    connected to one boundary port are explored in \chpref{chp:catStructure},
    where we elucidate the categorical structure of PNBs. For each transition
    $\aTrans \in \trans{\aPNB}$ with $\target{\aTrans} = \emptyset$, we have
    $(\setof{\aTrans}, \emptyset) \in \minsync{\aPNB}{\bPNB}$, and similarly
    for $\trans{\bPNB}$. In other words, transitions in the components are free
    to fire without affecting the other component as long as they do not
    connect to the common boundary. If they do, the transitions must
    synchronise with transitions in the other component respecting their
    interaction on the shared boundary.
\end{remark}

We can now give the definition of synchronous composition of nets with
boundaries:

\begin{definition}[Synchronous composition]\label{defn:sequentialCompositionPNB}
The composition of PNBs $\aTypedPNB$ and $\bTypedPNB$, is written $\abPNBComp
\withNetType{\aN}{\cN}$, and has the following structure:
\begin{itemize}
    \item $\places{\abPNBComp}$ is $\places{\aPNB} \disjointUnion
        \places{\bPNB}$,
    \item $\trans{\abPNBComp}$ is $\minsync{\aPNB}{\bPNB}$, the set of minimal
        synchronisations,
    \item $\pre{\aPNBSync} \defeq \pre{\aPNBTransSet} \disjointUnion
        \pre{\bPNBTransSet}$ and $\post{\aPNBSync} \defeq \post{\aPNBTransSet}
        \disjointUnion \post{\bPNBTransSet}$,
    \item $\source{\aPNBSync} \defeq \source{\aPNBTransSet}$ and
        $\target{\aPNBSync} \defeq \target{\bPNBTransSet}$,
    \item Contention is lifted to minimal synchronisations, as described in
        \defnref{defn:synchcontention}, but is subtle, as we discuss in
        \remref{rem:contentionpreserved}.
\end{itemize}
\end{definition}

In order for the composition to be well-defined, lifting the contention
relations from PNBs $\aPNB$ and $\bPNB$ as per \defnref{defn:synchcontention}
must yield a well-defined contention relation for $\abPNBComp$, which is
confirmed in~\cite[Definition 3.4]{Bruni2013}.

\begin{remark}\label{rem:contentionpreserved}
    We require that contention is \emph{explicitly} preserved by synchronous
    composition: if certain transitions are in contention in the component
    nets, transitions containing them in the nets' composition should also be
    in contention. Indeed, this is the reason for requiring an explicit
    contention relation, rather than relying on the connectivity of the
    transitions alone to determine contention. As we will demonstrate shortly
    (\exref{exm:nonCompositionalWithoutContention}), certain PNBs, when
    composed, form transitions in the composite that should be in contention,
    but are not, structurally. The intuition being that since transitions in a
    composition are formed of sets of transitions of the components, we should
    not be able to fire transitions of the composition that are conflicting as
    transitions of the components. Further examples, and the mathematical
    foundations of contention are given
    in~\cite{Sobocinski2013a}.
\end{remark}

An example of synchronous composition is illustrated in
\figref{fig:examplePNBcomposition}. Note that there is no synchronisation
containing $c$, it has no transition in $\aPNB$ with which to synchronise, and
that $t$ appears in two synchronisations in $\aPNB \comp \bPNB$: synchronising
both with $a$ \emph{and} $b$.

The two additional sources of contention relative to Petri nets, mentioned
earlier in this section can now be stated precisely:
\begin{enumerate}
    \item Connecting to the same boundary port leads to contention:
        \defnref{defn:contentionRelation}\ref{item:contentionItem1}
        and~\ref{item:contentionItem2}
    \item Contention is preserved in compositions, by \defnref{defn:synchcontention}
\end{enumerate}

We now demonstrate synchronous composition of two simple nets, showing why
explicit contention relations are required, rather than inferring contention
from PNB structure:

\begin{example}\label{exm:nonCompositionalWithoutContention}
Consider the two component nets $L$ in \figref{fig:componentL} and $R$ in
\figref{fig:componentR} and their composition in \figref{fig:composeLR}.  We
have that $\aTrans \contention_{L} \bTrans$ and $\cTrans
\contention_{R} \dTrans$.

\newcommand{\sync}[2]{\left(\setof{#1}, \setof{#2}\right)}

\begin{figure}[ht]
\centering
\begin{subfigure}{0.5\textwidth}
    \centering
    \begin{tikzpicture}[pnb]
        \node[pnbplace,tokens=1] (p0) [pnblabel=above:$\aPlace_0$] {};
        \node[pnbplace,tokens=1] (p1) [pnblabel=below:$\aPlace_1$,below=of p0] {};
        \drawBoundaries{0}{1}

        \labelledpnbarr{p0.out}{r1}{left:$\aTrans$}{o0i180}{}
        \labelledpnbarr{p1.out}{r1}{left:$\bTrans$}{o0i180}{}
    \end{tikzpicture}
    \caption{Component $L$.}
    \label{fig:componentL}
\end{subfigure}%
\begin{subfigure}{0.5\textwidth}
    \centering
    \begin{tikzpicture}[pnb]
        \node[pnbplace] (p2) [pnblabel=above:$\aPlace_2$]{};
        \node[pnbplace] (p3) [pnblabel=below:$\aPlace_3$,below=of p0] {};

        \drawBoundaries{1}{0}
        \rBAlignedWith{p2}{1}
        \rBAlignedWith{p3}{2}

        \labelledpnbarr{l1}{p2.in}{right:$\cTrans$}{o0i180}{}
        \labelledpnbarr{l1}{p3.in}{right:$\dTrans$}{o0i180}{}
        \labelledpnbarr{p2.out}{r1}{above:$\eTrans$}{o0i180}{}
        \labelledpnbarr{p3.out}{r2}{below:$\fTrans$}{o0i180}{}
    \end{tikzpicture}
    \caption{Component $R$.}
    \label{fig:componentR}
\end{subfigure}
\\
\begin{subfigure}{\textwidth}
    \centering
    \begin{tikzpicture}[pnb, node distance=3cm]
        \node[pnbplace,tokens=1] (p0) [pnblabel=above:$\inl{\aPlace_0}$]{};
        \node[pnbplace,tokens=1] (p1) [pnblabel=below:$\inl{\aPlace_1}$,below=of p0]{};
        \node[pnbplace] (p2) [pnblabel=above:$\inr{\aPlace_2}$, node distance=4cm,right=of p0] {};
        \node[pnbplace] (p3) [pnblabel=below:$\inr{\aPlace_3}$, below=of p2] {};

        \drawBoundaries[1.5]{0}{0}
        \rBAlignedWith{p2}{1}
        \rBAlignedWith{p3}{2}

        \labelledpnbarr{p0.out}{p2.in}{{[scale=0.7]above:$(\setof{\aTrans}, \setof{\cTrans})$}}{}{}
        \labelledpnbarr{p0.out}{p3.in}{{[pos=0.7,scale=0.7]above right:$(\setof{\aTrans}, \setof{\dTrans})$}}{o-20i160}{pos=0.8}
        \labelledpnbarr{p1.out}{p2.in}{{[pos=0.3, scale=0.7]above left:$(\setof{\bTrans}, \setof{\cTrans})$}}{o20i-160,pnbarrstyle1}{pos=0.2}
        \labelledpnbarr{p1.out}{p3.in}{{[scale=0.7]below:$(\setof{\bTrans}, \setof{\dTrans})$}}{}{}
        \labelledpnbarr{p2.out}{r1}{{[scale=0.7]above:$(\emptyset, \setof{\eTrans})$}}{}{}
        \labelledpnbarr{p3.out}{r2}{{[scale=0.7]above:$(\emptyset, \setof{\fTrans})$}}{}{}
    \end{tikzpicture}
    \caption{$L \comp R$.}
    \label{fig:composeLR}
\end{subfigure}
\hfill
\caption{Example component nets and their synchronous composition}
\label{fig:exampleComposition}
\end{figure}

Consider the pair of transitions $(\setof{\aTrans}, \setof{\cTrans})$ and
$(\setof{\bTrans}, \setof{\dTrans})$ in the composition. If contention was not
remembered, these two transitions would not be in contention in the composite
net (they share no common place/boundary ports), and could therefore be fired
concurrently. However, the underlying sets of transitions in the components
\emph{are} in contention (i.e.  $\setof{\aTrans,\bTrans}$ and
$\setof{\cTrans,\dTrans}$), since the transitions connect to the same boundary
port and thus they cannot fire concurrently. Fortunately, in the examples we
consider in this thesis, complications due to contention rarely play a
prominent role.
\end{example}

Graphically, synchronous composition can be intuitively understood as fusing the
transitions that are connected to the shared boundary ports in the components,
duplicating transitions when there is a choice in the other component. For
example, in \figref{fig:componentL}, $\aTrans$ is connected to the first shared
boundary port; in \figref{fig:componentR} there are two transitions connected
to the first boundary port ($\cTrans$ and $\dTrans$), thus in the composition,
shown in \figref{fig:composeLR}, there are two transitions formed from
$\aTrans$.

\subsection{Parallel composition of PNBs}

Whereas synchronous composition requires compatible component boundaries,
intuitively to allow the right boundary of one component to be connected to
the left boundary of the other, parallel composition has no such restriction,
simply stacking the components on top of each other.

\begin{definition}[Parallel composition]\label{defn:tensorCompositionPNB}

The parallel, or tensor, composition of PNBs $\aTypedPNB$ and $\bTypedPNBFree$,
is written\\ $\abPNBTensor \withNetType{\aN + \cN}{\bN + \dN}$, and has the
following structure:
\begin{itemize}
    \item $\places{\abPNBTensor} \defeq \places{\aPNB} \disjointUnion
        \places{\bPNB}$
    \item $\trans{\abPNBTensor} \defeq \trans{\aPNB} \disjointUnion
        \trans{\bPNB}$
    \item $\pre{\parens{\inl{\aTrans}}} \defeq \setBuilder{\inl{\aPlace}}{\aPlace
            \in \pre{\aTrans}}$ and $\pre{\parens{\inr{\aTrans}}} \defeq
            \setBuilder{\inr{\aPlace}}{\aPlace \in \pre{\aTrans}}$
    \item $\post{\parens{\inl{\aTrans}}} \defeq \setBuilder{\inl{\aPlace}}{\aPlace
            \in \post{\aTrans}}$ and $\post{\parens{\inr{\aTrans}}} \defeq
            \setBuilder{\inr{\aPlace}}{\aPlace \in \post{\aTrans}}$
    \item $\source{\parens{\inl{\aTrans}}} \defeq \source{\aTrans}$ and
            $\source{\parens{\inr{\aTrans}}} \defeq \setBuilder{\aPNBBPort +
        \aN}{\aPNBBPort \in \source{\aTrans}}$
    \item $\target{\parens{\inl{\aTrans}}} \defeq \target{\aTrans}$ and
        $\target{\parens{\inr{\aTrans}}} \defeq
        \setBuilder{\aPNBBPort + \bN}{\aPNBBPort \in \target{\aTrans}}$
    \item $\mathord{\contention} \defeq
        \setBuilder{\pairof{\inl{\aTrans}}{\inl{\bTrans}}}{\pairof{\aTrans}{\bTrans}
        \in \mathord{\contention_\aPNB}} \union
        \setBuilder{\pairof{\inr{\aTrans}}{\inr{\bTrans}}}{\pairof{\aTrans}{\bTrans}
        \in \mathord{\contention_\bPNB}}$
\end{itemize}
\end{definition}

Unlike synchronous composition, transitions in a tensor composition are only in
contention if they are in contention in one of the components. An example
tensor composition is illustrated in \figref{fig:exampleTensorComposition}.

\begin{figure}[ht]
\centering
\begin{subfigure}{0.5\textwidth}
    \centering
    \begin{tikzpicture}[pnb]
        \node[pnbplace, tokens=1] (p0) [pnblabel=above:$\aPlace_0$] {};

        \drawBoundaries[1]{1}{1}

        \labelledpnbarr{l1}{p0.in}{above:$\aTrans$}{o0i180}{}
        \labelledpnbarr{p0.out}{r1}{above:$\bTrans$}{o0i180}{}
    \end{tikzpicture}
    \caption{Component $T$.}
    \label{fig:componentT}
\end{subfigure}%
\begin{subfigure}{0.5\textwidth}
    \centering
    \begin{tikzpicture}[pnb]
        \node[pnbplace] (p1) [pnblabel=above:$\aPlace_1$]{};
        \node[pnbplace] (p2) [pnblabel=below:$\aPlace_2$,below=of p1]{};

        \drawBoundaries[1]{0}{1}
        \lBAlignedWith{p1}{1}
        \lBAlignedWith{p2}{2}


        \coordinate (join) at ([xshift=-0.5cm]r1);


        \labelledpnbarr{join}{r1}{above:$\dTrans$}{o0i180}{pos=0.01}
        \draw (p1.out) edge[pnbarr,o0i180] (join);
        \draw (p2.out) edge[pnbarr,o0i180] (join);

        \labelledpnbarr{l1}{l2}{right:$\cTrans$}{bend left=90}{}
    \end{tikzpicture}
    \caption{Component $B$.}
    \label{fig:componentB}
\end{subfigure}
\\
\begin{subfigure}{\textwidth}
    \centering
    \begin{tikzpicture}[pnb]
        \node[pnbplace,tokens=1] (p0) [pnblabel={[overlay=false]above:$\inl{\aPlace_0}$}]{};
        \node[pnbplace] (p1) [pnblabel={[overlay=false]above:$\inr{\aPlace_1}$}, below=of p0]{};
        \node[pnbplace] (p2) [pnblabel={[overlay=false]below:$\inr{\aPlace_2}$}, below=of p1] {};

        \drawBoundaries[1.2]{0}{0}

        \lBAlignedWith{p0}{1}
        \lBAlignedWith{p1}{2}
        \lBAlignedWith{p2}{3}
        \rBAlignedWith{p0}{1}
        \rBAlignedWith{$(p1)!.5!(p2)$}{2}

        \labelledpnbarr{l1}{p0.in}{above:$\inl{\aTrans}$}{o0i180}{}
        \labelledpnbarr{p0.out}{r1}{above:$\inl{\bTrans}$}{o0i180}{}

        \coordinate (join) at ([xshift=-0.5cm]r2);

        \labelledpnbarr{join}{r2}{{[yshift=0.2cm]above:$\inr{\dTrans}$}}{o0i180}{pos=0.01}
        \draw (p1.out) edge[pnbarr,o0i180] (join);
        \draw (p2.out) edge[pnbarr,o0i180] (join);

        \labelledpnbarr{l2}{l3}{right:$\inr{\cTrans}$}{bend left=90}{}
    \end{tikzpicture}
    \caption{$T \tensor B$.}
    \label{fig:composeTB}
\end{subfigure}
\hfill
\caption{Example component nets and their parallel composition.}
\label{fig:exampleTensorComposition}
\end{figure}

For convenience, and when there can be no confusion (e.g. when the places of
the components are disjoint), we label places of a composite PNB as $\aPlace$
instead of $\inl{\aPlace}$ or $\inr{\aPlace}$ and similarly for transitions.

\subsection{Connectedness, Purity and Simplicity}

We briefly explore several standard Petri net properties that PNBs \emph{do
not} satisfy; specifically, connectedness, purity and simplicity (described
in~\cite{Bernardinello1992}). These properties, assuming they have been stated
in the naive way for PNBs, are \emph{not} preserved by synchronous composition
and thus they are in-fact ill-defined for arbitrary PNBs.

When introducing ENs, we mentioned that in the literature, an EN should have no
isolated places or transitions---the net is \emph{connected}. However, we lift
this restriction for the ENs underlying PNBs: we may have places that are
unconnected to transitions and transitions unconnected to any places.

To see that connectedness is not preserved by composition, consider the
component nets in \figref{fig:connectedComponents}. Assuming the naive
extension of the definition from Petri nets, both $L$ and $R$ are connected
($R$ vacuously so), yet their synchronous composition is not connected: there is
no transition in $R$ for $\aTrans$ to synchronise with, therefore $\aTrans$ is
not present in any synchronisation, leaving $\aPlace$ unconnected in $L \comp
R$.

\begin{figure}[ht]
    \centering
    \begin{subfigure}{0.33\textwidth}
    \centering
    \begin{tikzpicture}[pnb]
        \node (p0) [pnbplace, pnblabel=above:$\aPlace$] {};
        \drawBoundaries{0}{1}

        \labelledpnbarr{p0.out}{r1}{$\aTrans$}{}{}
    \end{tikzpicture}
    \caption{Component $L$}
    \end{subfigure}%
    \begin{subfigure}{0.33\textwidth}
    \centering
    \begin{tikzpicture}[pnb]
        \node (p0) [pnbhidden] {};
        \drawBoundaries{1}{0}
    \end{tikzpicture}
    \caption{Component $R$}
    \end{subfigure}
    \begin{subfigure}{0.33\textwidth}
    \centering
    \begin{tikzpicture}[pnb]
        \node (p0) [pnbplace, pnblabel=above:$\aPlace$] {};
        \drawBoundaries{0}{0}
    \end{tikzpicture}
    \caption{Composition $L \comp R$}
    \end{subfigure}
    \caption{Connectedness is not preserved by composition}
    \label{fig:connectedComponents}
\end{figure}

Whereas the EN definition of connectedness only concerns places and
transitions, since PNB transitions can also connect to boundary ports, we might
additionally require that boundary ports are connected (i.e. each boundary port
has at least one transition connected to it). However, this is an unreasonable
restriction: recall that each boundary port allows the \emph{possibility} of
partially specifying those transitions that connect to it; it is therefore
entirely reasonable to allow no transition to be partially specified through a
particular boundary port.

Purity is the property of Petri nets that there are no self loops: a single
transition cannot be connected to both the out-port and the in-port of a single
place. Consider the components shown in \figref{fig:pureComponents}; each is
pure (again by the naive extension of the definition to PNBs), yet their
composition is impure, since there is a single transition connecting the
out-port of $\aPlace$ to its in-port.

\begin{figure}[ht]
    \centering
    \begin{subfigure}{0.65\textwidth}
        \begin{subfigure}{0.33\textwidth}
        \centering
        \begin{tikzpicture}[pnb]
            \node (p0) [pnbhidden] {};
            \drawBoundaries{0}{2}

            \labelledpnbarr{r1}{r2}{}{bend right=90, looseness=5}{}
        \end{tikzpicture}
        \caption{Component $L$}
        \end{subfigure}%
        \begin{subfigure}{0.33\textwidth}
            \centering
            \begin{subfigure}{\textwidth}
            \centering
            \begin{tikzpicture}[pnb]
                \node (p0) [pnbplace, pnblabel=above:$\aPlace$] {};
                \drawBoundaries{1}{1}

                \pnbarr{l1}{p0.in}
                \pnbarr{p0.out}{r1}
            \end{tikzpicture}
            \caption{Component $T$}
            \end{subfigure}%
            \\
            \begin{subfigure}{\textwidth}
            \centering
            \begin{tikzpicture}[pnb]
                \node (p0) [pnbhidden] {};
                \drawBoundaries{1}{1}

                \pnbarr{l1}{r1}
            \end{tikzpicture}
            \caption{Component $B$}
            \end{subfigure}
        \end{subfigure}%
        \begin{subfigure}{0.33\textwidth}
        \centering
        \begin{tikzpicture}[pnb]
            \node (p0) [pnbhidden] {};
            \drawBoundaries{2}{0}

            \labelledpnbarr{l1}{l2}{}{bend left=90, looseness=5}{}
        \end{tikzpicture}
        \caption{Component $R$}
        \end{subfigure}%
    \end{subfigure}%
    \begin{subfigure}{0.35\textwidth}
    \centering
    \begin{tikzpicture}[pnb]
        \node (p0) [pnbplace, pnblabel=above:$\aPlace$] {};
        \drawBoundaries{0}{0}

        \coordinate  (via) at ([yshift=-0.75cm]p0);

        \draw (p0.out) edge[pnbarr, o0i0, mark inside=1] (via);
        \draw (via) edge[pnbarr, o180i180] (p0.in);
    \end{tikzpicture}
    \caption{Composition\\$L \comp (T \tensor B) \comp R$}
    \end{subfigure}
    \caption{Purity is not preserved by composition}
    \label{fig:pureComponents}
\end{figure}

Finally, simplicity is the property that the set of places that a transition
connects to uniquely identifies it. Again, assuming the naive extension to PNBs
(i.e. also considering the boundary ports in the unique \emph{footprint}),
consider the components shown in \figref{fig:simpleComponents}. Each component
is simple, yet their composition is not: both transitions connect the out-port
of $\aPlace_1$ to the in-port of $\aPlace_2$, yet they are different
transitions, one is $(\setof{\aTrans}, \emptyset)$ and the other is
$(\setof{\bTrans}, \setof{\cTrans})$.

\begin{figure}[ht]
    \centering
    \begin{subfigure}{0.33\textwidth}
    \centering
    \begin{tikzpicture}[pnb]
        \node [pnbplace] (p0) {};

        \drawBoundaries[1]{1}{1}

        \labelledpnbarr{l1}{p0.in}{$\aTrans$}{bend left=45}{}
        \labelledpnbarr[l1p0]{l1}{p0.in}{below left:$\bTrans$}{bend right=45}{}
        \draw (l1p0) edge[pnbarr, out=-45,in=-170, looseness=1.4] (r1);
    \end{tikzpicture}
    \caption{Component $L$}
    \end{subfigure}%
    \begin{subfigure}{0.33\textwidth}
    \centering
    \begin{tikzpicture}[pnb]
        \node (px) {};

        \drawBoundaries{1}{0}

        \labelledpnbarr{l1}{[xshift=0.5cm]l1}{$\cTrans$}{}{}
    \end{tikzpicture}
    \caption{Component $R$}
    \end{subfigure}
    \begin{subfigure}{0.33\textwidth}
    \centering
    \begin{tikzpicture}[pnb]
        \node (p0) [pnbplace] {};

        \drawBoundaries[1]{1}{0}

        \labelledpnbarr{l1}{p0.in}{$\aTrans$}{bend left=45}{}

        \labelledpnbarr{l1}{p0.in}{{[xshift=0.4cm]below:$(\setof{\bTrans}, \setof{\cTrans})$}}{bend right=45}{}
    \end{tikzpicture}
    \caption{Composition $L \comp R$}
    \end{subfigure}
    \caption{Simplicity is not preserved by composition}
    \label{fig:simpleComponents}
\end{figure}

Thus, despite the fact that any PNB, $\aPNB \withNetType{0}{0}$, is
\emph{isomorphic} to a Petri net, there can be no guarantee that the Petri net
will necessarily be pure, connected or simple.

We now proceed to defining an LTS-based semantics for PNBs, in a similar
fashion to that of Petri nets.

\subsection{\TLTS{} Semantics of PNBs} \label{sec:ltsPNB}

Recall that the LTS semantics for Petri nets defined
in~\secref{sec:PNLTSSemantics} labels the LTS transition between two states
(markings) being the set of Petri net transitions that were fired to transform
between the markings.

For PNBs, we could use the same definition. However, as we will show later in this
thesis---with a category theoretic proof \secref{sec:functorPNBTLTS} and an
LTS-based proof~\secref{sec:compCheckingProof}---it is sufficient to forget the
identities of the fired transitions, and instead only record their
\emph{effect} of their interactions on the boundaries. Furthermore, since PNBs
have two boundaries, we use a \TLTS{} semantics, with one label per boundary
side.

As a first step towards defining the \TLTS{} semantics of PNBs, we show how to
map sets of boundary ports into binary strings: indeed, we use pairs of binary
strings to record the interactions of fired PNB transitions on the boundaries:
\begin{align*}
\ordinalSetToBinary{-} &: \powerset{\ordinal{\aN}} \to \B^\aN \\
\ordinalSetToBinary{O}_\bN &\defeq \begin{cases}1 \text{ if } \bN \in O \\
    0 \text { otherwise,} \end{cases} \text{ for } (0 \le \bN < \aN)\\
\end{align*}
that is, the $\bN^{\text{th}}$ character of the string is $1$ if the
$\bN^{\text{th}}$ boundary port is in the set of boundary ports, and $0$
otherwise. Now, we can define the \TLTS{} semantics of PNBs:

\begin{definition}[\TLTS{} semantics of a PNB]\label{defn:PNBlts}
    For a PNB $\aTypedPNB$, its \TLTS{} semantics is written as
    $\TLTSsemantics{\aPNB}$ and is a \LTSB{\aN}{\bN}:
    \[
        (\powerset{\places{\aPNB}}, \B^\aN \times \B^\bN, \aLTSTransitionRel)
    \]
    with $\aPNBMarking \LabelledTrans{\lbl{\aLbl}{\bLbl}} \bPNBMarking$ iff
    there is a set of PNB transitions, $\aPNBTransSet$, such that
    \[
        \aPNB_\aPNBMarking \to_\aPNBTransSet \aPNB_\bPNBMarking
        \text{ with } \ordinalSetToBinary{\source{\aPNBTransSet}} = \aLbl
        \text{ and } \ordinalSetToBinary{\target{\aPNBTransSet}} = \bLbl
    \]
\end{definition}

\begin{example}\label{ex:LTScomponentB}
    The \TLTS{} semantics for the simple PNB in \figref{fig:componentB} is
    given in \figref{fig:componentBLTS}.
\end{example}

\begin{figure}[ht]
\centering
\begin{tikzpicture}[nfa]
\node[state] (00)  {$\emptyset$};
\node[state] (10) [below=of 00] {$\setof{\aPlace_2}$};
\node[state] (01) [right=of 00] {$\setof{\aPlace_1}$};
\node[state] (11) [below=of 01] {$\setof{\aPlace_1,\aPlace_2}$};
\path (11) edge node {$\setof{\lbl{00}{1}, \lbl{11}{1}}$} (00)
           edge [loop below] node {$\setof{\lbl{00}{0}, \lbl{11}{0}}$} (11)
      (00) edge [loop above] node {$\setof{\lbl{00}{0}, \lbl{11}{0}}$} (00)
      (01) edge [loop above] node {$\setof{\lbl{00}{0}, \lbl{11}{0}}$} (01)
      (10) edge [loop below] node {$\setof{\lbl{00}{0}, \lbl{11}{0}}$} (10) ;
\end{tikzpicture}
\caption{\LTSB{2}{1} semantics of the PNB in \figref{fig:componentB}. }
\label{fig:componentBLTS}
\end{figure}

Note that the \TLTS{} semantics of any net $\aTypedPNB$ is always
\emph{reflexive}, that is, there exists a transition: $s
\LabelledTrans{\lbl{0^\aN}{0^\bN}} s$ for every state, $s$, since the empty set
of transitions is enabled, vacuously and is contention-free at every marking.

\subsection{Marked PNBs}\label{sec:markedPNB}

For a given PNB and pair of initial and final markings, we can
consider its \TLTS{} semantics as an (specially-labelled) NFA, as we did
for Petri nets (\defnref{defn:PnNFA}), by taking the
states corresponding to the appropriate markings as initial/accepting states of
the NFA. To do so, we first make precise the definition of a PNB with
initial/target markings:

\begin{definition}[Marked PNB]
    A marked PNB is a triple, $(\aPNB, \aPNBMarking, \bPNBMarking)$ consisting
    of a PNB, $\aPNB$, together with a particular initial ($\aPNBMarking$) and
    target ($\bPNBMarking$) marking.
\end{definition}

We can now define the \TNFA{} semantics of a \emph{marked} PNB as follows:

\begin{definition}[\TNFA{} semantics of a marked PNB]\label{defn:PNBTNFA}
    Given a marked PNB, $\amPNB$, the \TNFA{} semantics extends the \TLTS{}
    semantics: the initial state is the state corresponding to $\aPNBMarking$
    and the accepting states are the singleton set containing the state
    corresponding to $\bPNBMarking$. We write $\PNBToTNFA{\amPNB}$ for the
    \TNFA{} semantics of a marked PNB, $\aPNB$.
\end{definition}

\begin{example}
    Consider the marked PNB $(B, \setof{\aPlace_1, \aPlace_2}, \emptyset)$,
    where $B$ is taken from \figref{fig:componentB}. We construct
    $\PNBToTNFA{B}$ by taking the \TLTS{} shown in~\figref{fig:componentBLTS},
    and making the state corresponding to the empty marking as accepting, and
    that corresponding to the state marking both places as initial.
\end{example}

\subsection{Isomorphism of PNBs}

Suppose that $\aPNB,\bPNB \withNetType{\aN}{\bN}$ are PNBs. We say that a
mapping $f : \aPNB \to \bPNB$, comprised of two components: $f_P :
\places{\aPNB} \to \places{\bPNB}$ and $f_T : \trans{\aPNB} \to \trans{\bPNB}$,
is a homomorphism if the following hold, for all $\aTrans, \bTrans \in
\trans{\aPNB}$:
\begin{enumerate}
    \item $\pre{f_T(\aTrans)} =  \setBuilder{f_P(\aPlace)}{\aPlace \in
        \pre{\aTrans}}$,
    \item $\post{f_T(\aTrans)} = \setBuilder{f_P(\aPlace)}{\aPlace \in
        \post{\aTrans}}$,
    \item $\source{f_T(\aTrans)} = \source{\aTrans}$,
    \item $\target{f_T(\aTrans)} = \target{\aTrans}$,
    \item $f_T(\aTrans) \contention_N f_T(\bTrans) \iff \aTrans
        \contention_M \bTrans$.
\end{enumerate}

We say that a homomorphism $f$ is an isomorphism iff both components are
bijections, writing $\aPNB \PNBIso \bPNB$ when an isomorphism exists
between $\aPNB$ and $\bPNB$.

Isomorphic nets clearly have isomorphic markings (and thus \TLTS{} semantics):
\begin{lemma}
If $\aPNB$ and $\bPNB$ are nets, and $\aPNB \PNBIso \bPNB$, then
$\TLTSsemantics{\aPNB} \NFAIso \TLTSsemantics{\bPNB}$.
\end{lemma}
\begin{proof}
    Immediate: the state component of the LTS-isomorphism is the place
    component of the PNB-isomorphism; the PNB transition structure is
    preserved, the label component of the LTS-isomorphism is the identity
    function.
\end{proof}

\subsection{Read Arcs}\label{sec:PNBReadArcs}

In this thesis we introduce a slight modification of PNBs, adding a new arc
type (connection between transitions and places), called a read arc, as found
in contextual nets~\cite{Christensen1993}.

Read arcs allow for (concurrent) \emph{non-destructive} reads of a token at a
particular place. In standard PNBs, it is possible for a transition to check
for the presence of a token at a particular place: the transition can remove
the token from the place under consideration, placing it at a temporary
place, before another transition fires, placing the token back in the original
place. However, only one such remove/replace loop can be concurrently fired for
any single target place (each removal transition is in contention with every
other). We say that remove/replace loops are \emph{destructive reads} of a
token. An example of a net using remove/replace loops is illustrated in
\figref{fig:removeReplaceLoops}; its intuitive behaviour is that $\aPlace_1$ is
filled by $\bTrans$, emptied by $\cTrans$ and the presence of its token
can be checked by $\aTrans$ and $\dTrans$. Indeed, $\aTrans$ and $\dTrans$ both
attempt to remove $\aPlace_1$'s token, which cannot happen concurrently, hence
the presence of a token at $\aPlace_1$ cannot be signalled on the left and
right at once. Indeed, using remove/replace loops (and the extra required
places) in this manner leads to (undesired) additional behaviour, as can be
witnessed in the \TLTS{} semantics shown in \figref{fig:removeReplaceLTS}.

\begin{figure}[ht]
\centering
\begin{tikzpicture}[pnb]
\node[pnbplace] (p0) [pnblabel=above:$\aPlace_1$, rotate=90] {};
\node[pnbplace] (p1) [pnblabel=below:$\aPlace_0$, rotate=-90, left=of p0] {};
\node[pnbplace] (p2) [pnblabel=below:$\aPlace_2$, rotate=-90, right=of p0] {};

\labelledpnbarr[p0p1]{p0.out}{p1.in}{above:$\aTrans$}{bend right=90}{pos=0.7}
\labelledpnbarr{p1.out}{p0.in}{}{bend right=90}{pos=0.5}

\labelledpnbarr[p0p2]{p0.out}{p2.in}{above:$\dTrans$}{bend left=90}{pos=0.7}
\labelledpnbarr{p2.out}{p0.in}{}{bend left=90}{pos=0.5}

\drawBoundaries{2}{2}

\draw (l1) edge[pnbarr,o0i150] (p0p1);

\labelledpnbarr{l2}{p0.in}{below:$\bTrans$}{o-30i-90, looseness=1.5}{}
\labelledpnbarr{r1}{p0.out}{above:$\cTrans$}{o150i90, looseness=1.5}{}

\draw (r2) edge[pnbarr,o150i30] (p0p2);

\end{tikzpicture}
\caption{PNB with remove/replace loops}
\label{fig:removeReplaceLoops}
\end{figure}

\begin{figure}[ht]
    \centering
    \scalebox{0.8}{
    \begin{tikzpicture}[nfa]
        \node[state] (3)  {$010$};
        \node[state] (7) [above=of 3] {$000$};

        \node[state] (6) [below left=of 3] {$001$};
        \node[state] (2) [below=of 6] {$011$};

        \node[state] (5) [below right=of 3] {$100$};
        \node[state] (1) [below=of 5] {$110$};

        \node[state] (4) [below right=of 2] {$101$};
        \node[state] (0) [below=of 4] {$111$};

        \path (0) edge [loop right] node {$\lbl{00}{00}$} (0);
        \path (1) edge [loop right] node {$\lbl{00}{00}$} (1);
        \path (2) edge [loop left] node {$\lbl{00}{00}$} (2);
        \path (3) edge [loop right] node {$\lbl{00}{00}$} (3);
        \path (4) edge [loop right] node {$\lbl{00}{00}$} (4);
        \path (5) edge [loop right] node {$\lbl{00}{00}$} (5);
        \path (6) edge [loop left] node {$\lbl{00}{00}$} (6);
        \path (7) edge [loop right] node {$\lbl{00}{00}$} (7);

        \path (0) edge [bend left] node {$\lbl{00}{10}$} (4);
        \path (1) edge [bend left] node {$\lbl{00}{01}$} (4);
        \path (1) edge [bend left] node {$\lbl{00}{10}$} (5);
        \path (2) edge [bend left, pos=0.2] node {$\lbl{10}{00}$} (4);
        \path (2) edge [bend left] node {$\lbl{00}{10}$} (6);
        \path (3) edge [bend left] node {$\lbl{10}{00}$} (5);
        \path (3) edge [bend left, pos=0.8] node {$\lbl{00}{01}$} (6);
        \path (3) edge [bend left] node {$\lbl{00}{10}$} (7);
        \path (4) edge [bend left] node {$\lbl{01}{00}$} (0);
        \path (4) edge [bend left, pos=0.8] node {$\lbl{00}{00}$} (1);
        \path (4) edge [bend left] node {$\lbl{00}{00}$} (2);
        \path (5) edge [bend left] node {$\lbl{01}{00}$} (1);
        \path (5) edge [bend left, pos=0.2] node {$\lbl{00}{00}$} (3);
        \path (6) edge [bend left] node {$\lbl{01}{00}$} (2);
        \path (6) edge [bend left] node {$\lbl{00}{00}$} (3);
        \path (7) edge [bend left] node {$\lbl{01}{00}$} (3);
    \end{tikzpicture}
}
    \newcommand{\capt}{\TLTS{} semantics of the PNB with remove/replace loops shown in
    \figref{fig:removeReplaceLoops}.}
    \caption[\capt]{\capt{} For compact representation, markings are
    identified with binary strings: there is a $1$ at position $i$ iff place
    $\aPlace_i$ is marked.}
\label{fig:removeReplaceLTS}
\end{figure}

Adding read arcs to the definition of PNBs requires making simple modifications
to the definitions of contending/enabled transitions:

\begin{definition}[PNB with read arcs]
    A PNB with read arcs is a 10-tuple:
    $(\aPNBAllPlaces, \aPNBAllTrans, \pre{-}, \post{-}, l, r, \source{-},
    \target{-}, \readArc{-}, \contention)$, where:
    \begin{itemize}
        \item $(\aPNBAllPlaces, \aPNBAllTrans, \pre{-}, \post{-}, l, r,
            \source{-}, \target{-}, \contention)$, is a PNB,
        \item $\readArc{-} : \aPNBAllTrans \to \powerset{\aPNBAllPlaces}$
            connects a transition to places via read arcs,
        \item $\contention$ is a read-arc aware contention relation: a
            symmetric relation such that for $(\aTrans, \bTrans) \in
            \aPNBAllTrans \times \aPNBAllTrans$, $\aTrans \neq \bTrans$, for
            which any of the conditions of \defnref{defn:contentionRelation} or
            the following condition holds:
            \[
                \readArc{\aTrans} \intersection \prepost{\bTrans} \neq \emptyset
            \]
            then $\aTrans \contention \bTrans$.
    \end{itemize}
\end{definition}

\begin{remark}
    Observe that two transitions, $\aTrans, \bTrans$ may have
    $\readArc{\aTrans} \intersection \readArc{\bTrans} \neq \emptyset$,
    but $\aTrans \not\contention \bTrans$.  Indeed, this is what allows
    concurrent non-destructive reads.
\end{remark}

The graphical representation of a read arc is an undirected edge connecting to
the side of a place.  The example shown in \figref{fig:removeReplaceLoops} can
be encoded using read arcs as shown in \figref{fig:readArcs}. Now, $\aTrans$
and $\bTrans$ are not in contention, and are both enabled by the current
marking. Thus, the set $\setof{\aTrans, \bTrans}$ \emph{can} be fired
(preserving the net's marking).

\begin{figure}[ht]
    \centering
\begin{tikzpicture}[pnb]
\node[pnbplace] (p0) [pnblabel={[xshift=0.25cm]$\aPlace_1$}] {};

\drawBoundaries{2}{2}

\labelledpnbarr{l1}{p0.north}{below:$\aTrans$}{o0i90}{pos=0.3}
\labelledpnbarr{l2}{p0.in}{below:$\bTrans$}{o0i180}{}

\labelledpnbarr{p0.out}{r1}{below:$\cTrans$}{o0i180}{}
\labelledpnbarr{r2}{p0.south}{below:$\dTrans$}{o180i-90}{pos=0.3}

\end{tikzpicture}
\caption{PNB with read arcs}
\label{fig:readArcs}
\end{figure}

We tweak the definition of enabled transitions from that of
\defnref{defn:pnEnabledTrans}, to require that each place being read has a
token, and that a transition does not produce/consume a token at the same place
as it reads:

\begin{definition}[Enabled Transitions for PNBs with Read Arcs]
    \label{defn:enabledTransPNBReadArcs}
    For a PNB with read arcs, $\aPNB$, a transition $\aTrans \in \trans{\aPNB}$
    is enabled for a marking, $\aPNBMarking$, written
    $\enabledTrans{\aTrans}{\aPNBMarking}$, if:
    \begin{multicols}{2}
    \begin{enumerate}
        \item $\forall \aPlace \in \pre{\aTrans}$,
            $\markedAt{\aPNBMarking}{\aPlace} = 1$
        \item $\forall \bPlace \in \readArc{\aTrans}$,
            $\markedAt{\aPNBMarking}{\bPlace} = 1$
        \item $\forall \cPlace \in \post{\aTrans}$,
            $\markedAt{\aPNBMarking}{\cPlace} = 0$
        \item $\prepost{\aTrans} \intersection \readArc{\aTrans} = \emptyset$
    \end{enumerate}
\end{multicols}
\end{definition}

\begin{remark}
To see why these modifications are necessary, observe that PNBs \emph{without}
read arcs may contain \emph{self-conflicting} transitions: a transition can
share a place between its pre and post sets. However, sets containing a
self-conflicting transition \emph{are} contention-free by
\defnref{defn:pnContentionFreeSet}, but, they cannot be fired, since they are
not enabled. Indeed, to be enabled, a transition's pre-set must be marked and
the post set unmarked; a self-conflicting PNB transition would require a place
to be both marked and unmarked and therefore is never enabled.

On the other hand, PNBs with read arcs may contain a transition, $\aTrans$,
which attempts to consume the token from a place and also read the token, i.e.
there is some $\aPlace \in \pre{\aTrans}$ such that $\aPlace \in
\readArc{\aTrans}$. Such a transition is similarly \emph{never} enabled, by
\defnref{defn:enabledTransPNBReadArcs}. Intuitively, it does not make sense to
non-destructively read a token \emph{and} destructively remove a token from the
same place.
\end{remark}

We are now able to construct LTS semantics of PNBs with read arcs, by amending
the firing semantics of \defnref{defn:firePNBTransitions} to use the modified
definition of enabled transition sets (\defnref{defn:enabledTransPNBReadArcs}).
The \TLTS{} semantics of the PNB with read arcs shown in \figref{fig:readArcs}
is shown in \figref{fig:readArcsLTS}; it is clear that the undesirable
\emph{additional} behaviour due to using remove/replace loops is no longer
present.

\begin{figure}[ht]
    \centering
    \begin{tikzpicture}[nfa]
    \node[state] (0)  {$\emptyset$};
    \node[state] (1) [below=of 0] {$\setof{\aPlace_1}$};
    \path (0) edge [bend left] node {$\lbl{01}{00}$} (1)
              edge [loop right] node {$\lbl{00}{00}$} (0)
          (1) edge [bend left] node {$\lbl{00}{10}$} (0)
              edge [loop right] node {$\lbl{*0}{0*}$} (1)
          ;
    \end{tikzpicture}
    \caption{\TLTS{} semantics of the PNB with read arcs shown in
    \figref{fig:readArcs}.}
    \label{fig:readArcsLTS}
\end{figure}
