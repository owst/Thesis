\section{Labelled Transition Systems}

The notion of a labelled transition system (LTS) is a central conceptual tool
used to capture semantic models in this thesis. An LTS is a collection of
states, and (labelled) transitions relating pairs of states:

\begin{definition}[LTS]
An LTS is a 3-tuple: $(\aLTSAllStates,
\aLTSAllLabels, \aLTSTransitionRel)$, where $\aLTSAllStates$ is the set of
states, $\aLTSAllLabels$ the set of labels and $\aLTSTransitionRel{} \subseteq
\aLTSAllStates \times \aLTSAllLabels \times \aLTSAllStates$ is the labelled
transition relation. As standard, we write
\[
    (\aLTSState, \aLTSLabel, \bLTSState) \in{} \mathord{\aLTSTransitionRel}
    \,\text{ as }\,
    \aLTSState \LabelledTrans{\aLTSLabel} \bLTSState.
\]
\end{definition}

\begin{definition}[Finite LTS]
    An LTS, $(\aLTSAllStates, \aLTSAllLabels, \aLTSTransitionRel)$, is finite
    if both $\aLTSAllStates$ and $\aLTSAllLabels$ are finite.
\end{definition}

\begin{figure}[ht]
    \centering
    \begin{tikzpicture}[nfa]
        \node[state] (0)                    {$0$};
        \node[state] (1) [below right=of 0] {$1$};
        \node[state] (2) [below left=of 0]  {$2$};

        \path (0) edge              node {$a$} (1);
        \path (0) edge [swap]       node {$b$} (2);
        \path (1) edge [loop left]  node {$c$} (1);
        \path (2) edge [loop right] node {$d$} (2);
    \end{tikzpicture}
\caption{Example LTS}
\label{fig:exampleLTS}
\end{figure}

\begin{example}
An example finite LTS is illustrated in \figref{fig:exampleLTS}, with
$\aLTSAllStates = \setof{0,1,2}$, $\aLTSAllLabels = \setof{a,b,c,d}$, and\\
$\mathord{\aLTSTransitionRel} = \setof{(0,a,1), (0,b,2), (1,c,1), (2,d,2)}$.
\end{example}

The set of finite \emph{runs} of an LTS consists of all finite strings of
transitions where the states proceed from one transition to the next. The
\emph{trace} of a run is the sequence of labels of the underlying transitions:

\begin{definition}\label{defn:ltsRun}
    For an LTS, $(\aLTSAllStates, \aLTSAllLabels, \aLTSTransitionRel)$, a run
    is a string over $\aLTSTransitionRel$:
    \[\sequenceof{
            \aLTSState_0 \LabelledTrans{\aLTSLabel_0} \bLTSState_0,\,
            \aLTSState_1 \LabelledTrans{\aLTSLabel_1} \bLTSState_1,\,
            \dots,\,
            \aLTSState_{n-1} \LabelledTrans{\aLTSLabel_{n-1}} \bLTSState_{n-1}
        }
    \]
    such that $\aLTSState_i = \bLTSState_{i-1}$, for all $1 \le i < n$.
    The trace of a run simply forgets the states, giving a corresponding
    sequence of labels:
    \[
        \sequenceof{\aLTSLabel_0, \aLTSLabel_1, \dots, \aLTSLabel_{n-1}}
    \]
\end{definition}

\begin{example}
    The set of traces of the LTS in \figref{fig:exampleLTS} is the (infinite)
    set:
    \[
        \setof{\sequenceof{}, \sequenceof{a}, \sequenceof{b},
        \sequenceof{b,d,d,\dots,d}, \sequenceof{a,c,c,\dots,c}}
    \]
\end{example}

In this thesis we frequently use LTSs that have pairs of fixed-length binary
strings as labels, which we refer to as \TLTS{}s, or \LTSB{\aN}{\bN}s if $\aN$
and $\bN$ are the binary string lengths.

\begin{definition}
    An \LTSB{\aN}{\bN} is an LTS, $(\aLTSAllStates, \aLTSAllLabels,
    \aLTSTransitionRel)$, where $\aLTSAllLabels \subseteq \B^\aN \times
    \B^\bN$, where we write
    \[
        \lbl{x}{y} \,\text{ as syntactic sugar for }\, (x,y) \in \aLTSAllLabels
    \]
\end{definition}

\begin{example}
    Two example \TLTS{}s are illustrated in \figref{fig:exampleTLTSs}.
\end{example}

\begin{figure}[ht]
    \centering
    \begin{subfigure}{0.5\textwidth}
        \centering
        \begin{tikzpicture}[nfa]
            \node[state] (0)              {$0$};
            \node[state] (1) [below=of 0] {$1$};

            \path (0) edge [loop right, overlay] node {$\lbl{0}{1}$} (0);
            \path (0) edge [overlay]             node {$\lbl{1}{0}$} (1);
        \end{tikzpicture}
    \caption{\LTSB{1}{2}, $\aLTS$}
    \end{subfigure}%
    \begin{subfigure}{0.5\textwidth}
    \centering
        \begin{tikzpicture}[nfa]
            \node[state] (2)              {$2$};
            \node[state] (3) [below=of 2] {$3$};

            \path (2) edge [loop right, overlay] node {$\lbl{1}{0}$} (2);
            \path (2) edge [overlay]             node {$\lbl{0}{1}$} (3);
        \end{tikzpicture}
        \caption{\LTSB{2}{1}, $\bLTS$}
    \end{subfigure}%
\caption{Example \TLTS{}s}
\label{fig:exampleTLTSs}
\end{figure}

As a notational shorthand, we often write \TLTS{} labels containing one or more
`$\ast$' characters, to represent the \emph{set} of transitions where each
`$\ast$' is expanded into a pair of transitions, with a 1 and 0 at that
position in the label. For example, the label $\lbl{\ast1}{\ast}$ corresponds
to the set of labels $\setof{\lbl{01}{0},\lbl{01}{1},\lbl{11}{0},\lbl{11}{1}}$
for a $\LTSB{2}{1}$. When drawing LTSs, we often collapse multiple transitions
between the same source/target states, drawing a single transition with a
\emph{set} of labels, which should be understood as a set of singly-labelled
transitions. These labelling shorthand are illustrated in
\figref{fig:compactExpandedNotation}, where we give two presentations of the
same LTS\@.

\begin{figure}[ht]
    \centering
    \begin{subfigure}{0.5\textwidth}
        \centering
        \begin{tikzpicture}[nfa]
            \node[state] (0)              {$0$};
            \node[state] (1) [below right=of 0] {$1$};
            \node[state] (2) [above right=of 1] {$2$};

            \path (0) edge [swap] node {$\setof{\lbl{0}{1}, \lbl{1}{0}}$} (1);
            \path (1) edge [swap] node {$\lbl{*}{1}$} (2);
        \end{tikzpicture}
    \end{subfigure}%
    \begin{subfigure}{0.5\textwidth}
    \centering
        \begin{tikzpicture}[nfa]
            \node[state] (0)              {$0$};
            \node[state] (1) [below right=of 0] {$1$};
            \node[state] (2) [above right=of 1] {$2$};

            \path (0) edge [pos=0.3, bend left] node {$\lbl{1}{0}$} (1);
            \path (0) edge [swap, bend right] node {$\lbl{0}{1}$} (1);
            \path (1) edge [pos=0.7, bend left] node {$\lbl{0}{1}$} (2);
            \path (1) edge [swap, bend right] node {$\lbl{1}{1}$} (2);
        \end{tikzpicture}
    \end{subfigure}%
\caption{Compact vs Expanded label notation}
\label{fig:compactExpandedNotation}
\end{figure}

When considering LTSs we often want to ignore the particular label and state
names and instead concentrate on the transition connections, allowing us to
identify those LTSs with the same underlying \emph{structure}. We say that LTSs
that can be identified in this way are \emph{isomorphic}.

\subsection{Isomorphism of LTSs}

First, we describe how to map one LTS's transition structure onto another,
known as a \emph{homomorphism} between LTSs\footnote{A more general definition of LTS homomorphism
allows relabelling of transitions. However, we do not (and indeed, can not, since we must preserve
labels to preserve compositions: \defnref{defn:sequentialCompositionTLTS}) use such a
generalisation in this thesis. }:

\begin{definition}
Consider two LTSs, $(\aLTSAllStates_1, \aLTSAllLabels,
\aLTSTransitionRel_1)$ and $(\aLTSAllStates_2, \aLTSAllLabels,
\aLTSTransitionRel_2)$. A mapping, $f$, between the LTSs is a mapping on their states:
$f : \aLTSAllStates_1 \to \aLTSAllStates_2$. A mapping is said to be an
LTS homomorphism if for every $\aLTSState \LabelledTransNum{\aLTSLabel}{1}
\bLTSState$ we have $f(\aLTSState) \LabelledTransNum{\aLTSLabel}{2} f(\bLTSState)$.
\end{definition}

An LTS homomorphism, $f$, is an isomorphism iff $f$ is a bijection and we have\\
$f(\aLTSState) \LabelledTransNum{\aLTSLabel}{2} f(\bLTSState)$ iff we have $\aLTSState
\LabelledTransNum{\aLTSLabel}{1} \bLTSState$, i.e. there is a bijection between the transitions
that respects the action of $f$ on states. We write $\aLTS \LTSIso \bLTS$ when there exists an LTS
isomorphism between $\aLTS$ and $\bLTS$.

\subsection{\TLTS{} Compositions}

We define two composition operations on \TLTS{}s, synchronous and tensor, both
of which are modifications of the cartesian product construction on LTSs. Later
in the thesis, we will use these compositions to give \emph{compositional}
semantics of Petri Nets With Boundaries components.

First, we define the cartesian product construction on LTSs, which takes two
LTSs and generates a new LTS, with the set of states (transitions) being the
cartesian product of the underlying sets of states (transitions).

\begin{definition}[Cartesian Product of LTSs]
Given two LTS, $\aLTS_1$:
\(
    ( \aLTSAllStates_1
    , \aLTSAllLabels_1
    , \aLTSTransitionRel_1
    )
\), and $\aLTS_2$:
\(
    ( \aLTSAllStates_2
    , \aLTSAllLabels_2
    , \aLTSTransitionRel_2
    )
\) their cartesian product is:\\
\(
    ( \aLTSAllStates_1 \times \aLTSAllStates_2
    , \aLTSAllLabels_1 \times \aLTSAllLabels_2
    , \aLTSTransitionRel
    )
\), where
\[
(\aLTSState, \bLTSState) \LabelledTrans{(\sigma_1, \sigma_2)}
(\aLTSState', \bLTSState')
\,\text{ iff }\,
\aLTSState \LabelledTransNum{\sigma_1}{1} \aLTSState' \conjunction
\bLTSState \LabelledTransNum{\sigma_2}{2} \bLTSState'
\]
\end{definition}

Synchronous composition is defined for pairs of \TLTS{}s that share a common
``internal'' label size; it restricts transitions in the product to those that
agree on the shared label:

\begin{definition}[Synchronous Composition]
\label{defn:sequentialCompositionTLTS}
    Given a $\LTSB{\aN}{\bN}$ $\aLTS$ and a $\LTSB{\bN}{\cN}$ $\bLTS$,
    their synchronous composition, written $\aLTS \comp \bLTS$, is the
    $\LTSB{\aN}{\cN}$ $(\aLTSAllStates,
    \aLTSAllLabels, \aLTSTransitionRel\syncRestriction)$, where
    $(\aLTSAllStates, \aLTSAllLabels, \aLTSTransitionRel)$ is the cartesian
    product of $\aLTS$ and $\bLTS$, and $\aLTSTransitionRel\syncRestriction$
    is defined:
    \[
        (\aLTSState, \bLTSState)
        \LabelledTrans{\lbl{\aLTSLabel_1}{\aLTSLabel_4}}_{\syncRestriction}
        (\aLTSState', \bLTSState')
        \iff
        \exists\, \aLTSLabel_2, \aLTSLabel_3 \,.\,
        (\aLTSState, \bLTSState)
        \LabelledTrans{(\lbl{\aLTSLabel_1}{\aLTSLabel_2},\,\,
            \lbl{\aLTSLabel_3}{\aLTSLabel_4})}
        (\aLTSState', \bLTSState')
        \,
        \conjunction
        \,
        \aLTSLabel_2 = \aLTSLabel_3
    \]
    That is, we remove transitions whose labels do not agree on the $\bN$-sized
    internal boundary; furthermore, we hide the shared label component in the
    resulting transition. We sometimes write states of a synchronous
    composition as $\compStates{\aLTSState}{\bLTSState}$ rather than
    $(\aLTSState,\bLTSState)$.
\end{definition}

\begin{remark}\label{rem:exponentiationSyntax}
    For a $\LTSB{\aN}{\aN}$ $\aLTS$, we write $\aLTS^\cN$ for the
    (left-associated) $\cN$-fold synchronous composition $\aLTS \comp \aLTS
    \comp \ldots \comp \aLTS$.
\end{remark}

As we have seen, synchronous composition is only defined when the internal
boundary sizes match up. Tensor composition, however, is defined for any pair
of \TLTS{}s, regardless of their boundary sizes:

\begin{definition}[Tensor Composition]\label{defn:tensorCompositionTLTS}
    Given a $\LTSB{\aN}{\bN}$ $\aLTS$ and a $\LTSB{\cN}{\dN}$ $\bLTS$
    their tensor composition, written $\aLTS \tensor \bLTS$, is a
    $\LTSB{\aN + \cN}{\bN + \dN}$, defined as $(\aLTSAllStates, \aLTSAllLabels,
    \aLTSTransitionRel\tensorRestriction)$, where $(\aLTSAllStates,
    \aLTSAllLabels, \aLTSTransitionRel)$ is their cartesian product and
    $\aLTSTransitionRel\tensorRestriction$ is defined:
    \[
        (\aLTSState, \bLTSState)
        \LabelledTrans{\lbl{\aLTSLabel_1\aLTSLabel_3}{\aLTSLabel_2\aLTSLabel_4}}_{\tensorRestriction}
        (\aLTSState', \bLTSState')
        \iff
        (\aLTSState, \bLTSState)
        \LabelledTrans{(\lbl{\aLTSLabel_1}{\aLTSLabel_2},\,\,
            \lbl{\aLTSLabel_3}{\aLTSLabel_4})}
        (\aLTSState', \bLTSState')
    \]
    In other words, pairwise concatenation of the labels of the cartesian
    product. Again, we sometimes write states of the tensor composition as
    $\tensorStates{\aLTSState}{\bLTSState}$ rather than
    $(\aLTSState,\bLTSState)$.
\end{definition}

\begin{example}
An example of synchronous and tensor composition is given in \figref{fig:exampleTLTScompositions},
where we compose the LTSs $\aLTS$ and $\bLTS$ from \figref{fig:exampleTLTSs}. Note that while we
have drawn $\compStates{0}{3}$ and $\compStates{1}{2}$ in~\figref{fig:ltsCompAB}, in general we
will save space and omit such unconnected states.
\end{example}

\begin{figure}[ht]
    \centering
    \begin{subfigure}{0.5\textwidth}
        \centering
        \begin{tikzpicture}[nfa]
            \node[state] (02)               {$\compStates{0}{2}$};
            \node[state] (03) [below=of 02] {$\compStates{0}{3}$};
            \node[state] (12) [right=of 02] {$\compStates{1}{2}$};
            \node[state] (13) [below=of 12] {$\compStates{1}{3}$};

            \path (02) edge [loop above] node {$\lbl{0}{0}$} (02);
            \path (02) edge node {$\lbl{1}{1}$} (13);
        \end{tikzpicture}
        \caption{\LTSB{1}{1}, $\aLTS \comp \bLTS$}
        \label{fig:ltsCompAB}
    \end{subfigure}%
    \begin{subfigure}{0.5\textwidth}
        \centering
        \begin{tikzpicture}[nfa]
            \node[state] (02)               {$\tensorStates{0}{2}$};
            \node[state] (03) [below=of 02] {$\tensorStates{0}{3}$};
            \node[state] (12) [right=of 02] {$\tensorStates{1}{2}$};
            \node[state] (13) [below=of 12] {$\tensorStates{1}{3}$};

            \path (02) edge [loop above] node {$\lbl{01}{10}$} (02);
            \path (02) edge node {$\lbl{11}{00}$} (12);
            \path (02) edge node {$\lbl{00}{11}$} (03);
            \path (02) edge node {$\lbl{10}{01}$} (13);
        \end{tikzpicture}
        \caption{\LTSB{2}{2}, $\aLTS \tensor \bLTS$}
    \end{subfigure}%
\caption{Example \TLTS{} compositions}
\label{fig:exampleTLTScompositions}
\end{figure}

Given a finite LTS, we can consider only its runs that start and finish in
specified states. To do so, we identify a certain state as being \emph{initial}
and a collection of states as being \emph{accepting} and then only consider
runs of the LTS that start with the initial state and finish with an accepting
state. This, of course, is the familiar concept of a non-deterministic finite
automaton (NFA).
