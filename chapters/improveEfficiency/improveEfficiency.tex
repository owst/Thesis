\chapter{Efficient Compositional Reachability Checking}\label{chp:improveEfficiency}

In the previous chapter we introduced a technique for compositional statespace
generation for systems modelled using PNBs. We proved that it was correct, but
its performance left something to be desired --- indeed, even for reasonably
small examples, the generated intermediate \TNFA{}s were large, hindering
performance of \TNFA{} composition and thus the overall technique.

In this chapter, we describe improvements that can be made to the naive
compositional algorithm to improve its performance, such that it can be used to
efficiently check marking reachability. Briefly, the two key improvements are:
we exploit the fact that (weak) language equivalence is a congruence, to reduce
intermediate \TNFA{}s, and use memoisation to avoid repeated computation.

First, we introduce the notion of \emph{boundary protocol} and internal, or
$\tau$, transitions which must be preserved by any structural reduction
performed on intermediate \TNFA{}s, to ensure correctness of the resulting
performance improvements.

\section{Boundary protocol and $\tau$-transitions} \label{sec:boundaryProtocol}
\subimport{improve}{}

\section{Memoisation, Associativity and Fixed Points}\label{sec:memoisation}
\subimport{associativity}{}

\section{Reassociated Examples}\label{sec:reassociation}
\subimport{tweaks}{}

\section{Optimised Algorithm}
\subimport{timings}{}

\section{Poorly performing Examples}
\subimport{poorexamples}{}

\section{Summary}

In this chapter, we have introduced and evaluated improvements to our algorithm
for compositionally checking reachability in PNB systems, taking advantage of
weak-language equivalence and the fixed-points that can be exploited by
memoisation.  In summary, our approach embodies a simple divide-and-conquer and
memoisation strategy, i.e.\ a bottom-up \emph{dynamic programming} approach.
The component-wise specification is exploited to avoid re-computing local
reachability \TNFA{}s when particular components appear more than once, while
intermediate \TNFA{}s are minimised to reduce composition costs, possible since
\TNFA{} (weak) language-equivalence is a congruence w.r.t.\ the composition
operations.
