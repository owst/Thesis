For a simple marked PNB, the possibly empty collection of firing sequences (and
thus boundary interactions) that will take it from the initial marking to the
desired marking can be readily identified. For example, consider the simple PNB
that will be our running example, illustrated in \figref{fig:examplePNB}; we
can see that the firing sequences are those with either a single firing of
$\aTrans_1$, or a firing of $\aTrans_2$ then $\aTrans_3$ and $\aTrans_4$.
Indeed, we can abstract over the identities of the transitions that must be
fired, and only consider their interactions on the boundary ports, giving those
sequences with either a $\lbl{1}{0}$, or $\lbl{0}{0}$ followed by $\lbl{0}{1}$
and $\lbl{0}{0}$. We refer to this collection as the \emph{boundary protocol}
of the PNB\@. Indeed, the boundary protocol of a (marked) PNB is precisely the
\emph{language} of its \TNFA{} semantics.  Intuitively, the boundary protocol
encompasses all sequences of \emph{external interactions}, that take the
component between its local initial and target markings.

\begin{figure}[ht]
    \centering
    \begin{tikzpicture}[pnb]
        \node[pnbplace] (p0) [rotate=-90, tokens=1] {};
        \node[pnbplace] (p1) [rotate=-90, below right=of p0] {};
        \node[pnbplace] (p2) [rotate=-90, below=of p1] {};
        \node[pnbplace] (p3) [rotate=-90, below left=of p2, wantedTokenPlace] {};

        \drawBoundaries{1}{1}

        \labelledpnbarr[p0p3]{p0.out}{p3.in}{above left:$\aTrans_1$}{relative}{}
        \draw (p0p3) edge[barr, out=-120, in=0] (l1);

        \labelledpnbarr{p0.out}{p1.in}{above right:$\aTrans_2$}{out=-90, in=90}{}
        \labelledpnbarr[p1p2]{p1.out}{p2.in}{above right:$\aTrans_3$}{relative}{}
        \draw (p1p2) edge[barr, out=-60, in=180] (r1);
        \labelledpnbarr{p2.out}{p3.in}{below right:$\aTrans_4$}{out=-90, in=90}{}
    \end{tikzpicture}
    \caption{Example PNB}
    \label{fig:examplePNB}
\end{figure}

Since the boundary protocol of a PNB, $\aPNB$, corresponds to the language of
$\PNBToTNFA{\aPNB}$, we can apply language-preserving modifications to
$\PNBToTNFA{\aPNB}$ to obtain a more compact representation of $\aPNB$'s
boundary protocol. Furthermore, we are able to perform such language-preserving
modifications, whilst preserving compositions of \TNFA{}s:

\begin{theorem}[(Strong) \TNFA{} language equivalence is a
    congruence]\label{thm:langEquivCongruence}

    Suppose that:
    \begin{enumerate}[(i)]
        \item $\aNFA, \aNFA'$ are \NFAB{\aN}{\bN}s, with $\aNFA \langEquiv \aNFA'$,
        \item $\bNFA, \bNFA'$ are \NFAB{\bN}{\cN}s, with $\bNFA \langEquiv \bNFA'$,
        \item $\cNFA, \cNFA'$ are \NFAB{\cN}{\dN}s, with $\cNFA \langEquiv \cNFA'$.
    \end{enumerate}
    Then, the following hold:
\begin{enumerate}[(i)]
    \item \label{langEquivItem1} $\aNFA \comp \bNFA \langEquiv \aNFA' \comp \bNFA'$,
    \item \label{langEquivItem2} $\aNFA \tensor \cNFA \langEquiv \aNFA' \tensor \cNFA'$.
\end{enumerate}
\end{theorem}
\begin{proof}
    For~\ref{langEquivItem1}, suppose that $\word{\lbl{\aLbl}{\bLbl}} \in
    \langOf{\aNFA \comp \bNFA}$. Then, there exists a $\word{\cLbl}$, such that
    $\word{\lbl{\aLbl}{\cLbl}} \in \langOf{\aNFA}$ and
    $\word{\lbl{\cLbl}{\bLbl}} \in \langOf{\bNFA}$. By the asssumptions that
    $\aNFA \langEquiv \aNFA'$ and $\bNFA \langEquiv \bNFA'$, then also
    $\word{\lbl{\aLbl}{\cLbl}} \in \langOf{\aNFA'}$ and
    $\word{\lbl{\cLbl}{\bLbl}} \in \langOf{\bNFA'}$ and thus
    $\word{\lbl{\aLbl}{\bLbl}} \in \langOf{\aNFA' \comp \bNFA'}$, as required.
    For~\ref{langEquivItem2}, suppose that $\word{\lbl{\aLbl\cLbl}{\bLbl\dLbl}}
    \in \langOf{\aNFA \tensor \cNFA}$. Then, we have that
    $\word{\lbl{\aLbl}{\bLbl}} \in \langOf{\aNFA}$ and
    $\word{\lbl{\cLbl}{\dLbl}} \in \langOf{\cNFA}$. By the assumptions, we have
    that $\word{\lbl{\aLbl}{\bLbl}} \in \langOf{\aNFA'}$ and
    $\word{\lbl{\cLbl}{\dLbl}} \in \langOf{\cNFA'}$ and thus
    $\word{\lbl{\aLbl\cLbl}{\bLbl\dLbl}} \in \langOf{\aNFA' \tensor \cNFA'}$,
    as required.
\end{proof}

Observe that \thmref{thm:langEquivCongruence} is the (strong) language
equivalence analogue of \propref{prop:TLTSCatAxioms}\ref{seq-item1} and
\propref{prop:TLTSMonCatProps}\ref{tensor-item1}; indeed, both
language-equivalence and isomorphism are congruences. Intuitively, being a
congruence means that we are free to reduce component \TNFA{}s, whilst ensuring
that we preserve the language of their composition. Thus, to reduce the size of
\TNFA{}s generated by our technique, and therefore improve its performance, we
can replace any \TNFA{} that arises with one that is smaller, yet language
equivalent.

In fact, we can do even better, by considering the boundary protocol (and thus
\TNFA{} semantics) only \emph{up-to internal behaviour}. By ignoring internal
computations, we can reduce the size of \TNFA{}s even further and indeed, in
some cases, reach fixed-points of behaviour w.r.t. composition, leading to
\emph{linear} time complexity.  First, we introduce $\tau$-transitions as the
representation of internal behaviour that we wish to ignore.

\subsection{$\tau$-transitions in \TNFA{} semantics}

Recall that every state in the \NFAB{\aN}{\bN} semantics of a PNB, $\aPNB
\withNetType{\aN}{\bN}$ has a self-loop labelled with $\lbl{0^\aN}{0^\bN}$,
since $\emptyset \subseteq \trans\aPNB$ can always be fired with no effect on
the boundaries. In general, such-labelled \TNFA-transitions need not be
self-loops and we refer to them as
$\tauLabel{\aN}{\bN}$-transitions (recall that we drop the subscripts when $\aN$
and $\bN$ are clear from the context). Indeed, firing any (set of) net-transitions
that do not connect to the left or right boundaries will generate
$\tau$-transitions. One particular source of such net-transitions is
composition of PNB components with transitions that connect only to
\emph{shared} boundary ports.

Our semantic model does not distinguish between those $\tau$-transitions
originating from firing the empty set of transitions and those from firing a
set of transitions unconnected to the boundaries. In other words, we abstract
over internal behaviour in that ``no behaviour'' and ``no externally observable
behaviour'' are represented the same. It is precisely this abstraction that
allows us to reduce the generated statespace: statespace that cannot be
distinguished by observable interactions can simply be forgotten.

To describe the behaviour that we wish to preserve, we consider a notion of
boundary protocol that only incorporates observable boundary interactions. We
say that the \emph{weak boundary protocol} of a PNB is its boundary protocol
with all $\tau$ labels removed. The removal of $\tau$ labels is realised as the
unique monoid homomorphism:
\[
    \stripTau{-} :
\freeMonoidOn{\parens{\B^\aN \times \B^\bN}} \to \freeMonoidOn{\parens{\B^\aN \times
\B^\bN \setminus \tauLabel{\aN}{\bN}}}
\]
defined on individual elements, $e$, of $\B^\aN \times \B^\bN$ as follows:
\[
    \stripTau{e} = \begin{cases} \freeMonoidUnit \text{\ if } e =
        \tauLabel{\aN}{\bN}\\
        e \text {\ otherwise}
    \end{cases}
\]

\begin{definition}
    Given a \NFAB{\aN}{\bN}, $\aNFA$, the weak language of $\aNFA$, written
    $\weakLangOf{\aNFA}$, is:
    \[
        \weakLangOf{\aNFA} \defeq \setBuilder{\stripTau{x}}{x \in \langOf{\aNFA}}
    \]
\end{definition}

\begin{definition}
    For \NFAB{\aN}{\bN}s, $\aNFA$ and $\bNFA$, we say $\aNFA$ and $\bNFA$ are
    weak language-equivalent, written $\aNFA \weakLangEquiv \bNFA$ if:
    \[
    \weakLangOf{\aNFA} = \weakLangOf{\bNFA}
    \]
\end{definition}

As an example, consider the \TNFA{} semantics of the simple PNB in
\figref{fig:examplePNB}, which illustrated in \figref{fig:examplePNBTNFA}.

\begin{figure}[ht]
    \centering
    \begin{tikzpicture}[nfa]
        \node[state] (0) [init]                       {$0$};
        \node[state] (1) [below right=of 0]           {$1$};
        \node[state] (2) [below=of 1]                 {$2$};
        \node[state] (3) [accepting, below left=of 2] {$3$};

        \draw (0) edge [loop left]  node {$\lbl{0}{0}$} (0);
        \draw (0) edge              node {$\lbl{1}{0}$} (3);
        \draw (0) edge              node {$\lbl{0}{0}$} (1);
        \draw (1) edge [loop right] node {$\lbl{0}{0}$} (1);
        \draw (1) edge              node {$\lbl{0}{1}$} (2);
        \draw (2) edge [loop right] node {$\lbl{0}{0}$} (2);
        \draw (2) edge              node {$\lbl{0}{0}$} (3);
        \draw (3) edge [loop left]  node {$\lbl{0}{0}$} (3);
    \end{tikzpicture}
    \caption{\NFAB{1}{1} semantics, $\aNFA$, of the PNB in \figref{fig:examplePNB}}
    \label{fig:examplePNBTNFA}
\end{figure}

The language of this \TNFA{} is\footnote{Using a regular expression notation.}:
$\parens{\lbl{0}{0}}^\ast\parens{\lbl{1}{0}\growMid\lbl{0}{0}\parens{\lbl{0}{0}}^\ast\lbl{0}{1}\parens{\lbl{0}{0}}^\ast\lbl{0}{0}}\parens{\lbl{0}{0}}^\ast$,
whereas the weak language is much simpler:
$\parens{\lbl{1}{0}\growMid\lbl{0}{1}}$. A simple observation is that $\aNFA
\weakLangEquiv \bNFA$, where $\bNFA$ is illustrated in
\figref{fig:exampleTNFAWeakEquiv}.

\begin{figure}[ht]
    \centering
    \begin{tikzpicture}[nfa]
        \node[state] (0) [init]                       {$0$};
        \node[state] (1) [below=of 0]                 {$1$};
        \node[state] (2) [below left=of 1]            {$2$};
        \node[state] (3) [below right=of 1]           {$3$};
        \node[state] (4) [accepting, below left=of 3] {$4$};

        \draw (0) edge              node {$\lbl{0}{0}$} (1);
        \draw (1) edge [swap]       node {$\lbl{0}{1}$} (2);
        \draw (1) edge              node {$\lbl{1}{0}$} (3);
        \draw (2) edge [swap]       node {$\lbl{0}{0}$} (4);
        \draw (3) edge              node {$\lbl{0}{0}$} (4);
    \end{tikzpicture}
    \caption{$\bNFA$, weakly-equivalent to $\aNFA$ shown in
    \figref{fig:examplePNBTNFA}}
    \label{fig:exampleTNFAWeakEquiv}
\end{figure}

Now, we have shown how we can consider equivalence up-to ``internal behaviour''
of \TNFA{}s using weak-language equivalence; however, to improve the
performance of \TNFA{} composition (and thus our compositional technique), we
must reduce the size of component \TNFA{}s, by modifying their structure, yet
preserving \emph{weak language} to ensure correctness.

The first step of this procedure is to ignore $\tau$-transitions, by
\emph{closing} the \TNFA{} w.r.t.\ them, which requires the notion of the
\tauClosure{} of a single state. \tauClosure{} is closely related to
$\epsilon$-closure from automata theory, as we discuss in
\remref{rem:tautrans}. The \tauClosure{} of a single state is the set of states
that can be reached by taking zero or more transitions labelled by $\tau$:

\begin{definition}[State \tauClosure{}]
    For a state, $\aNFAState$, of a \NFAB{\aN}{\bN}, $( \aNFAAllStates
                                                      , \aNFAAllLabels
                                                      , \aNFATransitionRel
                                                      , \aNFAInitState
                                                      , \aNFAAcceptStates
                                                      )$, we define its
    \emph{\tauClosure{}}, written $\tauCloseOp{\aNFAState}$ as the least fixed-point of the
    following recursive definition:
    \[
        \tauCloseOp{\aNFAState} = \setof{\aNFAState} \union
        \bigUnion{\setBuilder{\tauCloseOp{\bNFAState}}{\aNFAState
        \LabelledTrans{\tau_{\pairof{\aN}{\bN}}} \bNFAState}}
    \]
\end{definition}

Given the \tauClosure{} of a single state, we can define the \tauClosure{} of an entire \TNFA{}.
Intuitively, the \tauClosure{} of a \TNFA{}, $\aNFA$, is formed by ``saturating'' $\aNFA$ with
transitions that remove $\tau$s from traces:
\begin{enumerate}
    \item Any state whose \tauClosure{} contains an accepting state is also accepting, allowing us
        to ignore trailing $\tau$s in any trace,
    \item The transition relation has new entries corresponding to the pre-closure of the
        transition relation w.r.t.\ \tauClosure{}, allowing us to ignore leading $\tau$s in any
        trace.
\end{enumerate}

Precisely, the definition is as follows:

\begin{definition}[\TNFA{} \tauClosure{}]
    For a \NFAB{\aN}{\bN}, $\aNFA$, formed of $( \aNFAAllStates
                            , \aNFAAllLabels
                            , \aNFATransitionRel
                            , \aNFAInitState
                            , \aNFAAcceptStates
                            )$, its \tauClosure{}, $\tauCloseOp{\aNFA}$, is:
    \[
     ( \aNFAAllStates
     , \aNFAAllLabels
     , \aNFATransitionRel \union \aNFATransitionRel'
     , \aNFAInitState
     , \aNFAAcceptStates \union \aNFAAcceptStates'
     )
    \] where:
     \begin{align*}
         \aNFAAcceptStates' &= \setBuilder{\aNFAState}{\aNFAState \in
            \aNFAAllStates, \tauCloseOp{\aNFAState} \intersection
            \aNFAAcceptStates \neq \emptyset}\\
         \aNFATransitionRel' &=
             \setBuilder{(\aNFAState, \aNFALabel, \bNFAState)}{\exists
                 \aNFAState' \in \tauCloseOp{\aNFAState}
                 \text{ s.t. }
                 \aNFAState' \LabelledTrans{\aNFALabel} \bNFAState}
     \end{align*}
\end{definition}

In general, it is obvious that \tauClosure{} changes the strong language of a
\TNFA{}:

\begin{lemma}[Strong language is not preserved by \tauClosure{}]
    For a \TNFA{}, $\aNFA$, it is not necessarily the case that:
    $\aNFA \langEquiv \tauCloseOp{\aNFA}$.
\end{lemma}
\begin{proof}
    For an example, consider again the \NFAB{1}{1}, $\aNFA$, in
    \figref{fig:examplePNBTNFA} and its \tauClosure{}, $\tauCloseOp{\aNFA}$, in
    \figref{fig:closedTNFA}. The languages of these \TNFA{} are clearly
    different; for example, the word $\sequenceof{\lbl{0}{1}}$ is in the
    language of $\tauCloseOp{\aNFA}$, but not of $\aNFA$, therefore, $\aNFA
    \not\langEquiv \tauCloseOp{\aNFA}$.
\end{proof}

\begin{figure}[ht]
    \centering
    \begin{tikzpicture}[nfa]
        \node[state] (0) [init]                       {$0$};
        \node[state] (1) [below right=of 0]           {$1$};
        \node[state] (2) [accepting, below=of 1]      {$2$};
        \node[state] (3) [accepting, below left=of 2] {$3$};

        \draw (0) edge [loop left]  node {$\lbl{0}{0}$} (0);
        \draw (0) edge [swap]       node {$\lbl{1}{0}$} (3);
        \draw (0) edge              node {$\lbl{0}{0}$} (1);
        \draw (0) edge [out=0, in=30, looseness=1.75] node {$\lbl{0}{1}$} (2);
        \draw (1) edge [loop right] node {$\lbl{0}{0}$} (1);
        \draw (1) edge              node {$\lbl{0}{1}$} (2);
        \draw (2) edge [loop right] node {$\lbl{0}{0}$} (2);
        \draw (2) edge              node {$\lbl{0}{0}$} (3);
        \draw (3) edge [loop left]  node {$\lbl{0}{0}$} (3);
    \end{tikzpicture}
    \caption{\NFAB{1}{1} of \figref{fig:examplePNBTNFA} after \tauClosure{}}
    \label{fig:closedTNFA}
\end{figure}

Indeed, there are certain cases where strong language is \emph{not} affected
(e.g.\ a \TNFA{} that contains no $\tau$-transitions, or has no accepting
states), but we are not concerned with such \TNFA{}; we therefore only prove
that weak language \emph{is} preserved.

First, before proving that weak language is preserved by \tauClosure{}, we
prove a technical lemma. By the construction of $\tauCloseOp{\aNFA}$, we have
that the only additional words in $\langOf{\tauCloseOp{\aNFA}}$, relative to
$\langOf{\aNFA}$, are words of $\aNFA$ with $\tau$ labels removed:

\begin{lemma}\label{lem:removeTaus}
    Let $\aNFA = ( \aNFAAllStates
                 , \aNFAAllLabels
                 , \aNFATransitionRel
                 , \aNFAInitState
                 , \aNFAAcceptStates
                 )$ and $\bNFA = \tauCloseOp{\aNFA} =
                    ( \aNFAAllStates
                    , \aNFAAllLabels
                    , \aNFATransitionRel_\tau
                    , \aNFAInitState
                    , \aNFAAcceptStates'
                    )$.
For any $\aNFAWord \in \langOf{\bNFA}$, with $\cardinalityof{\aNFAWord} = \aN$,
such that $\aNFAWord \not\in \langOf{\aNFA}$, we can construct a word
$\aNFAWord' \in \langOf{\aNFA}$ by inserting zero or more $\tau$ labels into
$\aNFAWord$.
\end{lemma}
\begin{proof}
By definition, $\aNFAWord \in \langOf{\bNFA}$ implies there are states
$\aNFAState_0, \aNFAState_1, \dots, \aNFAState_\aN \in \aNFAAllStates$, such
that there are transitions $\aNFAState_{i - 1}
\LabelledTrans{\aNFAWord_i}_{\tau} \aNFAState_i, \text{for } 1
\le i \le \aN$, with $\aNFAState_0 = \aNFAInitState$, and $\aNFAState_\aN \in
\aNFAAcceptStates'$. Indeed, by the definition of \tauClosure{}, each
\emph{single} transition of $\bNFA$, $\aNFAState_{i - 1}
\LabelledTrans{\aNFAWord_i}_{\tau} \aNFAState_i, \text{for } 1
\le i \le \aN$ is either a transition of $\aNFA$, or corresponds to a (not
necessarily unique) \emph{sequence} of one or more transitions of $\aNFA$
(between the same pair of states: $\aNFAState_{i - 1}$ and $\aNFAState_i$),
with the final transition of the sequence being labelled with $\aNFAWord_i$,
and \emph{every other} transition by $\tau$. Concatenating these (sequences of)
transitions gives a sequence of transitions from $\aNFAState_0$ (the initial
state), to $\aNFAState_\aN$ (an accepting state in $\bNFA$). If $\aNFAState_\aN
\in \aNFAAcceptStates$, we are done, otherwise, by the definition of
$\aNFAAcceptStates'$, there exists a sequence of $\tau$ transitions from
$\aNFAState_\aN$, that reach some $\aNFAState_\eN \in \aNFAAcceptStates$;
appending this final sequence of $\tau$ transitions, gives the required
$\aNFAWord' \in \langOf{\aNFA}$.
\end{proof}

A graphical presentation of the idea of the proof of \lemref{lem:removeTaus} is
given in \figref{fig:wordExpansion}, the transitions forming word $\aNFAWord
\in \langOf{\bNFA}$ are highlighted in blue. For each such transition, either
it is also a transition of $\aNFA$, or there will exist a sequence of
transitions in $\aNFA$, between the same start and end states.

\begin{figure}[ht]
    \centering
    \scalebox{0.75}{%
    \begin{tikzpicture}[nfa, node distance=0.75cm]
        \node (0) [state] {$\aNFAState_0$};

        \node (00) [state, below left=of 0]  {$0_0$};
        \node (01) [state, draw=none, below left=of 00] {\rotatebox{-90}{$\cdots$}};
        \node (02) [state, below right=of 01] {$0_\bN$};

        \draw (0)  edge[swap] node {$\tau$} (00);
        \draw (00) edge[swap] node {$\tau$} (01);
        \draw (01) edge[swap] node {$\tau$} (02);

        \node (1) [state, below right=of 02] {$\aNFAState_1$};

        \draw (02) edge [swap] node {$\aNFAWord_1$} (1);
        \draw (0)  edge [blue] node {$\aNFAWord_1$} (1);

        \node (10) [state, below left=of 1]       {$1_0$};
        \node (11) [state, draw=none, below left=of 10] {\rotatebox{-90}{$\cdots$}};
        \node (12) [state, below right=of 11]      {$1_\cN$};

        \draw (1)  edge[swap] node {$\tau$} (10);
        \draw (10) edge[swap] node {$\tau$} (11);
        \draw (11) edge[swap] node {$\tau$} (12);

        \node (2) [state, below right=of 12] {$\aNFAState_2$};

        \draw (12) edge [swap] node {$\aNFAWord_2$} (2);
        \draw (1)  edge [blue] node {$\aNFAWord_2$} (2);

        \node (20) [below left=of 2, rotate=45]  {$\dots$};
        \node (21) [below left=of 20] {};
        \node (22) [below right=of 21, rotate=-45] {$\dots$};

        \node (f) [state, below right=of 22] {$\aNFAState_{\aN - 1}$};

        \node (d) [rotate=-90] at ($(2)!.5!(f)$) {$\dots$};

        \node (f0) [below left=of f, rotate=45]  {$\dots$};
        \node (f1) [below left=of f0] {};
        \node (f2) [below right=of f1, rotate=-45] {$\dots$};

        \node (ff) [state, below right=of f2] {$\aNFAState_\aN$};
        \draw (f2) edge [swap] node {$\aNFAWord_\aN$} (ff);
        \draw (f)  edge [blue] node {$\aNFAWord_\aN$} (ff);

        \node (ff0) [state, below left=of ff]        {$\aN_0$};
        \node (ff1) [state, draw=none, below left=of ff0] {\rotatebox{-90}{$\cdots$}};
        \node (ff2) [state, below right=of ff1, accepting]      {$\aN_\dN$};

        \draw (ff)  edge[swap] node {$\tau$} (ff0);
        \draw (ff0) edge[swap] node {$\tau$} (ff1);
        \draw (ff1) edge[swap] node {$\tau$} (ff2);
    \end{tikzpicture}
}
    \caption{Expansion of a word using only additional $\tau$ labels}
    \label{fig:wordExpansion}
\end{figure}

Now we prove that weak language \emph{is} preserved by \tauClosure{}:

\begin{lemma}[Weak language is preserved by \tauClosure{}]
    For any \TNFA{}, $\aNFA$, we have: $\tauCloseOp{\aNFA} \weakLangEquiv
    \aNFA$.
\end{lemma}
\begin{proof}
By the definition of weak language equivalence, we must have the following:
\[
    \setBuilder{\stripTau{\aNFAWord}}{\aNFAWord \in \langOf{\aNFA}}
    =
    \setBuilder{\stripTau{\bNFAWord}}{\bNFAWord \in \langOf{\tauCloseOp{\aNFA}}}
\]
By construction, $\aNFAWord \in \langOf{\aNFA} \implies \aNFAWord \in
\langOf{\tauCloseOp{\aNFA}}$ since we have not \emph{removed} transitions or
initial/accepting states. The converse does not hold; however, for any
$\aNFAWord \in \langOf{\tauCloseOp{\aNFA}}$ such that $\aNFAWord \not\in
\langOf{\aNFA}$, then by \lemref{lem:removeTaus}, there exists a $\bNFAWord \in
\langOf{\aNFA}$, such $\aNFAWord$ and $\bNFAWord$ only differ in $\tau$s;
however, since weak language equivalence removes \emph{all} $\tau$ labels from
words, $\aNFAWord$ and $\bNFAWord$ will be considered equal, as required.
\end{proof}

\begin{remark}\label{rem:tautrans}
The reader may observe that $\tau$ transitions and \tauClosure{} are similar to
$\epsilon$-moves and closure of $\epsilon$-NFA in Automata theory. Indeed, they
are very similar: our lemma regarding weak-language preservation by
\tauClosure{} is the analogue of the theorem stating that $\epsilon$-closed
$\epsilon$-NFAs recognises the same language as the original. However, there
are subtle differences: $\epsilon$ is not considered as part of the alphabet of
a $\epsilon$-NFA and thus doesn't consume any of the input string when taking
the transition. Furthermore, $\epsilon$-closure removes all transitions labels
from the $\epsilon$-NFA, whereas our \tauClosure{} simply supplements the
existing transitions and accepting states. Indeed, we could have used the
standard definition of $\epsilon$-closure, making an alternative design
decision: the theorems required to prove correctness would be slightly altered
and indeed, we would need to insert ``artificial'' transitions to ensure the
important property of \emph{reflexivity} (discussed later, in
\defnref{defn:reflexiveTNFA}) was preserved, but essentially the choice is
inconsequential.
\end{remark}

We want to be sure that weak language equivalence is also a congruence w.r.t.\
\TNFA{} compositions. However, it is \emph{not} in general, which we
demonstrate by way of an example: consider the \TNFA{}s, $\aNFA$, shown in
\figref{fig:weakEquivTNFAA}, $\bNFA$, shown in \figref{fig:weakEquivTNFAB}, and
$\bNFA'$, in \figref{fig:weakEquivTNFABPrime}. It is easy to verify that $\bNFA
\weakLangEquiv \bNFA'$, thus if weak language equivalence was a congruence, we
should have $\aNFA \comp \bNFA \weakLangEquiv \aNFA \comp \bNFA'$. This does
not hold: the LHS, $\aNFA \comp \bNFA$, is illustrated in
\figref{fig:weakEquivTNFACompAB}, whilst the RHS is illustrated in
\figref{fig:weakEquivTNFACompABPrime}; these two compositions are not weak
language equivalent, since the latter's language is empty.

\begin{figure}[ht]
    \centering
    \begin{tikzpicture}[nfa]
        \node[state, init]                  (0) {$0$};
        \node[state, below=of 0]            (1) {$1$};
        \node[state, below=of 1]            (2) {$2$};
        \node[state, below=of 2, accepting] (3) {$3$};

        \path (0) edge node {$\lbl{0}{0}$} (1);
        \path (0) edge [loop left] node {$\lbl{0}{0}$} (0);
        \path (1) edge node {$\lbl{0}{1}$} (2);
        \path (1) edge [loop left] node {$\lbl{0}{0}$} (1);
        \path (2) edge node {$\lbl{0}{0}$} (3);
        \path (2) edge [loop left] node {$\lbl{0}{0}$} (2);
        \path (3) edge [loop left] node {$\lbl{0}{0}$} (3);
    \end{tikzpicture}
    \caption{\TNFA{}, $\aNFA$}
    \label{fig:weakEquivTNFAA}
\end{figure}

\begin{figure}[ht]
    \centering
    \begin{subfigure}{0.33\textwidth}
        \centering
        \begin{tikzpicture}[nfa]
            \node[state, init]                  (0) {$0$};
            \node[state, below=of 0]            (1) {$1$};
            \node[state, below=of 1]            (2) {$2$};
            \node[state, below=of 2, accepting] (3) {$3$};

            \path (0) edge [loop right] node {$\lbl{0}{0}$} (0);
            \path (0) edge node {$\lbl{0}{0}$} (1);
            \path (1) edge node {$\lbl{1}{0}$} (2);
            \path (1) edge [loop right] node {$\lbl{0}{0}$} (1);
            \path (2) edge node {$\lbl{0}{0}$} (3);
            \path (2) edge [loop right] node {$\lbl{0}{0}$} (2);
            \path (3) edge [loop right] node {$\lbl{0}{0}$} (2);
        \end{tikzpicture}
        \caption{\TNFA{}, $\bNFA$}
        \label{fig:weakEquivTNFAB}
    \end{subfigure}%
    \begin{subfigure}{0.33\textwidth}
        \centering
        \begin{tikzpicture}
            \node {$\weakLangEquiv$};
        \end{tikzpicture}
    \end{subfigure}%
    \begin{subfigure}{0.33\textwidth}
        \centering
        \begin{tikzpicture}[nfa]
            \node[state, init]                  (0) {$0$};
            \node[state, below=of 0, accepting] (1) {$1$};

            \path (0) edge node {$\lbl{1}{0}$} (1);
        \end{tikzpicture}
        \caption{\TNFA{}, $\bNFA'$}
        \label{fig:weakEquivTNFABPrime}
    \end{subfigure}%
    \caption{Component \TNFA{} that are weak language equivalent}
\end{figure}

\begin{figure}[ht]
    \centering
    \begin{subfigure}{0.55\textwidth}
        \centering
        \begin{tikzpicture}[nfa]
            \node[state, init]                        (00) {$\parens{0,0}$};
            \node[state, below left=of 00]            (01) {$\parens{0,1}$};
            \node[state, below right=of 00]           (10) {$\parens{1,0}$};
            \node[state, below left=of 10]            (11) {$\parens{1,1}$};
            \node[state, below=of 11]                 (22) {$\parens{2,2}$};
            \node[state, below left=of 22]            (23) {$\parens{2,3}$};
            \node[state, below right=of 22]           (32) {$\parens{3,2}$};
            \node[state, below left=of 32, accepting] (33) {$\parens{3,3}$};

            \foreach \rightlooper in {10,32}{%
                \path (\rightlooper) edge [loop right] node {$\lbl{0}{0}$} (\rightlooper);
            }

            \foreach \leftlooper in {00,01,11,22,23,33}{%
                \path (\leftlooper) edge [loop left] node {$\lbl{0}{0}$} (\leftlooper);
            }

            \path (11) edge node {$\lbl{1}{1}$} (22);

            \foreach \from/\to in {00/11, 00/10, 10/11, 22/33, 22/32, 32/33}{%
                \path (\from) edge node {$\lbl{0}{0}$} (\to);
            }
            \foreach \from/\to in {00/01, 01/11, 22/23, 23/33}{%
                \path (\from) edge [swap] node {$\lbl{0}{0}$} (\to);
            }
        \end{tikzpicture}
        \caption{\TNFA{}, $\aNFA \comp \bNFA$}
        \label{fig:weakEquivTNFACompAB}
    \end{subfigure}%
    \begin{subfigure}{0.15\textwidth}
        \centering
        \begin{tikzpicture}
            \node {$\not\weakLangEquiv$};
        \end{tikzpicture}
    \end{subfigure}%
    \begin{subfigure}{0.3\textwidth}
        \centering
        \begin{tikzpicture}[nfa]
            \node[state, init]                   (00) {$\parens{0,0}$};
        \end{tikzpicture}
        \caption{\TNFA{}, $\aNFA \comp \bNFA'$}
        \label{fig:weakEquivTNFACompABPrime}
    \end{subfigure}%
    \caption{Composed \TNFA{} that are not weak-language equivalent}
\end{figure}

The failure of compositionality is due to the missing $\tau$ transitions in
$\bNFA'$, whereby the $\lbl{0}{0}$ transitions in $\aNFA$ cannot ``synchronise''
with any transitions in $\bNFA'$ and thus there can be no transitions in the
composition. Indeed, since $\tau$ transitions represent ``internal'' behaviour,
their presence allows either component to perform internal behaviour, whilst
the other component ``does nothing'' (similarly represented by a $\tau$
transition), before synchronising on transitions that do have an effect on the
common boundary. Indeed, weak language equivalence is too coarse - it equates
\TNFA{}s that are not equivalent in the sense of their compatibility with a
common (composition) context. It is important to note that \tauClosure{} should
be thought of as abstracting over internal behaviour, rather than
``do-nothing'' behaviour, which is also represented by $\tau$-transitions.
Since weak-language equivalence ignores \emph{any} $\tau$-transition, whether
representing internal or do-nothing behaviour, we must ensure that do-nothing
behaviour of every state is always preserved to ensure compositionality.

\subsection{Reflexivity and Compositionality}

If were to ensure that all \TNFA{}s could perform a $\tau$ transition in
\emph{each} state, the example of the previous subsection would not fail, and
then indeed, weak language equivalence would be a congruence. We call \TNFA{}s that
can perform a $\tau$-transition in each state \emph{reflexive},
recognising that in each state, there is a self-loop, labelled by $\tau$:

\begin{definition}[Reflexive \TNFA{}]\label{defn:reflexiveTNFA}
    A $\NFAB{\aN}{\bN}$,
    $( \aNFAAllStates
     , \aNFAAllLabels
     , \aNFATransitionRel
     , \aNFAInitState
     , \aNFAAcceptStates
     )$ is said to be reflexive, if: $\forall \aNFAState \in \aNFAAllStates,
     \aNFAState \LabelledTrans{\tauLabel{\aN}{\bN}} \aNFAState$.
\end{definition}

Ensuring reflexivity ensures that the $\tau$ transitions that correspond to
``do nothing'' behaviour are preserved. Indeed, composing $\bNFA$ with the
reflexive \TNFA{}, $\aNFA''$, illustrated in
\figref{fig:weakEquivTNFABPrimePrime}, such that $\aNFA \weakLangEquiv
\aNFA''$, gives the required, weak-language equivalent composition:
$\aNFA \comp \bNFA \weakLangEquiv \aNFA \comp \bNFA''$, as illustrated in
\figref{fig:weakEquivTNFACompABPrimePrime}.

\begin{figure}[ht]
    \centering
    \begin{tikzpicture}[nfa]
        \node[state, init]                  (0) {$0$};
        \node[state, below=of 0, accepting] (1) {$1$};

        \path (0) edge [loop right] node {$\lbl{0}{0}$} (0);
        \path (0) edge node {$\lbl{1}{0}$} (1);
        \path (1) edge [loop right] node {$\lbl{0}{0}$} (1);
    \end{tikzpicture}
    \caption{\TNFA{}, $\bNFA''$}
    \label{fig:weakEquivTNFABPrimePrime}
\end{figure}

\begin{figure}
    \centering
    \begin{tikzpicture}[nfa]
        \node[state, init]                   (00) {$\parens{0,0}$};
        \node[state, below=of 00]            (10) {$\parens{1,0}$};
        \node[state, below=of 10]            (21) {$\parens{2,1}$};
        \node[state, accepting, below=of 21] (31) {$\parens{3,1}$};

        \foreach \looper in {00,10,21,31}{%
            \path (\looper) edge [loop right] node {$\lbl{0}{0}$} (\looper);
        }

        \path (00) edge node {$\lbl{0}{0}$} (10);
        \path (10) edge node {$\lbl{1}{1}$} (21);
        \path (21) edge node {$\lbl{0}{0}$} (31);
    \end{tikzpicture}
    \caption{\TNFA{}, $\aNFA \comp \bNFA'$}
    \label{fig:weakEquivTNFACompABPrimePrime}
\end{figure}

Indeed, weak language-equivalence \emph{is} a congruence w.r.t. \TNFA{}
compositions, if the \TNFA{}s are reflexive.

First, we prove two technical lemmas; the first says that we can place
arbitrarily many $\tau$ labels inside an accepted word of a reflexive \TNFA{}
and obtain another accepted word:

\begin{lemma} \label{lem:padReflexiveTNFA}
    Suppose $\aNFA$ is a reflexive \TNFA{} and that $\aNFAWord \in
    \langOf{\aNFA}$. We can obtain another word, $\aNFAWord' \in
    \langOf{\aNFA}$, that results from inserting finitely-many $\tau$ labels
    into $\aNFAWord$.
\end{lemma}
\begin{proof}
    Since $\aNFAWord \in \langOf{\aNFA}$, any prefix of $\aNFAWord$
    corresponds to a state of $\aNFA$, at which, since $\aNFA$ is reflexive,
    we are able to take a $\tau$-labelled transition (multiple times if
    necessary) and remain in the same state. Thus, at any prefix of (i.e.
    position within) $\aNFAWord$ we are able to insert finitely-many $\tau$
    transitions and finish in the same accepting state.
\end{proof}

The second says that, given two \emph{weak}-language-equivalent, reflexive
\TNFA{}s, and a word, $\aNFAWord$, in the language of one, we can find a word
in the intersection of the two \TNFA{}'s languages that is weakly-equivalent to
$\aNFAWord$. Since we know that the languages are equal when $\tau$ labels are
disregarded, when given a word in the \emph{language} of one, we can remove
existing $\tau$ labels and insert $\tau$ labels in (potentially) different
positions to obtain a word in the second \TNFA{}. Then we can pad both words
with $\tau$ labels, to obtain a third word in the intersection of the \TNFA{}'s
languages.

\begin{lemma} \label{lem:wordInIntersectionOfWeakEquivTNFAs}
    Suppose $\aNFA, \aNFA'$ are reflexive \TNFA{}s, such that $\aNFA
    \weakLangEquiv \aNFA'$ and $\aNFAWord \in \langOf{\aNFA}$. Then, there
    exists a $\bNFAWord \in \langOf{\aNFA} \intersection \langOf{\aNFA'}$, with
    $\stripTau{\aNFAWord} = \stripTau{\bNFAWord}$.
\end{lemma}
\begin{proof}
    By our assumption on $\aNFAWord$, we have that  $\stripTau{\aNFAWord} \in
    \weakLangOf{\aNFA}$, and furthermore, by our assumption that $\aNFA
    \weakLangEquiv \aNFA'$, $\stripTau{\aNFAWord} \in \weakLangOf{\aNFA'}$.
    Then, there must exist a $\aNFAWord' \in \langOf{\aNFA'}$ such that
    $\stripTau{\aNFAWord} = \stripTau{\aNFAWord'}$, i.e.\ $\aNFAWord$ and
    $\aNFAWord'$ are equal when disregarding $\tau$ labels.
    Then, since $\aNFA$ and $\aNFA'$ are both reflexive, by
    \lemref{lem:padReflexiveTNFA}, we can pad $\aNFAWord$ and $\aNFAWord'$ with
    $\tau$, to obtain a common $\bNFAWord \in \langOf{\aNFA} \intersection
    \langOf{\aNFA'}$, with $\stripTau{\aNFAWord} = \stripTau{\aNFAWord'} =
    \stripTau{\bNFAWord}$.
\end{proof}

\begin{theorem}[Weak \TNFA{} language equivalence is a
    congruence for reflexive \TNFA{}s]\label{thm:weakLangEquivCongruence}

    Suppose that:
    \begin{enumerate}[(i)]
        \item $\aNFA, \aNFA'$ are reflexive \NFAB{\aN}{\bN}s, with $\aNFA \weakLangEquiv \aNFA'$,
        \item $\bNFA, \bNFA'$ are reflexive \NFAB{\bN}{\cN}s, with $\bNFA \weakLangEquiv \bNFA'$,
        \item $\cNFA, \cNFA'$ are reflexive \NFAB{\cN}{\dN}s, with $\cNFA \weakLangEquiv \cNFA'$.
    \end{enumerate}
    Then, the following hold:
\begin{enumerate}[(i)]
    \item \label{weakLangEquivItem1} $\aNFA \comp \bNFA \weakLangEquiv \aNFA'
    \comp \bNFA'$, \item \label{weakLangEquivItem2} $\aNFA \tensor \cNFA
        \weakLangEquiv \aNFA' \tensor \cNFA'$.
\end{enumerate}
\end{theorem}
\begin{proof}
    For~\ref{weakLangEquivItem1}, suppose that $\aNFAWord \in
    \weakLangOf{\aNFA \comp \bNFA}$, then, by the definition of weak language,
    there exists $\word{\lbl{\aLbl}{\bLbl}} \in \langOf{\aNFA \comp \bNFA}$,
    such that $\stripTau{\word{\lbl{\aLbl}{\bLbl}}} = \aNFAWord$.
    Thus, there exists $\cLbl$ such that $\word{\lbl{\aLbl}{\cLbl}} \in
    \langOf{\aNFA}$ and $\word{\lbl{\cLbl}{\bLbl}} \in \langOf{\bNFA}$.

    Now, since $\aNFA, \aNFA', \bNFA$ and $\bNFA'$ are reflexive \TNFA{}s, with
    $\aNFA \weakLangEquiv \aNFA'$ and $\bNFA \weakLangEquiv \bNFA'$, by
    \lemref{lem:wordInIntersectionOfWeakEquivTNFAs}, we have:
    \begin{enumerate}
        \item $\word{\lbl{\aLbl'}{\cLbl'}} \in \langOf{\aNFA} \intersection
            \langOf{\aNFA'}$ with $\stripTau{\word{\lbl{\aLbl'}{\cLbl'}}} =
            \stripTau{\word{\lbl{\aLbl}{\cLbl}}}$, and
        \item $\word{\lbl{\cLbl''}{\bLbl'}} \in \langOf{\bNFA} \intersection
            \langOf{\bNFA'}$ with $\stripTau{\word{\lbl{\cLbl''}{\bLbl'}}} =
            \stripTau{\word{\lbl{\cLbl}{\bLbl}}}$,
    \end{enumerate}
    N.B. that $\cLbl' \neq \cLbl''$, but $\stripTau{\cLbl'} =
    \stripTau{\cLbl''}$, thus we need to modify $\word{\lbl{\aLbl'}{\cLbl'}}$
    and $\word{\lbl{\cLbl''}{\bLbl'}}$ to obtain a pair of labels with equal
    common boundary component; by \lemref{lem:padReflexiveTNFA}, we can pad
    $\word{\lbl{\aLbl'}{\cLbl'}}$ and $\word{\lbl{\cLbl''}{\bLbl'}}$ to obtain
    the required labels, namely $\word{\lbl{\aLbl''}{\cLbl'''}} \in
    \langOf{\aNFA'}$ and $\word{\lbl{\cLbl'''}{\bLbl''}} \in \langOf{\bNFA'}$,
    with $\stripTau{\word{\lbl{\aLbl''}{\cLbl'''}}} =
    \stripTau{\word{\lbl{\aLbl}{\cLbl}}}$ and
    $\stripTau{\word{\lbl{\cLbl'''}{\bLbl''}}} =
    \stripTau{\word{\lbl{\cLbl}{\bLbl}}}$. Thus we have that
    $\stripTau{\word{\lbl{\aLbl''}{\bLbl''}}} =
    \stripTau{\word{\lbl{\aLbl}{\bLbl}}}$; furthermore, since
    $\word{\lbl{\aLbl''}{\cLbl'''}} \in \langOf{\aNFA'}$ and
    $\word{\lbl{\cLbl'''}{\bLbl''}} \in \langOf{\bNFA'}$, it follows that
    $\word{\lbl{\aLbl''}{\bLbl''}} \in \langOf{\aNFA' \comp \bNFA'}$. Indeed,
    $\stripTau{\word{\lbl{\aLbl''}{\bLbl''}}} =
    \stripTau{\word{\lbl{\aLbl}{\bLbl}}} = \aNFAWord \in \weakLangOf{\aNFA'
    \comp \bNFA'}$, as required.

    For~\ref{langEquivItem2}, suppose that $\aNFAWord
    \in \weakLangOf{\aNFA \tensor \cNFA}$. Then, by the definition of weak
    language there exists a $\word{\lbl{\aLbl\cLbl}{\bLbl\dLbl}} \in
    \langOf{\aNFA \tensor \cNFA}$ such that
    $\stripTau{\word{\lbl{\aLbl\cLbl}{\bLbl\dLbl}}} = \aNFAWord$. Then, we have
    that $\word{\lbl{\aLbl}{\bLbl}} \in \langOf{\aNFA}$ and
    $\word{\lbl{\cLbl}{\dLbl}} \in \langOf{\cNFA}$.

    Now, since $\aNFA, \aNFA', \cNFA$ and $\cNFA'$ are reflexive \TNFA{}s, with
    $\aNFA \weakLangEquiv \aNFA'$ and $\cNFA \weakLangEquiv \cNFA'$, by
    \lemref{lem:wordInIntersectionOfWeakEquivTNFAs}, we have:
    \begin{enumerate}
        \item $\word{\lbl{\aLbl'}{\bLbl'}} \in \langOf{\aNFA} \intersection
            \langOf{\aNFA'}$ with $\stripTau{\word{\lbl{\aLbl'}{\bLbl'}}} =
            \stripTau{\word{\lbl{\aLbl}{\bLbl}}}$, and
        \item $\word{\lbl{\cLbl'}{\dLbl'}} \in \langOf{\cNFA} \intersection
            \langOf{\cNFA'}$ with $\stripTau{\word{\lbl{\cLbl'}{\dLbl'}}} =
            \stripTau{\word{\lbl{\cLbl}{\dLbl}}}$,
    \end{enumerate}

    Now, since $\word{\lbl{\aLbl'}{\bLbl'}} \in \langOf{\aNFA'}$ and
    $\word{\lbl{\cLbl'}{\dLbl'}} \in \langOf{\cNFA'}$ we have that
    $\word{\lbl{\aLbl'\cLbl'}{\bLbl'\dLbl'}} \in \langOf{\aNFA' \tensor
    \cNFA'}$, and furthermore,
    $\stripTau{\word{\lbl{\aLbl'\cLbl'}{\bLbl'\dLbl'}}} =
    \stripTau{\word{\lbl{\aLbl\cLbl}{\bLbl\dLbl}}} = \aNFAWord \in
    \weakLangOf{\aNFA' \tensor \cNFA'}$, as required.
\end{proof}

Now, before we can employ weak-language equivalence as a quotient in our
compositional reachability algorithm (\algref{alg:compositionalAlgorithm}), we
must assure ourselves that the initial \TNFA{} semantics are reflexive, and
that reflexivity is preserved by composition:

\begin{lemma}[\TNFA{} semantics of PNBs are always reflexive]
    \label{lem:TNFASemanticsReflexive}
    For a marked PNB, $\aPNB$, its \TNFA{} semantics, $\PNBToTNFA{\aPNB}$ is
    reflexive.
\end{lemma}
\begin{proof}
    For any marking of $\aPNB$, we can always fire the empty set of
    transitions, which has no effect on the boundaries. Thus, for every state
    of the \TNFA{}, we have a self-loop with $\tau$ label, i.e.\
    $\PNBToTNFA{\aPNB}$ is reflexive.
\end{proof}

\begin{lemma}[Reflexivity of \TNFA{}s is preserved under composition]
    If $\aNFA$, a \NFAB{\aN}{\bN}, $\bNFA$ a \NFAB{\bN}{\cN}, and $\cNFA$ a
    \NFAB{\cN}{\dN} are reflexive then:
    \begin{itemize}
        \item $\aNFA \comp \bNFA$ is reflexive
        \item $\aNFA \tensor \cNFA$ is reflexive
    \end{itemize}
\end{lemma}
\begin{proof}
    The states of $\aNFA \comp \bNFA$ and $\aNFA \tensor \cNFA$ are pairs of
    states of the underlying \TNFA{}s. Since the components are reflexive, in
    any state (pair of states), we can take a $\tau$ transition in both
    components, leading to a $\tau$ transition in the composite, as required.
\end{proof}

Now we have shown that we can arbitrarily replace reflexive \TNFA{}s with
(reflexive) weak-language equivalent \TNFA{}s, and retain the weak language of
the composite. At this point, it is worth noting that after quotienting by
weak-language equivalence, the states of a \TNFA{} semantics will not
necessarily correspond to markings of the underlying PNB; indeed, we only
preserve the (weak) boundary protocol.

In fact, we can loosen our restriction requiring reflexive \TNFA{}s, and
instead require only that \TNFA{}s are \emph{strong language equivalent} to
reflexive \TNFA{}s, i.e.\ if we have: $\aNFA \weakLangEquiv \aNFA' \langEquiv
\aNFA''$ where $\aNFA, \aNFA'$ are reflexive, then we should expect, for
some compatible $\bNFA$, that $\aNFA \comp \bNFA \weakLangEquiv \aNFA'' \comp
\bNFA$. However, we might immediately notice that reflexivity is not
necessarily preserved by (strong) language equivalence, as illustrated in
\figref{fig:reflexivityNotPreservedByLangEquiv}.

\begin{figure}[ht]
    \centering
    \begin{subfigure}{0.5\textwidth}
        \centering
        \begin{tikzpicture}[nfa]
            \node[state] (0) [init]                  {$0$};
            \node[state] (1) [below=of 0]            {$1$};
            \node[state] (2) [accepting, below=of 1] {$2$};

            \draw (0) edge [loop left]  node {$\lbl{0}{0}$} (0);
            \draw (0) edge              node {$\lbl{0}{0}$} (1);
            \draw (1) edge [loop left]  node {$\lbl{0}{0}$} (1);
            \draw (1) edge              node {$\lbl{1}{1}$} (2);
            \draw (2) edge [loop left]  node {$\lbl{0}{0}$} (2);
        \end{tikzpicture}
        \caption{$\aNFA$, which is reflexive}
    \end{subfigure}%
    \begin{subfigure}{0.5\textwidth}
        \centering
        \begin{tikzpicture}[nfa]
            \node[state] (0) [init]                  {$0$};
            \node[state] (1) [below=of 0]            {$1$};
            \node[state] (2) [accepting, below=of 1] {$2$};

            \draw (0) edge              node {$\lbl{0}{0}$} (1);
            \draw (1) edge [loop left]  node {$\lbl{0}{0}$} (1);
            \draw (1) edge              node {$\lbl{1}{1}$} (2);
            \draw (2) edge [loop left]  node {$\lbl{0}{0}$} (2);
        \end{tikzpicture}
        \caption{$\aNFA'$, such that $\aNFA' \langEquiv \aNFA$ but $\aNFA'$ is
        not reflexive}
    \end{subfigure}%
    \caption{Reflexivity is not preserved by language equivalence}
    \label{fig:reflexivityNotPreservedByLangEquiv}
\end{figure}

However, this does not turn out to cause trouble; since strongly equivalent
\TNFA{}s can replicate language, they must contain an \emph{equivalent}
transition(s) to simulate the reflexivity of a reflexive \TNFA{}. Indeed, we
can prove that weak-language equivalence \emph{up-to} strong-language
equivalence is a congruence:

\begin{theorem}
    Suppose that:
    \begin{itemize}
        \item $\aNFA, \aNFA', \aNFA''$ are \NFAB{\aN}{\bN} ($\aNFA$, $\aNFA'$
            reflexive), $\aNFA \weakLangEquiv \aNFA'$, and $\aNFA' \langEquiv
            \aNFA''$,
        \item $\bNFA, \bNFA', \bNFA''$ are \NFAB{\bN}{\cN} ($\bNFA$, $\bNFA'$
            reflexive), $\bNFA \weakLangEquiv \bNFA'$, and $\bNFA' \langEquiv
            \bNFA''$.
    \end{itemize}
    Then, we have:
    \[
        \aNFA \comp \bNFA \weakLangEquiv \aNFA'' \comp \bNFA''
    \]
\end{theorem}
\begin{proof}
    We have:
    \begin{align*}
        \aNFA \comp \bNFA &\weakLangEquiv \aNFA' \comp \bNFA' && \text{by
            \thmref{thm:weakLangEquivCongruence}}\\
        \aNFA' \comp \bNFA' &\langEquiv \aNFA'' \comp \bNFA'' && \text{by
            \thmref{thm:langEquivCongruence}}\\
        \aNFA' \comp \bNFA' &\weakLangEquiv \aNFA'' \comp \bNFA'' && \text{by
            definition of $\weakLangEquiv$} \\
        \aNFA \comp \bNFA &\weakLangEquiv \aNFA'' \comp \bNFA'' && \text{by
            transitivity of $\weakLangEquiv$}
    \end{align*}
\end{proof}

Thus, we are justified in using standard language-equivalence preserving
transformations to quotient \tauClosed{} \TNFA{} to improve performance.

Returning to an earlier example, consider again the \tauClosed{} \TNFA{} shown in
\figref{fig:closedTNFA} and observe that the language it accepts is simple, being given by the
regular expression:
$\parens{\lbl{0}{0}}^\ast\parens{\lbl{1}{0}\mid\lbl{0}{1}}\parens{\lbl{0}{0}}^\ast$.  Indeed, there
is a smaller \TNFA{} with the same language; in general, NFA minimisation techniques aim to find a
structurally smaller (but not necessarily minimal) NFA that recognises the same language; an
example of such a minimised \TNFA{} is shown in \figref{fig:examplePNBTNFAMinimised}.  This \TNFA{}
recognises the weak boundary protocol containing only two traces: $\setof{\sequenceof{\lbl{0}{1}},
\sequenceof{\lbl{1}{0}}}$; indeed, inspecting the original PNB (\figref{fig:examplePNB}) we can see
that (modulo internal firings) it can either perform a single $\lbl{0}{1}$ or $\lbl{1}{0}$
interaction to reach its desired marking.

\begin{figure}[ht]
    \centering
    \begin{tikzpicture}[nfa]
        \node[state] (0) [init]                  {$0$};
        \node[state] (1) [accepting, below=of 0] {$1$};

        \draw (0) edge [loop left]  node {$\lbl{0}{0}$} (0);
        \draw (0) edge              node {$\setof{\lbl{1}{0},\lbl{0}{1}}$} (1);
        \draw (1) edge [loop left]  node {$\lbl{0}{0}$} (1);
    \end{tikzpicture}
    \caption{Minimised \TNFA{} of \figref{fig:closedTNFA}}
    \label{fig:examplePNBTNFAMinimised}
\end{figure}

A final important point to note is that there is a single \TNFA{} up-to
weak-language, for a $\aPNB \withNetType{0}{0}$. Indeed, since there are no
boundaries, all transitions of the \TNFA{} will be $\tau$-labelled. Therefore,
if there is a (connected) accepting state, then the \TNFA{} is weak-language
equivalent to the \TNFA{} with a single state self-loop labelled with $\tau$,
such that the state is initial, and accepting. If there is no such accepting
state, the language is empty and thus is equivalent to that of the same
single-state \TNFA{}, but with no accepting state. These single-state \TNFA{}s
are illustrated in \figref{fig:00TNFAs}.

\begin{figure}[ht]
    \centering
    \begin{subfigure}{0.5\textwidth}
        \centering
        \begin{tikzpicture}[nfa]
            \node (0) [state, init, accepting] {$0$};
            \nfaArr[loop right][]{0}{0}{}{}
        \end{tikzpicture}
    \end{subfigure}%
    \begin{subfigure}{0.5\textwidth}
        \centering
        \begin{tikzpicture}[nfa]
            \node (0) [state, init] {$0$};
            \nfaArr[loop right][]{0}{0}{}{}
        \end{tikzpicture}
    \end{subfigure}%
    \caption{\TNFA{}s that the semantics of any $\aPNB \withNetType{0}{0}$ are
    weak-language equivalent to}
    \label{fig:00TNFAs}
\end{figure}
