\newcommand{\netMap}{\texttt{knownNetNFAs}}
\newcommand{\compMap}{\texttt{knownNFAComps}}
\newcommand{\lineref}[1]{(Line \ref{line:#1})}
\newcommand{\genOP}{\texttt{OP}}

We now introduce an improved version of \algref{alg:compositionalAlgorithm},
which incorporates the optimisations introduced in this chapter. We demonstrate
the encouraging performance of this algorithm, by evaluating it on the examples
of \chpref{chp:benchmarksAndLang} and \secref{sec:tweaks}.

The optimised algorithm is presented as \algref{alg:thealgorithm}; observe that
\TNFA{}s are reduced up-to weak-language equivalence (lines \ref{line:toNFA}
and \ref{line:reduce2}) and the memoisation map is checked for membership up-to
weak-language (since all \TNFA{}s are reduced w.r.t. weak-language we simply
check for language equivalence at \lineref{contains}).

% Fixes nested Calls
% http://tex.stackexchange.com/questions/16046/how-to-nest-call-in-algorithmicx
\renewcommand*\Call[2]{\textproc{#1}(#2)}
    \newcommand{\mytriple}{(n_1,n_2,\genOP)}
%
\begin{algorithm}[ht]
\hrule
\vspace{1ex}
\begin{algorithmic}[1]
    \Require \netMap{}, \compMap{} initially empty
    \Procedure{toNFA}{$t$}
        \If{$t$ is a PNB} \label{line:isPNB}
            \If{\Call{Contains}{$\netMap,t$}} \label{line:checkKnownNet}
                \State \textbf{return} $\netMap[t]$ \label{line:returnNetMap}
            \Else \label{line:notKnownNet}
                \State $n \gets \Call{reduce}{\Call{$\tau$-close}{\Call{netToNFA}{t}}}$ \label{line:toNFA}
                \State $\netMap[t] := n$ \label{line:updateNetMap}
                \State \textbf{return} $n$
            \EndIf
            \Else \Comment{$t$ is $(t_1, t_2, \genOP)$, where \genOP{} is `$\comp$' or `$\tensor$'} \label{line:isComp}
                \State $n_1 \gets \Call{toNFA}{t_1}$ \label{line:recurseT1}
                \State $n_2 \gets \Call{toNFA}{t_2}$ \label{line:recurseT2}
                \If{\Call{ContainsEquiv}{$\compMap,\mytriple$}} \label{line:contains}\Comment{Up-to $\weakLangEquiv$} \label{line:foundComp}
                    \State \textbf{return} $\compMap[\mytriple]$ \label{line:returnCompMap}
                \Else \label{line:notKnownComp}
                    \State $n \gets
                    \Call{reduce}{\Call{$\tau$-close}{n_1 \mathrel{\genOP} n_2}}$\label{line:reduce2}

                    \State $\compMap[\mytriple] := n$ \label{line:addComp}
                    \State \textbf{return} $n$
                \EndIf
        \EndIf
    \EndProcedure
\end{algorithmic}
\hrule
\caption{Optimised Algorithm}
\label{alg:thealgorithm}
\end{algorithm}

We can now evaluate this algorithm on the examples of
\chpref{chp:benchmarksAndLang} and \secref{sec:tweaks}, that is, the original
benchmark examples, except those with refactored versions using more-performant
composition associatations. Indeed, we present the new timing and memory
requirements in \tabref{tab:fasttimings}.

\begin{table}[ht]
\centering
\newcommand{\mycaption}{Checking reachability of markings in
\figref{fig:slowmarkings}, on examples of \secref{sec:tweaks}, using
\algref{alg:thealgorithm}}
\caption[\mycaption]{\mycaption. Key: T = Time (s), M = Maximum Resident Memory
(MB), \emph{TO} = Time Out (300s)}
\label{tab:fasttimings}
\makebox[\textwidth][c]{
\rowcolors{2}{gray!25}{white}
\begin{tabular}{ | c | c | c | }
\hline
Sys & T & M \\ \hline
\overtakeSys{2}	&	0.017	&	21.02 \\
\overtakeSys{8}	&	0.017	&	21.02 \\
\overtakeSys{32}	&	0.017	&	21.02 \\
\overtakeSys{512}	&	0.017	&	21.06 \\
\overtakeSys{4096}	&	0.017	&	21.10 \\
\overtakeSys{32768}	&	0.019	&	22.65 \\
\DACSys{2}	&	0.001	&	15.18 \\
\DACSys{8}	&	0.001	&	15.26 \\
\DACSys{32}	&	0.001	&	15.20 \\
\DACSys{512}	&	0.001	&	15.23 \\
\DACSys{4096}	&	0.002	&	15.81 \\
\DACSys{32768}	&	0.003	&	22.18 \\
\diningphilosophersSys{2}	&	0.008	&	20.02 \\
\diningphilosophersSys{8}	&	0.011	&	20.64 \\
\diningphilosophersSys{32}	&	0.011	&	20.64 \\
\diningphilosophersSys{512}	&	0.011	&	20.64 \\
\diningphilosophersSys{4096}	&	0.011	&	19.25 \\
\diningphilosophersSys{32768}	&	0.012	&	22.61 \\
\bufferSys{2}	&	0.001	&	14.19 \\
\bufferSys{8}	&	0.001	&	14.24 \\
\bufferSys{32}	&	0.002	&	14.22 \\
\bufferSys{512}	&	0.001	&	14.38 \\
\bufferSys{4096}	&	0.002	&	14.90 \\
\bufferSys{32768}	&	0.007	&	21.30 \\
\replicatorsSys{2}	&	0.002	&	15.22 \\
\replicatorsSys{8}	&	0.001	&	15.23 \\
\replicatorsSys{32}	&	0.001	&	15.22 \\
\replicatorsSys{512}	&	0.001	&	15.31 \\
\replicatorsSys{4096}	&	0.002	&	15.81 \\
\replicatorsSys{32768}	&	0.004	&	22.31 \\
\iteratedchoiceSys{2}	&	0.002	&	15.23 \\
\iteratedchoiceSys{8}	&	0.002	&	15.25 \\
\iteratedchoiceSys{32}	&	0.002	&	15.24 \\
\iteratedchoiceSys{512}	&	0.002	&	15.30 \\
\iteratedchoiceSys{4096}	&	0.002	&	15.86 \\
\iteratedchoiceSys{32768}	&	0.004	&	22.32 \\
\hline
\end{tabular}
\rowcolors{2}{gray!25}{white}
\begin{tabular}{ | c | c | c | }
\hline
Sys & T & M \\ \hline
\contree{2}{2}	&	0.001	&	15.22 \\
\contree{8}{8}	&	0.001	&	15.25 \\
\contree{32}{32}	&	0.002	&	15.23 \\
\contree{128}{128}	&	0.002	&	15.26 \\
\contree{512}{512}	&	0.011	&	15.33 \\
\contree{2048}{2048}	&	1.069	&	20.62 \\
\hartstoneSys{2}	&	0.013	&	19.33 \\
\hartstoneSys{4}	&	0.039	&	19.61 \\
\hartstoneSys{8}	&	2.013	&	21.41 \\
\hartstoneSys{10}	&	4.016	&	20.62 \\
\hartstoneSys{13}	&	10.445	&	25.74 \\
\hartstoneSys{16}	&	25.252	&	28.65 \\
\tokenringSys{2}	&	0.008	&	17.99 \\
\tokenringSys{4}	&	0.035	&	19.79 \\
\tokenringSys{8}	&	2.043	&	21.33 \\
\tokenringSys{10}	&	4.072	&	21.11 \\
\tokenringSys{13}	&	10.026	&	25.22 \\
\tokenringSys{16}	&	19.247	&	28.65 \\
\cyclicschedulerSys{2}	&	0.005	&	16.81 \\
\cyclicschedulerSys{4}	&	0.017	&	19.52 \\
\cyclicschedulerSys{8}	&	0.094	&	20.57 \\
\cyclicschedulerSys{10}	&	1.077	&	20.59 \\
\cyclicschedulerSys{13}	&	3.075	&	22.24 \\
\cyclicschedulerSys{16}	&	6.091	&	24.60 \\
\counterSys{2}	&	0.004	&	16.09 \\
\counterSys{4}	&	0.011	&	20.56 \\
\counterSys{8}	&	0.087	&	21.36 \\
\counterSys{10}	&	1.075	&	22.75 \\
\counterSys{13}	&	4.035	&	28.59 \\
\counterSys{16}	&	8.649	&	28.60 \\
\distree{1}{1}	&	0.002	&	15.60 \\
\distree{2}{2}	&	0.004	&	15.63 \\
\distree{3}{3}	&	12.455	&	32.45 \\
\distree{4}{4}	&	\emph{TO}	&	\emph{TO} \\
\distree{5}{5}	&	\emph{TO}	&	\emph{TO} \\
\distree{6}{6}	&	\emph{TO}	&	\emph{TO} \\
\hline
\end{tabular}
}
\end{table}


Inspecting the values in \tabref{tab:fasttimings} and contrasting with those in
\tabref{tab:slowtimings}, several points are clear:

\begin{itemize}
    \item For the majority of examples, the performance has \emph{vastly}
        improved; indeed, for several systems (e.g. \overtakeSys{-}), the time
        taken by the improved algorithm for parameter 32768 is less
        than the naive algorithm for paremeter 1.
    \item Indeed, we can observe the effects of reaching fixed points --- for
        most systems (e.g. \bufferSys{-}), the time/memory use is essentially
        constant\footnote{Our tool is written in Haskell, a language that uses
            garbage collection for automatic memory management, which may
            explain some of the slight variance in memory usage.}
    \item The performance is \emph{not} directly proportional to net size:
        indeed, \overtakeSys{-} is a ``large'' system, with scalable
        performance, whereas \counterSys{-} is a ``small'' system, with poor
        performance.
\end{itemize}

To help interpreting our results, we have plotted the data in
\figref{fig:penrosetimes}.

\begin{figure}[ht]
\centering
\begin{tikzpicture}[trim axis left, trim axis right]
\pgfplotsset{width=\textwidth}
\newcommand{\datadir}{chapters/improveEfficiency/individual_bench_data}
\newcommand{\nameprefix}{penrose_data}
\begin{loglogaxis}[
    cycle list name=exotic,
    xlabel={Problem Size},
    ylabel={Time (s)},
    grid=major,
    log basis x=2,
    legend style={%
        overlay,
        at={(1,1)},
        anchor=north east},
]
\foreach \name in
{disjunction-tree,token-ring,counter,hartstone,philo,over,iterated-choice,replicator,dac,buffer,conjunction-tree}{%
\addplot+ [smooth] table {\datadir/\nameprefix_\name.tsv};
\addlegendentryexpanded{\name}
}
\end{loglogaxis}
\end{tikzpicture}
\caption{Time vs Problem size for \algref{alg:thealgorithm}}
\label{fig:penrosetimes}
\end{figure}
