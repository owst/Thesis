\newcommand{\netMap}{\texttt{knownNetNFAs}}
\newcommand{\compMap}{\texttt{knownNFAComps}}
\newcommand{\lineref}[1]{(Line \ref{line:#1})}
\newcommand{\genOP}{\texttt{OP}}

We now introduce an improved version of \algref{alg:compositionalAlgorithm},
which incorporates the optimisations introduced in this chapter. We demonstrate
the encouraging performance of this algorithm, by evaluating it on the examples
of \chpref{chp:benchmarksAndLang} and \secref{sec:tweaks}.

The optimised algorithm is presented as \algref{alg:thealgorithm}; observe that
\TNFA{}s are reduced up-to weak-language equivalence (lines \ref{line:toNFA}
and \ref{line:reduce2}) and the memoisation map is checked for membership up-to
weak-language (since all \TNFA{}s are reduced w.r.t. weak-language we simply
check for language equivalence at \lineref{contains}).

% Fixes nested Calls
% http://tex.stackexchange.com/questions/16046/how-to-nest-call-in-algorithmicx
\renewcommand*\Call[2]{\textproc{#1}(#2)}
    \newcommand{\mytriple}{(n_1,n_2,\genOP)}
%
\begin{algorithm}[ht]
\hrule
\vspace{1ex}
\begin{algorithmic}[1]
    \Require \netMap{}, \compMap{} initially empty
    \Procedure{toNFA}{$t$}
        \If{$t$ is a PNB} \label{line:isPNB}
            \If{\Call{Contains}{$\netMap,t$}} \label{line:checkKnownNet}
                \State \textbf{return} $\netMap[t]$ \label{line:returnNetMap}
            \Else \label{line:notKnownNet}
                \State $n \gets \Call{reduce}{\Call{$\tau$-close}{\Call{netToNFA}{t}}}$ \label{line:toNFA}
                \State $\netMap[t] := n$ \label{line:updateNetMap}
                \State \textbf{return} $n$
            \EndIf
            \Else \Comment{$t$ is $(t_1, t_2, \genOP)$, where \genOP{} is `$\comp$' or `$\tensor$'} \label{line:isComp}
                \State $n_1 \gets \Call{toNFA}{t_1}$ \label{line:recurseT1}
                \State $n_2 \gets \Call{toNFA}{t_2}$ \label{line:recurseT2}
                \If{\Call{ContainsEquiv}{$\compMap,\mytriple$}} \label{line:contains}\Comment{Up-to $\weakLangEquiv$} \label{line:foundComp}
                    \State \textbf{return} $\compMap[\mytriple]$ \label{line:returnCompMap}
                \Else \label{line:notKnownComp}
                    \State $n \gets
                    \Call{reduce}{\Call{$\tau$-close}{n_1 \mathrel{\genOP} n_2}}$\label{line:reduce2}

                    \State $\compMap[\mytriple] := n$ \label{line:addComp}
                    \State \textbf{return} $n$
                \EndIf
        \EndIf
    \EndProcedure
\end{algorithmic}
\hrule
\caption{Optimised Algorithm}
\label{alg:thealgorithm}
\end{algorithm}

We can now evaluate this algorithm on the examples of
\chpref{chp:benchmarksAndLang} and \secref{sec:tweaks}, that is, the original
benchmark examples, except those with refactored versions using more-performant
composition associatations. Indeed, we present the new timing and memory
requirements in \tabref{tab:fasttimings}.

\rowcolors{2}{gray!25}{white}
\begin{tabular}{ | c | c || c | c | c | c | c | c | }
\hline
\multicolumn{2}{|c ||}{Problem}&\multicolumn{6}{c |}{Time} \\
name	&	size	&	LOLA	&	CLP	&	CNA	&	Penrose	&	TAAPL	&	MARCIE\\ \hline
buffer	&	2	&	\highlightedResult{0.000}	&	0.002	&	0.014	&	0.001	&	0.001	&	0.043 \\
buffer	&	8	&	\highlightedResult{0.001}	&	0.003	&	0.017	&	\highlightedResult{0.001}	&	0.002	&	0.044 \\
buffer	&	32	&	\highlightedResult{0.001}	&	0.013	&	0.824	&	0.002	&	0.005	&	0.048 \\
buffer	&	512	&	0.058	&	\failureResult{T}	&	\failureResult{M}	&	\highlightedResult{0.001}	&	\failureResult{T}	&	157.449 \\
buffer	&	4096	&	\failureResult{T}	&	\failureResult{T}	&	\failureResult{M}	&	\highlightedResult{0.002}	&	\failureResult{T}	&	\failureResult{M} \\
buffer	&	32768	&	\failureResult{T}	&	\failureResult{T}	&	\failureResult{M}	&	\highlightedResult{0.007}	&	\failureResult{T}	&	\failureResult{M} \\
over	&	2	&	\highlightedResult{0.001}	&	0.003	&	0.016	&	0.017	&	\failureResult{Q}	&	\failureResult{Q} \\
over	&	8	&	31.039	&	\highlightedResult{0.008}	&	1.071	&	0.017	&	\failureResult{Q}	&	\failureResult{Q} \\
over	&	32	&	\failureResult{M}	&	\failureResult{T}	&	\failureResult{M}	&	\highlightedResult{0.017}	&	\failureResult{Q}	&	\failureResult{Q} \\
over	&	512	&	\failureResult{M}	&	\failureResult{T}	&	\failureResult{M}	&	\highlightedResult{0.017}	&	\failureResult{Q}	&	\failureResult{Q} \\
over	&	4096	&	\failureResult{M}	&	\failureResult{T}	&	\failureResult{M}	&	\highlightedResult{0.017}	&	\failureResult{Q}	&	\failureResult{Q} \\
over	&	32768	&	\failureResult{M}	&	\failureResult{T}	&	\failureResult{M}	&	\highlightedResult{0.019}	&	\failureResult{Q}	&	\failureResult{Q} \\
dac	&	2	&	\highlightedResult{0.001}	&	0.002	&	0.015	&	\highlightedResult{0.001}	&	\highlightedResult{0.001}	&	0.045 \\
dac	&	8	&	\highlightedResult{0.001}	&	0.003	&	0.017	&	\highlightedResult{0.001}	&	\highlightedResult{0.001}	&	0.046 \\
dac	&	32	&	\highlightedResult{0.001}	&	0.005	&	0.028	&	\highlightedResult{0.001}	&	0.002	&	0.056 \\
dac	&	512	&	0.005	&	\failureResult{T}	&	255.847	&	\highlightedResult{0.001}	&	0.083	&	266.221 \\
dac	&	4096	&	2.462	&	\failureResult{T}	&	\failureResult{M}	&	\highlightedResult{0.002}	&	\failureResult{M}	&	\failureResult{M} \\
dac	&	32768	&	\failureResult{T}	&	\failureResult{T}	&	\failureResult{M}	&	\highlightedResult{0.003}	&	\failureResult{M}	&	\failureResult{M} \\
philo	&	2	&	\highlightedResult{0.001}	&	0.002	&	0.014	&	0.008	&	0.004	&	0.044 \\
philo	&	8	&	\highlightedResult{0.002}	&	0.003	&	0.016	&	0.011	&	\highlightedResult{0.002}	&	0.046 \\
philo	&	32	&	\failureResult{M}	&	\highlightedResult{0.003}	&	0.017	&	0.011	&	0.007	&	1.040 \\
philo	&	512	&	\failureResult{M}	&	0.020	&	0.086	&	\highlightedResult{0.011}	&	68.033	&	\failureResult{T} \\
philo	&	4096	&	\failureResult{M}	&	7.853	&	\failureResult{M}	&	\highlightedResult{0.011}	&	\failureResult{M}	&	\failureResult{T} \\
philo	&	32768	&	\failureResult{M}	&	\failureResult{T}	&	\failureResult{M}	&	\highlightedResult{0.012}	&	\failureResult{M}	&	\failureResult{T} \\
\nonCorbett{iter-choice}	&	2	&	\highlightedResult{0.001}	&	0.003	&	0.015	&	0.002	&	\highlightedResult{0.001}	&	0.044 \\
\nonCorbett{iter-choice}	&	8	&	0.006	&	5.025	&	19.062	&	\highlightedResult{0.002}	&	0.003	&	0.044 \\
\nonCorbett{iter-choice}	&	32	&	\failureResult{M}	&	\failureResult{T}	&	\failureResult{T}	&	\highlightedResult{0.002}	&	0.005	&	0.054 \\
\nonCorbett{iter-choice}	&	512	&	\failureResult{M}	&	\failureResult{T}	&	\failureResult{T}	&	\highlightedResult{0.002}	&	6.084	&	36.066 \\
\nonCorbett{iter-choice}	&	4096	&	\failureResult{M}	&	\failureResult{T}	&	\failureResult{T}	&	\highlightedResult{0.002}	&	\failureResult{M}	&	\failureResult{M} \\
\nonCorbett{iter-choice}	&	32768	&	\failureResult{M}	&	\failureResult{T}	&	\failureResult{T}	&	\highlightedResult{0.004}	&	\failureResult{M}	&	\failureResult{M} \\
\nonCorbett{replicator}	&	2	&	\highlightedResult{0.001}	&	\failureResult{/}	&	0.015	&	0.002	&	\highlightedResult{0.001}	&	\failureResult{T} \\
\nonCorbett{replicator}	&	8	&	\highlightedResult{0.001}	&	\failureResult{/}	&	0.016	&	\highlightedResult{0.001}	&	0.002	&	\failureResult{T} \\
\nonCorbett{replicator}	&	32	&	\highlightedResult{0.001}	&	\failureResult{/}	&	0.017	&	\highlightedResult{0.001}	&	0.004	&	\failureResult{T} \\
\nonCorbett{replicator}	&	512	&	0.002	&	\failureResult{/}	&	1.023	&	\highlightedResult{0.001}	&	112.248	&	\failureResult{T} \\
\nonCorbett{replicator}	&	4096	&	0.062	&	\failureResult{/}	&	64.046	&	\highlightedResult{0.002}	&	\failureResult{M}	&	\failureResult{T} \\
\nonCorbett{replicator}	&	32768	&	91.646	&	\failureResult{/}	&	\failureResult{M}	&	\highlightedResult{0.004}	&	\failureResult{M}	&	\failureResult{T} \\
conjunction-tree	&	1,1	&	\failureResult{NA}	&	\failureResult{NA}	&	\failureResult{NA}	&	\highlightedResult{0.001}	&	\highlightedResult{0.001}	&	0.043 \\
conjunction-tree	&	2,2	&	\failureResult{NA}	&	\failureResult{NA}	&	\failureResult{NA}	&	\highlightedResult{0.001}	&	\highlightedResult{0.001}	&	0.044 \\
conjunction-tree	&	3,3	&	\failureResult{NA}	&	\failureResult{NA}	&	\failureResult{NA}	&	\highlightedResult{0.001}	&	0.003	&	0.045 \\
conjunction-tree	&	4,4	&	\failureResult{NA}	&	\failureResult{NA}	&	\failureResult{NA}	&	\highlightedResult{0.001}	&	0.012	&	2.066 \\
conjunction-tree	&	5,5	&	\failureResult{NA}	&	\failureResult{NA}	&	\failureResult{NA}	&	\highlightedResult{0.001}	&	9.069	&	\failureResult{M} \\
conjunction-tree	&	6,6	&	\failureResult{NA}	&	\failureResult{NA}	&	\failureResult{NA}	&	\highlightedResult{0.001}	&	\failureResult{M}	&	\failureResult{M} \\
conjunction-tree	&	7,7	&	\failureResult{NA}	&	\failureResult{NA}	&	\failureResult{NA}	&	\highlightedResult{0.001}	&	\failureResult{M}	&	\failureResult{M} \\
\hline
\end{tabular}
\rowcolors{2}{gray!25}{white}
\begin{tabular}{ | c | c || c | c | c | c | c | c | }
\hline
\multicolumn{2}{|c ||}{Problem}&\multicolumn{6}{c |}{Time} \\
name	&	size	&	LOLA	&	CLP	&	CNA	&	Penrose	&	TAAPL	&	MARCIE\\ \hline
\nonCorbett{counter}	&	2	&	\highlightedResult{0.000}	&	\failureResult{/}	&	0.015	&	0.004	&	\failureResult{Q}	&	\failureResult{Q} \\
\nonCorbett{counter}	&	4	&	\failureResult{NA}	&	\failureResult{NA}	&	\failureResult{NA}	&	\highlightedResult{0.011}	&	\failureResult{Q}	&	\failureResult{Q} \\
\nonCorbett{counter}	&	8	&	\highlightedResult{0.001}	&	\failureResult{/}	&	\failureResult{/}	&	0.087	&	\failureResult{Q}	&	\failureResult{Q} \\
\nonCorbett{counter}	&	16	&	\highlightedResult{0.000}	&	\failureResult{/}	&	\failureResult{/}	&	1.075	&	\failureResult{Q}	&	\failureResult{Q} \\
\nonCorbett{counter}	&	32	&	\highlightedResult{0.001}	&	\failureResult{/}	&	\failureResult{/}	&	4.035	&	\failureResult{Q}	&	\failureResult{Q} \\
\nonCorbett{counter}	&	64	&	\highlightedResult{0.001}	&	\failureResult{/}	&	\failureResult{/}	&	8.649	&	\failureResult{Q}	&	\failureResult{Q} \\
hartstone	&	2	&	\failureResult{NA}	&	\failureResult{NA}	&	\failureResult{NA}	&	\highlightedResult{0.013}	&	\failureResult{/}	&	\failureResult{/} \\
hartstone	&	4	&	\failureResult{NA}	&	\failureResult{NA}	&	\failureResult{NA}	&	\highlightedResult{0.039}	&	\failureResult{/}	&	\failureResult{/} \\
hartstone	&	8	&	\failureResult{NA}	&	\failureResult{NA}	&	\failureResult{NA}	&	\highlightedResult{2.013}	&	\failureResult{/}	&	\failureResult{/} \\
hartstone	&	10	&	\failureResult{NA}	&	\failureResult{NA}	&	\failureResult{NA}	&	\highlightedResult{4.016}	&	\failureResult{/}	&	\failureResult{/} \\
hartstone	&	13	&	\failureResult{NA}	&	\failureResult{NA}	&	\failureResult{NA}	&	\highlightedResult{10.445}	&	\failureResult{/}	&	\failureResult{/} \\
hartstone	&	16	&	\failureResult{NA}	&	\failureResult{NA}	&	\failureResult{NA}	&	\highlightedResult{25.252}	&	\failureResult{/}	&	\failureResult{/} \\
token-ring	&	2	&	\highlightedResult{0.001}	&	0.002	&	0.017	&	0.008	&	\highlightedResult{0.001}	&	0.043 \\
token-ring	&	8	&	\highlightedResult{0.001}	&	0.007	&	0.071	&	0.035	&	0.002	&	0.043 \\
token-ring	&	10	&	\failureResult{NA}	&	\failureResult{NA}	&	\failureResult{NA}	&	2.043	&	\highlightedResult{0.001}	&	0.044 \\
token-ring	&	16	&	1.824	&	\failureResult{T}	&	\failureResult{T}	&	4.072	&	\highlightedResult{0.001}	&	0.045 \\
token-ring	&	32	&	\failureResult{M}	&	\failureResult{T}	&	\failureResult{T}	&	10.026	&	\highlightedResult{0.002}	&	0.047 \\
token-ring	&	64	&	\failureResult{M}	&	\failureResult{T}	&	\failureResult{T}	&	19.247	&	\highlightedResult{0.003}	&	0.048 \\
cyclic	&	2	&	\failureResult{NA}	&	\failureResult{NA}	&	\failureResult{NA}	&	0.005	&	\highlightedResult{0.002}	&	0.045 \\
cyclic	&	4	&	\failureResult{NA}	&	\failureResult{NA}	&	\failureResult{NA}	&	0.017	&	\highlightedResult{0.002}	&	0.044 \\
cyclic	&	8	&	\failureResult{NA}	&	\failureResult{NA}	&	\failureResult{NA}	&	0.094	&	\highlightedResult{0.004}	&	0.053 \\
cyclic	&	10	&	\failureResult{NA}	&	\failureResult{NA}	&	\failureResult{NA}	&	1.077	&	\highlightedResult{0.004}	&	0.069 \\
cyclic	&	13	&	\failureResult{NA}	&	\failureResult{NA}	&	\failureResult{NA}	&	3.075	&	\highlightedResult{0.004}	&	2.037 \\
cyclic	&	16	&	\failureResult{NA}	&	\failureResult{NA}	&	\failureResult{NA}	&	6.091	&	\highlightedResult{0.005}	&	19.235 \\
disjunction-tree	&	1,1	&	\failureResult{NA}	&	\failureResult{NA}	&	\failureResult{NA}	&	0.002	&	\highlightedResult{0.001}	&	0.043 \\
disjunction-tree	&	2,2	&	\failureResult{NA}	&	\failureResult{NA}	&	\failureResult{NA}	&	\highlightedResult{0.004}	&	\failureResult{/}	&	\failureResult{/} \\
disjunction-tree	&	3,3	&	\failureResult{NA}	&	\failureResult{NA}	&	\failureResult{NA}	&	\highlightedResult{12.455}	&	\failureResult{/}	&	\failureResult{/} \\
disjunction-tree	&	4,4	&	\failureResult{NA}	&	\failureResult{NA}	&	\failureResult{NA}	&	\failureResult{T}	&	\failureResult{/}	&	\failureResult{/} \\
disjunction-tree	&	5,5	&	\failureResult{NA}	&	\failureResult{NA}	&	\failureResult{NA}	&	\failureResult{T}	&	\failureResult{/}	&	\failureResult{M} \\
disjunction-tree	&	6,6	&	\failureResult{NA}	&	\failureResult{NA}	&	\failureResult{NA}	&	\failureResult{T}	&	\failureResult{M}	&	\failureResult{M} \\
\hline
\end{tabular}
\rowcolors{2}{gray!25}{white}
\begin{tabular}{ | c | c || c | c | c | c | c | c | }
\hline
\multicolumn{2}{|c ||}{Problem}&\multicolumn{6}{c |}{Max} \\
name	&	size	&	LOLA	&	CLP	&	CNA	&	Penrose	&	TAAPL	&	MARCIE\\ \hline
buffer	&	2	&	7.51	&	33.16	&	37.81	&	14.19	&	\highlightedResult{6.47}	&	506.98 \\
buffer	&	8	&	\highlightedResult{7.51}	&	33.30	&	38.45	&	14.24	&	11.07	&	507.24 \\
buffer	&	32	&	\highlightedResult{7.51}	&	34.49	&	48.09	&	14.22	&	29.68	&	509.56 \\
buffer	&	512	&	83.44	&	\failureResult{T}	&	\failureResult{M}	&	\highlightedResult{14.38}	&	\failureResult{T}	&	1087.10 \\
buffer	&	4096	&	\failureResult{T}	&	\failureResult{T}	&	\failureResult{M}	&	\highlightedResult{14.90}	&	\failureResult{T}	&	\failureResult{M} \\
buffer	&	32768	&	\failureResult{T}	&	\failureResult{T}	&	\failureResult{M}	&	\highlightedResult{21.30}	&	\failureResult{T}	&	\failureResult{M} \\
over	&	2	&	\highlightedResult{7.51}	&	33.36	&	38.41	&	21.02	&	\failureResult{Q}	&	\failureResult{Q} \\
over	&	8	&	3812.00	&	37.63	&	141.85	&	\highlightedResult{21.02}	&	\failureResult{Q}	&	\failureResult{Q} \\
over	&	32	&	\failureResult{M}	&	\failureResult{T}	&	\failureResult{M}	&	\highlightedResult{21.02}	&	\failureResult{Q}	&	\failureResult{Q} \\
over	&	512	&	\failureResult{M}	&	\failureResult{T}	&	\failureResult{M}	&	\highlightedResult{21.06}	&	\failureResult{Q}	&	\failureResult{Q} \\
over	&	4096	&	\failureResult{M}	&	\failureResult{T}	&	\failureResult{M}	&	\highlightedResult{21.10}	&	\failureResult{Q}	&	\failureResult{Q} \\
over	&	32768	&	\failureResult{M}	&	\failureResult{T}	&	\failureResult{M}	&	\highlightedResult{22.65}	&	\failureResult{Q}	&	\failureResult{Q} \\
dac	&	2	&	7.51	&	33.20	&	37.88	&	15.18	&	\highlightedResult{2.11}	&	507.23 \\
dac	&	8	&	7.51	&	33.28	&	38.85	&	15.26	&	\highlightedResult{2.37}	&	507.50 \\
dac	&	32	&	7.50	&	34.50	&	49.45	&	15.20	&	\highlightedResult{3.04}	&	512.98 \\
dac	&	512	&	20.62	&	\failureResult{T}	&	6012.00	&	\highlightedResult{15.23}	&	75.72	&	1062.78 \\
dac	&	4096	&	166.07	&	\failureResult{T}	&	\failureResult{M}	&	\highlightedResult{15.81}	&	\failureResult{M}	&	\failureResult{M} \\
dac	&	32768	&	\failureResult{T}	&	\failureResult{T}	&	\failureResult{M}	&	\highlightedResult{22.18}	&	\failureResult{M}	&	\failureResult{M} \\
philo	&	2	&	\highlightedResult{7.51}	&	33.25	&	37.91	&	20.02	&	8.74	&	507.24 \\
philo	&	8	&	\highlightedResult{8.86}	&	33.22	&	38.54	&	20.64	&	20.66	&	508.07 \\
philo	&	32	&	\failureResult{M}	&	33.53	&	40.87	&	\highlightedResult{20.64}	&	67.30	&	536.20 \\
philo	&	512	&	\failureResult{M}	&	41.69	&	290.77	&	\highlightedResult{20.64}	&	3063.57	&	\failureResult{T} \\
philo	&	4096	&	\failureResult{M}	&	172.76	&	\failureResult{M}	&	\highlightedResult{19.25}	&	\failureResult{M}	&	\failureResult{T} \\
philo	&	32768	&	\failureResult{M}	&	\failureResult{T}	&	\failureResult{M}	&	\highlightedResult{22.61}	&	\failureResult{M}	&	\failureResult{T} \\
\nonCorbett{iter-choice}	&	2	&	\highlightedResult{7.51}	&	33.21	&	38.13	&	15.23	&	8.54	&	507.24 \\
\nonCorbett{iter-choice}	&	8	&	36.37	&	465.17	&	1570.64	&	\highlightedResult{15.25}	&	17.69	&	507.50 \\
\nonCorbett{iter-choice}	&	32	&	\failureResult{M}	&	\failureResult{T}	&	\failureResult{T}	&	\highlightedResult{15.24}	&	55.05	&	512.76 \\
\nonCorbett{iter-choice}	&	512	&	\failureResult{M}	&	\failureResult{T}	&	\failureResult{T}	&	\highlightedResult{15.30}	&	864.47	&	1160.22 \\
\nonCorbett{iter-choice}	&	4096	&	\failureResult{M}	&	\failureResult{T}	&	\failureResult{T}	&	\highlightedResult{15.86}	&	\failureResult{M}	&	\failureResult{M} \\
\nonCorbett{iter-choice}	&	32768	&	\failureResult{M}	&	\failureResult{T}	&	\failureResult{T}	&	\highlightedResult{22.32}	&	\failureResult{M}	&	\failureResult{M} \\
\nonCorbett{replicator}	&	2	&	7.51	&	\failureResult{/}	&	37.84	&	15.22	&	\highlightedResult{7.24}	&	\failureResult{T} \\
\nonCorbett{replicator}	&	8	&	\highlightedResult{7.51}	&	\failureResult{/}	&	38.15	&	15.23	&	11.87	&	\failureResult{T} \\
\nonCorbett{replicator}	&	32	&	\highlightedResult{7.51}	&	\failureResult{/}	&	39.41	&	15.22	&	30.66	&	\failureResult{T} \\
\nonCorbett{replicator}	&	512	&	\highlightedResult{14.72}	&	\failureResult{/}	&	77.87	&	15.31	&	818.19	&	\failureResult{T} \\
\nonCorbett{replicator}	&	4096	&	86.85	&	\failureResult{/}	&	3256.00	&	\highlightedResult{15.81}	&	\failureResult{M}	&	\failureResult{T} \\
\nonCorbett{replicator}	&	32768	&	1524.50	&	\failureResult{/}	&	\failureResult{M}	&	\highlightedResult{22.31}	&	\failureResult{M}	&	\failureResult{T} \\
conjunction-tree	&	1,1	&	\failureResult{NA}	&	\failureResult{NA}	&	\failureResult{NA}	&	15.22	&	\highlightedResult{5.56}	&	506.98 \\
conjunction-tree	&	2,2	&	\failureResult{NA}	&	\failureResult{NA}	&	\failureResult{NA}	&	15.22	&	\highlightedResult{7.77}	&	506.98 \\
conjunction-tree	&	3,3	&	\failureResult{NA}	&	\failureResult{NA}	&	\failureResult{NA}	&	\highlightedResult{15.22}	&	20.44	&	507.49 \\
conjunction-tree	&	4,4	&	\failureResult{NA}	&	\failureResult{NA}	&	\failureResult{NA}	&	\highlightedResult{15.22}	&	136.62	&	589.97 \\
conjunction-tree	&	5,5	&	\failureResult{NA}	&	\failureResult{NA}	&	\failureResult{NA}	&	\highlightedResult{15.22}	&	1556.56	&	\failureResult{M} \\
conjunction-tree	&	6,6	&	\failureResult{NA}	&	\failureResult{NA}	&	\failureResult{NA}	&	\highlightedResult{15.22}	&	\failureResult{M}	&	\failureResult{M} \\
conjunction-tree	&	7,7	&	\failureResult{NA}	&	\failureResult{NA}	&	\failureResult{NA}	&	\highlightedResult{15.22}	&	\failureResult{M}	&	\failureResult{M} \\
\hline
\end{tabular}
\rowcolors{2}{gray!25}{white}
\begin{tabular}{ | c | c || c | c | c | c | c | c | }
\hline
\multicolumn{2}{|c ||}{Problem}&\multicolumn{6}{c |}{Max} \\
name	&	size	&	LOLA	&	CLP	&	CNA	&	Penrose	&	TAAPL	&	MARCIE\\ \hline
\nonCorbett{counter}	&	2	&	\highlightedResult{7.51}	&	\failureResult{/}	&	37.87	&	16.09	&	\failureResult{Q}	&	\failureResult{Q} \\
\nonCorbett{counter}	&	4	&	\failureResult{NA}	&	\failureResult{NA}	&	\failureResult{NA}	&	\highlightedResult{20.56}	&	\failureResult{Q}	&	\failureResult{Q} \\
\nonCorbett{counter}	&	8	&	\highlightedResult{7.51}	&	\failureResult{/}	&	\failureResult{/}	&	21.36	&	\failureResult{Q}	&	\failureResult{Q} \\
\nonCorbett{counter}	&	16	&	\highlightedResult{7.51}	&	\failureResult{/}	&	\failureResult{/}	&	22.75	&	\failureResult{Q}	&	\failureResult{Q} \\
\nonCorbett{counter}	&	32	&	\highlightedResult{7.51}	&	\failureResult{/}	&	\failureResult{/}	&	28.59	&	\failureResult{Q}	&	\failureResult{Q} \\
\nonCorbett{counter}	&	64	&	\highlightedResult{8.60}	&	\failureResult{/}	&	\failureResult{/}	&	28.60	&	\failureResult{Q}	&	\failureResult{Q} \\
hartstone	&	2	&	\failureResult{NA}	&	\failureResult{NA}	&	\failureResult{NA}	&	\highlightedResult{19.33}	&	\failureResult{/}	&	\failureResult{/} \\
hartstone	&	4	&	\failureResult{NA}	&	\failureResult{NA}	&	\failureResult{NA}	&	\highlightedResult{19.61}	&	\failureResult{/}	&	\failureResult{/} \\
hartstone	&	8	&	\failureResult{NA}	&	\failureResult{NA}	&	\failureResult{NA}	&	\highlightedResult{21.41}	&	\failureResult{/}	&	\failureResult{/} \\
hartstone	&	10	&	\failureResult{NA}	&	\failureResult{NA}	&	\failureResult{NA}	&	\highlightedResult{20.62}	&	\failureResult{/}	&	\failureResult{/} \\
hartstone	&	13	&	\failureResult{NA}	&	\failureResult{NA}	&	\failureResult{NA}	&	\highlightedResult{25.74}	&	\failureResult{/}	&	\failureResult{/} \\
hartstone	&	16	&	\failureResult{NA}	&	\failureResult{NA}	&	\failureResult{NA}	&	\highlightedResult{28.65}	&	\failureResult{/}	&	\failureResult{/} \\
token-ring	&	2	&	7.51	&	33.20	&	37.98	&	17.99	&	\highlightedResult{2.11}	&	507.24 \\
token-ring	&	8	&	7.51	&	39.96	&	89.81	&	19.79	&	\highlightedResult{2.18}	&	507.24 \\
token-ring	&	10	&	\failureResult{NA}	&	\failureResult{NA}	&	\failureResult{NA}	&	21.33	&	\highlightedResult{2.31}	&	507.50 \\
token-ring	&	16	&	318.08	&	\failureResult{T}	&	\failureResult{T}	&	21.11	&	\highlightedResult{2.42}	&	507.75 \\
token-ring	&	32	&	\failureResult{M}	&	\failureResult{T}	&	\failureResult{T}	&	25.22	&	\highlightedResult{2.50}	&	508.36 \\
token-ring	&	64	&	\failureResult{M}	&	\failureResult{T}	&	\failureResult{T}	&	28.65	&	\highlightedResult{2.61}	&	508.78 \\
cyclic	&	2	&	\failureResult{NA}	&	\failureResult{NA}	&	\failureResult{NA}	&	16.81	&	\highlightedResult{10.88}	&	507.24 \\
cyclic	&	4	&	\failureResult{NA}	&	\failureResult{NA}	&	\failureResult{NA}	&	19.52	&	\highlightedResult{16.17}	&	507.50 \\
cyclic	&	8	&	\failureResult{NA}	&	\failureResult{NA}	&	\failureResult{NA}	&	\highlightedResult{20.57}	&	27.29	&	510.95 \\
cyclic	&	10	&	\failureResult{NA}	&	\failureResult{NA}	&	\failureResult{NA}	&	\highlightedResult{20.59}	&	32.77	&	520.13 \\
cyclic	&	13	&	\failureResult{NA}	&	\failureResult{NA}	&	\failureResult{NA}	&	\highlightedResult{22.24}	&	40.95	&	603.45 \\
cyclic	&	16	&	\failureResult{NA}	&	\failureResult{NA}	&	\failureResult{NA}	&	\highlightedResult{24.60}	&	48.96	&	1041.87 \\
disjunction-tree	&	1,1	&	\failureResult{NA}	&	\failureResult{NA}	&	\failureResult{NA}	&	15.03	&	\highlightedResult{5.56}	&	506.98 \\
disjunction-tree	&	2,2	&	\failureResult{NA}	&	\failureResult{NA}	&	\failureResult{NA}	&	\highlightedResult{15.63}	&	\failureResult{/}	&	\failureResult{/} \\
disjunction-tree	&	3,3	&	\failureResult{NA}	&	\failureResult{NA}	&	\failureResult{NA}	&	\highlightedResult{32.45}	&	\failureResult{/}	&	\failureResult{/} \\
disjunction-tree	&	4,4	&	\failureResult{NA}	&	\failureResult{NA}	&	\failureResult{NA}	&	\failureResult{T}	&	\failureResult{/}	&	\failureResult{/} \\
disjunction-tree	&	5,5	&	\failureResult{NA}	&	\failureResult{NA}	&	\failureResult{NA}	&	\failureResult{T}	&	\failureResult{/}	&	\failureResult{M} \\
disjunction-tree	&	6,6	&	\failureResult{NA}	&	\failureResult{NA}	&	\failureResult{NA}	&	\failureResult{T}	&	\failureResult{M}	&	\failureResult{M} \\
\hline
\end{tabular}


Inspecting the values in \tabref{tab:fasttimings} and contrasting with those in
\tabref{tab:slowtimings}, several points are clear:

\begin{itemize}
    \item For the majority of examples, the performance has \emph{vastly}
        improved; indeed, for several systems (e.g. \overtakeSys{-}), the time
        taken by the improved algorithm for parameter 32768 is less
        than the naive algorithm for paremeter 1.
    \item Indeed, we can observe the effects of reaching fixed points --- for
        most systems (e.g. \bufferSys{-}), the time/memory use is essentially
        constant\footnote{Our tool is written in Haskell, a language that uses
            garbage collection for automatic memory management, which may
            explain some of the slight variance in memory usage.}
    \item The performance is \emph{not} directly proportional to net size:
        indeed, \overtakeSys{-} is a ``large'' system, with scalable
        performance, whereas \counterSys{-} is a ``small'' system, with poor
        performance.
\end{itemize}

To help interpreting our results, we have plotted the data in
\figref{fig:penrosetimes}.

\begin{figure}[ht]
\centering
\begin{tikzpicture}[trim axis left, trim axis right]
\pgfplotsset{width=\textwidth}
\newcommand{\datadir}{chapters/improveEfficiency/individual_bench_data}
\newcommand{\nameprefix}{penrose_data}
\begin{loglogaxis}[
    cycle list name=exotic,
    xlabel={Problem Size},
    ylabel={Time (s)},
    grid=major,
    log basis x=2,
    legend style={%
        overlay,
        at={(1,1)},
        anchor=north east},
]
\foreach \name in
{disjunction-tree,token-ring,counter,hartstone,philo,over,iterated-choice,replicator,dac,buffer,conjunction-tree}{%
\addplot+ [smooth] table {\datadir/\nameprefix_\name.tsv};
\addlegendentryexpanded{\name}
}
\end{loglogaxis}
\end{tikzpicture}
\caption{Time vs Problem size for \algref{alg:thealgorithm}}
\label{fig:penrosetimes}
\end{figure}
