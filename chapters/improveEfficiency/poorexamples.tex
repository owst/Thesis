\label{sec:poorexamples}
An immediate problem with examples that do not reach fixed points is the
monotonic size increase of the \TNFA{}-composition cache. One might imagine
that expiring old cache entries (for example, using a least-recently used (LRU)
strategy) would improve performance; however, if there is also a monotonic
increase in the \emph{size} of the \TNFA{}s being composed, the cost of
checking equivalence and minimisation will dominate.

Indeed, growth of \TNFA{}s does not have to be monotonic. In several examples,
the growth is ``intermediate'' in the sense that large \TNFA{}s are generated,
only to be quotiented later.
For example, in
the \tokenringSys{-} system (\secref{sec:example-token-ring}), intermediate
growth is caused by the sequence of \workerC{} components being able to receive
multiple tokens. However, when composing the \workerC{}s components with the
\injectorC{}, which can only input a single token into the system, the
behaviour is quotiented, vastly reducing the size of the resulting \TNFA{}.

Another example of a system exhibiting intermediate growth is the
\distree{\aN}{\bN} system, with depth $\aN$ and width $\bN$
(\secref{sec:example-trees}); consider the target marking that places a token
in each leaf place, having started with only a single token in the root place.
Intuitively, such a target marking is not reachable: at each internal
node of the tree, we can only pass a token to a single sub-tree, and thus we
can't pass tokens down to every leaf from the root. However, until the root
node is composed with the rest of the tree, the behaviour of the sub-tree is not
constrained to only receive a single token and thus has behaviour that allows
it to reach its target marking by receiving a token for each leaf component.

Indeed, the \distree{3}{3} system, illustrated in \figref{fig:distree33}, has a
component-wise specification (given in \secref{sec:example-trees}) that builds
the composite net by recursively building larger complete sub-trees. As an
detailed example, let us consider the \TNFA{}s obtained during such a
procedure.

\begin{figure}[ht]
    \centering
    \scalebox{0.8}{
\begin{tikzpicture}[pnb, node distance=2.2cm and 0.7cm]
    \node (1) [pnbplace, wantedTokenPlace, rotate=-90] {};

    \foreach \x in {2,...,27}{
        \pgfmathtruncatemacro{\prev}{\x - 1}
        \node (\x) [pnbplace, wantedTokenPlace, left=of \prev, rotate=-90] {};
    }

    \foreach \x in {2,5,...,26}{
        \pgfmathtruncatemacro{\prev}{\x - 1}
        \pgfmathtruncatemacro{\next}{\x + 1}
        \node (lvl1\x) [pnbplace, above=of \x, rotate=-90] {};

        \foreach \to in {\prev, \x, \next}{
            \draw (lvl1\x.out) edge [pnbarr,out=-90, in=90] (\to.in);
        }
    }

    \foreach \x in {5,14,23}{
        \pgfmathtruncatemacro{\prev}{\x - 3}
        \pgfmathtruncatemacro{\next}{\x + 3}
        \node (lvl2\x) [pnbplace, above=of lvl1\x, rotate=-90] {};

        \foreach \to in {\prev, \x, \next}{
            \draw (lvl2\x.out) edge [pnbarr,out=-90, in=90] (lvl1\to.in);
        }
    }

    \node (lvl314) [pnbplace, tokens=1, above=of lvl214, rotate=-90] {};

    \foreach \to in {5,14,23}{
        \draw (lvl314.out) edge [pnbarr,out=-90, in=90] (lvl2\to.in);
    }
\end{tikzpicture}
}
\caption{\distree{3}{3}}
\label{fig:distree33}
\end{figure}

The leaf sub-tree is illustrated in \figref{fig:leavesSubtree}; first, the
component PNBs are converted to their \TNFA{} semantics, as per
\figref{fig:leafComponentTNFAs}. Then, \TNFA{} compositions are performed, with
reduction up to weak-language being performed after each composition. The first
composition leads to the \TNFA{}\footnote{Here, we use $\aLbl$ to stand for
either 0 or 1 --- N.B. that $\lbl{\aLbl}{\aLbl} = \setof{\lbl{0}{0},
\lbl{1}{1}} \neq \setof{\lbl{0}{0}, \lbl{0}{1}, \lbl{1}{0}, \lbl{1}{1}} =
\lbl{*}{*}$.} that is not in \figref{fig:firstLeafCompositionTNFA}; notice
that the leaf components can reach their target marking ``in either order''
since tokens can either be consumed by a leaf, or passed to the next. Indeed,
it is this lack of order that leads to a blowup in \TNFA{} size. Therefore, a
simpler, language-equivalent \TNFA{} can be constructed, which disregards the
internal state of the composition.

\newcommand{\treeTree}[2]{%
\begin{tikzpicture}[wiringTree]
\Tree
    [. \node[draw, circle]{$\comp$};
        [. \node[draw, circle]{$\comp$};
            [. \node[draw, circle]{$\comp$};
                    [. \node[draw] {#1}; ]
                    [. \node[draw] {#1}; ]
            ]
            [. \node[draw] {#1}; ]
        ]
        [. \node[draw] {#2}; ]
    ]
\end{tikzpicture}
}

\makesavebox{\markedDisLeafBox}{%
    \disLeaf{1}
}

\begin{figure}[ht]
    \centering
    \treeTree{\usescalebox[0.8]{\markedDisLeafBox}}{\usescalebox[0.8]{\rtermOneBox}}
\caption{Leaves sub-tree of \distree{3}{3}}
\label{fig:leavesSubtree}
\end{figure}

\makesavebox{\markedDisLeafTNFABox}{%
    \begin{tikzpicture}[nfa]
        \node (0) [state, init] {};
        \node (1) [state, accepting, below=of 0] {};

        \nfaArr[loop left]{0}{0}{\aLbl}{\aLbl}
        \nfaArr{0}{1}{1}{0}
        \nfaArr[loop left]{1}{1}{\aLbl}{\aLbl}
    \end{tikzpicture}
}

\makesavebox{\rtermOneTNFABox}{%
\begin{tikzpicture}[nfa]
    \node (0) [state, init, accepting] {};

    \nfaArr[loop left]{0}{0}{0}{}
\end{tikzpicture}
}

\begin{figure}[ht]
    \centering
    \begin{tikzpicture}
        \matrix (m)[every node/.style={anchor=center}, matrix of nodes, column sep=2cm, row sep=0.5cm] {
        \usebox\markedDisLeafBox & \usebox\markedDisLeafTNFABox \\
        \usebox\rtermOneBox & \usebox\rtermOneTNFABox \\
        };
        \translateArr{m-1-1}{m-1-2}
        \translateArr{m-2-1}{m-2-2}
    \end{tikzpicture}
    \caption{\TNFA{} semantics of leaf components}
    \label{fig:leafComponentTNFAs}
\end{figure}

\newcommand{\gooTNFA}[1]{
    \begin{tikzpicture}[nfa]
        \node (0) [state, initial where=left, init] {};

        \nfaArr[loop above]{0}{0}{\aLbl}{\aLbl}

        \foreach \x in {1,...,#1}{%
            \ifthenelse{\x = #1}{
                \tikzset{wanted style/.style=accepting}
            }{}
            \pgfmathtruncatemacro{\prev}{\x - 1}
            \node (\x) [state, wanted style, right=of \prev] {};

            \nfaArr{\prev}{\x}{1}{0}
            \nfaArr[loop above]{\x}{\x}{\aLbl}{\aLbl}
        }
    \end{tikzpicture}
}

\newcommand{\fooTNFA}[1]{
    \begin{tikzpicture}[nfa]
        \node (0) [state, initial where=left, init] {};

        \nfaArr[loop above]{0}{0}{0}{}

        \foreach \x in {1,...,#1}{%
            \ifthenelse{\x = #1}{
                \tikzset{wanted style/.style=accepting}
            }{}
            \pgfmathtruncatemacro{\prev}{\x - 1}
            \node (\x) [state, wanted style, right=of \prev] {};

            \nfaArr{\prev}{\x}{1}{}
            \nfaArr[loop above]{\x}{\x}{0}{}
        }
    \end{tikzpicture}
}

\makesavebox{\twoDisLeafBox}{%
    \begin{tikzpicture}[nfa]
        \node (0) [state, init] {};
        \node (1) [state, below left=of 0] {};
        \node (2) [state, below right=of 0] {};
        \node (3) [state, accepting, below left=of 2] {};

        \nfaArr[swap]{0}{1}{1}{0}
        \nfaArr{0}{2}{1}{0}
        \nfaArr[swap]{1}{3}{1}{0}
        \nfaArr{2}{3}{1}{0}

        \foreach \x in {0,1,3}{%
            \nfaArr[loop left]{\x}{\x}{\aLbl}{\aLbl}
        }
        \nfaArr[loop right]{2}{2}{\aLbl}{\aLbl}
    \end{tikzpicture}
}

\makesavebox{\twoDisLeafReducedBox}{%
    \gooTNFA{2}
}

\makesavebox{\threeDisLeafBox}{%
    \begin{tikzpicture}[nfa]
        \node (0) [state, init] {};
        \node (1) [state, below right=of 0] {};
        \node (2) [state, below left=of 0] {};
        \node (4) [state, below=of 2] {};
        \node (3) [state, below=of 1] {};
        \node (5) [state, accepting, below left=of 3] {};

        \nfaArr{0}{1}{1}{0}
        \nfaArr[swap]{0}{2}{1}{0}
        \nfaArr{1}{3}{1}{0}
        \nfaArr[swap]{2}{3}{1}{0}
        \nfaArr[swap]{2}{4}{1}{0}
        \nfaArr[swap]{4}{5}{1}{0}
        \nfaArr{3}{5}{1}{0}

        \foreach \x in {0,2,4,5}{%
            \nfaArr[loop left]{\x}{\x}{\aLbl}{\aLbl}
        }
        \foreach \x in {1,3}{%
            \nfaArr[loop right]{\x}{\x}{\aLbl}{\aLbl}
        }
    \end{tikzpicture}
}

\makesavebox{\threeDisLeafReducedBox}{%
    \gooTNFA{3}
}

\makesavebox{\leavesBox}{%
    \fooTNFA{3}
}

\begin{figure}[ht]
    \centering
    \begin{tikzpicture}
        \node (1)              {\usebox\twoDisLeafBox};
        \node (2) [below=of 1] {\usebox\twoDisLeafReducedBox};

        \translateArr{1}{2}
    \end{tikzpicture}
    \caption{\TNFA{} composition of two leaf components}
    \label{fig:firstLeafCompositionTNFA}
\end{figure}

The reduced \TNFA{} for two leaf components is then composed with the
\TNFA{} of a single leaf component, leading to the \TNFA{} illustrated in
\figref{fig:secondLeafCompositionTNFA}. Again, the \TNFA{} can be simplified,
as shown.

\begin{figure}[ht]
    \centering
    \begin{tikzpicture}
        \node (1)              {\usebox\threeDisLeafBox};
        \node (2) [below=of 1] {\usebox\threeDisLeafReducedBox};

        \translateArr{1}{2}
    \end{tikzpicture}
    \caption{\TNFA{} composition of three leaf components}
    \label{fig:secondLeafCompositionTNFA}
\end{figure}

The final composition for the leaf sub-tree composes the reduced three-leaf
\TNFA{} with the \TNFA{} of the terminator, leading to the \TNFA{} shown in
\figref{fig:leavesSubtreeTNFA}, which is the \TNFA{} representing the
PNB-expression sub-tree of three leaf components.

\begin{figure}[ht]
    \centering
    \usebox\leavesBox
    \caption{\TNFA{} semantics of a leaves sub-tree}
    \label{fig:leavesSubtreeTNFA}
\end{figure}

\makesavebox{\idOneBox}{%
\begin{tikzpicture}[pnb]
    \drawBoundaries[1][2]{1}{1}

    \labelledpnbarr{l1}{r1}{}{barr}{}
\end{tikzpicture}
}

The deepest internal node sub-tree has the structure shown in
\figref{fig:deepestInternalNodeSubtreeStructure}; \texttt{leaves} is the
sub-tree of \figref{fig:leavesSubtree}. Again the component PNBs are converted
to their \TNFA{} semantics, shown in \figref{fig:internalComponentTNFAs}. The
compositions lead to intermediate \TNFA{}s shown in \figref{fig:tensorSubtree}
and \figref{fig:deepestInternalNodeSubtreeTNFA}. Indeed this \TNFA{} shows that
the boundary protocol of the deepest internal-node sub-tree simple requires
three tokens to be inserted to the sub-tree (and internally, passed to the
leaves), while passing additional tokens to the sub-tree's next sibling.

\begin{figure}[ht]
    \centering
\begin{tikzpicture}[wiringTree]
\Tree
    [. \node[draw, circle]{$\comp$};
        [. \node[draw] {\usebox\disNodeBox}; ]
        [. \node[draw, circle]{$\tensor$};
                [. \node[draw] {\usebox\idOneBox}; ]
                [. \node[draw] {\texttt{leaves}}; ]
        ]
    ]
\end{tikzpicture}
\caption{Deepest internal-node sub-tree structure}
\label{fig:deepestInternalNodeSubtreeStructure}
\end{figure}

\makesavebox{\disNodeTNFABox}{%
    \begin{tikzpicture}[nfa]
        \node (0) [state, init, accepting] {};
        \node (1) [state, below=of 0] {};

        \nfaArr[loop left]{0}{0}{\aLbl}{\aLbl0}
        \nfaArr[bend left]{0}{1}{1}{00}
        \nfaArr[loop left]{1}{1}{\aLbl}{\aLbl0}
        \nfaArr[bend left]{1}{0}{\aLbl}{\aLbl1}
    \end{tikzpicture}
}

\makesavebox{\idOneTNFABox}{%
    \begin{tikzpicture}[nfa]
        \node (0) [state, init, accepting] {};
        \nfaArr[loop left]{0}{0}{\aLbl}{\aLbl}
    \end{tikzpicture}
}

\begin{figure}[ht]
    \centering
    \begin{tikzpicture}[node distance=1.5cm]
        \matrix (m)[every node/.style={anchor=center}, matrix of nodes, column sep=2cm, row sep=0.5cm] {
        \usebox\disNodeBox & \usebox\disNodeTNFABox \\
        \usebox\idOneBox & \usebox\idOneTNFABox \\
        };

        \translateArr{m-1-1}{m-1-2}
        \translateArr{m-2-1}{m-2-2}
    \end{tikzpicture}
    \caption{\TNFA{} semantics of internal components}
    \label{fig:internalComponentTNFAs}
\end{figure}

\begin{figure}[ht]
    \centering
    \begin{tikzpicture}[nfa]
        \node (0) [state, initial where=left, init] {};

        \nfaArr[loop above]{0}{0}{\aLbl0}{\aLbl}

        \foreach \x in {1,2,3}{%
            \ifthenelse{\x = 3}{
                \tikzset{wanted style/.style=accepting}
            }{}
            \pgfmathtruncatemacro{\prev}{\x - 1}
            \node (\x) [state, wanted style, right=of \prev] {};

            \nfaArr{\prev}{\x}{\aLbl1}{\aLbl}
            \nfaArr[loop above]{\x}{\x}{\aLbl0}{\aLbl}
        }
    \end{tikzpicture}
    \caption{\TNFA{} of tensor sub-tree shown from
    \figref{fig:deepestInternalNodeSubtreeStructure}}
    \label{fig:tensorSubtree}
\end{figure}

\makesavebox{\nodeSubTreeTNFABox}{%
    \begin{tikzpicture}[nfa]
        \node (0) [state, initial where=left, init] {};

        \nfaArr[loop above]{0}{0}{\aLbl}{\aLbl}

        \foreach \x in {1,3,5}{%
            \ifthenelse{\x = 5}{
                \tikzset{wanted style/.style=accepting}
            }{}
            \pgfmathtruncatemacro{\prev}{\x - 1}
            \pgfmathtruncatemacro{\next}{\x + 1}

            \node (\x) [state, right=of \prev] {};
            \node (\next) [state, wanted style, right=of \x] {};

            \nfaArr{\prev}{\x}{1}{0}
            \nfaArr{\x}{\next}{\aLbl}{\aLbl}

            \nfaArr[loop above]{\x}{\x}{\aLbl}{\aLbl}
            \nfaArr[loop above]{\next}{\next}{\aLbl}{\aLbl}
        }

        \node (7) [state, right=of 6] {};
        \nfaArr{6}{7}{1}{0}
        \nfaArr[loop above]{7}{7}{\aLbl}{\aLbl}
    \end{tikzpicture}
}

\makesavebox{\nodeSubTreeReducedTNFABox}{%
    \gooTNFA{3}
}

\begin{figure}[ht]
    \centering
    \begin{tikzpicture}
        \node (1)              {\scalebox{0.75}{\usebox\nodeSubTreeTNFABox}};
        \node (2) [below=of 1] {\usebox\nodeSubTreeReducedTNFABox};

        \translateArr{1}{2}
    \end{tikzpicture}
    \caption{\TNFA{} semantics of deepest internal-node sub-tree}
    \label{fig:deepestInternalNodeSubtreeTNFA}
\end{figure}

Three deepest internal-node sub-trees are composed, as per
\figref{fig:compositionOfDeepestInternalSubtrees}, where \texttt{dins} is the
sub-tree shown in \figref{fig:deepestInternalNodeSubtreeStructure}.

Indeed, the reduced \TNFA{} shown in
\figref{fig:deepestInternalNodeSubtreeTNFA} has a repeated structure similar to
that of a single leaf component, albeit requiring 3 transitions to reach the
accepting state. Thus, one would expect that composing two such \TNFA{}s in
order to perform the compositions of
\figref{fig:compositionOfDeepestInternalSubtrees} will lead to similar blow up
in state space.

\begin{figure}[ht]
    \centering
    \treeTree{\texttt{dins}}{\usescalebox[0.8]{\rtermOneBox}}
    \caption{Composition of deepest internal node sub-trees}
    \label{fig:compositionOfDeepestInternalSubtrees}
\end{figure}

\makesavebox{\twoSubTreesTNFABox}{%
    \begin{tikzpicture}[nfa]
        \node (00) [state, init] {};
        \nfaArr[loop left]{00}{00}{\aLbl}{\aLbl}

        \nodeRow{1}{1}
        \nodeRow{2}{2}
        \nodeRow{3}{3}
        \nodeRow[0]{4}{2}
        \nodeRow[0]{5}{1}

        \node (60) [state, accepting, below right=of 50] {};
        \nfaArr[loop left]{60}{60}{\aLbl}{\aLbl}
        \nfaArr{50}{60}{1}{0}
        \nfaArr{51}{60}{1}{0}
    \end{tikzpicture}
}

\makesavebox{\twoSubTreesReducedTNFABox}{%
    \gooTNFA{6}
}

\begin{figure}[ht]
    \centering
    \begin{tikzpicture}
        \node (1)              {\scalebox{0.75}{\usebox\twoSubTreesTNFABox}};
        \node (2) [below=of 1] {\scalebox{0.75}{\usebox\twoSubTreesReducedTNFABox}};

        \translateArr{1}{2}
    \end{tikzpicture}
    \caption{\TNFA{} semantics of deepest internal-node sub-tree composition}
    \label{fig:internalNodeComposition}
\end{figure}

Indeed, this is the case: with the first composition leading to the \TNFA{}
illustrated in \figref{fig:internalNodeComposition}; again, all possible
orderings of the sub-trees reaching their target markings are represented in the
composite \TNFA{}.

\makesavebox{\threeSubTreesTNFABox}{%
    \begin{tikzpicture}[nfa]
        \node (00) [state, init] {};
        \nfaArr[loop left]{00}{00}{\aLbl}{\aLbl}

        \nodeRow{1}{1}
        \nodeRow{2}{2}
        \nodeRow{3}{3}
        \nodeRow[2]{4}{3}
        \nodeRow[2]{5}{3}
        \nodeRow[2]{6}{3}
        \nodeRow[0]{7}{2}
        \nodeRow[0]{8}{1}

        \node (90) [state, accepting, below right=of 80] {};
        \nfaArr[loop left]{90}{90}{\aLbl}{\aLbl}
        \nfaArr{80}{90}{1}{0}
        \nfaArr{81}{90}{1}{0}
    \end{tikzpicture}
}

\makesavebox{\threeSubTreesReducedTNFABox}{%
    \gooTNFA{9}
}

\begin{figure}[ht]
    \centering
    \begin{tikzpicture}
        \node (1)              {\scalebox{0.75}{\usebox\threeSubTreesTNFABox}};
        \node (2) [below=of 1] {\scalebox{0.5}{\usebox\threeSubTreesReducedTNFABox}};

        \translateArr{1}{2}
    \end{tikzpicture}
    \caption{\TNFA{} semantics of three composed internal-node sub-trees}
    \label{fig:threeInternalNodeSubtreesTNFA}
\end{figure}

Composing the reduced \TNFA{} of \figref{fig:internalNodeComposition}
with another internal-node \TNFA{} leads to the \TNFA{} shown in
\figref{fig:threeInternalNodeSubtreesTNFA}. Indeed, such a \TNFA{} represents the
semantics of the sub-tree containing three internal nodes (and thus, 9 leaf
components), as can be witnessed in the language of the reduced \TNFA{}.

The preceding pattern of large composite \TNFA{}s being generated, that are
later reduced, repeats for the final internal-node level of \distree{3}{3},
with the final composition leading to a (unminimised) \TNFA{} of $190$ nodes.
We show the intermediate \TNFA{} sizes in \tblref{tbl:productSizesDisTree}: for
any sub tree containing $\aN$ leaf components the minimised \TNFA{} semantics
will have $\aN + 1$ states. Only when composing with the root, does the \TNFA{}
drastically in size: since the root can only perform a single
$\lbl{}{1}$-labelled transition, thus the composite \TNFA{} only has 2 states,
neither of which are accepting, as illustrated in
\figref{fig:TNFARootComposed}. Therefore, the reduced final \TNFA{} only has a
single non-accepting state, indicating the target marking is not reachable.

\begin{figure}[ht]
    \centering
    \begin{tikzpicture}[nfa]
        \node (0) [state, init] {};
        \node (1) [state, below=of 0] {};

        \nfaArr[loop left]{0}{0}{}{}
        \nfaArr{0}{1}{}{}
        \nfaArr[loop left]{1}{1}{}{}
    \end{tikzpicture}
    \caption{\TNFA{} after composing with the root component}
    \label{fig:TNFARootComposed}
\end{figure}

\begin{table}
    \centering
    \newcommand{\hilight}[1]{\textbf{#1}}
\begin{tabular}{| r c r c | r | c | c |}
    \hline
    \multicolumn{5}{|c|}{\textbf{Product Size}} & \textbf{Minimised NFA size} &
    \textbf{Sub-tree level} \\
    \hline
    $2$ &$\times$ &$2$ &= &\hilight{4} &3 & \multirow{2}{*}{Leaf level} \\
    $3$ &$\times$ &$2$ &= &\hilight{6} &4 &\\
    \hline
    $4$ &$\times$ &$4$ &= &\hilight{16} &7 & \multirow{2}{*}{1st Node Level} \\
    $7$ &$\times$ &$4$ &= &\hilight{28} &10 & \\
    \hline
    $10$ &$\times$ &$10$ &= &\hilight{100} &19 & \multirow{2}{*}{2nd Node Level} \\
    $19$ &$\times$ &$10$ &= &\hilight{190} &28 & \\
    \hline
    $28$ &$\times$ &$2$ &= &\hilight{2} &1 & Root Level \\
    \hline
\end{tabular}
\caption{Intermediate Product Sizes for \distree{3}{3}}
\label{tbl:productSizesDisTree}
\end{table}
