\chapter{Categorical Structure}\label{chp:catStructure}

In this chapter we show that both PNBs and \TLTS{}s have a natural categorical structure. We
consider the categories of PNBs, \PNBCat, and \TLTS{}s, \TLTSCat. Our aim is to show that these
categories are in fact Product and Permutation Categories, known as PROPs, and that there is a
semantic functor $\PNBToTLTS{-} : \text{\PNBCat} \to \text{\TLTSCat}$ taking a \PNB{} to its
\TLTS{} statespace, preserving the PROP structure. By elucidating the PROP structure we show that
the two types of composition for PNBs/\TLTS{}s enjoy desirable properties (associativity,
identities), while functoriality ensures that the compositions are preserved when taking the
semantics of components, in a precise sense. Indeed, we demonstrate that the property of
\emph{compositionality}---that the semantics of a composite PNB is determined by the semantics of
the components---is simply an instance of functoriality.

\section{Preliminaries}

We briefly recall the requisite definitions of strict symmetric monoidal
categories and functors. For an in-depth introduction to symmetric monoidal
categories, and their intuitive graphical presentation,
see~\cite{Selinger2009}.

\begin{definition}[Category]
    A (locally small) category, $\aCat$, is comprised of:
    \begin{itemize}
        \item A collection of objects, $\obj{\aCat}$,
        \item A collection of morphisms; for each pair of objects, $\oA,
            \oB$ we have a set of morphisms between those objects, called
            their hom-set, written: $\arr{\aCat}{\oA}{\oB}$,
        \item For every object, $\oA$, we have the identity morphism:
            $\idArr{\oA} \in \arr{\aCat}{\oA}{\oA}$,
        \item For each $\aA \in \arr{\aCat}{\oA}{\oB}$ and
            $\aB \in \arr{\aCat}{\oB}{\oC}$, there is a
            composite morphism: $\aA \comp \aB \in
            \arr{\aCat}{\oA}{\oC}$.
    \end{itemize}

    Subject to the following:
    \begin{align*}
        \aA \comp \idArr{\oB} &= \aA = \idArr{\oA} \comp
        \aA \\
        \aA \comp \parens{\aB \comp \aC} &=
        \parens{\aA \comp \aB} \comp \aC
    \end{align*}
    for any $\oA, \oB, \oC, \oD \in \obj{\aCat}$, and
    $\aA \in \arr{\aCat}{\oA}{\oB}$, $\aB \in
    \arr{\aCat}{\oB}{\oC}$ and $\aC \in
    \arr{\aCat}{\oC}{\oD}$.
\end{definition}

\begin{definition}[Isomorphisms]
    A morphism $\aA \in \arr{\aCat}{\oA}{\oB}$ is an isomorphism
    if there exists $\aA^{-1} \in \arr{\aCat}{\oB}{\oA}$ subject
    to the conditions:
    \begin{align*}
        \aA \comp \aA^{-1} &= \idArr{\oA}\\
        \aA^{-1} \comp \aA &= \idArr{\oB}
    \end{align*}
\end{definition}

\begin{definition}[Product Category]
    For categories $\aCat$ and $\bCat$, we can lift their operations pointwise
    to form the product category, $\prodCat$, which has the following
    structure:
    \begin{align*}
        \obj{\prodCat} &= \obj\aCat \times \obj\bCat\\
        \arr{\prodCat}{\pairof{\oA}{\oB}}{\pairof{\oC}{\oD}}
        &= \arr\aCat\oA\oC \times \arr\bCat\oB\oD\\
        \idArr{\pairof{\oA}{\oB}} &=
        \pairof{\idArr{\oA}}{\idArr{\oB}} \\
        \pairof{\aA}{\aB} \comp \pairof{\aC}{\aD} &=
        \pairof{\aA \comp \aC}{\aB \comp \aD}
    \end{align*}
\end{definition}

\begin{definition}[Functor]
    A functor is a mapping between categories, $\aFunctor : \aCat \to \bCat$,
    subject to the conditions:
    \begin{align*}
        \aFunctor\parens{\idArr{\oA}} &=
        \idArr{\aFunctor\parens{\oA}}\\
        \aFunctor(\aA \comp \aB) &= \aFunctor\parens{\aA} \comp
        \aFunctor\parens{\aB}
    \end{align*}
\end{definition}

\begin{definition}[Bifunctor]
    A bifunctor is a functor whose domain category is a product category.
\end{definition}

\begin{definition}[Natural Transformation]
A natural transformation is a mapping between functors, $\natTrans : \aFunctor \to \bFunctor$,
comprised of a morphism, called a component, $\natTrans\parens{\oA} : \aFunctor\parens{\oA} \to
\bFunctor\parens{\oA} $, for each $\oA \in \obj{\aCat}$, such that for any $\aA \in
\arr{\aCat}{\oA}{\oB}$, the diagram in~\figref{fig:natTrans} commutes.
\end{definition}

\begin{figure}[ht]
    \centering
\begin{tikzpicture}[node distance=6cm]
\node[regular polygon, regular polygon sides=4, minimum size=6cm] (A) {};
\node at (A.corner 2) {$\aFunctor\parens{\oA}$};
\node at (A.corner 1) {$\aFunctor\parens{\oB}$};
\node at (A.corner 3) {$\bFunctor\parens{\oA}$};
\node at (A.corner 4) {$\bFunctor\parens{\oB}$};

\draw (A.corner 2) edge [commarr] node [label=left:$\natTrans\parens{\oA}$] {} (A.corner 3);
\draw (A.corner 1) edge [commarr] node [label=right:$\natTrans\parens{\oB}$] {} (A.corner 4);
\draw (A.corner 2) edge [commarr] node [label=above:$\aFunctor\parens{\aA}$] {} (A.corner 1);
\draw (A.corner 3) edge [commarr] node [label=below:$\bFunctor\parens{\aA}$] {} (A.corner 4);
\end{tikzpicture}
\caption{Natural transformation commuting diagram}
\label{fig:natTrans}
\end{figure}

\begin{definition}[Natural Isomorphism]
    A natural transformation, $\natTrans$, is a natural isomorphism if every
    component is an isomorphism.
\end{definition}

\begin{definition}[Monoidal Category]\label{defn:monCat}
    A monoidal category is a 6-tuple, $(\aCat, \tensor, \tensorUnit,
    \tensorAssoc, \tensorLeftID, \tensorRightID)$, where:
    \begin{itemize}
        \item $\aCat$ is a category,
        \item $\tensor : \aCat \times \aCat \to \aCat$ is a bifunctor, called
            tensor,
        \item $\tensorUnit \in \obj{\aCat}$ is the tensor unit,
        \item $\tensorAssocArgs{\oA}{\oB}{\oC} : (\oA \tensor \oB) \tensor \oC \to \oA \tensor (\oB
            \tensor \oC)$ is the associativity natural isomorphism,
        \item $\tensorLeftIDArg{\oA} : \tensorUnit \tensor \oA \to \oA$
            ($\tensorRightIDArg{\oA} : \oA \tensor \tensorUnit \to \oA $) are the
            left (right) identity natural isomorphisms.
    \end{itemize}
    Subject to the pentagon and triangle conditions, illustrated in~\figref{fig:pentagon}
    and~\figref{fig:triangle}.
\end{definition}

\begin{figure}[ht]
\centering
\begin{tikzpicture}
\node[regular polygon, regular polygon sides=5, minimum size=6cm, xscale=1.5] (A) {};
\node at (A.corner 2) {$((\oA \tensor \oB) \tensor \oC) \tensor \oD$};
\node at (A.corner 1) {$(\oA \tensor \oB) \tensor (\oC \tensor \oD)$};
\node at (A.corner 3) {$(\oA \tensor (\oB \tensor \oC)) \tensor \oD$};
\node at (A.corner 4) {$\oA \tensor ((\oB \tensor \oC)) \tensor \oD)$};
\node at (A.corner 5) {$\oA \tensor (\oB \tensor (\oC \tensor \oD))$};
\draw (A.corner 2) edge [commarr] node [label=left:$\tensorAssocArgs{\oA}{\oB}{\oC} \tensor \idArr{\oD}$] {} (A.corner 3);
\draw ($(A.corner 3)+(1.25,0)$) edge [commarr] node [label=below:$\tensorAssocArgs{\oA}{\parens{\oB
\tensor \oC}}{\oD}$] {} ($(A.corner 4)+(-1.25,0)$);
\draw (A.corner 4) edge [commarr] node [label=right:$\idArr{\oA} \tensor \tensorAssocArgs{\oB}{\oC}{\oD}$] {} (A.corner 5);
\draw (A.corner 2) edge [commarr] node [label=above left:$\tensorAssocArgs{\parens{\oA \tensor \oB}}{\oC}{\oD}$] {} (A.corner 1);
\draw (A.corner 1) edge [commarr] node [label=above right:$\tensorAssocArgs{\oA}{\oB}{\parens{\oC \tensor \oD}}$] {} (A.corner 5);
\end{tikzpicture}
\caption{Pentagon coherence condition}
\label{fig:pentagon}
\end{figure}

\begin{figure}[ht]
\centering
\begin{tikzpicture}
\node[regular polygon, regular polygon sides=3, minimum size=4cm] (A) {};
\node at (A.corner 2) {$\parens{\oA \tensor \tensorUnit} \tensor \oB$};
\node at (A.corner 1) {$\oA \tensor \parens{\tensorUnit \tensor \oB}$};
\node at (A.corner 3) {$\oA \tensor \oB$};
\draw (A.corner 2) edge [commarr] node [label=above left:$\tensorAssocArgs{\oA}{\tensorUnit}{\oB}$] {} (A.corner 1);
\draw (A.corner 1) edge [commarr] node [label=above right:$\idArr{\oA} \tensor \tensorLeftIDArg{\oB}$] {} (A.corner 3);
\draw ($(A.corner 2)+(0.5,0)$) edge [commarr] node [label=below:$\tensorRightIDArg{\oA} \tensor
\idArr{\oB}$] {} ($(A.corner 3)+(-0.25,0)$);
\end{tikzpicture}
\caption{Triangle coherence condition}
\label{fig:triangle}
\end{figure}

\begin{definition}[Strict Monoidal Category]
    A strict monoidal category, $(\aCat, \tensor, \tensorUnit, \tensorAssoc,
    \tensorLeftID, \tensorRightID)$, is a monoidal category where
    $\tensorAssoc, \tensorLeftID$ and $\tensorRightID$ are all identity mappings.
\end{definition}

\begin{definition}[Symmetric Monoidal Category]\label{defn:symMonCat}
    A symmetric monoidal category is a monoidal category with a braiding,
    $\braiding$, which is a natural isomorphism witnessing the commutativity
    of the tensor product:
    \[
        \braidingArgs{\oA}{\oB} : \oA \tensor \oB \to \oB \tensor \oA
    \]
    such that the symmetry is self-inverse (i.e.~\figref{fig:symmetryInverse} commutes) and the
    hexagon diagram commutes (\figref{fig:hexagon}).
\end{definition}

\begin{figure}[ht]
\centering
\begin{tikzpicture}
\node[regular polygon, regular polygon sides=3, minimum size=4cm] (A) {};
\node at (A.corner 2) {$\oA \tensor \oB$};
\node at (A.corner 1) {$\oB \tensor \oA$};
\node at (A.corner 3) {$\oA \tensor \oB$};
\draw (A.corner 2) edge [commarr] node [label=above left:$\braidingArgs{\oA}{\oB}$] {} (A.corner 1);
\draw (A.corner 1) edge [commarr] node [label=above right:$\braidingArgs{\oB}{\oA}$] {} (A.corner 3);
\draw ($(A.corner 2)+(0.5,0)$) edge [commarr] node [label=below:$\idArr{\oA \tensor \oB}$] {} ($(A.corner 3)+(-0.25,0)$);
\end{tikzpicture}
\caption{Symmetry is self-inverse}
\label{fig:symmetryInverse}
\end{figure}


\begin{figure}[ht]
\centering
\begin{tikzpicture}
\node[regular polygon, regular polygon sides=6, minimum size=5cm, xscale=2] (A) {};
\node at (A.corner 3) {$\parens{\oA \tensor \oB} \tensor \oC$};
\node at (A.corner 2) {$\parens{\oB \tensor \oA} \tensor \oC$};
\node at (A.corner 1) {$\oB \tensor \parens{\oA \tensor \oC}$};
\node at (A.corner 4) {$\oA \tensor \parens{\oB \tensor \oC}$};
\node at (A.corner 5) {$\parens{\oB \tensor \oC}\tensor \oA$};
\node at (A.corner 6) {$\oB \tensor \parens{\oC \tensor \oA}$};
\draw (A.corner 3) edge [commarr] node [label=above left:$\braidingArgs{\oA}{\oB} \tensor \idArr{\oC}$] {} (A.corner 2);
\draw ($(A.corner 2)+(0.75,0)$) edge [commarr] node [label=above:$\tensorAssocArgs{\oB}{\oA}{\oC}$]
{} ($(A.corner 1)+(-0.75,0)$);
\draw (A.corner 1) edge [commarr] node [label=above right:$\idArr{\oB} \tensor \braidingArgs{\oA}{\oC}$] {} (A.corner 6);
\draw (A.corner 3) edge [commarr] node [label=below left:$\tensorAssocArgs{\oA}{\oB}{\oC}$] {} (A.corner 4);
\draw ($(A.corner 4)+(0.75,0)$) edge [commarr] node [label=below:$\braidingArgs{\oA}{\oB \tensor \oC}$]
{} ($(A.corner 5)+(-0.75,0)$);
\draw (A.corner 5) edge [commarr] node [label=below right:$\tensorAssocArgs{\oB}{\oC}{\oA}$] {} (A.corner 6);
\end{tikzpicture}
\caption{Hexagon coherence condition}
\label{fig:hexagon}
\end{figure}


Symmetric monoidal categories with objects the natural numbers and tensor being
addition are known as PROPs~\cite{Lack2004}:

\begin{definition}[PROP]
    A PROP (Products and Permutations) is a strict monoidal category that has
    $\N$ as its objects, and $\tensor$ being addition on objects.
\end{definition}

Monoidal functors map between monoidal categories, whilst preserving the
monoidal structure:

\newcommand{\monFunctorCoherenceUnit}{\theta^1}
\newcommand{\monFunctorCoherenceTensor}{\theta^2}
\newcommand{\monFunctorCoherenceTensorArgs}[2]{\theta^2_{\parens{#1,#2}}}

\newcommand{\inA}[1]{\color{red}{#1}}
\newcommand{\inB}[1]{\color{green!50!black}{#1}}
\newcommand{\tensA}{\mathbin{\inA\tensor}}
\newcommand{\tensB}{\mathbin{\inB\tensor}}

\begin{definition}[Monoidal Functor]
    A monoidal functor is a functor between monoidal categories, $\inA\aCat$ and
    $\inB\bCat$, formed of three components,
    $\parens{\aFunctor,\monFunctorCoherenceUnit, \monFunctorCoherenceTensor}$
    where:
    \begin{itemize}
        \item $\aFunctor : {\inA\aCat} \to {\inB\bCat}$ is a functor,
        \item $\monFunctorCoherenceUnit : {\inB\tensorUnit} \to
            \aFunctor\parens{{\inA\tensorUnit}}$ is a morphism in $\inB\bCat$,
        \item $\monFunctorCoherenceTensorArgs{\oA}{\oB} :
            \fA \tensB \fB \to \aFunctor\parens{\oA \tensA \oB}$ is a
            natural transformation.
    \end{itemize}
    subject to the conditions
    illustrated in~\figref{fig:functorCoherence1} and~\figref{fig:functorCoherence2}.
\end{definition}

\begin{definition}[Strict Monoidal Functor]
    A monoidal functor, $\parens{\aFunctor,\monFunctorCoherenceUnit, \monFunctorCoherenceTensor}$,
    is strict iff $\monFunctorCoherenceUnit$ and $\monFunctorCoherenceTensor$ are identity
    mappings.
\end{definition}

\newcommand{\fAB}{\aFunctor\parens{\oA \tensA \oB}}
\newcommand{\fBC}{\aFunctor\parens{\oB \tensA \oC}}
\newcommand{\fABCL}{\aFunctor\parens{\parens{\oA \tensA \oB} \tensA \oC}}
\newcommand{\fABCR}{\aFunctor\parens{\oA \tensA \parens{\oB \tensA \oC}}}

\begin{figure}[ht]
\centering
\makebox[\textwidth][c]{
    \scalebox{0.95}{
\begin{tikzpicture}
\node[regular polygon, regular polygon sides=6, minimum size=5.5cm, xscale=2.5] (A) {};
\node at (A.corner 3) {$\parens{\fA \tensB \fB} \tensB \fC$};
\node at (A.corner 2) {$\fA \tensB \parens{\fB \tensB \fC}$};
\node at (A.corner 1) {$\fA \tensB \fBC$};
\node at (A.corner 6) {$\fABCR$};
\node at (A.corner 4) {$\fAB \tensB \fC$};
\node at (A.corner 5) {$\fABCL$};
\draw (A.corner 3) edge [commarr] node [label=above left:$\tensorAssocArgs{\fA}{\fB}{\fC}$] {} (A.corner 2);
\draw ($(A.corner 2)+(1.75,0)$) edge [commarr] node [label=above:$\idArr{\fA} \tensB \monFunctorCoherenceTensorArgs{\oB}{\oC}$]
{} ($(A.corner 1)+(-1.25,0)$);
\draw (A.corner 1) edge [commarr] node [label=above right:$\monFunctorCoherenceTensorArgs{\oA}{\parens{\oB \tensA \oC}}$] {} (A.corner 6);
\draw (A.corner 3) edge [commarr] node [label=below left:$\monFunctorCoherenceTensorArgs{\oA}{\oB}
\tensB \idArr{\oC}$] {} (A.corner 4);
\draw ($(A.corner 4)+(1.3,0)$) edge [commarr] node [label=below:$\monFunctorCoherenceTensorArgs{\parens{\oA \tensA \oB}}{\oC}$]
{} ($(A.corner 5)+(-1.25,0)$);
\draw (A.corner 5) edge [commarr] node [label=below right:$\aFunctor\parens{\tensorAssocArgs{\oA}{\oB}{\oC}}$] {} (A.corner 6);
\end{tikzpicture}
}
}
\caption{Monoidal functor coherence 1}
\label{fig:functorCoherence1}
\end{figure}

\begin{figure}[ht]
    \centering
\makebox[1.075\textwidth][c]{%
    \begin{subfigure}{0.5\textwidth}%
    \centering
\begin{tikzpicture}[node distance=6cm]
\node[regular polygon, regular polygon sides=4, minimum size=6cm] (A) {};
\node at (A.corner 2) {$\fA \tensB {\inB\tensorUnit{}}$};
\node at (A.corner 1) {$\fA$};
\node at (A.corner 3) {$\fA \tensB \aFunctor{\parens{\inA\tensorUnit{}}}$};
\node at (A.corner 4) {$\aFunctor\parens{\oA \tensA {\inA\tensorUnit}}$};

\draw (A.corner 2) edge [commarr, shorten <=1cm] node [label=above:$\tensorRightIDArg{\fA}$] {} (A.corner 1);
\draw (A.corner 4) edge [commarr] node [label=right:$\aFunctor\parens{\tensorRightIDArg{\oA}}$] {} (A.corner 1);
\draw (A.corner 2) edge [commarr] node [label=left:$\idArr{\fA} \tensB \monFunctorCoherenceUnit$] {} (A.corner 3);
\draw (A.corner 3) edge [commarr, shorten >=1cm, shorten <=1.25cm] node [label=below:$\monFunctorCoherenceTensorArgs{\oA}{{\inA\tensorUnit}}$] {} (A.corner 4);
\end{tikzpicture}
\end{subfigure}%
\hspace{0.075\textwidth}%
\begin{subfigure}{0.5\textwidth}%
    \centering
\begin{tikzpicture}[node distance=6cm]
\node[regular polygon, regular polygon sides=4, minimum size=6cm] (A) {};
\node at (A.corner 2) {${\inB\tensorUnit{}} \tensB \fA$};
\node at (A.corner 1) {$\fA$};
\node at (A.corner 3) {$\aFunctor{\parens{\inA\tensorUnit{}}} \tensB \fA$};
\node at (A.corner 4) {$\aFunctor\parens{{\inA\tensorUnit}\tensA \oA}$};

\draw (A.corner 2) edge [commarr, shorten <=1cm] node [label=above:$\tensorLeftIDArg{\fA}$] {} (A.corner 1);
\draw (A.corner 4) edge [commarr] node [label=right:$\aFunctor\parens{\tensorLeftIDArg{\oA}}$] {} (A.corner 1);
\draw (A.corner 2) edge [commarr] node [label=left:$\monFunctorCoherenceUnit \tensB \idArr{\fA}$] {} (A.corner 3);
\draw (A.corner 3) edge [commarr, shorten >=1cm, shorten <=1.25cm] node [label=below:$\monFunctorCoherenceTensorArgs{{\inA\tensorUnit}}{\oA}$] {} (A.corner 4);
\end{tikzpicture}
\end{subfigure}
}
\caption{Monoidal functor coherence 2}
\label{fig:functorCoherence2}
\end{figure}

A symmetric monoidal functor is then a monoidal functor that preserves the
braided structure:

\newcommand{\braidingArgsA}[2]{{\inA\braiding}_{\parens{#1,#2}}}
\newcommand{\braidingArgsB}[2]{{\inB\braiding}_{\parens{#1,#2}}}

\begin{definition}[Symmetric Monoidal Functor]
    A monoidal functor, $\aFunctor$ between symmetric monoidal categories
    $\inA\aCat$ and $\inB\bCat$ is symmetric if the diagram in~\figref{fig:braidingCoherence}
    commutes.
\end{definition}

\begin{figure}[ht]
    \[
    \]
    \centering
\begin{tikzpicture}[node distance=6cm]
\node[regular polygon, regular polygon sides=4, minimum size=6cm] (A) {};
\node at (A.corner 2) {$\fA \tensB \fB$};
\node at (A.corner 1) {$\fB \tensB \fA$};
\node at (A.corner 3) {$\aFunctor{\parens{\oA \tensA \oB}}$};
\node at (A.corner 4) {$\aFunctor{\parens{\oB \tensA \oA}}$};

\draw (A.corner 2) edge [commarr, shorten <=1.5cm, shorten >=1.5cm] node [label=above:$\braidingArgsB{\fA}{\fB}$] {} (A.corner 1);
\draw (A.corner 1) edge [commarr] node [label=right:$\monFunctorCoherenceTensorArgs{\oB}{\oA}$] {} (A.corner 4);
\draw (A.corner 2) edge [commarr] node [label=left:$\monFunctorCoherenceTensorArgs{\oA}{\oB}$] {} (A.corner 3);
\draw (A.corner 3) edge [commarr, shorten >=1cm, shorten <=1.25cm] node [label=below:$\aFunctor\parens{\braidingArgsA{\oA}{\oB}}$] {} (A.corner 4);
\end{tikzpicture}
\caption{Symmetric monoidal functor coherence}
\label{fig:braidingCoherence}
\end{figure}

\section{The category of PNBs} \label{sec:pnbCat}

As shown by {Bruni et al.}~\cite[Proposition 5.1]{Bruni2013}, PNBs form a
category. Before we give the structure of this category, we must describe
isomorphism classes of PNBs:

\begin{definition}[PNB Isomorphism Class]
    For a PNB, $\aPNB \withNetType{\aN}{\bN}$, its PNB isomorphism class,
    written $\isoClass{\aPNB} \withNetType{\aN}{\bN}$ is the set
    $\setBuilder{\bPNB}{\aPNB \PNBIso \bPNB}$.
\end{definition}

Now, \PNBCat{}, the category of \PNB{}s has the following structure:
\begin{itemize}
\item Objects are the natural numbers, \N,
\item Arrows from $\aN$ to $\bN$ are the \PNB{} isomorphism classes:
    $\isoClass{\aPNB} \withNetType{\aN}{\bN}$,
\item The identity morphism for $\aN \in \N$, is the net $\PNBid{\aN}
    \withNetType{\aN}{\aN}$,
    illustrated in \figref{fig:netIdn}. $\PNBid{\aN}$ has no places, $\aN$
    boundaries on the left and right, and transitions connecting the $i^{th}$
    left boundary to the $i^{th}$ right boundary,
\item The composition of morphisms $\aPNB \withNetType{\aN}{\bN}$ and $\bPNB
    \withNetType{\bN}{\cN}$, is $\aPNB \arrComp \bPNB \withNetType{\aN}{\cN}$,
    obtained using PNB composition, defined in
    \defnref{defn:sequentialCompositionPNB}.
\end{itemize}

To show that such a structure is well-defined, {Bruni et al.}~\cite[Proposition
5.1]{Bruni2013} proved:
\begin{enumerate}
    \item PNB composition is compatible with isomorphism classes:
        $\aPNB \comp \bPNB' \PNBIso \aPNB' \comp \bPNB$ with $\aPNB \PNBIso
        \aPNB'$ and $\bPNB \PNBIso \bPNB'$,
    \item PNB composition is associative up-to isomorphism: $\aPNB \comp
        \parens{\bPNB \comp \cPNB} \PNBIso \parens{\aPNB \comp \bPNB} \comp
        \cPNB$,
\item PNB composition has a unit up-to isomorphism: $\aPNB \comp \PNBid{}
    \PNBIso \aPNB \PNBIso \PNBid{} \comp \aPNB$.
\end{enumerate}

\begin{figure}[ht]
    \centering
    \begin{tikzpicture}[pnb]
        \drawBoundaries[2][4]{4}{4}

        \path node [dotsBPort] at (l3) {};
        \path node [dotsBPort] at (r3) {};

        \foreach \i in {1,2,4}{
            \draw (l\i) edge[pnbarr, mark inside=0.5] (r\i) {};
        }
\end{tikzpicture}
\caption{Net $\PNBid{\aN} \withNetType{\aN}{\aN}$}
\label{fig:netIdn}
\end{figure}

Furthermore, there is a (strict) monoidal structure on \PNBCat:

\begin{itemize}
    \item The tensor product is addition on objects and $\tensor$-composition
        (\defnref{defn:tensorCompositionPNB}) on morphisms,
    \item The unit is $0$,
    \item $\tensorAssoc, \tensorLeftID$ and $\tensorRightID$ are identities.
\end{itemize}

{Bruni et al.}~\cite{Bruni2013}[Proposition 5.2] assures us that tensor is
well-defined for equivalence classes of \PNB{}s, and is a bifunctor. Indeed,
addition is strictly associative with 0 being its strict identity, thus it
follows that \PNBCat{} is strict monoidal:

\begin{proposition}\label{prop:PNBMonCat}
    \PNBCat{} is a strict monoidal category.
\end{proposition}

We now show that there is a \emph{symmetric} monoidal structure on \PNBCat{}.
For any $\aN, \bN \in \N$, there is a \PNB{}, $\PNBbr{\aN}{\bN} \withNetType{\aN +
\bN}{\bN + \aN}$, illustrated in \figref{fig:netBr}. $\PNBbr{\aN}{\bN}$ has no
places, and $\aN + \bN$ of each of transitions, left boundaries and right boundaries. The
$\aN + \bN$ transitions are connected to boundary ports as follows: for $0 \le
x < \aN$, $\source{\aTrans_{x}}=x$ and $\target{\aTrans_{x}}=\bN + x$ and
for $0 \le y < \bN$, $\source{\aTrans_{\aN + y}}=\aN + y$ and
$\target{\aTrans_{\aN + y}}=y$.

\makeatletter
\NewDocumentCommand{\drawBraid}{oooo}{%
    \begin{tikzpicture}[pnb, boundaryPort-l5/.style={boundaryPortHidden},
                             boundaryPort-r5/.style={boundaryPortHidden},
                             boundaryPort-l6/.style={boundaryPortHidden},
                             boundaryPort-r6/.style={boundaryPortHidden}]

    \drawBoundaries[3][6]{10}{10}

    \foreach \i in {3,9}{
        \path node [dotsBPort] at (l\i) {};
        \path node [dotsBPort] at (r\i) {};
    }

    \IfValueT{#1}{%
        \@ifnotmtarg{#1}{%
            \draw[mirrorbrace] (l1.west) -- node [left] {$#1$} (l4.west);
        }}%
    \IfValueT{#2}{\@ifnotmtarg{#2}{\draw[mirrorbrace] (l7.west) -- node [left]
        {$#2$} (l10.west);}}
    \IfValueT{#3}{\@ifnotmtarg{#3}{\draw[brace] (r1.east) -- node [right]
        {$#3$} (r4.east);}}
    \IfValueT{#4}{\@ifnotmtarg{#4}{\draw[brace] (r7.east) -- node [right]
        {$#4$} (r10.east);}}

    \foreach \l/\r in {1/7,2/8,4/10}{
        \labelledpnbarr{l\l}{r\r}{}{out=-15, in=175}{pos=0.1}
        \labelledpnbarr{l\r}{r\l}{}{pnbarrstyle1, out=-15, in=175}{pos=0.1}
    }
\end{tikzpicture}
}
\makeatother

\begin{figure}[ht]
    \centering
    \drawBraid[\aN][\bN][\bN][\aN]
\caption{Net $\PNBbr{\aN}{\bN} \withNetType{\aN + \bN}{\bN + \aN}$}
\label{fig:netBr}
\end{figure}

$\PNBbr{\aN}{\bN}$ is an example of a PNB whose transitions are 1-1 with its
boundary ports; we refer to such PNBs as \emph{permutation} PNBs:

\begin{definition}[Permutation PNB]
    A \PNB, $\exPermPNB \withNetType{\aN}{\aN}$, is a Permutation PNB iff
    $\perm$ is a permutation on $\ordinal{\aN}$, $\places{\exPermPNB} =
    \emptyset$, and the following hold, with $\aTrans,\bTrans$
    ranging over $\trans{\exPermPNB}$:
    \begin{multicols}{2}
    \begin{itemize}
        \item $\cardinalityof{\source{\aTrans}} = 1$,
        \item if $\aTrans \neq \bTrans$ then $\source{\aTrans} \intersect
            \source{\bTrans} = \emptyset$,
        \item $\forall \aPNBBPort \in \ordinal{\aN}$, $\exists
            \cTrans \in \trans{\aPNB} \text{ s.t. }
            \source{\cTrans} = \setof{\aPNBBPort}$.
        \item $\cardinalityof{\target{\aTrans}} = 1$,
        \item if $\aTrans \neq \bTrans$ then $\target{\aTrans}
            \intersect \target{\bTrans} = \emptyset$,
        \item $\forall \aPNBBPort \in \ordinal{\bN}$, $\exists
            \cTrans \in \trans{\aPNB} \text{ s.t. }
            \target{\cTrans} = \setof{\aPNBBPort}$.
    \end{itemize}
    \end{multicols}
    That is, each transition is connected to exactly one left (right) boundary
    port, distinct transitions do not share left (right) boundary ports and for
    every left (right) boundary port, there is a transition that connects to
    that boundary port. The boundary connections of the transitions are
    determined by $\perm$: given $\aTrans \in \trans{\exPermPNB}$,
    $\target{\aTrans} = \setBuilder{\perm(\aPNBBPort)}{\source{\aTrans} =
    \setof{\aPNBBPort}}$.
\end{definition}

Permutation \PNB{}s embed permutations on $\ordinal{\aN}$; we refer to such
permutations as $\aN$-permutations. Indeed, we can compose permutations, both
sequentially and in parallel:

\begin{definition}[Synchronous composition of permutations]
    Given two $\aN$-permutations, $\aPerm$ and $\bPerm$, their synchronous
    composition is a $\aN$-permutation, written $\aPerm \comp \bPerm$, and
    simply composes the underlying bijections: $(\aPerm \comp \bPerm)(x) \defeq
    \bPerm(\aPerm(x))$.
\end{definition}

\begin{definition}[Parallel composition of permutations]
    \newcommand{\aVar}{x}
    Given a $\aN$-permutation, $\aPerm$, and $\bN$-permutation, $\bPerm$,
    their tensor composition is a $\parens{\aN + \bN}$-permutation, written $\aPerm
    \tensor \bPerm$, and defined:
    \[
    (\aPerm \tensor \bPerm)(\aVar)
    \defeq
    \begin{cases}
        \aPerm(\aVar) \text{ if } 0 \le \aVar < \aN\\
        \bPerm(\aVar - \aN) + \aN \text{ if } \aN \le \aVar < \aN + \bN
    \end{cases}
    \]
    that is, we directly apply $\aPerm$ if $\aVar$ is in its domain, else we
    apply a shift to $\aVar$, such that it is in the domain of $\bPerm$, before
    applying $\bPerm$ and the inverse shift.
\end{definition}

We later make use of two particular permutations, $\idPerm{\aN}$ and
$\swPerm{\bN}{\cN}$ (identity and ``swap'', or \emph{braid} permutations,
respectively):
\begin{example}
    $\idPerm{\aN}$ is a $\aN$-permutation, with: $\idPerm{\aN}(x) \defeq x$.
\end{example}
\begin{example}
    $\swPerm{\bN}{\cN}$ is a $\bN+\cN$-permutation, with:
    \[
    \swPerm{\bN}{\cN}(x) \defeq
    \begin{cases}
        x + \cN \text{ if } 0 \le x < \bN\\
        x - \bN \text{ otherwise.}
    \end{cases}
    \]
\end{example}

A simple lemma confirms that permutation PNBs are closed (up-to isomorphism) under synchronous and
tensor composition.

\begin{lemma}
    If $\aPermPNB \withNetType{\aN}{\aN}$, $\bPermPNB \withNetType{\aN}{\aN}$ and
    $\cPermPNB \withNetType{\bN}{\bN}$ are permutation PNBs, then:
    \begin{enumerate}
        \item \label{enum:PermClosedSeq}
            We have $\aPermPNB \comp \bPermPNB \withNetType{\aN}{\aN}$, with $\aPermPNB \comp
            \bPermPNB \PNBIso \PermPNB{\aPerm \comp \bPerm}$,
        \item \label{enum:PermClosedTen}
            We have $\aPermPNB \tensor \cPermPNB \withNetType{\aN+\bN}{\aN + \bN}$, with
            $\aPermPNB \tensor \cPermPNB \PNBIso \PermPNB{\aPerm \tensor \cPerm}$.
    \end{enumerate}
    \label{lem:permutationPNBClosedComposition}
\end{lemma}
\begin{proof}
    \newcommand{\abMinSync}{(\aPNBTransSing, \bPNBTransSing)}
    For part \ref{enum:PermClosedSeq}, we have that each $\aPNBBPort \in
    \ordinal{\bN}$ determines unique $\aTrans \in \trans\aPNB$ and
    $\bTrans \in \trans\bPNB$. Thus $\abMinSync$ is a minimal
    synchronisation, with $\source{\abMinSync}$ and $\target{\abMinSync}$ being
    unique, in particular, since $\target{\setof{\aTrans}} =
    \source{\setof{\bTrans}}$, we have: $\target{\bPNBTransSing} =
    \setBuilder{\bPerm(\aPerm(\aPNBBPort))}{\setof{\aPNBBPort} =
    \source{\aPNBTransSing}}$. For part \ref{enum:PermClosedTen}, observe that
    no boundary ports or transitions are created, and existing transitions'
    connections are preserved with those in $\cPermPNB$ being appropriately
    shifted.
\end{proof}

We prove 3 technical lemmas relating to PNB compositions with a permutation
PNB, $\aPermComp$: the first says that each transition of $\aPNB$ synchronises
with a uniquely defined set of transitions in $\exPermPNB$. The second says
that the transitions of $\aPermComp$ (that is, minimal synchronisations)
consist of single transitions from $\aPNB$. Finally, the third says that
transitions of $\aPermComp$ are in bijection with those of $\aPNB$.

\begin{lemma} \label{lem:uniqueTransSync}
    For $\exPermPNB \withNetType{\aN}{\aN}$, each $\aPNBTargetSet \in \powerset{\ordinal{\aN}}$
    (i.e. $\aPNBTargetSet$ is a set of left boundary ports of $\exPermPNB$) determines a unique
    $\bPNBTransSet \subseteq \trans{\exPermPNB}$ such that $\aPNBTargetSet =
    \source{\bPNBTransSet}$.
    \end{lemma}
\begin{proof}
    By induction on $\aPNBTargetSet$: in the base case of $\aPNBTargetSet = \emptyset$, $\emptyset$
    suffices. In the case of $\aPNBTargetSet = \aPNBTargetSet' \union \setof{\aPNBBPort}$, we apply
    the \IH{} to $\aPNBTargetSet'$, obtaining the unique $\bPNBTransSet' \subseteq
    \trans{\exPermPNB}$ such that $\aPNBTargetSet' = \source{\bPNBTransSet'}$. Then, by our
    assumptions on $\trans{\exPermPNB}$, there is a unique $\bTrans \in \trans{\exPermPNB}$ such
    that $\source{\bTrans} = \aPNBBPort$; therefore, $\bPNBTransSet' \union \bPNBTransSing$
    satisfies the requirements.
\end{proof}

\begin{lemma}\label{lem:singletonSynch}
    For $\aPNB \withNetType{\aN}{\bN}$ and $\exPermPNB \withNetType{\bN}{\bN}$,
    each $\aPNBSync \in \trans{\aPermComp}$ has $\cardinalityof{\aPNBTransSet}
    = 1$.
\end{lemma}
\begin{proof}
    \newcommand{\aPNBTransSetNoaTrans}{\aPNBTransSet \setminus \setof{\aTrans}}
    \newcommand{\bPNBTransSetNocTransSet}{\bPNBTransSet \setminus
        \cPNBTransSet}
    For a contradiction, assume that we have a $\aPNBSync \in
    \trans{\aPermComp}$ with $\cardinalityof{\aPNBTransSet} > 1$. Take any
    $\aTrans \in \aPNBTransSet$, and using \lemref{lem:uniqueTransSync}
    identify the (possibly empty) unique set $\cPNBTransSet \subseteq
    \trans{\exPermPNB}$ such that $\target{\aTrans} =
    \source{\cPNBTransSet}$; since $\cPNBTransSet$ is unique in
    $\trans{\exPermPNB}$, we must have that $\cPNBTransSet \subseteq
    \bPNBTransSet$.

    Then, since $\aPNBTransSet$ and $\bPNBTransSet$ are contention-free, we
    have that $\target{(\aPNBTransSetNoaTrans)} = \target{\aPNBTransSet}
    \setminus \target{\aTrans} = \source{\bPNBTransSet} \setminus
    \source{\cPNBTransSet} = \source{(\bPNBTransSetNocTransSet)}$, and clearly,
    $(\aPNBTransSetNoaTrans, \bPNBTransSetNocTransSet) \subseteq \aPNBSync$.
    In other words, $\aPNBSync$ is not a \emph{minimal} synchronisation,
    contradicting $\aPNBSync \in \trans{\aPermComp}$.
\end{proof}

\begin{lemma} \label{lem:singletonInCompWithPerPNB}
    For $\aPNB \withNetType{\aN}{\bN}$ and $\exPermPNB \withNetType{\bN}{\bN}$,
    there is a bijection between $\trans{\aPNB}$ and $\trans{\aPermComp}$. In
    particular, each $\aTrans \in \trans{\aPNB}$ determines a unique
    $\aPNBSync \in \trans{\aPermComp}$ such that $\aPNBTransSet =
    \setof{\aTrans}$.
\end{lemma}
\begin{proof}
    We have that $\trans{\aPermComp}$ is the set of all minimal
    synchronisations between $\aPNB$ and $\exPermPNB$. By
    \lemref{lem:singletonSynch} every $\aPNBSync \in \trans{\aPermComp}$
    has $\cardinalityof{\aPNBTransSet} = 1$, i.e. $\aPNBTransSet = \setof{\aTrans}$ for some
    $\aTrans$, such that,
    by \lemref{lem:uniqueTransSync}, $\bPNBTransSet$ is uniquely
    determined by $\target{\aTrans}$. In the other direction, suppose that there is no $\aPNBSync
    \in \trans{\aPermComp}$ such that $\aPNBTransSet = \setof{\aTrans}$; by
    \lemref{lem:uniqueTransSync} we have the unique set of transitions $\aPNBTransSet \subseteq
    \trans{\exPermPNB}$ such that $\target{\aTrans} = \source{\aPNBTransSet}$. Clearly
    $(\setof{\aTrans}, \aPNBTransSet)$ is a minimal synchronisation and thus must appear in
    $\trans{\aPermComp}$, a contradiction.
\end{proof}

Indeed, we can easily obtain the symmetric versions of the previous lemmas, for composition with a
permutation PNB on the left; the proofs of which follow the same arguments:

\begin{lemma} \label{lem:uniqueTransSyncSym}
    For $\exPermPNB \withNetType{\aN}{\aN}$, each $\aPNBTargetSet \in \powerset{\ordinal{\aN}}$
     determines a unique $\aPNBTransSet \subseteq \trans{\exPermPNB}$ such that $\aPNBTargetSet =
    \target{\aPNBTransSet}$.

    For $\exPermPNB \withNetType{\aN}{\aN}$ and $\bPNB \withNetType{\aN}{\bN}$, each
    $\aTrans \in \trans{\bPNB}$, determines a unique $\aPNBTransSet
    \subseteq \trans{\exPermPNB}$ such that $\target{\aPNBTransSet} =
    \source{\aTrans}$.
    \qed
\end{lemma}

\begin{lemma} \label{lem:singletonSynchSym}
    For $\exPermPNB \withNetType{\aN}{\aN}$ and $\bPNB \withNetType{\aN}{\bN}$, each
    $\aPNBSync \in \trans{\exPermPNB \comp \bPNB}$ has
    $\cardinalityof{\bPNBTransSet} = 1$.  \qed
\end{lemma}

\begin{lemma} \label{lem:singletonInCompWithPerPNBSym}
    For $\exPermPNB \withNetType{\aN}{\aN}$ and $\bPNB \withNetType{\aN}{\bN}$,
    there is a bijection between $\trans{\bPNB}$ and $\trans{\exPermPNB \comp
    \bPNB}$. In particular, each $\aTrans \in \trans{\bPNB}$ determines a
    unique $\aPNBSync \in \trans{\exPermPNB \comp \bPNB}$ such that
    $\bPNBTransSet = \setof{\aTrans}$.
\end{lemma}

We may now prove a proposition relating tensor and the ``swap'' PNB; the
intuition is provided graphically in \figref{fig:braidingNatural}, where both
compositions should be isomorphic.

\makeatletter
\NewDocumentCommand{\singleComponent}{oom}{%
\begin{tikzpicture}[pnb]
    \drawBoundaries[1.8][2.725]{4}{4}

    \node (lbl) at (current bounding box.center) {$#3$};

    \foreach \i in {3}{
        \path node [boundaryPort,fill=white,rotate=90] at (l\i) {$\dots$};
        \path node [boundaryPort,fill=white,rotate=90] at (r\i) {$\dots$};
    }

    \IfValueT{#1}{\@ifnotmtarg{#1}{\draw[mirrorbrace] (l1.west) --
        node [left, overlay] {$#1$} (l4.west);}}
    \IfValueT{#2}{\@ifnotmtarg{#2}{\draw[brace] (r1.east) --
        node [right,overlay] {$#2$} (r4.east);}}
\end{tikzpicture}
}
\makeatother

\makesavebox{\nPNBl}{\singleComponent[\aN][\bN]{\aPNB}}
\makesavebox{\PNBl}{\singleComponent[\cN][\dN]{\bPNB}}
\makesavebox{\brPNBr}{\drawBraid[][][\dN][\bN]}

\makesavebox{\nPNBr}{\singleComponent[\cN][\dN]{\bPNB}}
\makesavebox{\PNBr}{\singleComponent[\aN][\bN]{\aPNB}}
\makesavebox{\brPNBl}{\drawBraid[\aN][\cN][][]}

\begin{figure}[ht]
    \centering
    \begin{subfigure}{0.5\textwidth}
    \centering
    \begin{tikzpicture}[node distance=0.5cm]
        \node (N) [anchor=north] {\usebox{\nPNBl}};
        \node (br) [right=0.2cm of N.north east, anchor=north west] {\usebox{\brPNBr}};
        \node (M) [left=0.2cm of br.south west, anchor=south east] {\usebox{\PNBl}};
    \end{tikzpicture}
    \caption{Composition $(\abPNBTensor) \comp \PNBbr{\bN}{\dN}$}
    \end{subfigure}%
    \begin{subfigure}{0.5\textwidth}
    \begin{tikzpicture}[node distance=0.5cm]
        \node (M) [anchor=north] {\usebox{\nPNBr}};
        \node (br) [left=0.2cm of M.north west, anchor=north east] {\usebox{\brPNBl}};
        \node (N) [right=0.2cm of br.south east, anchor=south west] {\usebox{\PNBr}};
    \end{tikzpicture}
    \caption{Composition $\PNBbr{\aN}{\cN} \comp (\bPNB \tensor \aPNB)$}
    \end{subfigure}
    \caption{\propref{prop:braidingNatural}, graphically; (a) is isomorphic to
(b).}
\label{fig:braidingNatural}
\end{figure}

\begin{proposition}\label{prop:braidingNatural}
    For $\aPNB \withNetType{\aN}{\bN}$ and $\bPNB \withNetType{\cN}{\dN}$, the
    following holds:
    \[
        (\abPNBTensor) \comp \PNBbr{\bN}{\dN}
        \PNBIso
        \PNBbr{\aN}{\cN} \comp (\bPNB \tensor \aPNB)
    \]
\end{proposition}
\begin{proof}
    For any $\aN, \bN \in \N$, $\PNBbr{\aN}{\bN}$ has no places, thus we can
    give an isomorphism on places that simply maps between the tagged
    places of $\aPNB$ and $\bPNB$, from the top (bottom) left component, to the
    bottom (top) right component:
    \begin{align*}
        \inl{\parens{\inl \aPlace}} &\mapsto \inr{\parens{\inr \aPlace}} \quad \text{for } \aPlace \in \places{\aPNB}\\
        \inl{\parens{\inr \bPlace}} &\mapsto \inr{\parens{\inl \bPlace}} \quad \text{for } \bPlace \in \places{\bPNB}
    \end{align*}

    To show the transitions are also isomorphic, consider any $\aPNBSync \in \trans{(\abPNBTensor)
    \comp \PNBbr{\bN}{\dN}}$; we can apply \lemref{lem:singletonSynch}, obtaining that
    $\aPNBTransSet = \setof{\inl{\aTrans}}$ or $\aPNBTransSet = \setof{\inr{\aTrans}}$. Indeed, in
    the former case $\aTrans \in \trans{\aPNB}$ and $\aTrans \in \trans{\bPNB}$ in the latter.

    Assume without loss of generality that $i = 0$.
    Then, $\source{\inl{\aTrans}} = \source{\aTrans}$, and
    $\target{\inl{\aTrans}} = \target{\aTrans}$; indeed, since $\aPNBSync$
    is a synchronisation, we have that $\target{\setof{\inl{\aTrans}}} =
    \source{\bPNBTransSet}$, thus, by the definition of the permutation
    underlying $\PNBbr{\bN}{\dN}$, $\target{\bPNBTransSet} =
    \setBuilder{\aPNBBPort + \dN}{\aPNBBPort \in \target{\aTrans}}$. This
    gives us $\source{\aPNBSync} = \source{\aTrans}$ and $\target{\aPNBSync}
    =  \setBuilder{\aPNBBPort + \dN}{\aPNBBPort \in \target{\aTrans}}$.

    Furthermore, by the definition of $\tensor$, and
    \lemref{lem:singletonInCompWithPerPNBSym}, we also have a $\bPNBSync \in
    \trans{\PNBbr{\aN}{\cN} \comp (\bPNB \tensor \aPNB}))$, with
    $\bPNBTransSet' = \setof{\inr{\aTrans}}$. Now, $\source{\inr{\aTrans}}
    = \setBuilder{\aPNBBPort + \cN}{\aPNBBPort \in \source{\aTrans}}$ and
    $\target{\inr{\aTrans}} = \setBuilder{\aPNBBPort + \dN}{\aPNBBPort \in
    \target{\aTrans}}$ Again, since $\bPNBSync$ is a synchronisation, we
    have that $\target{\aPNBTransSet'} = \source{\setof{\inr{\aTrans}}}$. By
    the definition of $\PNBbr{\aN}{\cN}$ we have that $\source{\aPNBTransSet'}
    = \setBuilder{\aPNBBPort - \cN}{\aPNBBPort \in \source{\setof{(\aTrans,
    1)}}}$, which, by the definition of $\source{\inr{\aTrans}}$, gives
    $\source{\aPNBTransSet'} = \source{\aTrans}$. Therefore,
    $\source{\bPNBSync} = \source{\aTrans}$ and $\target{\bPNBSync} =
    \setBuilder{\aPNBBPort + \dN}{\aPNBBPort \in \target{\aTrans}}$.

    Indeed, we have shown that $\aPNBSync \in (\abPNBTensor) \comp
    \PNBbr{\bN}{\dN}$ determines $\bPNBSync \in \PNBbr{\aN}{\cN} \comp
    (\bPNB \tensor \aPNB)$ such that $\source{\aPNBSync} = \source{\bPNBSync}$
    and $\target{\aPNBSync} = \target{\bPNBSync}$, as required. The opposite
    direction uses a similar argument.
\end{proof}

The following isomorphisms assure us that we can equivalently braid $\aN$ past
$\bN + \cN$ in one step or two individual steps, and similarly for $\aN + \bN$
past $\cN$:

\begin{proposition}\label{prop:braidingHexagonAxioms}
    We have the following isomorphisms:
    \begin{enumerate}[(a)]
        \item $\PNBbr{\aN}{\bN + \cN} \PNBIso (\PNBbr{\aN}{\bN} \tensor
            \PNBid{\cN}) \comp (\PNBid{\bN} \tensor \PNBbr{\aN}{\cN})$
        \item $\PNBbr{\aN + \bN}{\cN} \PNBIso (\PNBid{\aN} \tensor
            \PNBbr{\bN}{\cN}) \comp ( \PNBbr{\aN}{\cN}\tensor \PNBid{\bN})$
    \end{enumerate}
\end{proposition}
\begin{proof}
    By definition; observe that we may consider $\PNBid{\aN}$ as a trivial
    permutation PNB, $\PermPNB{id(\aN)}$. Then, it only remains to verify that
    the corresponding compositions on permutations are equal:

    \begin{enumerate}[(a)]
        \item $\swPerm{\aN}{\bN+\cN} = (\swPerm{\aN}{\bN} \tensor \idPerm{\cN})
            \comp (\idPerm{\bN} \tensor \swPerm{\aN}{\cN})$
        \item $\swPerm{\aN + \bN}{\cN} = (\idPerm{\aN} \tensor
            \swPerm{\bN}{\cN}) \comp (\swPerm{\aN}{\cN}\tensor \idPerm{\bN})$
    \end{enumerate}

    which is immediate by definition.
\end{proof}

\begin{proposition}\label{prop:braidingSymmetric}
    We have the following isomorphism: $\PNBbr{\aN}{\bN} \comp \PNBbr{\bN}{\aN}
    \PNBIso \PNBid{\aN + \bN}$
\end{proposition}
\begin{proof}
    Immediate, relying on the fact that composing $\aPerm$ and $\aPerm^{-1}$ is
    equal to the identity permutation.
\end{proof}

We can now show that \PNBCat is a strict \emph{symmetric} monoidal category.

\begin{proposition}\label{prop:PNBSymMonCat}
    \PNBCat{} is a symmetric monoidal category.
\end{proposition}
\begin{proof}
    \propref{prop:braidingNatural} confirms that the braiding is a natural
    isomorphism, \propref{prop:braidingHexagonAxioms} assures us that the
    braiding satisfies the hexagon axioms of \defnref{defn:symMonCat}
    (simplified since our associators are identities) and
    \propref{prop:braidingSymmetric} says that the braiding is symmetric.
\end{proof}

Finally, it follows that \PNBCat{} is a PROP:

\begin{proposition}\label{prop:PNBPROP}
    \PNBCat{} is a PROP.
\end{proposition}
\begin{proof}
    By \cite[Proposition 5.1]{Bruni2013} and propositions \ref{prop:PNBMonCat}
    and \ref{prop:PNBSymMonCat}.
\end{proof}

While it is informative to expose the PROP structure of \PNBCat{} (illustrating that PNBs are an
instance of a very general structure), it is the underlying monoidal category structure that is
most "useful" later in this thesis. Indeed, associativity of both composition types ensures that we
are free to compose from left-to-right, or right-to-left, or any grouping in between, as we do for
example in~\figref{fig:BufferExprTreeAssocs}, while identity of $\comp$-composition allows us to
pass \emph{signals} past a (composite) component, for example in~\figref{fig:tokenringschematic}.

\section{The category of \TLTS{}s}

We can show that \TLTS{}s form a category, in a similar fashion to \PNB{}s. In
the following, we define isomorphism classes of \TLTS{}s, and then show that
composition is compatible with such classes and furthermore, is
associative and has identities up-to isomorphism.

\begin{definition}[PNB Isomorphism Class]
    For a \TLTS{}, $\aTLTS \withNetType{\aN}{\bN}$, its \TLTS{} isomorphism
    class, written $\isoClass{\aTLTS} \withNetType{\aN}{\bN}$ is the set
    $\setBuilder{\bTLTS}{\aTLTS \TLTSIso \bTLTS}$.
\end{definition}

In the following, $\idTLTS{\aN} \withNetType{\aN}{\aN}$, is a \TLTS{},
illustrated in \figref{fig:idTLTS}, with a single state, and transitions the
self-loops labelled by $\setBuilder{x}{x \in \B^n}$

\begin{proposition}\label{prop:TLTSCatAxioms}
    The following isomorphisms hold:
    \begin{enumerate}[(i)]
        \item \label{seq-item1} Given \TLTS{}s $\aTLTS, \aTLTS'
            \withNetType{\aN}{\bN}$, $\bTLTS, \bTLTS' \withNetType{\bN}{\cN}$,
            with $\aTLTS \NFAIso \aTLTS'$ and $\bTLTS \NFAIso \bTLTS'$, we have
            that $\aTLTS \comp \bTLTS \NFAIso \aTLTS' \comp \bTLTS'$,
        \item \label{seq-item2} For \TLTS{}s $\aTLTS \withNetType{\aN}{\bN}$,
            $\bTLTS \withNetType{\bN}{\cN}$, $\cTLTS \withNetType{\cN}{\dN}$,
            we have $(\aTLTS \comp \bTLTS) \comp \cTLTS \NFAIso \aTLTS \comp
            (\bTLTS \comp \cTLTS)$,
        \item \label{seq-item3} For any \TLTS{} $\aTLTS
            \withNetType{\aN}{\bN}$, we have $\idTLTS{\aN} \comp \aTLTS \NFAIso
            \aTLTS \NFAIso \aTLTS \comp \idTLTS{\bN}$.
    \end{enumerate}
\end{proposition}
\begin{proof}
    For \ref{seq-item1} we have that $\compStates{\aNFAState}{\bNFAState}
    \LabelledTrans{\lbl{\aLbl}{\bLbl}} \compStates{\aNFAState'}{\bNFAState'}$ is a transition of
    $\aTLTS \comp \bTLTS$ iff $\aNFAState \LabelledTrans{\lbl{\aLbl}{\cLbl}} \aNFAState'$ and
    $\bNFAState \LabelledTrans{\lbl{\cLbl}{\bLbl}} \bNFAState'$ are transitions in $\aTLTS$ and
    $\bTLTS$, respectively, for some $\cLbl \in \B^\bN$. By the definition of \TLTS{} isomorphism,
    we have corresponding transitions in $\aTLTS'$ and $\bTLTS'$, which, by the definition of
    \TLTS{} composition gives the required transition in $\aTLTS' \comp \bTLTS'$. The converse
    argument is similar.

    For \ref{seq-item2} we have that $\compStates{\compStates{\aNFAState}{\bNFAState}}{\cNFAState}
    \LabelledTrans{\lbl{\aLbl}{\dLbl}}
    \compStates{\compStates{\aNFAState'}{\bNFAState'}}{\cNFAState'}$ is a transition of
    $\parens{\aTLTS \comp \bTLTS} \comp \cTLTS$ iff $\compStates{\aNFAState}{\bNFAState}
    \LabelledTrans{\lbl{\aLbl}{\cLbl}} \compStates{\aNFAState'}{\bNFAState'}$ and $\cNFAState
    \LabelledTrans{\lbl{\cLbl}{\dLbl}} \cNFAState'$ are transitions in $\aTLTS \comp \bTLTS$ and
    $\cTLTS$, respectively, for some $\cLbl \in \B^\cN$, and $\aNFAState
    \LabelledTrans{\lbl{\aLbl}{\bLbl}} \aNFAState'$ and $\bNFAState
    \LabelledTrans{\lbl{\bLbl}{\cLbl}} \bNFAState'$ are transitions in $\aTLTS$ and $\bTLTS$,
    respectively, for some $\bLbl \in \B^\bN$.  These transitions give
    $\compStates{\bNFAState}{\cNFAState} \LabelledTrans{\lbl{\bLbl}{\dLbl}}
    \compStates{\bNFAState'}{\cNFAState'}$ in $\bTLTS \comp \cTLTS$, and thus
    $\compStates{\aNFAState}{\compStates{\bNFAState}{\cNFAState}}
    \LabelledTrans{\lbl{\aLbl}{\dLbl}}
    \compStates{\aNFAState'}{\compStates{\bNFAState'}{\cNFAState'}}$ in $\aTLTS \comp
    \parens{\bTLTS \comp \cTLTS}$, as required.

    For \ref{seq-item3}, we have $\compStates{s}{\aNFAState} \LabelledTrans{\lbl{\aLbl}{\bLbl}}
    \compStates{s}{\aNFAState'}$ in $\idTLTS{\aN} \comp \aTLTS$ (where $s$ is the single state of
    $\idTLTS{\aN}$) iff we have $s \LabelledTrans{\lbl{\aLbl}{\aLbl}} s$ in $\idTLTS{\aN}$ and
    $\aNFAState \LabelledTrans{\lbl{\aLbl}{\bLbl}} \aNFAState'$ in $\aTLTS$, giving
    $\compStates{\aNFAState}{s'} \LabelledTrans{\lbl{\aLbl}{\bLbl}} \compStates{\aNFAState'}{s'}$
    in $\aTLTS \comp \idTLTS{\bN}$ (assuming $s'$ is the single state of $\idTLTS{\bN}$), as
    required. Again, the converse follows a similar argument.
\end{proof}

Now, \TLTSCat{}, the category of \TLTS{}s has the following structure:
\begin{itemize}
\item Objects are the natural numbers, \N,
\item Arrows from $\aN$ to $\bN$ are the \LTSB{\aN}{\bN} isomorphism classes,
\item The identity morphism for $\aN \in \N$, is $\idTLTS{\aN}$,
\item The composition of morphisms $\aTLTS \withNetType{\aN}{\bN}$ and $\bTLTS
    \withNetType{\bN}{\cN}$, is $\aTLTS \arrComp \bTLTS
    \withNetType{\aN}{\cN}$, obtained using the variant of the product
    construction on LTSs described in \defnref{defn:sequentialCompositionTLTS}.
\end{itemize}

\begin{figure}[ht]
    \centering
    \begin{tikzpicture}[nfa]
        \node [state] (p0) {};
        \path (p0) edge[loop right] node {$\setBuilder{b}{b \in \B^n}$} (p0);
\end{tikzpicture}
\caption{\TLTS{} $\idTLTS{\aN} \withNetType{\aN}{\aN}$}
\label{fig:idTLTS}
\end{figure}

\begin{proposition}\TLTSCat{} is a category.
    \label{prop:TLTSCategory}
\end{proposition}
\begin{proof}
    \propref{prop:TLTSCatAxioms}\ref{seq-item1} ensures that \TLTS{}
    composition is well-defined on equivalence classes of \TLTS{}s, whilst
    \ref{seq-item2} and \ref{seq-item3} ensure that composition is associative
    and has $\idTLTS{\aN}$ as identity.
\end{proof}

We have the following (strict) monoidal structure on \TLTSCat{}:
\begin{itemize}
    \item Tensor is addition on objects and the modification of the product
        construction (\defnref{defn:tensorCompositionTLTS}) on morphisms,
    \item The identity object is $0$,
    \item The associator and left/right unitors are identity natural
        isomorphisms.
\end{itemize}

We must assure ourselves that $\tensor$ is well-defined on equivalence
classes of \TLTS{}s, and is a bifunctor:

\begin{proposition}\label{prop:TLTSMonCatProps} The following isomorphisms hold:
    \begin{enumerate}[(i)]
        \item \label{tensor-item1} Given \TLTS{}s $\aTLTS, \aTLTS'
            \withNetType{\aN}{\bN}$, $\bTLTS, \bTLTS' \withNetType{\cN}{\dN}$,
            with $\aTLTS \NFAIso \aTLTS'$ and $\bTLTS \NFAIso \bTLTS'$, we have
            that $\aTLTS \tensor \bTLTS \NFAIso \aTLTS' \tensor \bTLTS'$,
        \item \label{tensor-item2} For any objects $\aN$, $\bN$, we have that
            $\idTLTS{(\aN + \bN)} \NFAIso \idTLTS{\aN} \tensor
            \idTLTS{\bN}$
        \item \label{tensor-item3} For \TLTS{}s $\aTLTS
            \withNetType{\aN}{\bN}$, $\bTLTS \withNetType{\bN}{\cN}$, $\cTLTS
            \withNetType{\dN}{\eN}$, $\dTLTS \withNetType{\eN}{\fN}$, we have\\
            $(\aTLTS \comp \bTLTS) \tensor (\cTLTS \comp \dTLTS) \NFAIso
            (\aTLTS \tensor \cTLTS) \comp (\bTLTS \tensor \dTLTS)$
    \end{enumerate}
\end{proposition}
\begin{proof}
    For \ref{tensor-item1}, the argument follows that of
    \propref{prop:TLTSCatAxioms}\ref{seq-item1}. For \ref{tensor-item2} we use
    that $\B^{(\aN + \bN)} = \setBuilder{xy}{x \in \B^\aN, y \in \B^\bN}$ and
    the obvious isomorphism between $x$ and $(x,x)$.  Finally, for
    \ref{tensor-item3}, we have a transition
    \(
    \tensorStates{\compStates{\aNFAState_1}{\bNFAState_1}}{\compStates{\aNFAState_2}{\bNFAState_2}}
    \LabelledTrans{\lbl{\aLbl\bLbl}{\cLbl\dLbl}}
    \tensorStates{\compStates{\aNFAState_1'}{\bNFAState_1'}}{\compStates{\aNFAState_2'}{\bNFAState_2'}}
    \) in $(\aTLTS \comp \bTLTS \tensor \cTLTS \comp \dTLTS)$ iff we have
    transitions \(
    \compStates{\aNFAState_1}{\bNFAState_1}
    \LabelledTrans{\lbl{\aLbl}{\cLbl}}
    \compStates{\aNFAState_1'}{\bNFAState_1'}
    \) in $\aTLTS \comp \bTLTS$ and
    \(
    \compStates{\aNFAState_2}{\bNFAState_2}
    \LabelledTrans{\lbl{\bLbl}{\dLbl}}
    \compStates{\aNFAState_2'}{\bNFAState_2'}
    \) in $(\cTLTS \comp \dTLTS)$, iff there exists $\eLbl \in \B^\bN$ and
    $\fLbl \in \B^\eN$ such that we have
    \(
    \aNFAState_1 \LabelledTrans{\lbl{\aLbl}{\eLbl}} \aNFAState_1'
    \) in $\aTLTS$,
    \(
    \bNFAState_1 \LabelledTrans{\lbl{\eLbl}{\cLbl}} \bNFAState_1'
    \) in $\bTLTS$,
    \(
    \aNFAState_2 \LabelledTrans{\lbl{\bLbl}{\fLbl}} \aNFAState_2'
    \) in $\cTLTS$,
    \(
    \bNFAState_2 \LabelledTrans{\lbl{\fLbl}{\dLbl}} \bNFAState_2'
    \) in $\dTLTS$. Then, we have
    \(
    \tensorStates{\aNFAState_1}{\aNFAState_2}
    \LabelledTrans{\lbl{\aLbl\bLbl}{\eLbl\fLbl}}
    \tensorStates{\aNFAState_1'}{\aNFAState_2'}
    \) in $\aTLTS \tensor \cTLTS$, and
    \(
    \tensorStates{\bNFAState_1}{\bNFAState_2}
    \LabelledTrans{\lbl{\eLbl\fLbl}{\cLbl\dLbl}}
    \tensorStates{\bNFAState_1'}{\bNFAState_2'}
    \) in $\bTLTS \tensor \dTLTS$, giving
    \(
    \compStates{\tensorStates{\aNFAState_1}{\aNFAState_2}}{\tensorStates{\aNFAState_1'}{\aNFAState_2'}}
    \LabelledTrans{\lbl{\aLbl\bLbl}{\cLbl\dLbl}}
    \compStates{\tensorStates{\bNFAState_1}{\bNFAState_2}}{\tensorStates{\bNFAState_1'}{\bNFAState_2'}}
    \) in $(\aTLTS \tensor \cTLTS) \comp (\bTLTS \tensor \dTLTS)$,
    as required.
\end{proof}

This is enough to prove that \TLTSCat{} is (strict) monoidal:

\begin{proposition}\label{prop:TLTSMonCat}
    \TLTSCat{} is a strict monoidal category.
\end{proposition}
\begin{proof}
    The tensor product is addition on objects, which is strictly associative,
    and has 0 as identity. The required coherence conditions of
    \defnref{defn:monCat} are then identities. Finally, by
    \propref{prop:TLTSMonCatProps}, we have the required proof of
    bifunctoriality.
\end{proof}

As we did for \PNB{}s, we may also embed $\aN$-permutations into a \TLTS{}:

\begin{definition}[Permutation \TLTS{}]
    A \TLTS, $\exPermTLTS \withNetType{\aN}{\aN}$, is a Permutation \TLTS{} iff
    $\perm$ is a $\aN$-permutation, it has a single state, $\aLTSState$, and
    transitions:
    \[
    \setBuilder{\aLTSState \LabelledTrans{\aLbl} \aLTSState}{\aLbl \in
        \setBuilder{\lbl{\ordinalSetToBinary{\aOrdinal}}{\ordinalSetToBinary{\perm(\aOrdinal)}}}{\aOrdinal
        \in \powerset{\ordinal{\aN}}}
    }
    \]
    $\exPermTLTS$ is illustrated in \figref{fig:exPermTLTS}.
\end{definition}

\begin{figure}[ht]
    \centering
    \begin{tikzpicture}[nfa]
        \node [state] (p0) {};
        \path (p0) edge[loop right] node
            {$\setBuilder{\lbl{\ordinalSetToBinary{\aOrdinal}}{\ordinalSetToBinary{\perm(\aOrdinal)}}}{\aOrdinal
                \in \powerset{\ordinal{\aN}}}$}
        (p0);
\end{tikzpicture}
\caption{\TLTS{} $\exPermTLTS \withNetType{\aN}{\aN}$}
\label{fig:exPermTLTS}
\end{figure}

A specific permutation \TLTS{} is that which embeds the $\swPerm{\aN}{\bN}$
permutation: for $\aN, \bN \in \N$, there is a \TLTS{}, $\TLTSbr{\aN}{\bN}
\withNetType{\aN + \bN}{\bN + \aN}$, illustrated in \figref{fig:TLTSbr}.
$\TLTSbr{\aN}{\bN}$ has a single state, and has self-loops labelled with
$\setBuilder{\lbl{\aLbl\bLbl}{\bLbl\aLbl}}{\aLbl \in \B^\aN, \bLbl \in
\B^\bN}$.

\begin{figure}[ht]
    \centering
    \begin{tikzpicture}[nfa]
        \node [state] (p0) {};
        \path (p0) edge[loop right] node
        {$\setBuilder{\lbl{\aLbl\bLbl}{\bLbl\aLbl}}{\aLbl \in \B^\aN, \bLbl \in
        \B^\bN}$} (p0);
\end{tikzpicture}
\caption{\TLTS{} $\TLTSbr{\aN}{\bN} \withNetType{\aN+\bN}{\bN+\aN}$}
\label{fig:TLTSbr}
\end{figure}

\begin{proposition}\label{prop:TLTSbraidingNatural}
    For $\aTLTS \withNetType{\aN}{\bN}$ and $\bTLTS \withNetType{\cN}{\dN}$, the
    following holds:
    \[
        (\abTLTSTensor) \comp \TLTSbr{\bN}{\dN}
        \NFAIso
        \TLTSbr{\aN}{\cN} \comp (\bTLTS \tensor \aTLTS)
    \]
\end{proposition}
\begin{proof}
    Immediate from the definitions of \TLTS{ } composition and $\TLTSbr{-}{-}$.
    Letting $\cNFAState$ be the single state of $\TLTSbr{\bN}{\dN}$ and
    $\dNFAState$ be that of $\TLTSbr{\aN}{\cN}$, we have:
    \begin{flalign*}
        &&\compStates{\tensorStates{\aNFAState}{\bNFAState}}{\cNFAState}
        \LabelledTrans{\lbl{\aLbl\cLbl}{\dLbl\bLbl}}
        \compStates{\tensorStates{\aNFAState'}{\bNFAState'}}{\cNFAState}
        &\in \abTLTSTensor \comp \TLTSbr{\bN}{\dN}\\
        \iff
        &&\tensorStates{\aNFAState}{\bNFAState}
        \LabelledTrans{\lbl{\aLbl\cLbl}{\bLbl\dLbl}}
        \tensorStates{\aNFAState'}{\bNFAState'}
        &\in \abTLTSTensor \\
        \iff
        &&\aNFAState
        \LabelledTrans{\lbl{\aLbl}{\bLbl}}
        \aNFAState'
        &\in \aTLTS \text{, and }\\
        &&
              \bNFAState
              \LabelledTrans{\lbl{\cLbl}{\dLbl}}
              \bNFAState'
              &\in \bTLTS\\
        \iff
        &&\tensorStates{\bNFAState}{\aNFAState}
        \LabelledTrans{\lbl{\cLbl\aLbl}{\dLbl\bLbl}}
        \tensorStates{\bNFAState'}{\aNFAState'}
        &\in \bTLTS \tensor \aTLTS\\
        \iff
        &&\compStates{\tensorStates{\dNFAState}{\bNFAState}}{\aNFAState}
        \LabelledTrans{\lbl{\aLbl\cLbl}{\dLbl\bLbl}}
        \compStates{\tensorStates{\dNFAState}{\bNFAState'}}{\aNFAState'}
        &\in \TLTSbr{\aN}{\cN} \comp \bTLTS \tensor \aTLTS
    \end{flalign*}
\end{proof}

\begin{proposition}\label{prop:TLTSbraidingHexagonAxioms}
    The following are isomorphisms:
    \begin{enumerate}[(a)]
        \item $\TLTSbr{\aN}{\bN + \cN} \NFAIso (\TLTSbr{\aN}{\bN} \tensor
            \idTLTS{\cN}) \comp (\idTLTS{\bN} \tensor \TLTSbr{\aN}{\cN})$
        \item $\TLTSbr{\aN + \bN}{\cN} \NFAIso (\idTLTS{\aN} \tensor
            \TLTSbr{\bN}{\cN}) \comp ( \TLTSbr{\aN}{\cN}\tensor \idTLTS{\bN})$
    \end{enumerate}
\end{proposition}
\begin{proof}
    We give a proof for (b), the proof of (a) is similar:

    Observe that $\idTLTS{\aN}$,
    $\TLTSbr{\bN}{\cN}$, $\TLTSbr{\aN}{\cN}$ and $\idTLTS{\bN}$ all have a
    single state, thus their composition also has a single state.

    The transitions of $\TLTSbr{\aN + \bN}{\cN}$ are labelled by:
    \[
        \setBuilder{\lbl{\aLbl\bLbl}{\bLbl\aLbl}}
                {\aLbl \in \B^{\aN+\bN}, \bLbl \in \B^{\cN}} =
    \setBuilder{\lbl{\cLbl\dLbl\bLbl}{\bLbl\cLbl\dLbl}}
               {\cLbl \in \B^\aN, \dLbl \in \B^\bN, \bLbl \in \B^{\cN}}
           \]

    On the RHS, we have that $(\idTLTS{\aN} \tensor \TLTSbr{\bN}{\cN})$ has
    transitions labelled with:
    \[
    \setBuilder{\lbl{\cLbl\dLbl\bLbl}{\cLbl\bLbl\dLbl}}
                {\cLbl \in \B^\aN, \dLbl \in \B^\bN, \bLbl \in \B^{\cN}}
    \]
    and $(\TLTSbr{\aN}{\cN}\tensor \idTLTS{\bN})$ has transitions labelled
    with:
    \[
        \setBuilder{\lbl{\cLbl\bLbl\dLbl}{\bLbl\cLbl\dLbl}}
                {\cLbl \in \B^\aN, \dLbl \in \B^\bN, \bLbl \in \B^{\cN}}
    \]
    thus
    $(\idTLTS{\aN} \tensor \TLTSbr{\bN}{\cN}) \comp (\TLTSbr{\aN}{\cN}\tensor
    \idTLTS{\bN})$ has transitions labelled by
    \[
        \setBuilder{\lbl{\cLbl\dLbl\bLbl}{\bLbl\cLbl\dLbl}}
                {\cLbl \in \B^\aN, \dLbl \in \B^\bN, \bLbl \in \B^{\cN}}
    \] as required.
\end{proof}

\begin{proposition}\label{prop:TLTSSymmetry}
    We have the following isomorphism:
    \[
        \TLTSbr{\aN}{\bN} \comp \TLTSbr{\bN}{\aN}
        \NFAIso
        \idTLTS{\aN + \bN}
    \]
\end{proposition}
\begin{proof}
    $\TLTSbr{\aN}{\bN}$ has transitions labelled by
    \[
        \setBuilder{\lbl{\aLbl\bLbl}{\bLbl\aLbl}}
                {\aLbl \in \B^\aN, \bLbl \in \B^\bN}
    \]
    and $\TLTSbr{\bN}{\aN}$ has transitions labelled by
    \[
        \setBuilder{\lbl{\bLbl\aLbl}{\aLbl\bLbl}}
                {\bLbl \in \B^\bN, \aLbl \in \B^\aN}
    \]
    by the definition of composition, $\TLTSbr{\aN}{\bN} \comp
    \TLTSbr{\bN}{\aN}$ has transitions labelled by
    \[
        \setBuilder{\lbl{\aLbl\bLbl}{\aLbl\bLbl}}
                {\aLbl \in \B^\aN, \bLbl \in \B^\bN}
        = \setBuilder{\lbl{\cLbl}{\cLbl}}{\cLbl \in \B^{\aN+\bN}}
    \] as required.
\end{proof}

We can now prove that \TLTSCat{} is a strict symmetric monoidal category, and
indeed, a PROP:

\begin{proposition}\label{prop:TLTSSymMonCat}
    \TLTSCat{} is a strict symmetric monoidal category.
\end{proposition}
\begin{proof}
    By Propositions \ref{prop:TLTSbraidingNatural},
    \ref{prop:TLTSbraidingHexagonAxioms} and \ref{prop:TLTSSymmetry}.
\end{proof}

\begin{proposition}
    \TLTSCat{} is a PROP.
\end{proposition}
\begin{proof}
    By propositions \ref{prop:TLTSCategory}, \ref{prop:TLTSMonCat} and
    \ref{prop:TLTSSymMonCat}.
\end{proof}

\section{Mapping between \PNBCat{} and \TLTSCat{}}

Given $\aPNB \withNetType{\aN}{\bN}$, we can generate its \TLTS{} statespace,
$\PNBToTLTS{\aPNB} \withNetType{\aN}{\bN}$, using the firing semantics
described in \secref{sec:PNBfiring}.

This mapping respects the identity \PNB{}:

\begin{proposition}\label{prop:functorId}
    For any $\aN \in \N$, $\PNBToTLTS{\PNBid{\aN}} \NFAIso \idTLTS{\aN}$.
\end{proposition}
\begin{proof}
    There are no places in $\PNBid{\aN}$, giving only one possible state in
    $\PNBToTLTS{\PNBid{\aN}}$, corresponding to the empty marking.  Each
    transition of $\PNBid{\aN}$ is always enabled, allowing arbitrary subsets
    of transitions to be fired, giving transitions in
    $\PNBToTLTS{\PNBid{\aN}}$ labelled by each element of $\setBuilder{x}{x \in
    \B^\aN}$. Therefore, we have $\PNBToTLTS{\PNBid{\aN}} = \idTLTS{\aN}$, as
    required.
\end{proof}

and further, respects the compositions of \PNB{}s:

We abuse notation when referring to states of the \TLTS{} corresponding to a
PNB composed of $\aPNB$ and $\bPNB$. Such states are markings of the
underlying \PNB; that is, some $M \subseteq \powerset{\places{\aPNB}
\disjointUnion \places{\bPNB}}$, indeed, we can partition $M$ into $x$, a
marking of $\aPNB$ and $y$, a marking of $\bPNB$, justifying our notation $(x,
y)$.

\begin{proposition}\label{prop:functorComps}
    For any pair of \PNB{}s, $\aPNB \withNetType{\aN}{\bN}$ and $\bPNB
    \withNetType{\bN}{\cN}$, we have that
    \[
    \PNBToTLTS{\aPNB \comp \bPNB}
    \NFAIso
    \PNBToTLTS{\aPNB} \comp \PNBToTLTS{\bPNB}
    \]
\end{proposition}
\begin{proof}
    The states of $\PNBToTLTS{\aPNB \comp \bPNB}$ are of
    the form $\powerset{\places{\aPNB} \disjointUnion \places{\bPNB}}$, isomorphic
    to those of $\PNBToTLTS{\aPNB} \comp \PNBToTLTS{\bPNB}$, which are
    of the form $\powerset{\places{\aPNB}} \times 2^{\places{\bPNB}}$.
    By~\cite[Theorem 3.8]{Bruni2013}, transitions,
    \(
    (\aNFAState, \bNFAState)
    \LabelledTrans{\lbl{\aLbl}{\bLbl}}
    (\aNFAState', \bNFAState')
    \), exist in
    $\PNBToTLTS{\aPNB \comp \bPNB}$ iff, for $\cLbl \in \B^\bN$, there
    are transitions $\aNFAState \LabelledTrans{\lbl{\aLbl}{\cLbl}}
    \aNFAState'$ in $\PNBToTLTS{\aPNB}$ and $\bNFAState
    \LabelledTrans{\lbl{\cLbl}{\bLbl}} \bNFAState'$ in
    $\PNBToTLTS{\bPNB}$, corresponding to transitions in $\PNBToTLTS{\aPNB}
    \comp \PNBToTLTS{\bPNB}$.
\end{proof}

It follows that $\PNBToTLTS{-} : \text{\PNBCat} \to \text{\TLTSCat}$ is a functor:

\begin{proposition}
    $\PNBToTLTS{-}$ is a functor: identity on objects and firing semantics
    (\defnref{defn:firePNBTransitions}) on morphisms.
\end{proposition}
\begin{proof}
    By propositions \ref{prop:functorId} and \ref{prop:functorComps}.
\end{proof}

\begin{lemma} \label{lem:functorTensorArrows}
    For any pair of \PNB{}s, $\aPNB \withNetType{\aN}{\bN}$ and $\bPNB
    \withNetType{\cN}{\dN}$, we have:

    $(\aNFAState, \bNFAState) \LabelledTrans{\lbl{\aLbl\cLbl}{\bLbl\dLbl}}
    (\aNFAState', \bNFAState') \in \PNBToTLTS{\aPNB \tensor \bPNB}$ iff there
    are $\aNFAState \LabelledTrans{\lbl{\aLbl}{\bLbl}} \aNFAState' \in
    \PNBToTLTS{\aPNB}$ and $\bNFAState \LabelledTrans{\lbl{\cLbl}{\dLbl}}
    \bNFAState' \in \PNBToTLTS{\bPNB}$.
\end{lemma}
\begin{proof}
    $(\Rightarrow)$ Suppose we have $(\aNFAState, \bNFAState)
    \LabelledTrans{\lbl{\aLbl\cLbl}{\bLbl\dLbl}} (\aNFAState', \bNFAState')$
    in $\PNBToTLTS{\aPNB \tensor \bPNB}$. There is a contention-free
    set of enabled transitions, $\aPNBTransSet \in \trans{\aPNB \tensor
    \bPNB}$. Indeed, we can partition $\aPNBTransSet$ into (contention-free,
    enabled) $\bPNBTransSet \subseteq \trans{\aPNB}$ and $\cPNBTransSet
    \subseteq \trans{\bPNB}$, with $\ordinalSetToBinary{\source{\bPNBTransSet}}
    = \aLbl$, $\ordinalSetToBinary{\target{\bPNBTransSet}} = \bLbl$,
    $\ordinalSetToBinary{\source{\cPNBTransSet}} = \cLbl$ and
    $\ordinalSetToBinary{\target{\cPNBTransSet}} = \dLbl$, which correspond to
    the required transitions in $\PNBToTLTS{\aPNB}$ and $\PNBToTLTS{\bPNB}$.

    $(\Leftarrow)$ Suppose we have $\aNFAState
    \LabelledTrans{\lbl{\aLbl}{\bLbl}} \aNFAState'$ in $\PNBToTLTS{\aPNB}$
    and $\bNFAState \LabelledTrans{\lbl{\cLbl}{\dLbl}} \bNFAState'$ in
    $\PNBToTLTS{\bPNB}$; then, there exist contention-free, enabled
    $\bPNBTransSet \subseteq \trans{\aPNB}$ and $\cPNBTransSet \subseteq
    \trans{\bPNB}$, with $\ordinalSetToBinary{\source{\bPNBTransSet}} =
    \aLbl$, $\ordinalSetToBinary{\target{\bPNBTransSet}} = \bLbl$,
    $\ordinalSetToBinary{\source{\cPNBTransSet}} = \cLbl$, and
    $\ordinalSetToBinary{\target{\cPNBTransSet}} = \dLbl$. By definition,
    $\bPNBTransSet \disjointUnion \cPNBTransSet \subseteq \trans{\aPNB \tensor
    \bPNB}$, giving the required transition in $\PNBToTLTS{\aPNB \tensor
    \bPNB}$.
\end{proof}

\begin{proposition} \label{prop:functorTensors}
    For any pair of \PNB{}s, $\aPNB \withNetType{\aN}{\bN}$ and $\bPNB
    \withNetType{\cN}{\dN}$, we have that
    \[
    \PNBToTLTS{\aPNB} \tensor \PNBToTLTS{\bPNB}
    \TLTSIso
    \PNBToTLTS{\aPNB \tensor \bPNB}
    \]
\end{proposition}
\begin{proof}
    Similar to the proof of \propref{prop:functorComps}, employing
    \lemref{lem:functorTensorArrows} to show equality of labelled transitions.
\end{proof}

\begin{proposition}
    $\PNBToTLTS{-}$ is a strict monoidal functor.
    \label{prop:PNBToTLTSMonFunctor}
\end{proposition}
\begin{proof}
    The isomorphic \TLTS{}s of \propref{prop:functorTensors} form\footnote{Recall that morphisms of
    \TLTSCat{} are isomorphism classes of \TLTS{}s.} a natural transformation with each component
    being an identity map. Similarly, the unit morphism is an identity map, since $\PNBToTLTS{-}$
    is the identity mapping on objects.
\end{proof}

\begin{proposition} \label{prop:functorBraiding}
    We have a \TLTS{} isomorphism, $\TLTSbr{\aN}{\bN} \TLTSIso
    \PNBToTLTS{\PNBbr{\aN}{\bN}}$.
\end{proposition}
\begin{proof}
    The \PNB{} $\PNBbr{\aN}{\bN}$ (shown in \figref{fig:netBr}) has no places,
    hence its \TLTS{} statespace has a single place corresponding to the
    empty marking. Since the same permutation is embedded in the \PNB{} and
    \TLTS{}, by definition, the generated \TLTS{} statespace will have
    transitions labelled the same as those in $\TLTSbr{\aN}{\bN}$, as required.
\end{proof}

\begin{proposition}
    $\PNBToTLTS{-}$ is a strict symmetric monoidal functor.
\end{proposition}
\begin{proof}
    Since $\tensorAssoc$ is formed of identities, the commutativity
    of~\figref{fig:braidingCoherence} is reduced to requiring equality of the top and bottom
    morphisms; since morphisms of \TLTSCat{} are equivalence classes, this requirement is satisfied
    by~\propref{prop:functorBraiding}.
\end{proof}

This statement is tantamount to saying that $\PNBToTLTS{-}$ is a homomorphism
of PROPs.

\begin{proposition}
    $\PNBToTLTS{-}$ is a homomorphism of PROPs.
\end{proposition}
\begin{proof}
    Immediate.
\end{proof}

Observe that \propref{prop:functorComps} and \propref{prop:functorTensors} are
precisely equivalent to the standard definition of \emph{compositionality},
which states that the semantics of a composite PNB is determined by the
semantics of the component PNBs. Put differently, there is no emergent
behaviour when composing the semantics of the components --- the semantics
embody a precise account of all possible behaviours of the components. Indeed,
we revisit compositionality, in its standard setting, in
\propref{prop:compsem}.

\section{Encoding Reachability}

We can annotate \PNB{}s with a pair of markings, encoding a particular initial
and target marking, as described in \secref{sec:markedPNB}. Intuitively, such
an annotated \PNB{} represents a reachability problem: can the target marking
be reached from the initial marking?

\subsection{The category of \mPNB{}s}

\mPNB{}s form a strict symmetric monoidal category, \mPNBCat, by lifting the
various constructions from \PNB{}s to \mPNB{}s. We do not offer proofs for the
definitions in this section, since they follow simply from the corresponding
definitions on \PNB{}s:

\begin{definition}\label{defn:mPNBComp}
    Synchronous composition of \mPNB{}s, $\amPNB$ and $\bmPNB$ is defined as:
    $\amPNB \comp \bmPNB \defeq (\aPNB \comp \bPNB,
    \aPNBMarking \disjointUnion \aPNBMarking',
    \bPNBMarking \disjointUnion \bPNBMarking')$
\end{definition}

The category of \mPNB{}s has structure:

\begin{itemize}
\item Objects are the natural numbers, \N,
\item Arrows from $\aN$ to $\bN$ are the isomorphism classes of \mPNB{}s:
    $\isoClass{\amPNB \withNetType{\aN}{\bN}}$,
\item The identity morphism for $\aN \in \N$, is
    $(\PNBid{\aN}, \emptyset, \emptyset)$
\item The composition of morphisms is per \defnref{defn:mPNBComp}.
\end{itemize}

\begin{proposition}
    \mPNBCat{} is a category. \qed
\end{proposition}

\begin{definition}\label{defn:mPNBTensor}
    Tensor composition of \mPNB{}s, $\amPNB$ and $\bmPNB$ is defined as:
    $\amPNB \tensor \bmPNB \defeq (\aPNB \tensor \bPNB,
    \aPNBMarking \disjointUnion \aPNBMarking',
    \bPNBMarking \disjointUnion \bPNBMarking')$
\end{definition}

\mPNBCat{} has a monoidal structure:

\begin{itemize}
    \item The tensor product is addition on objects and the operation described
        in \defnref{defn:mPNBTensor} on morphisms,
    \item The unit is $0$,
    \item The associator and left/right unitors are identity natural
        isomorphisms.
\end{itemize}

\begin{proposition}
    \mPNBCat{} is a strict monoidal category. \qed
\end{proposition}

Furthermore, \mPNBCat{} has a symmetric structure, given by the following
morphism:
\[
    (\PNBbr{\aN}{\bN}, \emptyset, \emptyset) \withNetType{\aN + \bN}{\bN + \aN}
\]

\begin{proposition}
    \mPNBCat{} is a symmetric monoidal category. \qed
\end{proposition}

\subsection{The category of \TNFA{}s}

A \TNFA{} is a \TLTS{}, together with a chosen initial state and a set of accepting states, as
defined in \secref{sec:TNFA}.

We then have that \TNFACat{}, the category of \TNFA{}s has structure:

\begin{itemize}
\item Objects are the natural numbers, \N,
\item Arrows from $\aN$ to $\bN$ are the isomorphism classes of \TNFA{}s:
    $\isoClass{\aTNFA \withNetType{\aN}{\bN}}$,
\item The identity morphism for $\aN \in \N$, is
    $(\idTLTS{\aN}, 0, \setof{0})$, where $0$ is the state of $\idTLTS{\aN}$,
\item The composition of morphisms is per \defnref{defn:sequentialCompositionTNFA}.
\end{itemize}

\begin{proposition}
    \TNFACat{} is a category.\qed
\end{proposition}

We have a monoidal structure on \TNFACat{}:

\begin{itemize}
    \item The tensor product is addition on objects and the operation described
        in \defnref{defn:tensorCompositionTNFA} on morphisms,
    \item The unit object is $0$,
    \item The associator and left/right unitors are identity natural
        isomorphisms.
\end{itemize}

\begin{proposition}
    \TNFACat{} is a strict monoidal category. \qed
\end{proposition}

Furthermore, \TNFACat{} has a symmetric structure, given by the following
morphism:
\[
    (\TLTSbr{\aN}{\bN}, 0, \setof{0}) \withNetType{\aN + \bN}{\bN + \aN}
\] where $0$ is the single state of $\TLTSbr{\aN}{\bN}$.

\begin{proposition}
    \TNFACat{} is a symmetric monoidal category. \qed
\end{proposition}

\subsection{Mapping between \mPNBCat{}, \TNFACat{}, \PNBCat{} and
\TLTSCat{}}\label{sec:functorPNBTLTS}

\mPNB{}s can be given a \TNFA{} semantics:

\begin{definition}[\TNFA{} semantics of \mPNB{}s]
    For a \mPNB{}, $\amPNB$, its \TNFA{} semantics, $\mPNBToTNFA{\amPNB}$, is
    $(\aTLTS, \aNFAInitState, \aNFAAcceptStates)$, where $\aTLTS$ is the
    \TLTS{} semantics of $\aPNB$, $\aNFAInitState$ is the state
    corresponding to $\aPNBMarking$, and $\aNFAAcceptStates$ is the (singleton
    set containing the) state corresponding to $\bPNBMarking$.
\end{definition}

\begin{proposition}
    $\mPNBToTNFA{-} : \mPNBCat \to \TNFACat$ is a strict symmetric monoidal
    functor, identity on objects and \TNFA{} semantics on morphisms. \qed
\end{proposition}

Furthermore, we have two forgetful (strict symmetric monoidal) functors,
mapping \mPNBCat{} to \PNBCat{}, and \TNFACat{} to \TLTSCat{}. The forgetful
functor on \mPNBCat, $\mPNBToPNB : \mPNBCat \to \PNBCat$ is identity on
objects, and forgets the markings on morphisms:
\[
    \mPNBToPNB(\amPNB) \defeq \aPNB
\]
Similarly, we can forget the initial and accepting states of a \TNFA, to obtain
a \TLTS. The forgetful functor $\TNFAToTLTS : \TNFACat \to \TLTSCat$ is
identity on objects, and on arrows:
\[
    \TNFAToTLTS(\aTNFA) \defeq \aTLTS
\]

Indeed, it is equivalent to take the \TNFA{} semantics of a \mPNB{}, before
forgetting the initial and accepting states, or to forget the initial/target
markings of the \mPNB before taking the \TLTS{} semantics. This equivalence is
embodied in the commuting diagram of \figref{fig:categoryRelationships}.

\begin{figure}[ht]
    \centering
\begin{tikzpicture}
    \node (mPNB) {\mPNBCat{}};
    \node (PNB) [below=of mPNB] {\PNBCat{}};
    \node (TNFA) [right=of mPNB] {\TNFACat{}};
    \node (TLTS) [below=of TNFA] {\TLTSCat{}};

    \draw [->] (mPNB) to node [above] {$\mPNBToTNFA{-}$} (TNFA);
    \draw [->] (mPNB) to node [left] {$\mPNBToPNB$} (PNB);
    \draw [->] (PNB) to node [below] {$\PNBToTLTS{-}$} (TLTS);
    \draw [->] (TNFA) to node [right] {$\TNFAToTLTS$} (TLTS);
\end{tikzpicture}
\caption{Commuting diagram illustrating the categories' relationships}
\label{fig:categoryRelationships}
\end{figure}

\subsection{Summary}

In this chapter we have illustrated the categorical structure of PNBs and
\TLTS{}s and the relationship between these categories. We highlighted that the
functorial relationship between the categories \PNBCat{} and \TLTSCat{} is
equivalent to the property of compositionality and lifted the constructions to
\emph{marked} PNBs and corresponding \TNFA{}s leading to the categories
\mPNBCat{} and \TNFACat{} and illustrated their inter-relations.
