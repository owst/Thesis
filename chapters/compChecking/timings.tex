\section{Performance of the Compositional Algorithm}\label{sec:performance}

\newcommand{\exampleRef}[5][0]{%
\item[] {%
\begin{tabular}{r l}%
    \emph{system}:& {\bf #2} (\secref{sec:example-#3})\\
    \emph{initial}:& \parbox[t]{0.75\linewidth}{#4} \\
    \emph{target} $(\mathlarger{\mathlarger{\ifthenelse{#1=1}{\yesReachable}{\noReachable}}})$:& \parbox[t]{0.75\linewidth}{#5} \\
\end{tabular}}}

We now turn to examining the performance of \algref{alg:PNBAlgorithm}, using
the example systems introduced in \secref{sec:benchmarks}.  To do so, we must
first specify the particular initial and target markings that we will use for
each system. These markings are specified in \figref{fig:slowmarkings} (some
are expected to be reachable, some not).

\begin{figure}[ht]
\begin{multicols}{2}
\begin{itemize}
    \exampleRef[1]{\bufferSys{-}}{buffer}
        {``empty'': only tokens in upper places}
        {``full'': only tokens in lower places}
    \exampleRef{\tokenringSys{-}}{token-ring}
        {System ready: workers idle}
        {System started: \emph{all} workers working}
    \exampleRef[1]{\contree{-}{-}}{trees}
        {Root contains the only token}
        {Leaves each contain a token}
    \exampleRef{\distree{-}{-}}{trees}
        {Root contains the only token}
        {Leaves each contain a token}
    \exampleRef[1]{\cliqueSys{-}}{cliques}
        {Injector contains the only token}
        {Final (rightmost) place contains the only token}
    \exampleRef[1]{\powersetSys{-}}{powersets}
        {Injector contains the only token}
        {All places except root contain a token}
    \exampleRef{\overtakeSys{-}}{over}
        {All locks and interfaces unlocked, each car ready}
        {All cars overtaking}
    \exampleRef[1]{\hartstoneSys{-}}{hartstone}
        {Controllers, tasks and master all ready}
        {Each task working}
    \exampleRef[1]{\iteratedchoiceSys{-}}{iter-choice}
        {\addtokC{} contains the only token}
        {Alternating taken/not-taken marking, \addtokC{} has no token}
    \exampleRef[1]{\replicatorsSys{-}}{replicator}
        {\addtokC{} contains the only token}
        {\taketokC{} contains the only token}
    \exampleRef{\DACSys{-}}{dac}
        {Controller and workers ready}
        {Every worker and controller awaiting a join}
    \exampleRef[1]{\diningphilosophersSys{-}}{philos}
        {Philos ready, forks present}
        {Every fork taken}
    \exampleRef[1]{\cyclicschedulerSys{-}}{cyclic}
        {Injector contains token, each scheduler ready}
        {Each task working}
    \exampleRef[1]{\counterSys{-}}{counter}
        {Counter set to 0: all components zero}
        {Counter set to n: tester recognises counter is full}
\end{itemize}
\end{multicols}
\newcommand{\capt}{Example markings used in benchmarking
\algref{alg:PNBAlgorithm}.}
\caption[\capt]{\capt{} Those marked $\yesReachable$ are expected to be
    reachable and those $\noReachable$ are not}
\label{fig:slowmarkings}
\end{figure}

Given these marked examples, we execute our implementation, recording the total
time taken, averaged over 5 runs. The results are illustrated in
\tabref{tab:slowtimings}.

\begin{table}[ht]
\centering
\newcommand{\mycaption}{Checking reachability of markings in \figref{fig:slowmarkings}, using \algref{alg:PNBAlgorithm}}
\caption[\mycaption]{\mycaption. Key: M = Maximum \# States in a Composition, T = Time (s), R? = Reachable?}
\label{tab:slowtimings}
\makebox[\textwidth][c]{
\rowcolors{2}{gray!25}{white}
\begin{tabular}{ | c | c | c | c | }
\hline
Sys & M & T & R? \\ \hline
\bufferSys{3}	&	(8,1)	&	0.004	&	$\yesReachable$ \\
\bufferSys{6}	&	(64,1)	&	0.027	&	$\yesReachable$ \\
\bufferSys{9}	&	(512,1)	&	0.258	&	$\yesReachable$ \\
\bufferSys{12}	&	(4096,1)	&	3.270	&	$\yesReachable$ \\
\tokenringSys{1}	&	(12,1)	&	0.017	&	$\yesReachable$ \\
\tokenringSys{2}	&	(72,1)	&	0.092	&	$\noReachable$ \\
\tokenringSys{4}	&	(2592,1)	&	5.422	&	$\noReachable$ \\
\tokenringSys{5}	&	(15552,1)	&	49.881	&	$\noReachable$ \\
\contree{1}{10}	&	(2,1024)	&	1.506	&	$\yesReachable$ \\
\contree{2}{4}	&	(104,104)	&	3.560	&	$\yesReachable$ \\
\contree{4}{2}	&	(10,4)	&	0.013	&	$\yesReachable$ \\
\contree{12}{2}	&	(2050,4)	&	19.455	&	$\yesReachable$ \\
\distree{1}{8}	&	(2,256)	&	0.372	&	$\yesReachable$ \\
\distree{2}{3}	&	(2,16384)	&	10.741	&	$\noReachable$ \\
\distree{3}{2}	&	(2,4096)	&	2.188	&	$\noReachable$ \\
\distree{15}{1}	&	(2,32768)	&	26.002	&	$\noReachable$ \\
\cliqueSys{2}	&	(16,1)	&	0.030	&	$\yesReachable$ \\
\cliqueSys{4}	&	(64,1)	&	0.134	&	$\yesReachable$ \\
\cliqueSys{7}	&	(512,1)	&	2.314	&	$\yesReachable$ \\
\cliqueSys{9}	&	(2048,1)	&	20.200	&	$\yesReachable$ \\
\powersetSys{3}	&	(2,8)	&	0.006	&	$\yesReachable$ \\
\powersetSys{6}	&	(2,64)	&	0.055	&	$\yesReachable$ \\
\powersetSys{9}	&	(2,512)	&	0.686	&	$\yesReachable$ \\
\powersetSys{12}	&	(2,4096)	&	12.410	&	$\yesReachable$ \\
\overtakeSys{1}	&	(12,7)	&	0.068	&	$\yesReachable$ \\
\overtakeSys{3}	&	(20,102)	&	0.596	&	$\noReachable$ \\
\overtakeSys{4}	&	(20,594)	&	3.985	&	$\noReachable$ \\
\overtakeSys{5}	&	(20,3462)	&	32.502	&	$\noReachable$ \\
\hline
\end{tabular}
\rowcolors{2}{gray!25}{white}
\begin{tabular}{ | c | c | c | c | }
\hline
Sys & M & T & R? \\ \hline
\hartstoneSys{2}	&	(6,8)	&	0.039	&	$\noReachable$ \\
\hartstoneSys{4}	&	(6,32)	&	0.136	&	$\noReachable$ \\
\hartstoneSys{8}	&	(6,512)	&	2.696	&	$\noReachable$ \\
\hartstoneSys{10}	&	(6,2048)	&	12.174	&	$\noReachable$ \\
\iteratedchoiceSys{1}	&	(2,16)	&	0.008	&	$\yesReachable$ \\
\iteratedchoiceSys{2}	&	(2,256)	&	0.042	&	$\yesReachable$ \\
\iteratedchoiceSys{3}	&	(2,4096)	&	0.790	&	$\yesReachable$ \\
\iteratedchoiceSys{4}	&	(2,65536)	&	17.548	&	$\yesReachable$ \\
\replicatorsSys{1}	&	(2,4)	&	0.003	&	$\yesReachable$ \\
\replicatorsSys{3}	&	(2,32)	&	0.014	&	$\yesReachable$ \\
\replicatorsSys{6}	&	(2,851)	&	0.527	&	$\yesReachable$ \\
\replicatorsSys{8}	&	(2,7655)	&	7.557	&	$\yesReachable$ \\
\DACSys{10}	&	(65,5)	&	0.107	&	$\noReachable$ \\
\DACSys{25}	&	(350,5)	&	0.930	&	$\noReachable$ \\
\DACSys{50}	&	(1325,5)	&	6.178	&	$\noReachable$ \\
\DACSys{75}	&	(2925,5)	&	19.702	&	$\noReachable$ \\
\diningphilosophersSys{1}	&	(8,1)	&	0.038	&	$\yesReachable$ \\
\diningphilosophersSys{2}	&	(8,8)	&	0.114	&	$\yesReachable$ \\
\diningphilosophersSys{4}	&	(72,8)	&	1.146	&	$\yesReachable$ \\
\diningphilosophersSys{6}	&	(648,8)	&	12.689	&	$\yesReachable$ \\
\cyclicschedulerSys{1}	&	(2,5)	&	0.016	&	$\noReachable$ \\
\cyclicschedulerSys{2}	&	(2,18)	&	0.036	&	$\yesReachable$ \\
\cyclicschedulerSys{4}	&	(2,278)	&	0.512	&	$\yesReachable$ \\
\cyclicschedulerSys{6}	&	(2,4438)	&	11.273	&	$\yesReachable$ \\
\counterSys{1}	&	(2,1)	&	0.003	&	$\yesReachable$ \\
\counterSys{2}	&	(4,1)	&	0.010	&	$\yesReachable$ \\
\counterSys{4}	&	(16,1)	&	0.056	&	$\yesReachable$ \\
\counterSys{8}	&	(256,1)	&	1.238	&	$\yesReachable$ \\
\hline
\end{tabular}
}
\end{table}


Consider the first entry in \tabref{tab:slowtimings}, that of $\bufferSys{3}$,
which we evaluated by hand at the end of \secref{sec:compStatespaceGen}:
the maximum composition size encountered there (illustrated in
\figref{fig:exprtreestep3}) was $(8,1)$, agreeing with the recorded result.

For all systems, small parameters lead to run-times of $< 1$ second; but
small increments in the parameter sizes lead to vastly slower run-times, of
many seconds, accompanied with large composition sizes. Indeed, moving from a
parameter of $4$ to $5$ for $\overtakeSys{-}$ is catastrophic for the
algorithm's performance, with a much larger \TNFA{} composition encountered.

Indeed, the largest encountered compositions grows exponentially with the size
of the system - indeed, this should be intuitively clear, since the number of
markings of a net (and thus the number of states of its \TNFA{}) is exponential
in the number of places.

A final point to note is that in some cases, in particular,
\tokenringSys{-},\overtakeSys{-}, and \cyclicschedulerSys{-}, the result for
$\aN = 1$ differs to that for higher values of $\aN$. Indeed, inspecting the
desired markings for these systems, we observe that they specify the target
marking as a behaviour of all components, which can/cannot be reached for
multiple components, but however \emph{is} easily reachable for single
components.

Summarising these results, and to set the scene for the following chapter, we
can extract some key points that we will address:
\begin{enumerate}
    \item Without mitigation, the statespace explosion leads to NFA sizes that
        grow exponentially. We will investigate \emph{weak language preserving}
        reductions to reduce NFA sizes, while preserving correctness
        (\secref{sec:boundaryProtocol}).
    \item The behaviour of \bufferSys{3} should intuitively be no different to
        that of \bufferSys{4}, yet the current compositional algorithm does not
        exploit this. We will investigate \emph{fixed-points} of behaviour in
        such systems, and use \emph{memoisation} to avoid repeated work in
        their presence (\secref{sec:memoisation}).
    \item While PNB composition \emph{is} associative (up to isomorphism)
        w.r.t. the generated \TNFA{}, we will see that when employing
        reduction, it is \emph{not} associative w.r.t. the (size of)
        intermediate \TNFA{}s generated and thus the \emph{performance} of the
        algorithm. We will show that the performance is subtly affected by
        reassociating certain compositions (\secref{sec:reassociation}).
\end{enumerate}
