\chapter{Compositional Statespace Generation}\label{chp:compChecking}

In this chapter we introduce a technique for \emph{compositional} generation of
the global statespace of systems modelled using the PNB specifications of
\chpref{chp:benchmarksAndLang}, \emph{without} first forming the corresponding
composite global net. We show that because the boundary interactions of
component nets are recorded in their statespaces, we are able to reconstruct
the global statespace from the just the component statespaces.

Our goal is to use our method of compositional statespace generation
(\secref{sec:compStatespaceGen}) as the basis of a compositional approach for
checking reachability. However, the initial compositional approach we introduce
here makes no attempt to limit the effects of statespace explosion. Existing
approaches also consider the global statespace, but use techniques such as
unfoldings or partial-order reduction to avoid generating the entire
statespace. We will use compositionality to avoid the statespace explosion
problem, but first show that we can indeed generate the global statespace in a
compositional manner. While we prove our compositional technique correct
(\secref{sec:compCheckingProof}), evaluating its performance (\secref{sec:performance})
shows degradation due to statespace explosion. However, as we will demonstrate
in \chpref{chp:improveEfficiency}, with suitable optimisations, our
compositional technique is \emph{particularly efficient} for several systems.

\subimport{}{compositionalreachability}

\subimport{}{timings}

\subimport{}{proof}

\section{Conclusion}

In this chapter we introduced a technique for \emph{compositional} statespace
generation of systems specified using PNBs, using it to form the basis of
a reachability check. We proved the technique correct and illustrated the
performance of the technique using the example systems of
\chpref{chp:benchmarksAndLang}. Our results highlighted that without
mitigation, statespace explosion leads to poor performance, even in small
systems. To combat this, in the next chapter, we show that by exploiting
the component-wise specification of systems, we can mitigate the statespace
explosion problem and obtain good performance.
