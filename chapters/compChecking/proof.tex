\section{Proof of Correctness}\label{sec:compCheckingProof}

In the previous section we gave an intuitive view of the first step of our
compositional approach to checking reachability in systems modelled by \DSL{}
expressions. To complete this chapter, we now prove that
\algref{alg:compositionalAlgorithm} is correct. Correctness in this case refers
to obtaining isomorphic \TNFA{}s by \algref{alg:compositionalAlgorithm} and
\algref{alg:naiveAlgorithm}.

First, we give a formal definition of PNB expressions. As discussed in
\remref{rem:evalExpr}, we evaluate \DSL{} expressions to PNB
expressions, which are formed of two types of composition and component nets.
Therefore, we represent PNB expressions using \emph{wiring expressions}; a
wiring expression is the abstract syntax tree of a PNB expression, where
internal nodes are labelled with either $\comp$ or $\tensor$, and leaves are
variables. The grammar of wiring expressions is simply:
\[
    T \bnfEq x \bnfSep T \comp T \bnfSep T \tensor T
\]
where we again take `$\comp$' and `$\tensor$' to be left associative, with
`$\tensor$' binding tighter than `$\comp$'.

Now, a wiring expression, $\wiringExpr$, together with an assignment map,
$\wiringVarAssign$, taking variables to \emph{marked} PNBs, can be evaluated
recursively to obtain a single marked PNB, which we write as
$\wiringExprToPNB{\wiringExpr}{\wiringVarAssign}$. However, as mentioned
in~\remref{rem:compositionMarkedPNB}, before we can define this evaluation, we
require definitions of synchronous and tensor composition on \emph{marked} PNBs:
\begin{align*}
    \parens{\aPNB, \aPNBMarking, \bPNBMarking} \comp
    \parens{\bPNB, \cPNBMarking, \dPNBMarking} &=
    \parens{\aPNB \comp \bPNB, \aPNBMarking \disjointUnion \cPNBMarking,
        \bPNBMarking \disjointUnion \dPNBMarking} \\
    \parens{\aPNB, \aPNBMarking, \bPNBMarking} \tensor
    \parens{\bPNB, \cPNBMarking, \dPNBMarking} &=
    \parens{\aPNB \tensor \bPNB, \aPNBMarking \disjointUnion \cPNBMarking,
        \bPNBMarking \disjointUnion \dPNBMarking}
\end{align*}
we assure ourselves that the definition is correct by a simple lemma:
\begin{lemma}\label{lem:compositeMarkings}
    For PNBs, $\aPNB$ and $\bPNB$, $\aPNBMarking$ and $\bPNBMarking$ are
    markings of $\aPNB$ and $\bPNB$, respectively, iff $\aPNBMarking
    \disjointUnion \bPNBMarking$ is a marking of $\aPNB \comp \bPNB$ (and
    $\aPNB \tensor \bPNB$).
\end{lemma}
\begin{proof}
    Immediate, since $\places{\aPNB \comp \bPNB} = \places{\aPNB}
    \disjointUnion \places{\bPNB} = \places{\aPNB \tensor \bPNB}$.
\end{proof}

Now we may define $\wiringExprToPNB{\wiringExpr}{\wiringVarAssign}$:
\begin{align*}
    \wiringExprToPNB{x}{\wiringVarAssign} &= \wiringVarAssign(x) \\
    \wiringExprToPNB{\wiringExpr_1 \comp \wiringExpr_2}{\wiringVarAssign} &=
        \wiringExprToPNB{\wiringExpr_1}{\wiringVarAssign}
        \comp
        \wiringExprToPNB{\wiringExpr_2}{\wiringVarAssign} \\
    \wiringExprToPNB{\wiringExpr_1 \tensor \wiringExpr_2}{\wiringVarAssign} &=
        \wiringExprToPNB{\wiringExpr_1}{\wiringVarAssign}
        \tensor
        \wiringExprToPNB{\wiringExpr_2}{\wiringVarAssign}
\end{align*}
where we implicitly assume that variable assignments are compatible with
$\wiringExpr$, in the sense that only nets with compatible boundaries are
composed. However, recall that in \secref{sec:lang}, we introduced a type
system for the more expressive language, \DSL{}, which disallows invalid
expressions, and could be trivially adapted to wiring expressions.

\begin{example}\label{ex:wiringDecomp}
The following are the wiring expression and variable assignment corresponding
to~\eqref{eq:buf3}:
\[
    \wiringExpr
    =
    \wiringVar_1 \comp
    \wiringVar_2 \comp
    \wiringVar_2 \comp
    \wiringVar_2 \comp \wiringVar_3
\qquad
    \wiringVarAssign
    =
    \setof{
    \wiringVar_1 \mapsto \lendC{},
    \wiringVar_2 \mapsto \bufferC{},
    \wiringVar_3 \mapsto \rendC{}
    }
\]
observe that $\wiringExprToPNB{\wiringExpr}{\wiringVarAssign}$ is isomorphic to
the PNB shown in \figref{fig:markedbuffer3}.
\end{example}

At this point, we have defined the operation to collapse a wiring expression
with corresponding variable assignment, into a single composite \emph{PNB}. We
now turn to defining an operation to collapse a wiring expression into a single
\emph{\TNFA{}}.

Given a wiring expression $\wiringExpr$ and variable assignment
$\wiringVarAssign$ we can recursively translate $\wiringExpr$ into a \TNFA{},
which we write as $\wiringExprToTNFA{\wiringExpr}{\wiringVarAssign}$. Recall
from~\defnref{defn:PNBTNFA}, that for a marked PNB $\aPNB$, $\PNBToTNFA{\aPNB}$
is the corresponding \TNFA{} encoding reachability. We define
$\wiringExprToTNFA{\wiringExpr}{\wiringVarAssign}$ as follows:
\begin{align*}
    \wiringExprToTNFA{x}{\wiringVarAssign} &= \PNBToTNFA{\wiringVarAssign(x)} \\
    \wiringExprToTNFA{\wiringExpr_1 \comp \wiringExpr_2}{\wiringVarAssign} &=
        \wiringExprToTNFA{\wiringExpr_1}{\wiringVarAssign}
        \comp
        \wiringExprToTNFA{\wiringExpr_2}{\wiringVarAssign} \\
    \wiringExprToTNFA{\wiringExpr_1 \tensor \wiringExpr_2}{\wiringVarAssign} &=
        \wiringExprToTNFA{\wiringExpr_1}{\wiringVarAssign}
        \tensor
        \wiringExprToTNFA{\wiringExpr_2}{\wiringVarAssign}
\end{align*}

Finally, we can state and prove correctness: converting a wiring expression to
a PNB and \emph{then} constructing its \TNFA{} semantics is equivalent to
\emph{directly constructing} the \TNFA{} semantics of the wiring expression.
This property is known as \emph{compositionality}.

\def\amPNB{\parens{\aPNB, \aPNBMarking, \bPNBMarking}}
\def\bmPNB{\parens{\bPNB, \cPNBMarking, \dPNBMarking}}
\begin{proposition}[Compositionality]\label{prop:compsem}
    $\wiringExprToTNFA{\wiringExpr}{\wiringVarAssign}
     \NFAIso \PNBToTNFA{\wiringExprToPNB{\wiringExpr}{\wiringVarAssign}}$
\end{proposition}
\begin{proof}
    We proceed by induction on the structure of $\wiringExpr$:
    \begin{itemize}
        \item In the base case of a variable, the result is trivial: the
            \TNFA{}s on both sides are equal by definition.
        \item In the case of $\wiringExpr_1 \comp \wiringExpr_2$, we must show
            that $\wiringExprToTNFA{\wiringExpr_1 \comp
            \wiringExpr_2}{\wiringVarAssign} \NFAIso
            \PNBToTNFA{\wiringExprToPNB{\wiringExpr_1 \comp
            \wiringExpr_2}{\wiringVarAssign}}$. After expanding the definitions
            of $\wiringExprToTNFA{-}{\wiringVarAssign}$ and
            $\wiringExprToPNB{-}{\wiringVarAssign}$, we obtain:
            \begin{equation}\label{eqn:foo}
                \wiringExprToTNFA{\wiringExpr_1}{\wiringVarAssign}
                    \comp
                    \wiringExprToTNFA{\wiringExpr_2}{\wiringVarAssign}
                 \NFAIso \PNBToTNFA{\wiringExprToPNB{\wiringExpr_1}{\wiringVarAssign}
                    \comp
                    \wiringExprToPNB{\wiringExpr_2}{\wiringVarAssign}}
            \end{equation}
            To proceed, we apply the \IH{} to $\wiringExpr_1$ and $\wiringExpr_2$,
            obtaining
            \begin{equation}\label{eqn:foo1}
                \wiringExprToTNFA{\wiringExpr_1}{\wiringVarAssign}
             \NFAIso \PNBToTNFA{\wiringExprToPNB{\wiringExpr_1}{\wiringVarAssign}}
            \end{equation}
            and
            \begin{equation}\label{eqn:foo2}
            \wiringExprToTNFA{\wiringExpr_2}{\wiringVarAssign}
             \NFAIso \PNBToTNFA{\wiringExprToPNB{\wiringExpr_2}{\wiringVarAssign}}
            \end{equation}
            Substituting~\ref{eqn:foo1} and~\ref{eqn:foo2} into~\ref{eqn:foo}
            gives:
            \[
              \PNBToTNFA{\wiringExprToPNB{\wiringExpr_1}{\wiringVarAssign}}
              \comp
              \PNBToTNFA{\wiringExprToPNB{\wiringExpr_2}{\wiringVarAssign}}
              \NFAIso
              \PNBToTNFA{\wiringExprToPNB{\wiringExpr_1}{\wiringVarAssign}
                         \comp
                         \wiringExprToPNB{\wiringExpr_2}{\wiringVarAssign}}
            \]
            Now, letting, $\amPNB =
            \wiringExprToPNB{\wiringExpr_1}{\wiringVarAssign}$ and
            $\bmPNB =
            \wiringExprToPNB{\wiringExpr_2}{\wiringVarAssign}$, we obtain:
            \[
              \PNBToTNFA{\amPNB} \comp \PNBToTNFA{\bmPNB}
              \NFAIso
              \PNBToTNFA{\amPNB \comp \bmPNB}
            \]
            Now, by \propref{prop:functorComps}, the underlying
            \TLTS{}s are isomorphic, thus it only remains to show that states
            are initial/accepting in the LHS iff they are in the RHS\@.
            A state of $\PNBToTNFA{\amPNB} \comp \PNBToTNFA{\bmPNB}$, is
            $\pairof{\aNFAState}{\bNFAState} \in \powerset{\places{\aPNB}}
            \times 2^{\places{\bPNB}}$, and
            is initial iff $\aNFAState = \aPNBMarking$ and $\bNFAState =
            \cPNBMarking$; applying the state component of the \TLTS{}
            isomorphism, we obtain the state
            $\setBuilder{\inl{\aPlace}}{\aPlace \in \aNFAState} \union
            \setBuilder{\inr{\bPlace}}{\bPlace \in \bNFAState}$, i.e.
            $\aPNBMarking \disjointUnion \bPNBMarking$, which is
            initial in $\PNBToTNFA{\amPNB \comp \bmPNB}$ by the definition of
            composition on marked PNBs. In the reverse direction, a state of
            $\PNBToTNFA{\amPNB \comp \bmPNB}$, is $\cNFAState \in
            \powerset{\places{\aPNB} \disjointUnion \places{\bPNB}}$, and
            initial iff $\cNFAState = \aPNBMarking \disjointUnion
            \cPNBMarking$. The state component of the isomorphism partitions
            this state into $\pairof{\aPNBMarking}{\cPNBMarking} \in
            \powerset{\places{\aPNB}} \times 2^{\places{\bPNB}}$, which is
            initial in $\PNBToTNFA{\amPNB} \comp \PNBToTNFA{\bmPNB}$.
            The case for accepting states follows the same argument.
        \item Finally, the case for $\wiringExpr_1 \tensor \wiringExpr_2$, is
            similar, but using \propref{prop:functorTensors} in place of
            \propref{prop:functorComps}.
    \end{itemize}
\end{proof}
