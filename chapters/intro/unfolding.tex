% What is an (the point of) unfolding

The \emph{unfolding} of a Petri net is a structure representing all of its
possible behaviours, originally introduced by {Nielsen et
al.}~\cite{Nielsen1981} under the name of \emph{Occurrence Nets}. Unfoldings
were later described in more detail by Engelfriet~\cite{Engelfriet1991}, who
used the name \emph{branching processes}. Essentially, a branching process
represents some partial ``history'' of a Petri net from some initial marking,
as another Petri net, but with the condition that at each conflict of
transitions that could be fired, each alternative is represented by a copy of
the (thus-far constructed) branching process. A branching process is thus an
acyclic net without ``backward conflict'', where two transitions output to the
same place; the lack of backwards conflict implies that each set of places in
the unfolding has a unique trace of transitions that were fired to place a
token in each place. An unfolding is therefore a compact representation of all
possible computations of its underlying net.

Intuitively, an unfolding is an tree-like structure: branching points in the
net signify \emph{choices} of enabled-transition firing. The advantage of an
unfolding (in fact, itself a Petri net), over a state graph, is that
concurrency is represented explicitly, rather than implicitly. Simple
structural properties of an unfolding determine whether or not a given pair of
transitions can fire concurrently in the original net, or are causally related
(i.e. the firing of one can lead to the other becoming enabled). While these
properties can be derived from the state graph, it is certainly a more arduous
task to do so. Furthermore, the well-known ``state-explosion'' problem leads to
interleaving semantics (e.g. state graphs) of concurrent systems being large,
even for small or simple systems. By explicitly representing points of choice
and synchronisation (and not representing all possible interleavings of
concurrent transition firings), unfoldings give much smaller representations.
Indeed, unfoldings are often referred to as \emph{partial-order} semantics,
since arbitrary transitions may not be causally related (i.e. firing one
eventually enables the other), when the transitions are concurrent.

\begin{example}\label{ex:unfolding}
    An example Petri net is shown in \figref{fig:exampleNetToUnfold}, and its
    unfolding is illustrated in \figref{fig:exampleUnfolding}. Note that the
    unfolding is infinite --- we only show a small, finite prefix of it. The
    graphical notation we use is not-standard, but should be intuitive; we
    introduce this notation later in the thesis, in
    \secref{sec:PNBgraphicalNotation}. Consider the transitions $\aTrans_1$ and
    $\aTrans_2$; these transitions \emph{can} be fired concurrently, and in the
    induced partial (causal) order of the unfolding, there is no relation
    between the transitions, which witnesses this fact. Intuitively, this fact
    is observed since there is no path from the post-set of $\aTrans_1$ to the
    pre-set of $\aTrans_2$. Note however, that $\aTrans_3$ \emph{is}
    causally-related to both of these transitions.  Furthermore, since there is
    a (forward) conflict (i.e. firing choice) between $\aTrans_6$ and
    $\aTrans_7$, the unfolding contains a branching point at these transitions.
\end{example}

\begin{figure}[ht]
\centering
\begin{tikzpicture}[pnb]
    \node[pnbplace] (p0) [pnblabel={above:$\aPlace_0$}, tokens=1, rotate=-90] {};
    \node[pnbplace] (p1) [pnblabel={below:$\aPlace_1$}, right=of p0, rotate=-90, tokens=1] {};
    \node[pnbplace] (p2) [pnblabel={above:$\aPlace_2$}, below=of p0, rotate=-90] {};
    \node[pnbplace] (p3) [pnblabel={below:$\aPlace_3$}, below=of p1, rotate=-90] {};
    \node[pnbplace] (p4) [pnblabel={above:$\aPlace_4$}, rotate=-90] at ([yshift=-2cm]$(p2) !.5! (p3)$) {};

    \node[pnbplace] (p5) [node distance=4cm, pnblabel={above:$\aPlace_5$}, below=of p2, rotate=180] {};
    \node[pnbplace] (p6) [node distance=4cm, pnblabel={below:$\aPlace_6$}, below=of p3] {};

    \coordinate (cjoin) at ([yshift=1cm]p4);

    \labelledpnbarr{p0.out}{p2.in}{right:$\aTrans_1$}{o90i270}{}
    \labelledpnbarr{p1.out}{p3.in}{left:$\aTrans_2$}{o90i270}{}

    \labelledpnbarr{p2.out}{cjoin}{}{o-90i90}{draw=none}
    \labelledpnbarr{p3.out}{cjoin}{}{o-90i90}{draw=none}
    \labelledpnbarr{cjoin}{p4.in}{right:$\aTrans_3$}{pos=0,o90i270}{pos=0}

    \labelledpnbarr{p4.out}{p5.in}{left:$\aTrans_4$}{o-120i20}{}
    \labelledpnbarr{p4.out}{p6.in}{right:$\aTrans_5$}{o-60i160}{}

    \coordinate (ljoin) at ([yshift=0.75cm]$(p0.in) !.2! (p1.in)$);
    \coordinate (rjoin) at ([yshift=0.75cm]$(p0.in) !.8! (p1.in)$);

    \begin{pgfinterruptboundingbox}
        \labelledpnbarr{p5.out}{ljoin}{above right:$\aTrans_6$}{o150i110}{pos=1}
    \end{pgfinterruptboundingbox}

    \labelledpnbarr{ljoin}{p0.in}{}{o-110i90}{draw=none}
    \labelledpnbarr{ljoin}{p1.in}{}{o-45i135}{draw=none}

    \begin{pgfinterruptboundingbox}
        \labelledpnbarr{p6.out}{rjoin}{above left:$\aTrans_7$}{o30i70}{pos=1}
    \end{pgfinterruptboundingbox}

    \labelledpnbarr{rjoin}{p0.in}{}{o-135i45}{draw=none}
    \labelledpnbarr{rjoin}{p1.in}{}{o-60i90}{draw=none}

    \drawBoundaries[1.5][1.5]{0}{0}

\end{tikzpicture}
\caption{An example Petri net}
\label{fig:exampleNetToUnfold}
\end{figure}

\begin{figure}[ht]
\newcommand{\drawOneLevel}[2]{
    \node (start#2) [draw=none,#1] {};
    \node[pnbplace] (p0#2) [pnblabel={above:$\aPlace_0$}, below left=of start#2, rotate=-90] {};
    \node[pnbplace] (p1#2) [pnblabel={below:$\aPlace_1$}, below right=of start#2, rotate=-90] {};
    \node[pnbplace] (p2#2) [pnblabel={above:$\aPlace_2$}, below=of p0#2, rotate=-90] {};
    \node[pnbplace] (p3#2) [pnblabel={below:$\aPlace_3$}, below=of p1#2, rotate=-90] {};
    \node[pnbplace] (p4#2) [pnblabel={above:$\aPlace_4$}, rotate=-90] at ([yshift=-2cm]$(p2#2) !.5! (p3#2)$) {};

    \node[pnbplace] (p5#2) [node distance=4cm, pnblabel={above:$\aPlace_5$}, below=of p2#2, rotate=-90] {};
    \node[pnbplace] (p6#2) [node distance=4cm, pnblabel={below:$\aPlace_6$}, below=of p3#2, rotate=-90] {};

    \coordinate (cjoin#2) at ([yshift=1cm]p4#2);

    \labelledpnbarr{p0#2.out}{p2#2.in}{right:$\aTrans_1$}{o90i270}{}
    \labelledpnbarr{p1#2.out}{p3#2.in}{left:$\aTrans_2$}{o90i270}{}

    \labelledpnbarr{p2#2.out}{cjoin#2}{}{o-90i90}{draw=none}
    \labelledpnbarr{p3#2.out}{cjoin#2}{}{o-90i90}{draw=none}
    \labelledpnbarr{cjoin#2}{p4#2.in}{right:$\aTrans_3$}{pos=0,o90i270}{pos=0}

    \labelledpnbarr{p4#2.out}{p5#2.in}{above left:$\aTrans_4$}{o-120i90}{}
    \labelledpnbarr{p4#2.out}{p6#2.in}{above right:$\aTrans_5$}{o-60i90}{}
}

\newcommand{\connect}[5]{
    \coordinate (foo#1#2#3) at ([yshift=1cm]$ (#2) !.5! (#3) $);

    \labelledpnbarr{#1.out}{foo#1#2#3}{#4:#5}{o-90i90}{}
    \labelledpnbarr{foo#1#2#3}{#2.in}{}{o-90i90}{}
    \labelledpnbarr{foo#1#2#3}{#3.in}{}{o-90i90}{}
}

\newcommand{\dotnode}[1]{
    \node (dots#1) [node distance=1cm, below=of #1, rotate=90] {$\bm\cdots$};
}

\centering
\begin{tikzpicture}[node distance=1.75cm]
    \coordinate (foo);
    \drawOneLevel{below=of foo}{1};
    \drawOneLevel{below left of=p51}{2};
    \drawOneLevel{below right of= p61}{3};

    \connect{p51}{p02}{p12}{above left}{$\aTrans_6$};
    \connect{p61}{p03}{p13}{above right}{$\aTrans_7$};

    \foreach \d in {52,62,53,63}{
        \dotnode{p\d};
    }

    \drawBoundaries{0}{0}
\end{tikzpicture}
\caption{Unfolding of the Petri net in \figref{fig:exampleNetToUnfold}}
\label{fig:exampleUnfolding}
\end{figure}

Using unfoldings as a method of checking reachability in Petri nets was
pioneered by McMillan~\cite{McMillan1992}, using the semantics proposed by
{Nielsen et al.}~\cite{Nielsen1981} and Engelfriet~\cite{Engelfriet1991}.

McMillan observed that by considering a suitable \emph{prefix} of the
unfolding, which represents all reachable markings of the original net, the
full (often infinite) unfolding need not be constructed. Such prefixes are
known as \emph{finite (marking-) complete prefixes}. McMillan's algorithm used
such prefixes to check reachability (precisely, coverability) of a set of
places, $\aPNBTargetSet$, by supplementing the original net with a new
transition, $\aTrans$, whose pre-set was $\aPNBTargetSet$. If, during the
execution of the algorithm (that is, while the constructed unfolding was not a
\emph{complete} prefix) $\aTrans$ was found to be enabled, then
$\aPNBTargetSet$ \emph{was} indeed coverable. If the complete prefix was
constructed before $\aTrans$ was encountered, the marking was not coverable.

McMillan's algorithm, while correct and simple, is inefficient in terms of the
size of the complete prefix generated, and has been subsequently improved by
several authors. Esparza et al.~\cite{Esparza1996} improved the algorithm to
construct smaller (complete) prefixes, by showing that more transitions in the
unfolding can correctly be considered as ``cutoff'' points of the unfolding,
which are not unfolded further. Heljanko~\cite{Heljanko1999} further refined
the work of {Esparza et al.} to generate even smaller prefixes in some examples
(though the author questioned the benefits of the size reduction vs. the
creation effort involved). Khomenko and Koutny~\cite{Khomenko2001} suggested an
alternative to the algorithm of {Esparza et al.}, using a different method of
computing possible extensions of a (incomplete) unfolding, that is more
efficient in the examples they present. Heljanko et al.~\cite{Heljanko2002}
showed how the approach of {Esparza et al.} may be parallelised for further
increases in efficiency. More recent work by {Bonet et al.}~\cite{Bonet2008}
employed heuristic search algorithms in an attempt to ``direct'' the unfolding
procedure towards the target transition, rather than existing approaches using
simple search strategies. Khomenko et al.~\cite{Khomenko2003a}
generalised the definition of complete prefix of an unfolding, that is
applicable to nets with e.g. equivalent markings, or extra data that is to be
considered along with markings. Finally, {Neumair et al.}~\cite{Neumair2004}
show that even unbounded Petri nets (those without finitely many reachable
states) may still have their statespaces by suitable extensions of unfoldings.

In addition to McMillan's approach (and its improvements), several methods
exist for checking the reachability problem in nets that use (complete prefixes
of) unfoldings. Esparza and Schr\"{o}ter~\cite{Esparza2001} surveyed 4 such
algorithms. They recommend an algorithm to use based on whether the
marking is expected to be reachable or not. If reachability is not expected,
they suggest the algorithm of Heljanko~\cite{Heljanko1999a}, which translates
the reachability problem into a logic program and checks for the existence of a
stable model of the program. If the marking is expected to be reachable, they
suggest using (an improvement of) McMillan's algorithm.

An important point to note regarding unfoldings, and indeed, finite complete
prefixes, is that they carry more information about the computations of nets,
than merely marking reachability. For instance, it is possible to check for
deadlock, as shown in McMillan's seminal paper~\cite{McMillan1992}.
Furthermore, automata-theoretic approaches to LTL model checking can be
adapted~\cite{Heljanko2000, Wallner1998} to unfoldings.

For an overview of the extensive field surrounding unfoldings and the
corresponding model checking approaches, see~\cite{Esparza2008}
and~\cite{Esparza2010}.
