% What is compositional model checking?
Compositional approaches to model checking decompose monolithic systems into
smaller components, which are checked, giving \emph{local} results. These local
results are combined to form a global result for the original monolithic
system. Such approaches are known as \emph{divide and conquer} --- the idea
being that the smaller components are easier to check, and local results carry
less information than the local component. Thus, by decomposing (checking, and
composing local results), the amount of information that must be processed is
(vastly) reduced.

% Examples in the literature
Clarke et al.~\cite{Clarke1989} describe a compositional approach to model
checking systems with temporal logic specifications. They observe that the main
difficulty is the preservation of local properties at a global level, and
introduce a framework using \emph{interface processes} to model the
environment, when checking individual components, to ensure that components
only exhibit globally-valid behaviours.

Compositional approaches using process algebra techniques date back at least to
Milner's seminal CCS~\cite{Milner1980}. Yeh and Young~\cite{Yeh1991} outline a
reachability analysis that allows the simplification of ``intermediate''
components, that helps alleviate the state explosion problem, by replacing
components with semantically-equivalent, but smaller components. Similarly,
Valmari~\cite{Valmari1993,Valmari1996} describes compositional state space
generation, a divide and conquer approach to global state space generation that
employs minimisation of local state spaces to avoid state explosion. A CSP
model is used, modelling synchronous communication of labelled Petri nets;
indeed, this method of decomposition restricts the possible minimisations,
since the global synchronisations must be preserved.  Furthermore, Valmari does
not use any form of memoisation to prevent repeated work.  Later,
Valmari~\cite{Valmari1994} presented a similar decomposition technique, using
shared ``boundary'' places, instead of shared transition labels. This allowed
for more natural LTSs: transitions were labelled with the effect on boundary
places rather than net transition names. A similar approach was taken by
Kindler~\cite{Kindler1997}, who introduced Petri net components, with
distinguished input/output places that are fused to form larger components.
Kindler gave a fully-abstract semantics, whereby the semantics of
components under composition agrees with that of closed components.  Very
recent work by {Baldan et al.}~\cite{Baldan2015} has elucidated compositional
encodings between (a)synchronous process-calculi and a variant of Petri nets,
with distinguished places/transitions used for synchronisation.  Indeed, such
an encoding gives ``technology transfer'': techniques for model
checking properties in process-calculi can inform those in Nets and vice-versa.

An important consideration for compositional model checking approaches was
noted by {Graf and Steffan}~\cite{Graf1990} and {Cheung and
Kramer}~\cite{Cheung1993}: local minimisation can be inefficient if it does not
explicitly exclude behaviours that will never be performed by the context
of a component. Both use interface constraints similar to the interface
processes of {Clarke et al.}~\cite{Clarke1989} to prevent spurious behaviours
of components, with {Cheung and Kramer} giving an algorithm for interface
generation without requiring an LTS for the context.  {Krimm and
Mounier}~\cite{Krimm1997} employed these ideas for compositional statespace
generation for LOTOS programs, including large, complex examples.

In the area of Petri nets, {Lee et al.}~\cite{Lee2000}, and
Rakow~\cite{Rakow2008} employ \emph{slicing} to decompose a Petri net into
concurrent units, in order to check reachability on a per-slice basis, thus
avoiding some of the inherent state explosion due to interleavings. Whereas
{Lee et al.} and Rakow use a \emph{static} slicing (i.e. without considering
the markings of a net, just its structure), {Llorens et al.}~\cite{Llorens2008}
propose a \emph{dynamic} approach, which takes account of an initial marking of
the net. However, slices are not \emph{true decompositions} (i.e. disjoint:
transitions (and places) of the original net should appear in only one slice),
since they necessarily overlap, with transitions being shared among (many)
slices. Furthermore, if the marking to be checked involves a large proportion
of places in the net, little benefit will be gained by using slicing.


Christensen~\cite{Christensen2000} uses a compositional approach to generate
the statespace of Petri net components that synchronise via shared places or
transitions. By generating a \emph{synchronisation graph} capturing component
communications, in addition to \emph{local} reachability graphs of components,
the full statespace can be implicitly represented, without generating all
concurrent transition interleavings. A similar approach was described by
{Notomi and Murata}~\cite{Notomi1994}, however, whereas at each step of
Christensen's algorithm the local reachability graphs are only unfolded as far
as the next \emph{global} synchronisation, {Notomi and Murata}'s approach
generates full local reachability graphs, before reducing w.r.t.
synchronisations and combining them. Thus {Notomi and Murata}'s approach may
explore local reachability graph states that are never explored in the global
statespace, due to non-synchronisation.  Furthermore, while Notomi and Murata
describe reduction by merging states that are equivalent w.r.t.
synchronisations, they do not give an algorithm for determining the set of such
states. A similar reduction strategy is employed by {Bucholz and
Kemper}~\cite{Buchholz2002}, who abstract over \emph{autonomous regions} (subnets
delimited by synchronisations), simplifying the representation of
local reachability graphs.
