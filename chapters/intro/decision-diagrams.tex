\label{sec:decision-diagrams}

In this section we move on to lower-level optimisation techniques,
concentrating on efficient symbolic representation of nets and their semantics.

% BDDs for model checking
Symbolic state space representations aim to be more efficient (in terms of
memory-requirements) than explicit representations, often leading to vast
improvements in the performance of corresponding model checkers. Rather than
storing and exploring states one-by-one, symbolic model checkers store and
explore multiple states at once. The most prominent initial work on symbolic
model checking is due to McMillan, in his PhD thesis, and later published
together with {Burch et al.}~\cite{Burch1992}. {Burch et al.} gave a symbolic
method of model checking $\mu$-calculus formulae, using (Ordered) Binary
Decision Diagrams (OBDDs); by checking $\mu$-calculus formulae, they were able
to generalise previous BDD-based model checking techniques targetting concrete
instances: {Coudert et al.}~\cite{Coudert1990} model checked Mealy
machines~\cite{Mealy1955} (automata with state-dependent output), while {Browne
et al.}~\cite{Dill1986} described a technique for model checking circuits
against CTL specifications.

% What are BDDs?
Ordered Binary Decision Diagrams (OBDDs, more commonly just BDDs) were
popularised by Bryant~\cite{Bryant1986}, who showed that BDDs canonically
represented functions over the Booleans, whilst allowing efficient
manipulation, e.g. performing conjunction or checking for tautology. A simple
example is illustrated in \figref{fig:simpleBDD}, showing a
representation\footnote{A dashed line represents that the source variable was
false, a solid line, true.} of the formula $(a \conjunction b) \disjunction (a
\conjunction c)$, which could represent the collection sets of states
$\setof{\setof{a,b}, \setof{a,c}}$ where a particular property holds.

\begin{figure}[ht]
    \centering
    \begin{tikzpicture}[bdd, node distance=1.5cm]
        \node (a) [var]                    {$a$};
        \node (b) [var, below=of a]        {$b$};
        \node (c) [var, below=of b]        {$c$};
        \node (f) [leaf, below left=of c]  {$\bot$};
        \node (t) [leaf, below right=of c] {$\top$};

        \falseEdge{a}{f}
        \trueEdge{a}{b}
        \trueEdge{b}{t}
        \falseEdge{b}{c}
        \trueEdge{c}{t}
        \falseEdge{c}{f}
    \end{tikzpicture}
    \newcommand{\capt}{BDD representing $(a \conjunction b) \disjunction (a
    \conjunction c)$}
    \caption[\capt]{\capt: dashed edges represent false, solid represent true.}
    \label{fig:simpleBDD}
\end{figure}

Using suitable encodings of states and transitions, entire state graphs can be
represented, and properties (for example, reachability) can be checked using a
fixed-point operation on the underlying BDDs. Detailed introductions to BDDs,
their use and algorithms can be found by Andersen~\cite{Andersen1997} and
{Drechsler and Sieling}~\cite{Drechsler2001}.

% BDDs for model checking Petri nets
Symbolic BDD representations can be applied to Petri nets.  {Pastor et
al.}~\cite{Pastor1994} gave encodings of the markings, transitions and firing
rules of 1-safe Petri nets as BDDs. These encodings were used to give the first
symbolic method of calculating all reachable states of the net, and
furthermore, verification of liveness and safeness properties.  Pastor and
other collaborators later improved the algorithm~\cite{Pastor1998, Pastor1999},
using more efficient representations of Petri net markings (using a suitable
encoding method, structurally-related places could be represented using fewer
variables).  {Semenov and Yakelov}~\cite{Semenov1995} took a different
optimisation approach, combining the symbolic exploration method with the
unfolding technique to discover ``clusters'' of places that are never
simultaneously marked; such clusters have minimal BDD size, regardless of
variable ordering and thus assigning ``close'' variables to elements of
clusters helps minimise the size of the resulting BDD.

{Miner and Ciado}~\cite{Miner1999} improved the efficiency of the statespace
generation technique of {Pastor et al.}. Using Multi Decision Diagrams
(MDDs)---a generalisation of BDDs introduced by {Srinivasan et
al.}~\cite{Srinivasan1990}, which encode functions of the form $\N^n \to \N$
rather than $\B^n \to \B$ as in BDDs---and suitably partitioning the input
net's places into \emph{localities} (similar to the clusters of {Semenov and
Yakelov}), Miner and Ciardo were able to generate the statespace with fewer
iterations and a corresponding reduction in time. In later work, {Ciardo et
al.}~\cite{Ciardo1999} further refined the use of event localities, to further
improve the algorithm's performance. The benefit of using higher-level decision
diagrams has been exhibited, for example by {Strehl and
Thiele}~\cite{Strehl1999} and Tovchigrechko~\cite{Tovchigrechko2008}.

% Generalise BDDs to MTBDDs
While BDDs \emph{can} encode functions with (co-)domains other than the
Booleans, using a direct encoding can be both more intuitive and storage-efficient.
While MDDs fully generalise BDDs to functions on the naturals, a finer
generalisation is to encode functions with co-domains other than the Booleans.
This natural generalisation of BDDs was originally observed by Clarke et
al.~\cite{Clarke1993} and {Bahar et al.}~\cite{Bahar1993}. Such generalised
BDDs, known as Multi Terminal BDDs (MTBDDs) have been used for a variety of
purposes, including: efficiently representing matrices~\cite{Fujita1997}, model
checking of probabilistic timed-automata~\cite{Wang2005} and for representing
specifications of parallel programs~\cite{Ciesinski2008}.

% Problems of BDD

A key problem related to the use of BDDs is that a particular ordering for the
BDD's variables must be chosen a priori. Unfortunately, choosing an optimal
variable ordering for a BDD is an NP-complete problem~\cite{Bollig1996}.
However, heuristics have been developed, which attempt to minimise the size of
a particular BDD. Static approaches attempt to determine a fixed variable
order, during construction of the BDD; for example, the variable-interleaving
technique of Fujii et al.~\cite{Fujii1993}, which aims to place
semantically-related (Fujii et al. consider the topology of the logic circuit)
variables near each other in the BDD. On the other hand, the dynamic approach
of Felt et al.~\cite{Felt1993} recognises that statically-determined variable
orders may not be optimal after operations have been applied to the BDD.
Instead, they monitor the size of intermediate BDDs and rearrange small
``windows'' of consecutive variables to reduce the resulting BDD's size.

More recently, so-called \emph{exact} algorithms have been further
investigated, to improve the non-optimal results of heuristic algorithms.
Indeed, {Ebendt et al.}~\cite{Ebendt2005} improve on the original algorithm of
{Friedman and Supowit}~\cite{Friedman1987} and improved algorithm of {Drechsler
et al.}~\cite{Drechsler1998}, by using the A$^\ast$ search algorithm to explore
(subsets of) the set of all variable orderings, and using local heuristics to
choose the best next variable at each step.

A recent, detailed overview of BDD optimisation techniques can be found in the
book by {Ebendt et al.}~\cite{Ebendt2005a}. Finally, while it is common that
(variants of) BDDs are used symbolic model checking, as pointed out by {Biere
et al.}~\cite{Biere1999}, it is not the case that symbolic methods imply the use
of BDDs, viable alternatives certainly do exist.
