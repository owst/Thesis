Partial-order reduction methods aim to prevent the statespace explosion caused
by representing all possible interleavings of truly-concurrent events. The
method relies on determining those transitions that have low mutual
interference; two transitions interfere if firing one enables or disables the
other, or if the resulting marking changes when the order of firing of the two
transitions is swapped. A reduced state space can be built and explored if sets
of transitions with low mutual interference are only fired in only one of the
possible orderings. Since the marking reached after firing the set of
transitions is the same, regardless of the chosen order (if it were not, there
would be a pair of transitions that interfere), the particular chosen order
does not matter. The name \emph{partial-order} methods is used since those
transitions that do not interfere can be considered as unrelated by some
\emph{dependency} ordering.  As an example, consider the example net shown
in~\figref{fig:lowInterferenceNet}, originally due to Wolf~\cite{Wolf2014}. The
transition $\aTrans_4$ does not interfere with $\aTrans_2$ or $\aTrans_3$ and
therefore we can only consider one of the three possible interleavings. The
corresponding statespace, illustrated in~\figref{fig:lowInterferenceNetLTS},
contains 6 redundant transitions and 4 redundant states, highlighted with
dashed lines.

\begin{figure}[ht]
\centering
\begin{tikzpicture}[node distance=2cm]
    \node[pnbplace] (p0) [tokens=1] {};
    \node[pnbplace] (p1) [above right=of p0] {};
    \node[pnbplace] (p2) [below right=of p0] {};
    \node[pnbplace] (p3) [right=of p1] {};
    \node[pnbplace] (p4) [right=of p3] {};
    \node[pnbplace] (p5) [below right=of p4] {};
    \node[pnbplace] (p6) [below left=of p5] {};

    \xHalfwayYFirst{p0}{p1}{join1}
    \labelledpnbarr{p0.out}{join1}{$\aTrans_1$}{}{pos=1}
    \labelledpnbarr{join1}{p1.in}{}{o0i180}{draw=none}
    \labelledpnbarr{join1}{p2.in}{}{o0i180}{draw=none}

    \xHalfwayYFirst{p5}{p4}{join2}
    \labelledpnbarr{join2}{p5.in}{$\aTrans_5$}{}{pos=0}
    \labelledpnbarr{p4.out}{join2}{}{o0i180}{draw=none}
    \labelledpnbarr{p6.out}{join2}{}{o0i180}{draw=none}

    \labelledpnbarr{p1.out}{p3.in}{$\aTrans_2$}{}{}
    \labelledpnbarr{p3.out}{p4.in}{$\aTrans_3$}{}{}
    \labelledpnbarr{p2.out}{p6.in}{$\aTrans_4$}{}{}

    \drawBoundaries{0}{0}
\end{tikzpicture}
\caption{Example Petri net}
\label{fig:lowInterferenceNet}
\end{figure}

\begin{figure}[ht]
\centering
\makebox[\textwidth][c]{
\scalebox{0.9}{
\begin{tikzpicture}[nfa]
    \node (0) [state]             {};
    \node (1) [state, right=of 0] {};

    \node (4) [state, dashed, right=of 1] {};
    \node (5) [state, dashed, right=of 4] {};

    \node (2) [state, above=of 4] {};
    \node (3) [state, right=of 2] {};

    \node (6) [state, dashed, below=of 4] {};
    \node (7) [state, dashed, right=of 6] {};

    \node (8) [state, right=of 5] {};
    \node (9) [state, right=of 8] {};

    \foreach \x/\y/\l/\style in {%
0/1/$\aTrans_1$/solid,%
1/2/$\aTrans_2$/solid,%
1/4/$\aTrans_2$/dashed,%
1/6/$\aTrans_4$/dashed,%
2/3/$\aTrans_3$/solid,%
4/5/$\aTrans_4$/dashed,%
6/7/$\aTrans_2$/dashed,%
3/8/$\aTrans_4$/solid,%
5/8/$\aTrans_3$/dashed,%
7/8/$\aTrans_3$/dashed,%
8/9/$\aTrans_5$/solid%
}{\path [\style] (\x) edge node [swap] {\l} (\y);}

\end{tikzpicture}
}
}
\caption{Statespace of the net in \figref{fig:lowInterferenceNet}}
\label{fig:lowInterferenceNetLTS}
\end{figure}

To explore a reduced statespace using the partial order method, the essential
idea is to fire only a subset of the enabled transitions at any particular
marking. Indeed, the trivial case would be to take the set of all
enabled transitions, but firing such a set offers no possibility of exploring a
reduced statespace. However, it is not necessarily the case that firing smaller
sets of transitions leads to a smaller explored statespace~\cite{Valmari1998}.
For a trivial example, consider the simple net
in~\figref{fig:concurrentEnabledNet}, both $\aTrans_1$ and $\aTrans_2$ are
enabled (and do not interfere). If we fire both simultaneously, the explored
statespace will have two states with one connecting transition; however, if we
pick a smaller set of transitions (i.e.  firing $\aTrans_1$ or $\aTrans_2$
alone), the explored statespace will have 3 states and 2 transitions since the
transitions will have been interleaved.

\begin{figure}[ht]
\centering
\begin{tikzpicture}[pnb]
    \node[pnbplace] (p0) [tokens=1] {};
    \node[pnbplace] (p1) [right=of p0] {};
    \node[pnbplace] (p2) [tokens=1, below=of p0] {};
    \node[pnbplace] (p3) [below=of p1] {};

    \labelledpnbarr{p0.out}{p1.in}{$\aTrans_1$}{}{}
    \labelledpnbarr{p2.out}{p3.in}{$\aTrans_2$}{}{}

    \drawBoundaries{0}{0}
\end{tikzpicture}
\caption{Example Petri net}
\label{fig:concurrentEnabledNet}
\end{figure}

Several related partial order reduction approaches exist in the literature, with
slight variations on the set of transitions considered in each case. So called
\emph{stubborn}~\cite{Valmari1991}, \emph{persistent}~\cite{Godefroid1990} and
\emph{ample}~\cite{Peled1993} set methods each specify the (sub)set of the
enabled transitions for a state that should be explored. Here, we only present
an intuitive overview of the stubborn set method; details of the precise
differences between the approaches can be found in surveys by
Valmari~\cite{Valmari1998}, Godefroid~\cite{Godefroid1996a} and {Clarke et
al.}~\cite{Clarke1999}. Additionally, in recent research, {Valmari and
Hansen}~\cite{Valmari2011} investigated optimality of stubborn sets and the
effect of particular choices of dependency relations on transitions.

As a general definition (c.f. \cite{Valmari1998}), a set of transitions
$\aPnTransSet$ is \emph{stubborn} iff:
\begin{enumerate}
    \item If firing some sequence of transitions \emph{not} in $\aPnTransSet$
        allows some $\aTrans \in \aPnTransSet$ to be fired, reaching a marking
        $\aPnMarking$, then we can also fire that same sequence \emph{after}
        firing $\aTrans$, to also reach $\aPnMarking$,
    \item There is at least one $\aTrans \in \aPnTransSet$ that is enabled
        after firing any length sequence of transitions \emph{not} in
        $\aPnTransSet$.
\end{enumerate}

In other words, the firing of some transition in $\aPnTransSet$ commutes with
the firing of transitions not in $\aPnTransSet$. Thus, the search
algorithm is justified in choosing the ordering where (the enabled transitions
of) $\aPnTransSet$ are fired first: the marking reached in either order is the
same.

However, to ensure that the possible statespace reduction enabled by the
stubborn set method is sound w.r.t. a particular net property, the definition
of stubborn sets must take into account the particular property being checked.
For example, considering the trivial example net shown
in~\figref{fig:concurrentEnabledNet}, if the property to check is reachability
of the marking with a token in top-left and bottom-right places only, then the
set of stubborn sets cannot include $\setof{\aTrans_1, \aTrans_2}$ since firing
this set does not preserve reachability of the target marking.  Whereas early
papers related to stubborn sets focussed on preserving deadlock,
Schmidt~\cite{Schmidt1999} detailed an encoding to efficiently check other
Petri net properties, including reachability, using the stubborn set method.

The stubborn set method was later improved by Godefroid and
co-authors~\cite{Godefroid1993,Wolper1993}, particularly by combining stubborn
sets and \emph{sleep}~\cite{Godefroid1990} sets for additional improvements.
The sleep set method attempts to avoid visiting states that have already been
explored by an alternative interleaving of transitions, by tracking the set of
transitions fired to reach a particular state. In later work, {Alur et
al.}~\cite{Alur1997} demonstrated efficiency improvements when combining
efficient representation via decision diagrams (see
\secref{sec:decision-diagrams}) and partial order reductions using ample sets.
Recent work by {Hansen and Wang}~\cite{Hansen2011} has shown that combining
stubborn sets with compositional approaches (see
\secref{sec:compositionalApproaches}) to determine transition dependencies can
improve the reduction effect.

An alternative to standard reachability graph representations, due to {Vernadat
et al.}~\cite{Vernadat1996} is the notion of a \emph{covering step graph}
(CSG). A CSG groups independent events into single transition step, which is
fired atomically, and thus implicitly avoids the state explosion due to
interleaving. CSGs can be combined with the persistent set method, as shown by
{Ribet et al.}~\cite{Ribet2002}.

In-depth surveys of stubborn set-like methods were presented by both
Valmari~\cite{Valmari1998} and Godefroid~\cite{Godefroid1996, Godefroid1996a},
examining the practical applicability and effectiveness of stubborn set methods
for various Petri net model checking problems.
