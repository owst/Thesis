\penrose's query ports are the same as read arcs in Petri nets, originally
introduced by Christensen~\cite{Christensen1993}. Nets with read arcs (also
known as contextual nets) allow the modelling of concurrent checking (but not
consuming) of a token in a particular place, a natural operation in many
models. While a read-write loop on a place is similar to a read arc (and gives
the same reachable markings~\cite{Baldan2008}), several such loops cannot be
fired concurrently, since they all consume a token from the same place.

{Vogler et al.}~\cite{Vogler1998} give an algorithm for generating the unfolding
of(certain restricted classes of) contextual nets, which was later generalised
by {Baldan et al.}~\cite{Baldan2008}. A practical reachability and deadlock
checking algorithm for contextual nets was introduced by {Rodriguez and
Schwoon}~\cite{Rodriguez2012} that leverages the high performance of SAT
solvers for efficiency.

Very recent work by {Rodriguez et al.}~\cite{Rodriguez2013} introduced
Contextual Merged Processes, which are merged processes applied to contextual
nets, giving more compact representation compared to merged processes or
contextual nets alone.
