We briefly consider two related approaches, which solve Petri net model
checking problems, by faithful translation into an alternate domain that
possesses efficient algorithms. By translating to, and using efficient solving
in the target domain, the statespace explosion can be avoided to a certain
extent. The two approaches we cover are SAT and Linear/Integer Programming.

% What is SAT

The Boolean Satisfiability (SAT) problem is concerned with determining the
existence of a satisfying variable assignment $V$ of a Boolean function, $F$.
For example, the formula $F = x \disjunction (y \conjunction z)$ is satisfied
by $V = \setof{x \mapsto 1, y \mapsto 1, z \mapsto 0}$, whereas the formula $(x
\conjunction y) \conjunction (\neg x \conjunction z)$ is \emph{not}
satisfiable. Using suitable encodings of the markings and transition relation
of a net, it is possible to encode reachability problems as SAT problems. To
restrict the size of input formulae, SAT solving is often used in a
\emph{bounded} context, where counter-examples are only considered up to some
maximal size. Indeed, by \emph{unrolling} the transition relation of a net
$\aN$ times, Petri net model checking can be bounded by searching for occurrence
sequences reaching the target marking with length $\leq \aN$. For example, in
the net of \figref{fig:net}, the marking $\setof{\aPlace_2, \aPlace_3}$ can be
reached in bound $3$.

% How is it used to check Petri nets
A SAT-based approach for bounded checking of 1-safe Petri nets was introduced
by {Ogata et al.}~\cite{Ogata2004} that included checking of reachability and
liveness properties, and illustrated the importance of compact encodings for
efficient SAT solving. A related approach was used by
Heljanko~\cite{Heljanko2001}, who gave an encoding into \emph{boolean
circuits}, which generalise boolean formulae, highlighting the efficiency
implications of using process~\cite{Devill1987}, step or interleaving
semantics.

% What is linear programming?
Linear programming is the minimising of a linear equation, subject to linear
constraints. Indeed, with suitable encodings of the markings and flow relation
of an acyclic net, reachability problems can be approximated by determining the
existence of a solution to a system of linear equations. For example, consider
the net, $\aPNB$, shown in \figref{fig:net}.

\begin{figure}[ht]
    \centering
    \begin{tikzpicture}[pnb]
    \node[pnbplace] (p0) [pnblabel=$\aPlace_0$, tokens=1] {};
    \node[pnbplace] (p2) [pnblabel=$\aPlace_2$, below right=of p0] {};
    \node[pnbplace] (p1) [pnblabel=below:$\aPlace_1$, tokens=1, below left=of p2] {};

    \node[pnbplace] (p3) [pnblabel=$\aPlace_3$, right=of p2] {};
    \node[pnbplace] (p4) [pnblabel=$\aPlace_4$, above right=of p3, xshift=0.5cm] {};
    \node[pnbplace] (p5) [pnblabel=below:$\aPlace_5$, below right=of p3, xshift=0.5cm] {};

    \labelledpnbarr{p0.out}{p2.in}{left:$\aTrans_0$}{}{}
    \labelledpnbarr{p1.out}{p2.in}{left:$\aTrans_1$}{}{}
    \labelledpnbarr{p2.out}{p3.in}{above:$\aTrans_2$}{}{}

    \path let \p1 = ($(p3.out)!.6!(p4.in)$), \p2 = ($(p4.in)!.5!(p5.in)$) in coordinate (mid) at  (\x1, \y2);

    \labelledpnbarr{p3.out}{mid}{{[pos=0.9]$\aTrans_3$}}{}{pos=1}

    \draw (mid) edge[o0i180] (p4.in);
    \draw (mid) edge[o0i180] (p5.in);

    \drawBoundaries{0}{0}
    \end{tikzpicture}
    \caption{Example net, $\aPNB$}
    \label{fig:net}
\end{figure}

The \emph{incidence matrix} of $\aPNB$ is a $\cardinalityof{\places{\aPNB}}
\times \cardinalityof{\trans{\aPNB}}$ matrix, where $I_{i, j}$ denotes the
effect of firing transition $\aTrans_j$ on the token count\footnote{1 signifies
that a token is added, -1 that it is taken away and 0 denotes no effect.} of
place $\aPlace_i$:
\[
    IM =
\begin{bmatrix}
    -1 &  0 &  0 &  0\\
     0 & -1 &  0 &  0\\
     1 &  1 & -1 &  0\\
     0 &  0 &  1 & -1\\
     0 &  0 &  0 &  1\\
     0 &  0 &  0 &  1\\
\end{bmatrix}
\]
The initial/target markings are column vectors with
$\cardinalityof{\places{\aPNB}}$ rows:
\[
    M_I =
\begin{bmatrix}
    1\\
    1\\
    0\\
    0\\
    0\\
    0\\
\end{bmatrix}
\qquad
    M_T =
\begin{bmatrix}
    0\\
    0\\
    0\\
    1\\
    1\\
    1\\
\end{bmatrix}
\]
And finally, the \emph{marking}, or \emph{state equation}~\cite{Murata1989} is:
\begin{equation}\label{eqn:marking}
    M_T = M_I + IM \cdot
\begin{bmatrix}
    x_0\\
    x_1\\
    x_2\\
    x_3\\
\end{bmatrix}
\end{equation}
where the final column vector is referred to as the \emph{Parikh vector}, with
its $x_i$ representing the number of occurrences of transition $i$ in a firing
sequence starting in $M_I$ and reaches $M_T$. Marking equations are thus
algebraic representations of the reachable markings of a net.

For the example net of \figref{fig:net}, equation~\ref{eqn:marking} has a
solution of $x_0\mapsto 1, x_1\mapsto1, x_2\mapsto2, x_3\mapsto1$,
corresponding to the firing sequence:
$\sequenceof{\aTrans_0,\aTrans_2,\aTrans_3,\aTrans_1,\aTrans_2}$.

% How can it be used to check Petri nets?

\newcommand{\mandr}{{Melzer and R\"{o}mer}}

\mandr~\cite{Melzer1997} used linear programming to optimise the search
strategy for certain instances of a deadlock-checking unfolding algorithm due
to McMillan~\cite{McMillan1992}; indeed, since unfoldings are acyclic, the
linear programming approach is sound. In later work, {Khomenko and
Koutny}~\cite{Khomenko2000, Khomenko2007} improved \mandr's approach and
extended it to check other properties such as reachability; furthermore, by
encoding causality/conflict properties derived from the unfolding in the system
of linear constraints, the solver's search space was reduced, exhibiting
an impressive speed-up.

However, as Esparza~\cite{Esparza2000} notes, since the linear programming
solutions (over-)approximate reachable markings, the target marking may not in
fact be reachable. However, the inverse implication does hold: if there are no
solutions, the marking is certainly not reachable. Detailed overviews of the
applications of linear programming techniques for analysing (P/T) nets can be
found by {Desel}~\cite{Desel1998b} and {Silva et al.}~\cite{Silva1998}.
