\emph{Model checking} is the systematic process of determining whether a
\emph{model} of some system conforms to its expected behaviour or
specification. For example, a system designer may wish to ensure that their
system is \emph{not} able to reach a problematic configuration: a lift system
should not be able to be in the configuration where the doors are open
\emph{and} the lift is moving.

Due to their complex and often non-intuitive behaviour, concurrent systems are
frequently scrutinised using automated model checking. Indeed, the vast number
of possible interleavings of component behaviour, even for simple systems,
makes it difficult to \emph{manually} reason about the (lack of a certain)
behaviour of the global system.

\begin{figure}[ht]
\centering
\begin{tikzpicture}[pnb, node distance=3cm]
    \node[pnbplace] (p00) [pnblabel=Ready$_1$,tokens=1] {};

    \node[pnbplace] (tok) [tokens=1, below right=of p00, rotate=-90] {};

    \node[pnbplace] (p01) [pnblabel=Working$_1$, below left=of tok, rotate=180] {};
    \node[pnbplace] (p10) [pnblabel=below:Ready$_2$,above right=of tok, tokens=1, rotate=180] {};
    \node[pnbplace] (p11) [pnblabel=below:Working$_2$, below right=of tok] {};

    \labelledpnbarr[p00p01]{p00.out}{p01.in}{}{red, bend left=45}{pos=0.8}
    \labelledpnbarr[p01p00]{p01.out}{p00.in}{}{ao, bend left=45}{pos=0.5}
    \labelledpnbarr[p10p11]{p10.out}{p11.in}{}{blue, bend right=45}{pos=0.8}
    \labelledpnbarr[p11p10]{p11.out}{p10.in}{}{bend right=45}{pos=0.5}

    \draw (tok.out) edge[pnbarr, red, o-90i45] (p00p01);
    \draw (p01p00) edge[pnbarr, ao, o60i90] (tok.in);
    \draw (tok.out) edge[pnbarr, blue, o-90i135] (p10p11);
    \draw (p11p10) edge[pnbarr, o120i90] (tok.in);

    \drawBoundaries{0}{0}
\end{tikzpicture}
\caption{Example mutual exclusion system modelled as a Petri net}
\label{fig:mutexNet}
\end{figure}

A prevalent modelling tool for concurrent or distributed systems is \emph{Petri
nets}. With an intuitive graphical presentation, yet rigorous underlying
mathematical definition, Petri nets are powerful, yet accessible. As an
example, the net illustrated in~\figref{fig:mutexNet} models a \emph{mutual
exclusion} system, where two workers compete to obtain the token required to
let them work. Either the first worker (red transition) or second (blue
transition) can start first, but they cannot start at the same time. Once
either worker has started, the other cannot. Once the worker has
finished, it replaces the token, with the green or black transitions, allowing
either worker to restart. An obvious property to check of this model is that
both workers cannot be started at the same time.

For small system models, it is often sufficient to give the system
specification in a monolithic manner: the global structure can be
specified in one. However, for anything other than the most basic of systems,
this approach is infeasible --- instead, a \emph{compositional}, or
component-wise approach is often used. In a component-wise approach, a
system is designed as a collection of logical \emph{components}, which are
composed to form the global system. By building systems in this way,
designers need only reason about local behaviour of components and the
interactions between components --- the system is structured as a
collection of maximally independent, decoupled entities.

The very reason to use automated model checking is to avoid having to explore
large statespaces or check properties by hand. However, model checking is not
immune to statespace explosion, and indeed, naively applying a compositional
approach to system specification does not help alleviate the problem. For
loosely coupled components, the numerous possible \emph{interleavings} of their
behaviour leads to large statespaces being generated and explored.  While
existing model checking techniques may allow the specification of nets in terms
of components, the techniques used for checking properties of the system all
consider the composed, \emph{global} net---their approach is \emph{monolithic}.

\newcommand{\drawComp}[2]{%
    \pgfmathtruncatemacro{\prev}{#1 - 1}
    \node (p#1-2) [anchor=north, right=of p\prev-2] {\rotatebox{90}{$\cdots$}};
    \node[pnbplace] (p#1-1) [rotate=-90, above=of p#1-2, pnblabel=$\aPlace_{1,#2}$] {};
    \node[pnbplace] (p#1-0) [rotate=-90, above=of p#1-1, tokens=1, pnblabel=$\aPlace_{0,#2}$] {};
    \node[draw=blue, pnbplace, pnblabel=$\aPlace_{\bN,#2}$] (p#1-3) [rotate=-90, below=of p#1-2] {};

    \labelledpnbarr{p#1-0.out}{p#1-1.in}{}{relative}{}
    \labelledpnbarr{p#1-1.out}{p#1-2}{}{relative}{}
    \labelledpnbarr{p#1-2}{p#1-3.in}{}{relative}{}
}%
\begin{figure}[ht]
\centering
\begin{tikzpicture}[pnb]
    \coordinate (p-1-2);
    \drawComp{0}{0}
    \drawComp{1}{1}
    \node (p2-2) [right=of p1-2] {};
    \path let \p1 = (p2-2), \p2=($(p0-0)!.5!(p0-3)$) in node (hdots) at (\x1, \y2) {$\cdots$};
    \drawComp{3}{\aN}

    \drawBoundaries{0}{0}
\end{tikzpicture}
\caption{System of components with no interaction}
\label{fig:componentSystem}
\end{figure}

As an extreme example of the statespace explosion problem, consider a system
formed of $\aN$ components, each of which is able to reach $\bN$ local states
and \emph{does not} interact with the other components; such a system has a
reachable statespace with $\bN^\aN$ states. An example of this system is
illustrated in~\figref{fig:componentSystem}. Now consider checking reachability
of the global configuration that places a single token in each of the blue
highlighted places; intuitively, to reach this global state, we must have fired
\emph{all} the transitions in the net, in some order that respects the
individual components' dependency ordering. Due to interleaving, there are a
large number of possible ways to reach the target state: if a reachability
checker uses a breadth-first search strategy, or generates the entire
statespace, a large majority of these interleavings will be needlessly
explored.

An alternative approach is to take a \emph{localised} view: the \emph{global}
target state can be considered as a \emph{local} target state for \emph{each
component} --- to reach the global target state, each component must reach its
local target state, whilst correctly interacting with the other components.
Indeed, from the knowledge of the interactions each component must make to
reach its local target marking, we can check the compatibility of these
component interactions, and determine if the global marking is reachable. For
the example system we are considering, there are no interactions between
components, so we simply check that each component can reach its local target
state. By this change of view, we avoid exploring all possible interleavings,
instead checking reachability of the components \emph{in isolation}.
Furthermore, since the components and target markings are repeated, we can
check reachability in \emph{a single component} and deduce that the global
system \emph{can} reach its target marking: a single component is able to reach
its target marking without interacting with its neighbours, and all $\aN$
components are the same.

However, in order to use a localised, component-based approach for Petri nets,
we must be able to specify the \emph{components} of a global net system. To do
so, we use an \emph{algebra of Petri nets}: we specify complex systems in terms
of compositions of (small) component nets. By using an algebra that allows
\emph{partial specification} of transitions by connecting to \emph{boundary
ports}, the full structure of composite nets can be built up through
composition. As components are composed, the transitions connected to their
boundary ports are fused to form fully-specified transitions.  By giving a
semantics that represents component reachability, while recording the
corresponding boundary interactions, we can capture the reachability of the
entire system, \emph{without} first generating the composite net. Furthermore,
we can \emph{reduce} the representation of a component's behaviour, while
\emph{preserving} the representation of the interactions necessary to reach the
target marking; when we compose the (reduced) component semantics, we
\emph{represent} global reachability, \emph{without} generating the full
statespace.

In other words, by taking advantage of information about the structure of the
components, and how they are connected, we can check reachability vastly more
efficiently.

Traditional approaches to model checking such systems instead take a global
view, operating on the composite structure, and essentially ignoring any
information about \emph{how the system was formed}.

In this thesis, we investigate and advocate compositional system specification
and an alternative approach to reachability checking that \emph{does} use the
structural compositional information to its advantage, in order to vastly
improve efficiency in many examples.
