An alternative approach to aid the minimisation of a Petri net's reachability
graph is to directly minimise or quotient the Petri net, intrinsically reducing
its reachability graph. By removing redundant transitions or places, the net
came become structurally smaller and thus lead to a smaller statespace to
explore. Indeed, such simple fusion/elimination rules were discussed by
Murata~\cite{Murata1989}, which were used to quotient Petri nets, whilst
preserving their behavioural properties.

{Esparza and Schr\"oter}~\cite{Esparza2001a} introduced a technique for
checking LTL properties of 1-safe nets that combined global and local reduction
techniques. Globally, dead places/transitions (places that are never marked and
thus never enable their outgoing transitions) and implicit places (places that
are redundant w.r.t the enabling/disabling of a particular outgoing transition)
are identified and removed by the (linear programming) marking equations they
satisfy. Locally, generalisations of Berthelot's~\cite{Berthelot1986}
\emph{agglomeration} rules are used to remove certain intermediate places and
their transitions, by bypassing such places. {Haddad et al.}~\cite{Haddad2006}
also generalised Berthelot's approach, but without the restriction to 1-safe
nets. Furthermore, by deriving structural conditions for reduction from the
behavioural properties (i.e.  firing sequences) to be preserved, {Haddad et
al.} were able to use weaker structural conditions, enhancing the effect of the
reduction.

More recently, Rakow~\cite{Rakow2009} also presented an approach for checking
LTL properties on 1-safe nets, but used a decomposition into a ``kernel'' net,
containing only the places mentioned in the LTL formula, and (reduced)
environment subnets recording the remainder of the net's influence on the
kernel.

Originally proposed by Olderog~\cite{Olderog1989}, the notion of
\emph{bisimulation} can be lifted from LTSs, to the places of a Petri net.
Olderog's approach did not in fact lead to well-defined bisimulations, as
{Autant et al.}~\cite{Autant1991} observed, while showing how to generalise the
definition, and give a correct notion of place-bisimulation for Petri nets.
Indeed, {Autant et al.} used the induced equivalence relation to define the
quotient of a net (i.e. a structurally smaller, but bisimilar net), however,
they gave no algorithm for calculating such quotients in order to reduce a
given net. In a later paper~\cite{Autant1994}, {Autant et al.} did provide a
simple algorithm to calculate such bisimulations, extending their approach to
labelled nets with $\tau$ (internal) transitions. {Schnoebelen and
Sidorova}~\cite{Schnoebelen2000} refined {Autant et al.}'s approach, showing
that by considering a set of markings as relevant (as opposed to all markings),
the effectiveness of the reduction can be improved.
