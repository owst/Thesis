Compositional algebraic systems have the property that behaviour of composite
systems is determined only by the behaviour of their component systems.  While
Petri nets are sometimes considered as being inherently non-compositional,
early work by Mazurkiewicz~\cite{Mazurkiewicz1988} defined a compositional
algebra of nets, based on fusion of named transitions. In the algebra of nets
we use in this thesis, Petri Nets With Boundaries,
(PNBs)~\cite{Sobocinski2010,Bruni2013} it is also transitions that are fused,
however, the composition operations are quite different in nature: PNB
compositions are not commutative, and operate on transitions connected to local
\emph{boundary ports} rather than by (global) name. Indeed, Mazurkiewicz's
composition is a commutative parallel composition in the spirit of CSP and CCS.

Similar operations were used for the development of the Petri Box calculus
(PBC), a process algebra of labelled Petri nets~\cite{Best1992}, with a
compositional Petri net semantics. The PBC features two kinds of composition:
the first is a control-flow-style sequential composition that utilises certain
places labelled as \emph{entry} or \emph{exit} places in order to enforce a
computation order, the second a synchronising composition, introducing new
transitions based on the \emph{global} fusion of transitions with conjugate
labels, whilst preserving the original transitions. The composition operations
of PNBs, instead, are closely related to the geometry of nets, with no
control-flow style composition and only \emph{local} synchronisation (that
fuses transitions) through shared boundary ports. {Best and
Koutny}~\cite{Koutny1999} later introduced the Box algebra, a generalisation of
the PBC, giving a (compositional) operational and denotational semantics and a
general view of various composition operations as specific instances of generic
transition refinement/relabelling operations. A tutorial-style
overview of the Box algebra can be found in~\cite{Best1995}.

Reisig's~\cite{Reisig2009} simple composition of nets (SCN) is an
elegantly simple way of composing nets and is conceptually quite close to our
work. His nets, similarly to PNBs, have left and right interfaces that are made
up of ports and ought not to be confused with notions of input and output,
rather reflecting the structural geometry of nets. Differently, in SCN the
interfaces typically expose places, whereas PNB interfaces expose only
transitions. Another difference is that in PNB, composition $N_1\mathrel{;}N_2$
is only defined when the right interface of $N_1$ is equal to the left
interface of $N_2$. While~\cite{Reisig2009} demonstrates that the operation is
very natural for composing real systems, the compositional semantic aspects of
the theory have not, so far, been developed.

Component-wise construction of nets was emphasised by
Kindler~\cite{Kindler1997}, who worked with a partial order semantics. The
interfaces are a set of input and output places, which are connected with a
transition when composed. The semantics was shown to be compositional with
respect to this operation. Since the composition introduces additional
transitions, it is not always clear how to \emph{divide} a net into components
with input and output places. A similar approach was taken by {Baldan et
al.}~\cite{Baldan2001}, who introduced \emph{Open nets}: nets with certain
identified input and output places. Composition of Open nets is defined in
terms of a pushout in the category of Open nets, realised as joining a common
subnet. Another related approach is that of {Priese and
Wimmel}~\cite{Priese1998}, who use a combination of interface places \emph{and}
transitions to define \emph{Petri nets with interface}, with composition
operators that join interface places and transitions. An algebra of nets is
defined, with combinators to form complex nets from basic components.

In this thesis, we will give a compositional semantics to the algebra of Petri
nets with boundaries, by a translation into the algebra of non deterministic
finite automata with boundaries (\secref{sec:TNFA}). This algebra is an instance of the algebra of
Span(Graph)~\cite{Katis1997}, developed by Walters and collaborators: in fact, a translation from
Petri nets to this algebra was already present in~\cite{Katis1997a}. In more recent
work~\cite{Sobocinski2010, Bruni2011, Bruni2013}, the algebra of Span(Graph) was lifted to the
level of nets in a compositional way, and the resulting behavioural equivalences and connections
with process algebra were explored.
