Summarising, the contributions presented in this thesis are as follows:
\begin{enumerate}
    \item Categorial structure of PNBs and their semantics: we show the
        well-known property of compositionality in a new light, as the
        functoriality of a mapping between suitable categories.
    \item We introduce \emph{contextual PNBs}, adding \emph{read arcs}, which
        naturally model behaviour that \emph{non-destructively} reads the token
        state of a place.
    \item Type-checked specification language: we show that by using a suitable
        programming language, we can compositionally construct systems to be
        modelled using PNBs, whilst ensuring that only correct compositions are
        expressible.
    \item \emph{Compositional} statespace generation for PNB-specified systems:
        we show that the statespace of a PNB-specified system can be
        compositionally generated, and furthermore, used to check
        reachability, \emph{without} constructing the global net.
    \item We show that compositional specifications can be exploited, to attack
        the statespace explosion problem, and improve the efficiency of
        reachability checking of systems modelled using PNBs. We show that by
        considering \emph{weak language equivalence} of PNB semantics, we can
        reduce the representation size of PNB semantics, whilst ensuring global
        behaviour is preserved. Furthermore, \emph{memoisation} allows us to
        avoid repeated computations.
    \item Compositional specification of existing benchmarks, in more natural,
        component-wise style, with formal, explicit specification of repeated
        structure.
\end{enumerate}

These contributions have already been partially presented in three co-authored
papers~\cite{Sobocinski2013,Sobocinski2014,Rathke2014}. The individual contributions of the author
of this thesis are the majority of the programming/implementation effort, while the theory was a
joint effort.

The contributions demonstrated in this thesis are structured as follows:
