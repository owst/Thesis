\label{sec:symmetryreduction}
% What is it?
Another technique, known as \emph{symmetry reduction}, exploits symmetries in
the structure of a net (and thus its reachability graph): the goal is, roughly,
to build a reduced, quotiented reachability graph that satisfies the same
temporal properties as the original, with only one representative of each
equivalence class of states.

% Example
As a simple example of the symmetry-reduction technique, consider the input net
illustrated in \figref{fig:exampleNet}; the symmetry of this net is plain to
see: the net is formed of two identical components. The LTS semantics of this
net (omitting self-loops and labels for simplicity) is shown in
\figref{fig:LTSSemanticsOfExampleNet}. The statespace explosion problem is
self-evident: the semantics contains all interleavings of the concurrent
transitions. However, taking the symmetry of the input net into account, we may
identify sets of LTS states, giving the LTS semantics shown
in~\figref{fig:quotientedLTSSemantics}. Intuitively, since the ``components''
of \figref{fig:exampleNet} do not interact, we are free to ``permute'' them,
removing the difference e.g. between the left component firing all its
transitions before any in the right and vice-versa. Indeed, taking $[x]$ to be
the partition of states equivalent (under the particular symmetry) to $x$, we
have that marking $y$ is reachable in the net if there is a path from the state
representing the initial marking, $i$, to $y$ in the LTS if there is a path
from $[i]$ to $[y]$ in the quotiented LTS. Since the quotiented LTS has fewer
states and paths, it is more efficient to check reachability using this LTS,
compared to the full LTS.

\begin{figure}[ht]
\centering
\begin{tikzpicture}[node distance=2cm]
    \node[pnbplace] (p00) [pnblabel={$\aPlace_0$}, tokens=1, rotate=-90] {};
    \node[pnbplace] (p01) [pnblabel={$\aPlace_1$}, below=of p00, rotate=-90] {};
    \node[pnbplace] (p02) [pnblabel={$\aPlace_2$}, below=of p01, rotate=-90] {};
    \node[pnbplace] (p03) [pnblabel={$\aPlace_3$}, below=of p02, rotate=-90] {};

    \node[pnbplace] (p10) [pnblabel={$\bPlace_0$}, right=of p00, tokens=1, rotate=-90] {};
    \node[pnbplace] (p11) [pnblabel={$\bPlace_1$}, below=of p10, rotate=-90] {};
    \node[pnbplace] (p12) [pnblabel={$\bPlace_2$}, below=of p11, rotate=-90] {};
    \node[pnbplace] (p13) [pnblabel={$\bPlace_3$}, below=of p12, rotate=-90] {};

    \foreach \x in {0,1,2} {
        \pgfmathtruncatemacro{\next}{\x + 1}

        \foreach \y in {0,1}{
            \labelledpnbarr{p\y\x}{p\y\next}{}{relative}{}
        }
    }

    \drawBoundaries{0}{0}
\end{tikzpicture}
\caption{Example Petri net}
\label{fig:exampleNet}
\end{figure}

\newcommand{\betweenAnd}[3]{%
    \path (#1) edge node [swap] {} (#2);
    \path (#1) edge node        {} (#3);
}

\begin{figure}[ht]
\centering
\scalebox{0.75}{%
\begin{tikzpicture}[nfa]
    \node (00) [state] {$00$};

    \node (10) [state, below left=of 00] {$10$};
    \node (01) [state, below right=of 00] {$01$};

    \node (20) [state, below left=of 10] {$20$};
    \node (11) [state, below right=of 10] {$11$};
    \node (02) [state, below right=of 01] {$02$};

    \node (30) [state, below left=of 20] {$30$};
    \node (21) [state, below right=of 20] {$21$};
    \node (12) [state, below right=of 11] {$12$};
    \node (03) [state, below right=of 02] {$03$};

    \node (31) [state, below right=of 30] {$31$};
    \node (22) [state, below right=of 21] {$22$};
    \node (13) [state, below right=of 12] {$13$};

    \node (32) [state, below right=of 31] {$32$};
    \node (23) [state, below right=of 22] {$23$};

    \node (33) [state, below right=of 32] {$33$};

    \foreach \x/\y in {%
30/31,%
31/32,%
32/33,%
03/13,%
13/23,%
23/33%
}{\path (\x) edge node [swap] {} (\y);}

    \foreach \x/\y/\z in {%
00/10/01,%
10/20/11,%
20/30/21,%
01/11/02,%
11/21/12,%
02/12/03,%
21/31/22,%
12/22/13,%
22/32/23%
}{\betweenAnd{\x}{\y}{\z}}

\end{tikzpicture}
}
\caption{LTS semantics of the net in \figref{fig:exampleNet}}
\label{fig:LTSSemanticsOfExampleNet}
\end{figure}

\begin{figure}[ht]
\centering
\scalebox{0.75}{
\begin{tikzpicture}[nfa]
    \node (00) [state] {$00$};

    \node (10) [state, below=of 00] {$01,10$};

    \node (11) [state, below left=of 10] {$11$};
    \node (20) [state, below right=of 10] {$02,20$};

    \node (21) [state, below=of 11] {$12,21$};
    \node (30) [state, below=of 20] {$03,30$};

    \node (22) [state, below=of 21] {$22$};
    \node (31) [state, below=of 30] {$13,31$};

    \node (32) [state, below right=of 22] {$23,32$};
    \node (33) [state, below=of 32] {$33$};

    \foreach \x/\y in {%
00/10,%
10/11,%
10/20,%
20/30,%
30/31,%
31/32,%
20/21,%
21/31,%
11/21,%
21/22,%
22/32,%
32/33%
}{\path (\x) edge node [swap] {} (\y);}
\end{tikzpicture}
}
\caption{Quotient of LTS semantics in~\figref{fig:LTSSemanticsOfExampleNet}}
\label{fig:quotientedLTSSemantics}
\end{figure}


% History
The first technique for exploiting symmetry in model checking was due to
Starke, in the context of checking reachability of Petri
Nets~\cite{Starke1991}.  Essentially, the reachability graph construction
algorithm only considers new states (i.e.  markings) if an equivalent state
(w.r.t. the symmetry of the net) has not already been visited.

Further initial results on model checking using symmetry reduction were
obtained by {Clarke et al.}~\cite{Clarke1993a}, showing that symmetry reduction
preserves satisfaction of CTL* formulas (preservation of formulas being the key
property for the validity of symmetry reduction).  {Emerson and
Sistla}~\cite{Emerson1993} presented theoretical results in the area of
$\mu$-calculus model checking, while {Ip and Dill}~\cite{Ip1996} investigated
the more practical problem of how to identify symmetries to be
exploited.

% Difficulties

There are two key problems~\cite{Clarke1998} that must be solved in order to
use symmetries:
\begin{enumerate}
    \item How to determine that two states, $s$ and $s'$ belong to the same
        orbit (set of equivalent states), known as the \emph{orbit problem},
    \item How to determine the symmetry group, from the input model.
\end{enumerate}

Indeed, for explicit-state representations, one can exploit a \emph{canonical
representative} for each set of equivalent states. The transition relation is
then explored up-to representatives, ensuring that only one representative
state of each equivalence set is explored. Model checking using explicit-state
systems is efficient if it is efficient to compute a canonical representative,
or similarly, if two states are equivalent. Unfortunately, under arbitrary
symmetries, checking state equivalence is as hard as the graph isomorphism
problem~\cite{Clarke1998}, a computationally difficult problem that has no
P-time algorithm, yet has not been shown to be NP-complete~\cite{Read1977}.
Several successful explicit-state model checkers exploit
symmetry~\cite{Sistla2000,Leuschel2003,Holzmann1997}.

To determine the symmetry group, one approach is to explicitly specify the
symmetries in the input model, such as the \emph{scalarset} data-type in the
input language of {Ip and Dill}'s~\cite{Ip1996}, or components that may be
permuted. Alternatively, an implicit approach attempts to determine the
symmetry groups by examining the input model, for example,
Schmidt~\cite{Schmidt2000} uses a partition refinement algorithm, to build the
coarsest symmetry relation on input Petri nets that preserves the flow relation
of the net in order to calculate the symmetries of a Petri net. The complexity
of Schmidt's method is investigated by Junttila~\cite{Junttila2001}.
The performance of three alternative symmetry-reduction implementations is
contrasted by Schmidt~\cite{Schmidt2000c}, who examines the trade off between
reduction time and effect on reachability checking performance.

Summarising, for systems with inherent symmetry (i.e. those with collections
of similar sub-systems), symmetry reduction can provide large reductions in the
number of states that must be explored. However, the advantage can only be
realised if efficient methods of detecting symmetry and checking equivalence of
states under it are available.

Thorough survey articles covering symmetry-reduction have been compiled by both
{Wahl and Donaldson}~\cite{Wahl2010} and {Miller et al.}~\cite{Miller2006}.
