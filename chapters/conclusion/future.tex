\section{Future Work}

We now briefly discuss several avenues of work to extend the compositional
approach introduced in this thesis.

\begin{enumerate}[leftmargin=*]
    \item \emph{What is a good decomposition? What characterises good performance?}

        When discussing our example systems in \secref{sec:benchmarks}, we did
        not discuss \emph{how} we arrived at the particular decompositions. In
        practice, the decompositions were natural, and were generated by hand.
        However, an automated decomposition search might be preferred, which
        could detect and abstract out repeated components. In our original
        paper~\cite{Sobocinski2013} introducing our technique, we described a
        naive decomposition algorithm; unfortunately, in practice the
        decomposition algorithm tended to not obtain the natural \emph{by hand}
        decompositions, and did not give generally acceptable performance. We
        will further investigate decomposition as a method of applying our
        technique to existing \emph{monolithic} models. The size of a PNB's
        boundaries and the number of places it contains influences the size of
        the corresponding \TNFA{} and thus the performance of our technique. As
        per our pre-print paper~\cite{Rathke2013} the \emph{rank width} of the
        underlying hypergraph is important in determining behaviour; we will
        further investigate structurally characterisations of PNBs.

    \item \emph{How to produce witnesses?}

        A drawback of using $\tau$-closure and minimisation to combat the
        statespace explosion problem is that we cannot immediately produce
        witnessing transitions that confirm the yes/no answer to the
        reachability question. The difficulty would be to preserve the minimal
        information required to generate a witnessing transition sequence,
        without forcing all possibly valid sequences to be remembered.

    \item \emph{Directed exploration of statespaces}

        In the style of {Bonet et al.}~\cite{Bonet2008}, the conversion of a
        PNB expression to a \TNFA{} might be \emph{directed} towards
        sub-expressions with not-reachable markings. Recall that if any
        component's local marking is unreachable, the global marking is
        unreachable; if the evaluation can quickly identify such components,
        then it can quickly determine unreachability in the global net.

    \item \emph{Prevention of invalid intermediate behaviour}

        As noted in \secref{sec:poorexamples} and by {Graf and
        Steffan}~\cite{Graf1990} intermediate results may exhibit behaviours
        that are ultimately not performed in the composite system. Here, the
        ideas of contextual bisimulation~\cite{Larsen1987}, or interface
        components of {Clarke et al.}~\cite{Clarke1989} will help alleviate
        intermediate \TNFA{} sizes. However, the main difficulty will be
        determining the correct restrictions that the context should enforce
        --- for example, in the case of \tokenringSys{-}, how should the
        procedure determine that only a single token should be emitted from the
        context? Additionally, lazy evaluation of \TNFA{} may be helpful, such
        that only the structure that takes part in successful computation is
        evaluated. For example, since the token ring context only emits a
        single token, the token ring itself should not evaluate transitions
        that rely on more than one token being emitted.

\item \emph{Unfoldings/Merged processes for PNBs}
        \newcommand{\compl}[1]{\bar #1}

        Similarly to how an unfolding is a (suitably restricted) Petri net, we
        expect that with suitable restrictions, the unfolding of a PNB will
        itself be a PNB. Furthermore, we expect that compositionality should
        also hold for unfoldings of PNB components (with a suitable equivalence
        relation).

        We can foresee at least three subtle points that must be accounted for:
        \begin{enumerate}
            \item Unfoldings will likely require complement places, to record
                token \emph{absence}
            \item Composition of unfoldings will induce additional contention
            \item Equivalence must be suitably defined, isomorphism is likely
                too strong
        \end{enumerate}
        To demonstrate these points, consider the left PNB in
        \figref{fig:examplePNBs}; we postulate that its unfolding will
        necessarily contain a \emph{complement} place, $\compl\aPlace$, and
        transitions to ensure that when the $\aPlace$ is empty, $\compl\aPlace$
        is full and vice-versa. An unfolding is \emph{seeded} with a place
        for each place assigned a token by the initial marking; if there is no
        such places, the unfolding will be empty. Furthermore, there should be
        additional \emph{contention} between transitions of composed PNB
        unfoldings: consider synchronously composing the two leftmost
        unfoldings shown in~\figref{fig:unfoldingExamplePNBs}, the composition
        should not allow for example the dashed transition in one component
        unfolding to synchronise with a solid transition in the other
        component. Since unfoldings record the \emph{history} of a net,
        allowing such synchronisations would be akin to one component going
        back in time. Finally, the notion of equivalence would need to be
        suitably defined; using a naive approach, the composition of unfoldings
        is \emph{not} isomorphic to the unfolding of a composed net. For
        example, composing the two leftmost unfoldings
        in~\figref{fig:unfoldingExamplePNBs} should obtain an equivalent PNB to
        the rightmost PNB, which is the unfolding of the composite net of
        \figref{fig:examplePNBs}; immediately, the composition of unfoldings
        contains too many places, arising from the complement places
        $\compl\aPlace$ and $\compl\bPlace$.

        \begin{figure}[ht]
            \centering
            \begin{tikzpicture}[pnb]
                \node (p) [pnbplace, rotate=-90, pnblabel=below:$\aPlace$] {};
                \drawBoundaries{0}{2}

                \labelledpnbarr{p.out}{r2}{}{o-90i180}{}
                \labelledpnbarr{p.in}{r1}{}{o90i180}{}
            \end{tikzpicture}
            \hspace{1cm}
            \begin{tikzpicture}[pnb]
                \node (p) [pnbplace, rotate=90, tokens=1, pnblabel=below:$\bPlace$] {};
                \drawBoundaries{2}{0}

                \labelledpnbarr{p.out}{l1}{}{o90i0}{}
                \labelledpnbarr{p.in}{l2}{}{o-90i0}{}
            \end{tikzpicture}%
            \hspace{1cm}
            \begin{tikzpicture}[pnb]
                \node (p) [pnbplace, rotate=-90, pnblabel=below:$\aPlace$] {};
                \node (q) [pnbplace, rotate=90, tokens=1, pnblabel=below:$\bPlace$, right=of p] {};
                \drawBoundaries{0}{0}

                \labelledpnbarr{q.out}{p.in}{}{bend right=90}{}
                \labelledpnbarr{p.out}{q.in}{}{bend right=90}{}
            \end{tikzpicture}%
            \caption{Two example PNBs and their composition}
            \label{fig:examplePNBs}
        \end{figure}
        \begin{figure}[ht]
            \centering
            \begin{tikzpicture}[pnb]
                \node (p0) [pnbplace, rotate=-90, pnblabel=below:$\compl\aPlace$] {};
                \node (p1) [pnbplace, rotate=-90, below=of p0, pnblabel=below:$\aPlace$] {};
                \node (p2) [pnbplace, rotate=-90, below=of p1, pnblabel=below:$\compl\aPlace$] {};
                \node (p3) [pnbplace, rotate=-90, below=of p2, pnblabel=below:$\aPlace$] {};

                \node (p4) [below=of p3.out] {\rotatebox{90}{$\dots$}};
                \drawBoundaries{0}{2}

                \labelledpnbarr[p0p1]{p0.out}{p1.in}{}{relative}{}
                \draw (p0p1) edge[pnbarr, o-30i150] (r1);
                \labelledpnbarr[p1p2]{p1.out}{p2.in}{}{relative}{}
                \draw (p1p2) edge[pnbarr, o-30i150] (r2);
                \labelledpnbarr[p2p3]{p2.out}{p3.in}{}{relative, pnbarrstyle2}{}
                \draw (p2p3) edge[pnbarrstyle2, o45i-135] (r1);
                \labelledpnbarr[p3p4]{p3.out}{p4.north}{}{relative, pnbarrstyle2}{}
                \draw (p3p4) edge[pnbarrstyle2, o45i-135] (r2);

            \end{tikzpicture}
            \hspace{1cm}
            \begin{tikzpicture}[pnb]
                \node (p0) [pnbplace, rotate=-90, pnblabel=$\bPlace$] {};
                \node (p1) [pnbplace, rotate=-90, below=of p0, pnblabel=$\compl\bPlace$] {};
                \node (p2) [pnbplace, rotate=-90, below=of p1, pnblabel=$\bPlace$] {};
                \node (p3) [pnbplace, rotate=-90, below=of p2, pnblabel=$\compl\bPlace$] {};

                \node (p4) [below=of p3.out] {\rotatebox{90}{$\dots$}};
                \drawBoundaries{2}{0}

                \labelledpnbarr[p0p1]{p0.out}{p1.in}{}{relative}{}
                \draw (p0p1) edge[pnbarr, o-150i30] (l1);
                \labelledpnbarr[p1p2]{p1.out}{p2.in}{}{relative}{}
                \draw (p1p2) edge[pnbarr, o-150i30] (l2);
                \labelledpnbarr[p2p3]{p2.out}{p3.in}{}{relative, pnbarrstyle2}{}
                \draw (p2p3) edge[pnbarrstyle2, o135i-45] (l1);
                \labelledpnbarr[p3p4]{p3.out}{p4.north}{}{relative, pnbarrstyle2}{}
                \draw (p3p4) edge[pnbarrstyle2, o135i-45] (l2);
            \end{tikzpicture}%
            \hspace{1cm}
            \begin{tikzpicture}[pnb]
                \node (p0) [pnbplace, rotate=-90, pnblabel=$\bPlace$] {};
                \node (p1) [pnbplace, rotate=-90, below=of p0, pnblabel=$\aPlace$] {};
                \node (p2) [pnbplace, rotate=-90, below=of p1, pnblabel=$\bPlace$] {};
                \node (p3) [pnbplace, rotate=-90, below=of p2, pnblabel=$\aPlace$] {};

                \node (p4) [below=of p3.out] {\rotatebox{90}{$\dots$}};
                \drawBoundaries{0}{0}

                \labelledpnbarr{p0.out}{p1.in}{}{relative}{}
                \labelledpnbarr{p1.out}{p2.in}{}{relative}{}
                \labelledpnbarr{p2.out}{p3.in}{}{relative}{}
                \labelledpnbarr{p3.out}{p4.north}{}{relative}{}
            \end{tikzpicture}%
            \caption{Unfoldings of the PNBs in \figref{fig:examplePNBs}}
            \label{fig:unfoldingExamplePNBs}
        \end{figure}
\item \emph{Applying existing statespace avoidance techniques to PNBs}

    Since our approach to avoiding statespace explosion (minimisation w.r.t.
    weak language-equivalence) is orthogonal to existing approaches discussed
    in \secref{sec:related}, we will investigate integrating these methods, to
    improve the performance of our technique, especially when fixed points of
    behaviour are \emph{not found}.

\end{enumerate}
