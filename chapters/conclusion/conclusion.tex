\chapter{Conclusion}\label{chp:conclusion}

In this thesis, we investigated and advocated compositional system
specification and an alternative approach to reachability checking that uses
the structural compositional information to its advantage, in order to vastly
improve efficiency in many examples.

The contributions presented in this thesis were:
\begin{enumerate}
    \item The elucidation of the categorical structure of PNBs and their
        semantics: we showed the well-known property of compositionality in a
        new light, as an instance of functoriality for suitable categories.
    \item The introduction of \emph{contextual PNBs}, which naturally model
        behaviour that \emph{non-destructively} reads the token state of a
        place.
    \item The motivation and introduction of a type-checked specification
        programming language for PNBs, which ensures that only correct
        compositions are expressible.
    \item We demonstrated \emph{compositional} statespace generation for PNB
        systems, and that it can be used to check reachability, \emph{without}
        constructing the global net.
    \item We showed that compositional specifications can be exploited, to
        attack the statespace explosion problem, and improve the efficiency of
        reachability checking of systems modelled using PNBs. We showed that by
        considering \emph{weak language equivalence} of PNB semantics, we are
        able to reduce the representation size of PNB semantics, whilst
        ensuring global behaviour is preserved. We demonstrated that
        \emph{memoisation} allows us to avoid repeated computation.
    \item We gave compositional specification of existing benchmarks in a more
        natural component-wise specification, with  explicit specification of
        repeated structure.
\end{enumerate}
Many of these contributions had been introduced in the author's papers
\cite{Sobocinski2013,Sobocinski2014,Rathke2014}.

\chpref{chp:prelims} contained the required preliminaries. In
\chpref{chp:catStructure}, the categorical structure of PNBs and the LTS that
form their semantics was presented, exposing the notion of compositionality as
functoriality. \chpref{chp:benchmarksAndLang} introduced the example systems
that we used to demonstrate and evaluate our technique, and a specification DSL
that uses a static type system to ensure that only valid component-wise
specifications can be constructed. In \chpref{chp:compChecking} we introduced a
compositional technique for generating the statespace of systems specified
using our DSL, and thus checking marking reachability. We proved the technique
correct and give some example timings of a tool implementing the technique.
\chpref{chp:improveEfficiency} showed how to exploit the fact that language
equivalence is a congruence to vastly improve the performance of our
reachability-checking technique, introducing the notion of internal behaviour
that we \emph{ignore} in order to aggressively prune statespace. We proved the
more-efficient algorithm correct. In \chpref{chp:comparisonAndDiscussion}, we
discussed the implementation of our technique, and compared and discussed its
performance relative to current state-of-the-art tools.

\subimport{}{future}
